\section{Mathematical framework}

%% --------------------------------------------------------------------- %%
%% Mathtools
\subsection{General tools}

The macros and functions of this paragraph are defined in \verb!pnl_mathtools.h!.

\subsubsection{Constants} A few mathematical constants are provided by the
library. Most of them are actually already defined in {\tt math.h}, {\tt
  values.h} or {\tt limits.h} and a few others have been added.
\begin{describeconst}
  \constentry{M_E}{$e^1$}
  \constentry{M_LOG2E}{$\log_2 e$}
  \constentry{M_LOG10E}{$\log_{10} e$}
  \constentry{M_LN2}{$\log_e 2$}
  \constentry{M_LN10}{$\log_e 10$}
  \constentry{M_PI}{$\pi$}
  \constentry{M_2PI}{$2 \pi$}
  \constentry{M_PI_2}{$\pi/2$}
  \constentry{M_PI_4}{$\pi/4$}
  \constentry{M_1_PI}{$1/\pi$}
  \constentry{M_2_PI}{$2/\pi$}
  \constentry{M_2_SQRTPI}{$2/\sqrt{\pi}$}
  \constentry{M_SQRT2PI}{$sqrt{2\pi}$}
  \constentry{M_SQRT2}{$\sqrt{2}$}
  \constentry{M_EULER}{$\gamma = \lim_{n \rightarrow \infty } \left( \sum_{k=1}^{n} \frac {1}{k} - \ln(n) \right)$}
  \constentry{M_SQRT1_2}{$1/\sqrt{2}$}
  \constentry{M_1_SQRT2PI}{$1/\sqrt{2 \pi}$}
  \constentry{M_SQRT2_PI}{$\sqrt{2/\pi}$}
  \constentry{INT_MAX}{$2147483647$}
  \constentry{MAX_INT}{INT_MAX}
  \constentry{DBL_MAX}{$1.79769313486231470e+308$}
  \constentry{DOUBLE_MAX}{DBL_MAX}
  \constentry{DBL_EPSILON}{$2.2204460492503131e-16$}
  \constentry{PNL_NEGINF}{$-\infty$}
  \constentry{PNL_POSINF}{$+\infty$}
  \constentry{PNL_INF}{$+\infty$}
  \constentry{NAN}{Not a Number}
\end{describeconst}

\subsubsection{A few macros}
\begin{itemize}
\item \describemacro{PNL_IS_ODD}{int n}
  \sshortdescribe Returns $1$ if \var{n} is odd and $0$ otherwise.
\item \describemacro{PNL_IS_EVEN}{int n}
  \sshortdescribe Returns $1$ if \var{n} is even and $0$ otherwise.
\item \describemacro{PNL_ALTERNATE}{int n}
  \sshortdescribe Returns $(-1)^{\var{n}}$.
\item \describemacro{MIN}{x,y}
  \sshortdescribe Returns the minimum of \var{x} and \var{y}.
\item \describemacro{MAX}{x,y}
  \sshortdescribe Returns the maximum of \var{x} and \var{y}.
\item \describemacro{ABS}{x}
  \sshortdescribe Returns the absolute value of \var{x}.
\item \describemacro{SIGN}{x}
  \sshortdescribe Returns the sign of \var{x} (-1 if x < 0, 0 otheriwse).
\item \describemacro{SQR}{x}
  \sshortdescribe Returns \var{$x^2$}.
\item \describemacro{CUB}{x}
  \sshortdescribe Returns \var{$x^3$}.
\end{itemize}

\subsubsection{Functions}
\begin{itemize}
\item \describefun{int}{intapprox}{double s}
  \sshortdescribe Returns the nearest integer with the convention ({\tt
  intapprox(1.5)=1}). This function is similar to the \var{round} function
  (provided by the C library) but the result is typed as an integer instead of a
  double.

\item \describefun{double}{trunc}{double s}
  \sshortdescribe Returns the nearest integer not greater than the absolute
  value of \var{s}. This function is part of C99.

\item \describefun{double}{Cnp}{int n, int p}
  \sshortdescribe Computes the binomial coefficient $\binom{n}{p}$ in double
  precision.

\item \describefun{double}{pnl_fact}{int n}
  \sshortdescribe Computes $n! = \Gamma(n+1)$ in double precision.

\item \describefun{double}{pnl_pow_i}{double x, int n}
  \sshortdescribe Computes $x^n$ for $n \in \N$ using a squaring method.

\item \describefun{double}{lgamma}{double x}
  \sshortdescribe Computes $\log(\Gamma(x))$. This function is part of C99.

\item \describefun{double}{tgamma}{double x}
  \sshortdescribe Computes $\Gamma(x)$. This function is part of C99.

\item \describefun{void}{pnl_qsort}{void *a, int n, int es, int lda, int *t,
  int ldt, int use_index, int (*cmp)(void const *, void const *)}
  \sshortdescribe Sorts the array \var{a} using the comparison function
  \var{cmp}. \var{n} is the number of elements in \var{a}, each element being of
  size \var{es}. \var{t} is an array of integers of length \var{n} used to store
  the permutation when \var{use_index=TRUE}. \var{lda} and \var{ldt} are the
  leading dimensions of the arrays \var{a} and \var{t} and are used to sort
  matrices column-wise. 

\item \describefun{double}{pnl_nan}{}
  \sshortdescribe Returns \var{Nan}

\item \describefun{double}{pnl_posinf}{}
  \sshortdescribe Returns \var{+ infinity} 

\item \describefun{double}{pnl_neginf}{}
  \sshortdescribe Returns \var{- infinity} 

\item \describefun{int}{pnl_isnan}{double x}
  \sshortdescribe Returns \var{1} if \var{x = Nan}

\item \describefun{int}{pnl_isfinite}{}
  \sshortdescribe Returns \var{1} if \var{x != Inf}

\item \describefun{int}{pnl_isinf}{}
  \sshortdescribe Returns \var{+1} if \var{x = +Inf}, \var{-1} if \var{x = -Inf},
  \var{0} otherwise.
\end{itemize}



% --------------------------------------------------------------------- %
% Complex
\subsection{Complex numbers}
\subsubsection{Short Description}

The complex type and related functions are defined in the header
\verb!pnl_complex.h!.\\

The first native implementation of complex numbers in the C language appeared in
C99, which is unfortunately not available on all platforms. For this reason, we
provide here an implementation of complex numbers.

\begin{verbatim}
typedef struct {
    double r; /*!< real part */
    double i; /*!< imaginary part */
} dcomplex;
\end{verbatim}


\subsubsection{Constants}

\begin{describeconst}
  \constentry{CZERO}{$0$ as a complex number}
  \constentry{CONE}{$1$ as a complex number}
  \constentry{CI}{$I$ the unit complex number}
\end{describeconst}

\subsubsection{Functions}
\begin{itemize}
\item \describefun{double}{Creal}{\refstruct{dcomplex} z}
  \sshortdescribe $ \mathrm{R}(z) $ 

\item \describefun{double}{Cimag}{\refstruct{dcomplex} z}
  \sshortdescribe  $ \mathrm{Im}(z) $  

\item \describefun{dcomplex}{Cadd}{\refstruct{dcomplex} z, \refstruct{dcomplex} b}
  \sshortdescribe \var{ z+b }  

\item \describefun{dcomplex}{CRadd}{\refstruct{dcomplex} z, double b}
  \sshortdescribe \var{ z+b }  

\item \describefun{dcomplex}{RCadd}{double b, \refstruct{dcomplex} z}
  \sshortdescribe \var{ b+z }  

\item \describefun{dcomplex}{Csub}{\refstruct{dcomplex} z, \refstruct{dcomplex} b}
  \sshortdescribe \var{ z-b }

\item \describefun{dcomplex}{CRsub}{\refstruct{dcomplex} z, double b}
  \sshortdescribe \var{ z-b }

\item \describefun{dcomplex}{RCsub}{double b, \refstruct{dcomplex} z}
  \sshortdescribe \var{ b-z }

\item \describefun{dcomplex}{Cminus}{\refstruct{dcomplex} z}
  \sshortdescribe \var{ -z }  

\item \describefun{dcomplex}{Cmul}{\refstruct{dcomplex} z, \refstruct{dcomplex} b}
  \sshortdescribe \var{ z*b }  

\item \describefun{dcomplex}{RCmul}{double x, \refstruct{dcomplex} z}
  \sshortdescribe \var{ x*z }

\item \describefun{dcomplex}{CRmul}{\refstruct{dcomplex} z, double x}
  \sshortdescribe \var{ z * x }

\item \describefun{dcomplex}{CRdiv}{\refstruct{dcomplex} z, double x}
  \sshortdescribe \var{ z/x }

\item \describefun{dcomplex}{RCdiv}{double x, \refstruct{dcomplex} z}
  \sshortdescribe \var{ x/z }

\item \describefun{dcomplex}{Complex}{double x, double y}
  \sshortdescribe \var{x + i y}  

\item \describefun{dcomplex}{Complex_polar}{double r, double theta}
  \sshortdescribe  \var{ r exp(i theta) }  

\item \describefun{dcomplex}{Conj}{\refstruct{dcomplex} z}
  \sshortdescribe $\overline{z}$  

\item \describefun{dcomplex}{Cinv}{\refstruct{dcomplex} z}
  \sshortdescribe \var{ 1/z }  

\item \describefun{dcomplex}{Cdiv}{\refstruct{dcomplex} z, \refstruct{dcomplex} w}
  \sshortdescribe \var{ z/w }  

\item \describefun{double}{Csqr_norm}{\refstruct{dcomplex} z}
  \sshortdescribe $ \mathrm{Re}(z)^2 + \mathrm{im}(z)^2 $  

\item \describefun{double}{Cabs}{\refstruct{dcomplex} z}
  \sshortdescribe \var{|z|}  

\item \describefun{dcomplex}{Csqrt}{\refstruct{dcomplex} z}
  \sshortdescribe \var{ sqrt(z) },  square root (with positive real part)  

\item \describefun{dcomplex}{Clog}{\refstruct{dcomplex} z}
  \sshortdescribe \var{log(z)}  

\item \describefun{dcomplex}{Cexp}{\refstruct{dcomplex} z}
  \sshortdescribe \var{ exp(z) }  

\item \describefun{dcomplex}{CIexp}{double t}
  \sshortdescribe \var{ exp( it ) }

\item \describefun{dcomplex}{Cpow}{\refstruct{dcomplex} z, \refstruct{dcomplex} w}
  \sshortdescribe $ z^w$, power function  

\item \describefun{dcomplex}{Cpow_real}{\refstruct{dcomplex} z, double x}
  \sshortdescribe $ z^x$, power function  

\item \describefun{dcomplex}{Ccos}{\refstruct{dcomplex} z}
  \sshortdescribe \var{ cos(g)}  

\item \describefun{dcomplex}{Csin}{\refstruct{dcomplex} z}
  \sshortdescribe \var{sin(g)}  

\item \describefun{dcomplex}{Ctan}{\refstruct{dcomplex} z}
  \sshortdescribe \var{tan(z)}

\item \describefun{dcomplex}{Ccotan}{\refstruct{dcomplex} z}
  \sshortdescribe \var{cotan(z)}

\item \describefun{dcomplex}{Ccosh}{\refstruct{dcomplex} z}
  \sshortdescribe \var{ cosh(g)}  

\item \describefun{dcomplex}{Csinh}{\refstruct{dcomplex} z}
  \sshortdescribe \var{sinh(g)}  

\item \describefun{dcomplex}{Ctanh}{\refstruct{dcomplex} z}
  \sshortdescribe $\tanh(z) = \frac{1 - \expp{-2z} }{1 + \expp{-2z} }$  

\item \describefun{dcomplex}{Ccotanh}{\refstruct{dcomplex} z}
  \sshortdescribe $\cotanh(z) = \frac{1 + \expp{-2z} }{1 - \expp{-2z} }$  

\item \describefun{double}{Carg}{\refstruct{dcomplex} z}
  \sshortdescribe \var{arg(z) }

\item \describefun{dcomplex}{Ctgamma}{\refstruct{dcomplex} z}
  \sshortdescribe \var{ Gamma(z)}, the Gamma function  

\item \describefun{dcomplex}{Clgamma}{\refstruct{dcomplex} z}
  \sshortdescribe \var{ log(Gamma (z))}, the logarithm of the Gamma function

\item \describefun{void}{Cprintf}{\refstruct{dcomplex} z}
  \sshortdescribe Prints a complex number on the standard output
\end{itemize}

Most algebraic operations on complex numbers are implemented using the
following naming for the functions
\begin{itemize}
\item All these function names begin in {\tt C_op_}, 
\item The small letters {\tt a, b} denote two complex numbers whereas {\tt d} is a real number, 
\item The letter {\tt i} denotes the multiplication by the pure imagniary
  number $\imath$, 
\item The letter {\tt c} indicates that the next coming number is conjugated.
\item The letters {\tt p, m} denote the two standard operations {\it plus} and
  {\it minus} respectively.
\end{itemize}
For example C_op_idamcb is $\imath d \left( a - \overline{b} \right)$. So
functions are :
\begin{itemize}
\item \describefun{dcomplex}{C_op_apib}{\refstruct{dcomplex} a, \refstruct{dcomplex} b}
  \sshortdescribe $ a+\imath b  $.
\item \describefun{dcomplex}{C_op_apcb}{\refstruct{dcomplex} a, \refstruct{dcomplex} b}
  \sshortdescribe $ a+\overline{ b}  $.
\item \describefun{dcomplex}{C_op_amcb}{\refstruct{dcomplex} a, \refstruct{dcomplex} b}
  \sshortdescribe $ a-\overline{b}  $.
\item \describefun{dcomplex}{C_op_amib}{\refstruct{dcomplex} a, 
    \refstruct{dcomplex} b}
    \sshortdescribe \var{a - i b}
\item \describefun{dcomplex}{C_op_dapb}{double d, \refstruct{dcomplex} a, 
    \refstruct{dcomplex} b}
  \sshortdescribe $ d(a+ b)  $.
\item \describefun{dcomplex}{C_op_damb}{double d, \refstruct{dcomplex} a, 
    \refstruct{dcomplex} b}
  \sshortdescribe $ d(a-b)  $.
\item \describefun{dcomplex}{C_op_dapib}{double d, \refstruct{dcomplex} a, 
    \refstruct{dcomplex} b}
  \sshortdescribe $ d(a+\imath b)  $.
\item \describefun{dcomplex}{C_op_damib}{double d, \refstruct{dcomplex} a, 
    \refstruct{dcomplex} b}
  \sshortdescribe $ d(a-\imath b)  $.
\item \describefun{dcomplex}{C_op_dapcb}{double d, \refstruct{dcomplex} a, 
    \refstruct{dcomplex} b}
  \sshortdescribe $ d\left(a+\overline{b}\right)  $.
\item \describefun{dcomplex}{C_op_damcb}{double d, \refstruct{dcomplex} a, 
    \refstruct{dcomplex} b}
  \sshortdescribe $ d\left(a-\overline{b}\right)  $.
\item \describefun{dcomplex}{C_op_idapb}{double d, \refstruct{dcomplex} a, 
    \refstruct{dcomplex} b}
  \sshortdescribe $\imath d\left(a+b\right) $.
\item \describefun{dcomplex}{C_op_idamb}{double d, \refstruct{dcomplex} a, 
    \refstruct{dcomplex} b}
  \sshortdescribe $\imath  d\left(a-b\right) $.
\item \describefun{dcomplex}{C_op_idapcb}{double d, \refstruct{dcomplex} a, 
    \refstruct{dcomplex} b}
  \sshortdescribe $ \imath d\left(a+\overline{b}\right) $.
\item \describefun{dcomplex}{C_op_idamcb}{double d, \refstruct{dcomplex} a, 
    \refstruct{dcomplex} b}
  \sshortdescribe $ \imath  d\left(a-\overline{b}\right) $.
\end{itemize}
%% --------------------------------------------------------------------- %%




\subsection{Special functions}

The special function approximations are defined in the header \verb!pnl_specfun.h!.\\

Most of these functions rely on the {\it Cephes} library which uses its own
error mechanism which can be activated or deactivated using the two following
functions
\begin{itemize}
  \item \describefun{void}{pnl_deactivate_mtherr}{}
    \sshortdescribe Deactivates Cephes error mechanism
  \item \describefun{void}{pnl_activate_mtherr}{}
    \sshortdescribe Activates Cephes error mechanism
\end{itemize}


\subsubsection{Real Bessel functions}

\begin{itemize}
\item \describefun{double}{pnl_bessel_i}{double v, double x}
  \sshortdescribe   Modified Bessel function of the first
  kind of order \var{v}.
\item \describefun{double}{pnl_bessel_i_scaled}{double v, double x}
  \sshortdescribe   Modified Bessel function of the first
  kind of order \var{v} divided by $e^{|x|}$.
\item \describefun{double}{pnl_bessel_rati}{double v, double x}
  \sshortdescribe Ratio of modified Bessel functions of the first kind : $I_{v+1}(x) /
  I_v (x)$.
\item \describefun{double}{pnl_bessel_j}{double v, double x}
  \sshortdescribe    Bessel function of the first
  kind of order \var{v}.
\item \describefun{double}{pnl_bessel_j_scaled}{double v, double x}
  \sshortdescribe    Bessel function of the first
  kind of order \var{v}. Same function as \reffun{pnl_bessel_j}.
\item \describefun{double}{pnl_bessel_y}{double v, double x}
  \sshortdescribe   Modified Bessel function of the second
  kind of order \var{v}.
\item \describefun{double}{pnl_bessel_y_scaled}{double v, double x}
  \sshortdescribe   Modified Bessel function of the second
  kind of order \var{v}. Same function as \reffun{pnl_bessel_y}.
\item \describefun{double}{pnl_bessel_k}{double v, double x}
  \sshortdescribe   Bessel function of the third
  kind of order \var{v}.
\item \describefun{double}{pnl_bessel_k_scaled}{double v, double x}
  \sshortdescribe   Bessel function of the third
  kind of order \var{v} multiplied by $e^{x}$.
\item \describefun{dcomplex}{pnl_bessel_h1}{double v, double x}
  \sshortdescribe   Hankel function of the first kind of
  order \var{v}. 
\item \describefun{dcomplex}{pnl_bessel_h1_scaled}{double v, double x}
  \sshortdescribe  Hankel function of the first kind of order
  \var{v}  and divided by $e^{I x}$.
\item \describefun{dcomplex}{pnl_bessel_h2}{double v, double x}
  \sshortdescribe  Hankel function of the second kind of
  order \var{v}. 
\item \describefun{dcomplex}{pnl_bessel_h2_scaled}{double v, double x}
  \sshortdescribe  Hankel function of the second kind of
  order \var{v}  and multiplied by $e^{I x}$.
\end{itemize}

\subsubsection{Complex Bessel functions}

\begin{itemize}
\item \describefun{dcomplex}{pnl_complex_bessel_i}{double v, dcomplex z}
  \sshortdescribe  Complex Modified Bessel function of the first
  kind of order \var{v}.
\item \describefun{dcomplex}{pnl_complex_bessel_i_scaled}{double v, dcomplex z}
  \sshortdescribe  Complex Modified Bessel function of the first
  kind of order \var{v} divided by $e^{|Creal(z)|}$.
\item \describefun{dcomplex}{pnl_complex_bessel_rati}{double v, dcomplex x}
  \sshortdescribe Ratio of complex modified Bessel functions of the first kind : $I_{v+1}(x) /
  I_v (x)$.
\item \describefun{dcomplex}{pnl_complex_bessel_j}{double v, dcomplex z}
  \sshortdescribe  Complex  Bessel function of the first
  kind of order \var{v}.
\item \describefun{dcomplex}{pnl_complex_bessel_j_scaled}{double v, dcomplex z}
  \sshortdescribe  Complex  Bessel function of the first
  kind of order \var{v} divided by $e^{|Cimag(z)|}$.
\item \describefun{dcomplex}{pnl_complex_bessel_y}{double v, dcomplex z}
  \sshortdescribe  Complex Modified Bessel function of the second
  kind of order \var{v}.
\item \describefun{dcomplex}{pnl_complex_bessel_y_scaled}{double v, dcomplex z}
  \sshortdescribe  Complex Modified Bessel function of the second
  kind of order \var{v} divided by $e^{|Cimag(z)|}$.
\item \describefun{dcomplex}{pnl_complex_bessel_k}{double v, dcomplex z}
  \sshortdescribe  Complex Bessel function of the third
  kind of order \var{v}.
\item \describefun{dcomplex}{pnl_complex_bessel_k_scaled}{double v, dcomplex z}
  \sshortdescribe  Complex Bessel function of the third
  kind of order \var{v} multiplied by $e^{z}$.
\item \describefun{dcomplex}{pnl_complex_bessel_h1}{double v, dcomplex z}
  \sshortdescribe  Complex Hankel function of the first kind of
  order \var{v}. 
\item \describefun{dcomplex}{pnl_complex_bessel_h1_scaled}{double v, dcomplex z}
  \sshortdescribe  Complex  Hankel function of the first kind of order
  \var{v}  and divided by $e^{I z}$.
\item \describefun{dcomplex}{pnl_complex_bessel_h2}{double v, dcomplex z}
  \sshortdescribe  Complex  Hankel function of the second kind of
  order \var{v}. 
\item \describefun{dcomplex}{pnl_complex_bessel_h2_scaled}{double v, dcomplex z}
  \sshortdescribe  Complex  Hankel function of the second kind of
  order \var{v}  and multiplied by $e^{I z}$.
\end{itemize}

\subsubsection{Error functions}

\begin{itemize}
\item \describefun{double}{pnl_sf_erf}{double x}
  \sshortdescribe Computes $\frac{2}{\pi}\int_0^\infty \expp{-t^2} dt$
\item \describefun{double}{pnl_sf_erfc}{double x}
  \sshortdescribe Computes $1. - \frac{2}{\pi}\int_0^\infty \expp{-t^2} dt$
\item \describefun{double}{pnl_sf_log_erf}{double x}
  \sshortdescribe Computes $\log$ \reffun{pnl_sf_erf}$(x)$
\item \describefun{double}{pnl_sf_log_erfc}{double x}
  \sshortdescribe Computes $\log$ \reffun{pnl_sf_erfc}$(x)$
\end{itemize}

\subsubsection{Gamma functions}

For $x>0$, the Gamma Function is defined by
\begin{equation*}
  \Gamma(x)=\int_0^{\infty} \expp{-u} u^{x-1} du.
\end{equation*}

\begin{itemize}
\item \describefun{double}{pnl_sf_gamma}{double x}
  \sshortdescribe   Computes $\Gamma(x), x \geq 0$
\item \describefun{double}{pnl_sf_log_gamma}{double x}
  \sshortdescribe   Computes $\log(\Gamma(x)), x \geq 0$
\end{itemize}

\subsubsection{Incomplete Gamma functions}

For $a \in \R$ and $x>0$, the Incomplete Gamma Function is defined by
\begin{equation*}
  \Gamma(a, x)=\int_x^{\infty} \expp{-u} u^{a-1} du.
\end{equation*}
A relation similar to the one existing for the standard Gamma function holds
\begin{equation*}
  \Gamma\paren{a, x}= \frac{- x^{a} \expp{-x} + \Gamma (a+1, x)}{a}.
\end{equation*}
\begin{align*}
  \Gamma(a)&=\int_0^{\infty} u^{a-1} \expp{-u}du\\ 
  P(a, x) &= \frac{\Gamma(a) - \Gamma(a, x)}{\Gamma(a)} =
  \frac{1}{\Gamma(a)} \int_0^x u^{a-1} \expp{-u}  du\\ 
  Q(a, x) &= 1-P(a, x) =\frac{\Gamma(a, x)}{\Gamma(a)} =
  \frac{1}{\Gamma(a)} \int_x^{\infty} \expp{-u} u^{a-1} du. 
\end{align*}

\begin{itemize}
\item \describefun{double}{pnl_sf_gamma_inc}{double a, double x}
  \sshortdescribe   Computes $\Gamma(a, x), \quad a \in \R , x \geq 0$
\item \describefun{void}{pnl_sf_gamma_inc_P}{double a, double x}
  \sshortdescribe  Computes $P(a, x), \quad a > 0 , x \geq 0$
\item \describefun{void}{pnl_sf_gamma_inc_Q}{double a, double x}
  \sshortdescribe  Computes $Q(a, x), \quad a > 0 , x \geq 0$
\end{itemize}

\subsubsection{Exponential integrals}
For $x>0$ and $n \in \N$, the  function $E_n$ is defined by
\begin{equation*}
  E_n\paren{x}=\int_{1}^{\infty} \expp{-x u} u^{-n} du
\end{equation*}

This function is linked to the Incomplete Gamma function by 
\begin{equation*}
  E_n\paren{x}=\int_{x}^{\infty}
  \expp{-xu} (xu)^{-n} x^{n-1} d(xu)=x^{n-1} \int_{x}^{\infty}
  \expp{-t} t^{-n} dt =  x^{n-1}  \Gamma\paren{1-n, x}, 
\end{equation*}
from which we can deduce
\begin{equation*}
  n E_{n+1}(x)  =   \expp{-x} - x E_n(x).
\end{equation*}
For $n>1$, the series expansion is given by
\begin{equation*}
  E_n(x)=x^{n-1}
  \Gamma(1-n)+\recaco{-\frac{1}{1-n}+\frac{x}{2-n}-\frac{x^2}{2(3-n)}
    +\frac{x^3}{6(4-n)}-\dots}.     
\end{equation*}
The asymptotic behaviour is given by
\begin{equation*}
  E_n(x)=\frac{\expp{-x}}{x}\recaco{1-\frac{n}{x}+\frac{n(n+1)}{x^2}+\dots}. 
\end{equation*}
The special case $n=1$ gives 
\begin{equation*}
  E_1(x) = \int_x^\infty \frac{e^{-u}}{u}\, du, \quad |\mathrm{Arg}(x)| \ge \pi. 
\end{equation*}
For any complex number $x$ with positive real part, this can be written
\begin{equation*}
  E_1(x) = \int_1^\infty \frac{e^{-ux}}{u}\, du, \quad \Re(x) \ge 0. 
\end{equation*}
By integrating the Taylor expansion of $\expp{-t}/t$, and extracting the
logarithmic singularity, we can derive the following series representation for
$E_1(x)$, 
\begin{equation*}
  {E_1}(x) =-\gamma-\ln x-\sum_{k=1}^{\infty}\frac{(-1)^k x^k}{k\; k!}
  \qquad |\mathrm{Arg}(x)| < \pi. 
\end{equation*}
The function $E_1$ is linked to the exponential integral $Ei$
\begin{equation*}
  Ei(x)=\int_{-\infty}^x\frac{e^u}u\, du=-\int_{-x}^{\infty}
  \frac{e^{-u}}{u}\, du \quad \forall x \neq 0. 
\end{equation*}
The above definition  can be used for positive values of $x$, but the integral
has to be understood in terms of its Cauchy principal value, due to the
singularity of the integrand at zero.
\begin{equation*}
  {Ei}(-x) = -{E}_1(x) , \quad \Re(x) \ge 0.
\end{equation*}
We deduce, 
\begin{equation*}
  Ei(x) = \gamma + \ln x+ \sum_{k=1}^{\infty} \frac{x^k}{k\; k!}, \quad x>0.
\end{equation*}
For $x \in \R$ 
\begin{equation*}
  \Gamma(0, x)=\left\{
    \begin{array}{l}
      -Ei(-x)-\imath \pi  \quad  x<0, \\
      -Ei(-x) \quad x>0. 
    \end{array}\right.
\end{equation*}

\begin{itemize}
\item \describefun{double}{pnl_sf_expint_En}{int n, double x}
  \sshortdescribe   Computes for  $ n\geq 0, x \geq 0$ and $x>0$ if $n=0$ or $1$.
  \begin{equation*}
    E_n(x)= \int_1^{\infty} u^{-n} \expp{-x u} du.
  \end{equation*}
\item \describefun{double}{pnl_sf_expint_Ei}{double x}
  \sshortdescribe   Computes for  $x>0$  the principal value of
  \begin{equation*}
    Ei(x)= \int_{-\infty}^x \frac{\expp{t}}{t}  du.
  \end{equation*}
\end{itemize}

\subsubsection{Hypergeometric functions}

\begin{itemize}
\item \describefun{double}{pnl_sf_hyperg_2F1}{double a, double b, double c,
    double x}
\sshortdescribe Computes the Gauss hypergeometric function $2F1(a,b,c,x)$. 
\item \describefun{double}{pnl_sf_hyperg_1F1}{double a, double b, double x}
  \sshortdescribe Computes the hypergeometric function $1F1(a,b,x) = M(a,b,x)$
\item \describefun{double}{pnl_sf_hyperg_2F0}{double a, double b, double x}
  \sshortdescribe Computes the hypergeometric function $2F0(a,b,x)$. For
  $x<0$, $2F0 (a,b,x) = (-x)^{-a} U(a,1+a-b,-\frac{1}{x})$.
\item \describefun{double}{pnl_sf_hyperg_0F1}{double c, double x}
  \sshortdescribe Computes the hypergeometric function $0F1(c,x)$
\item \describefun{double}{pnl_sf_hyperg_U}{double a, double b, double x}
    \sshortdescribe Computes the confluent hypergeometric function $U(a,b,x)$
\end{itemize}

%% --------------------------------------------------------------------- %%
%% financial functions

\subsection{Financial functions}
\subsubsection{Short Description}

The financial functions are defined in the header \verb!pnl_finance.h!.\\

\subsubsection{Functions}


\begin{itemize}
\item
  \describefun{int}{pnl_cf_call_bs}{double s, double k, double T, double r, 
    double divid, double sigma, double \ptr ptprice, double \ptr ptdelta}
  \sshortdescribe Computes the price and delta of a call option $(\var{s} -
  \var{k})_+$ in the Black-Scholes model with volatility \var{sigma},
  instantaneous interest rate \var{r}, maturity \var{T} and dividend rate
  \var{divid}. The parameters \var{ptprice} and \var{ptdelta} are respectively
  set to the price and delta on output.

\item
  \describefun{int}{pnl_cf_put_bs}{double s, double k, double T, double r, 
    double divid, double sigma, double \ptr ptprice, double \ptr ptdelta}
  \sshortdescribe Computes the price and delta of a put option $(\var{k} - 
  \var{s})_+$ in the Black-Scholes model with volatility \var{sigma},
  instantaneous interest rate \var{r}, maturity \var{T} and dividend rate
  \var{divid}.  The parameters \var{ptprice} and \var{ptdelta} are respectively
  set to the price and delta on output.

\end{itemize}


\begin{itemize}
\item 
  \describefun{double}{pnl_bs_call}{double s, double k, double T,
    double r, double divid, double sigma}
  \sshortdescribe Computes the price of a call option with spot \var{s}
  and strike \var{k} in the Black-Scholes model with volatility \var{sigma},
  instantaneous interest rate \var{r}, maturity \var{T} and dividend rate
  \var{divid}.

\item 
  \describefun{double}{pnl_bs_put}{double s, double k, double T,
    double r, double divid, double sigma}
  \sshortdescribe Computes the price a put option with spot \var{s}
  and strike \var{k} in the Black-Scholes model with volatility \var{sigma},
  instantaneous interest rate \var{r}, maturity \var{T} and dividend rate
  \var{divid}.

\item 
  \describefun{double}{pnl_bs_call_put}{int iscall, double s, double k,
    double T, double r, double divid, double sigma}
  \sshortdescribe Computes the price of a put option if \var{iscall=0} or a
  call option if \var{iscall=1} with spot \var{s} and strike \var{k} in the
  Black-Scholes model with volatility \var{sigma}, instantaneous interest rate
  \var{r}, maturity \var{T} and dividend rate \var{divid}.

\item 
  \describefun{double}{pnl_bs_vega}{double s, double k, double T,
    double r, double divid, double sigma}
  \sshortdescribe Computes the vega of a put or call option with spot \var{s}
  and strike \var{k} in the Black-Scholes model with volatility \var{sigma},
  instantaneous interest rate \var{r}, maturity \var{T} and dividend rate
  \var{divid}.

\item 
  \describefun{double}{pnl_bs_gamma}{double s, double k, double T, double r,
    double divid, double sigma}
  \sshortdescribe Computes the gamma of a put or call option with spot \var{s}
  and strike \var{k} in the Black-Scholes model with volatility \var{sigma},
  instantaneous interest rate \var{r}, maturity \var{T} and dividend rate
  \var{divid}.
\end{itemize}

Practitioners do not speak in terms of option prices, but rather compare
prices in terms of their implied Black \& Scholes volatilities. So this
parameter is very useful in practice. Here, we propose two functions to
compute $\sigma_{impl}$ : the first one is for one up-let, maturity,
strike, option price.  the second function is for a list of strikes and
maturities, a matrix of prices (with strikes varying row-wise).

\begin{itemize}
\item 
  \describefun{double}{pnl_bs_implicit_vol}{int is_call, double Price, double s,
    double K, double T, double r, double divid, int *error}
  \sshortdescribe Computes the implied volatility of a put option if
  \var{iscall=0} or a call option if \var{iscall=1} with spot \var{s} and
  strike \var{K} in the Black-Scholes model with instantaneous interest rate
  \var{r}, maturity \var{T} and dividend rate \var{divid}. On output
  \var{error} is \var{OK} if the computation of the implied volatility succeeded
  or \var{FAIL} if it failed.

\item 
  \describefun{int}{pnl_bs_matrix_implicit_vol}{const \refstruct{PnlMatInt}
    \ptr iscall, const \refstruct{PnlMat} \ptr Price, double s, 
    double r, double divid, const \refstruct{PnlVect} \ptr K, 
    const \refstruct{PnlVect} \ptr T, \refstruct{PnlMat} \ptr Vol}
  \sshortdescribe Computes the matrix of implied volatilities \var{Vol(i,j)}
  of a put option if \var{iscall(i,j)=0} or a call option if
  \var{iscall(i,j)=1} with spot \var{s} and strike \var{K(j)} in the
  Black-Scholes model with instantaneous interest rate \var{r}, maturity
  \var{T(j)} and dividend rate \var{divid}. This function returns the number
  of failures, when everything succeeded it returns $0$.
\end{itemize}

