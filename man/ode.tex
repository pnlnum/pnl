\section{Ordinary differential equations}
\subsection{Overview}

To use these functionalities, you should include \verb!pnl/pnl_integration.h!.

These functions are designed for numerically solving $n-$dimensional first order
ordinary differential equation of the general form
\begin{equation*}
  \frac{dy_i}{dt}(t) = f_i(t, y_1(t), \cdots, y_n(t)) 
\end{equation*}
The system of equations is defined by the following structure
\describestruct{PnlODEFunc}
\begin{lstlisting}
typedef struct
{
  void (*F) (int neqn, double t, const double *y, double *yp, void *params);
  int neqn; 
  void *params;
} PnlODEFunc ;
\end{lstlisting}

\begin{itemize}
\item \describevar{int}{neqn} 
  \sshortdescribe Number of equations
\item \describevar{void \ptr}{params} 
  \sshortdescribe An untyped structure used to pass extra
  arguments to the function \var{f} defining the system
\item \describefun*{void}{(\ptr\ F)}{int neqn, double t, const double \ptr
    y, double \ptr yp, void \ptr params}
  \sshortdescribe After calling the fuction, \var{yp} should be defined as
  follows \var{yp_i = f_i(neqn, t, y, params)}. \var{y} and \var{yp} should be
  both of size \var{neqn}
\end{itemize}
We provide the following macro to evaluate a \refstruct{PnlODEFunc}
at a given point
\begin{lstlisting}
#define PNL_EVAL_ODEFUNC(Fstruct, t, y, yp) \
    (*((Fstruct)->F))((Fstruct)->neqn, t, y, yp, (Fstruct)->params)
\end{lstlisting}

\subsection{Functions}

\begin{itemize}
\item \describefun{int}{pnl_ode_rkf45}{\refstruct{PnlODEFunc} \ptr f, double
    \ptr y, double t, double t_out, double relerr, double abserr, int \ptr flag}
  \sshortdescribe This function computes the solution of the system defined by
  the \refstruct{PnlODEFunc} \var{f} at the point \var{t_out}. On input,
  $\var{(t,y)}$ should be the initial condition, \var{abserr,relerr} are the
  maximum absolute and relative errors for local error tests (at each step,
    \var{abs(local error)} should be less that \var{relerr * abs(y) + abserr}).
  Note that if \var{abserr = 0} or \var{relerr = 0}  on input, an optimal value
  for these variables is computed inside the function The function returns an
  error \var{OK} or \var{FAIL}. In case of an \var{OK} code, the \var{y}
  contains the solution computed at \var{t_out}, in case of a \var{FAIL} code,
  \var{flag} should be examined to determine the reason of the error. Here are
  the different possible values for \var{flag}
  \begin{itemize}
  \item \var{flag = 2} : integration reached \var{t_out}, it indicates
    successful return and is the normal mode for continuing integration.
 \item \var{flag = 3} : integration was not completed because relative error
   tolerance was too small. relerr has been increased appropriately for
   continuing.
 \item \var{flag = 4} : integration was not completed because more than 3000
   derivative evaluations were needed. this is approximately 500 steps.
 \item \var{flag = 5} : integration was not completed because solution vanished
   making a pure relative error test impossible. must use non-zero abserr to
   continue.  using the one-step integration mode for one step is a good way to
   proceed.
 \item \var{flag = 6} : integration was not completed because requested accuracy
   could not be achieved using smallest allowable stepsize. user must increase
   the error tolerance before continued integration can be attempted.
 \item \var{flag = 7} : it is likely that rkf45 is inefficient for solving this
   problem. too much output is restricting the natural stepsize choice. use the
   one-step integrator mode. see \reffun{pnl_ode_rkf45_step}.
 \item \var{flag = 8} : invalid input parameters this indicator occurs if any of
   the following is satisfied -   neqn <= 0, t=tout,  relerr or abserr <= 0.
  \end{itemize}
\item \describefun{int}{pnl_ode_rkf45_step}{\refstruct{PnlODEFunc} \ptr f,
    double \ptr y, double \ptr t, double t_out, double \ptr relerr, double
    abserr, double \ptr work, int \ptr iwork, int \ptr flag} 
  \sshortdescribe Same as \reffun{pnl_ode_rkf45} but it only computes one step
  of integration in the direction of \var{t_out}. \var{work} and \var{iwork} are
  working arrays of size \var{3 + 6 * neqn} and \var{5} respectively and should
  remain untouched between successive calls to the function. 
  On output \var{t} holds the point at which integration stopped and \var{y} the
  value of the solution at that point.
\end{itemize}

% vim:spelllang=en:spell:

