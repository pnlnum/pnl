
\documentclass[a4paper,11pt,twoside]{article}
\usepackage{a4wide}
\usepackage{t1enc,lmodern}
\usepackage{hyperref}
\usepackage{textcomp,color}
\usepackage{verbatim,moreverb,alltt}
\usepackage{makeidx}
\usepackage[latin1]{inputenc}
\usepackage[english]{babel}
\usepackage[strings]{underscore}
\usepackage{amsmath,amsfonts}
\usepackage{listings}
\lstset{language=C, basicstyle=\small, breaklines=true}

\synctex=1

%% --------------------------------------------
\hbadness=10000
\emergencystretch=\hsize
\tolerance=9999
\parindent=0pt
%% --------------------------------------------

\newcommand{\R}{{\mathbb R}}
\newcommand{\C}{{\mathbb C}}
\newcommand{\N}{{\mathbb N}}
\newcommand{\ptr}{\textasteriskcentered}
\newcommand{\cD}{{\mathcal D}}
\newcommand{\ind}[1]{1_{\left\{#1\right\}}}

%%\renewcommand{\exp}[1]{\operatorname{e}^{ #1 } }
\newcommand{\cotan}{\operatorname{cotan}}
\newcommand{\cotanh}{\operatorname{cotanh}}
\newcommand{\expp}[1]{\operatorname{e}^{ #1 } }
\newcommand{\paren}[1]{\left( #1 \right)}
\newcommand{\recaco}[1]{\left [ #1 \right ]}
\newcommand{\crochet}[1]{\left \langle #1 \right \rangle}
\newcommand{\acolade}[1]{\left\{ #1 \right\}}
\newcommand{\real}[1]{\operatorname{Re}(#1)}

\makeatletter

% For pdflatex generation
\ifx\HCode\undefined%

\def\var#1{{\tt #1}}
% Describing functions
\newcommand{\describefunNoStar}[3]{%
  \index{#2}\label{#2} {#1~{\bf #2}~(#3)}}
\newcommand{\describefunStar}[3]{%
  {#1~{\bf #2}~(#3)}}
\newcommand{\describefun}{\@ifstar
  \describefunStar%
  \describefunNoStar%
}

% Describing struct
\newcommand{\describestruct}[1]{%
  % \expandafter\newcommand\csname #1\endcsname{\refstruct{#1}\space}%
  \index{Structs!#1}\label{#1}}

% Describing variables
\newcommand{\describevar}[2]{{#1~{\bf #2}}}

\newcommand{\describemacro}[2]{%
  \index{#1}\label{#1} {{\bf #1}~(#2)}}

\newcommand{\constentry}[2]{%
  \index{#1}\label{#1}{\bf #1} &&  #2 \\ }

\definecolor{Red}{rgb}{1,0,0}
\definecolor{Blue}{rgb}{0,0,1}
\def\refstruct#1{\hyperref[#1]{\color{Red} #1}}
\def\reffun#1{\hyperref[#1]{#1}}
\def\refmacro{\reffun}
\newenvironment{describeconst}{%
  \noindent\begin{tabular}{lp{1cm}l}}{\end{tabular}}
\def\shortdescribe{\unskip\vskip1ex{\color{Blue} Description~}}
\def\sshortdescribe{\unskip\newline\hskip1em {\color{Blue} Description~}}
\def\parameters{\unskip\newline\hskip1em {\color{Blue} Parameters~}}
\def\example{\unskip\newline\hskip1em {\color{Blue} Example~}}

% Html related stuff
\else%

\def\var#1{\HCode{<span class='var'>}#1\HCode{</span>}}
% Describing functions
\newcommand{\describefunStar}[3]{%
{\HCode{<span class='ret'>}#1\HCode{</span>}~{\bf \HCode{<span class='fun'>}#2\HCode{</span>}}~(\HCode{<span class='args'>}#3\HCode{</span>})}}
\newcommand{\describefunNoStar}[3]{%
\index{#2}\label{#2} {\HCode{<span class='ret'>}#1\HCode{</span>}~{\bf \HCode{<span class='fun'>}#2\HCode{</span>}}~(\HCode{<span class='args'>}#3\HCode{</span>})}}
\newcommand{\describefun}{\@ifstar
  \describefunStar%
  \describefunNoStar%
}

% Describing variables
\newcommand{\describevar}[2]{{#1~{\bf #2}}}

\newcommand{\describemacro}[2]{%
  \index{#1}\label{#1} {{\bf #1}~(#2)}}

\newcommand{\constentry}[2]{%
  \index{#1}\label{#1}\HCode{<span class='struct'>}#1\HCode{</span>} && #2 \\ }

% Describing struct
\newcommand{\describestruct}[1]{%
  % \expandafter\newcommand\csname #1\endcsname{\refstruct{#1}\space}%
  \index{Structs!#1}\label{#1}}

\def\refstruct#1{\hyperref[#1]{\HCode{<span class='struct'>}#1\HCode{</span>}}}
\def\reffun#1{\hyperref[#1]{#1}}
\def\refmacro{\reffun}
\newenvironment{describeconst}{%
  \noindent\begin{tabular}{lp{1cm}l}}{\end{tabular}}
\def\shortdescribe{\unskip\vskip1ex{\HCode{<span class='description'>}Description~\HCode{</span>}}}
\def\sshortdescribe{\unskip\newline\hskip1em \HCode{<span class='description'>}Description~\HCode{</span>}}
\def\parameters{\unskip\newline\hskip1em \HCode{<span class='description'>}Parameters~\HCode{</span>}}
\def\example{\unskip\newline\hskip1em \HCode{<span class='description'>}Example~\HCode{</span>}}

\fi
\makeatother

% Macro for Pnl types
\def\PnlSpMat{\refstruct{PnlSpMat}\space}
\def\PnlMat{\refstruct{PnlMat}\space}
\def\PnlMatInt{\refstruct{PnlMatInt}\space}
\def\PnlMatComplex{\refstruct{PnlMatComplex}\space}
\def\PnlHmat{\refstruct{PnlHmat}\space}
\def\PnlArray{\refstruct{PnlArray}\space}
\def\PnlBandMat{\refstruct{PnlBandMat}\space}
\def\PnlBasis{\refstruct{PnlBasis}\space}
\def\PnlCell{\refstruct{PnlCell}\space}
\def\PnlCmplxFunc{\refstruct{PnlCmplxFunc}\space}
\def\PnlHmat{\refstruct{PnlHmat}\space}
\def\PnlList{\refstruct{PnlList}\space}
\def\PnlMat{\refstruct{PnlMat}\space}
\def\PnlMatComplex{\refstruct{PnlMatComplex}\space}
\def\PnlMatInt{\refstruct{PnlMatInt}\space}
\def\PnlMorseMat{\refstruct{PnlMorseMat}\space}
\def\PnlObject{\refstruct{PnlObject}\space}
\def\PnlPermutation{\refstruct{PnlPermutation}\space}
\def\PnlRng{\refstruct{PnlRng}\space}
\def\PnlSpMat{\refstruct{PnlSpMat}\space}
\def\PnlSparseFactorization{\refstruct{PnlSparseFactorization}\space}
\def\PnlSparseMat{\refstruct{PnlSparseMat}\space}
\def\PnlTridiagMat{\refstruct{PnlTridiagMat}\space}
\def\PnlTridiagMatLU{\refstruct{PnlTridiagMatLU}\space}
\def\PnlVect{\refstruct{PnlVect}\space}
\def\PnlVectCompact{\refstruct{PnlVectCompact}\space}
\def\PnlVectComplex{\refstruct{PnlVectComplex}\space}
\def\PnlVectInt{\refstruct{PnlVectInt}\space}
\def\PnlRnFuncR{\refstruct{PnlRnFuncR}\space}
\def\dcomplex{\refstruct{dcomplex}\space}

\setcounter{tocdepth}{2}
\title{Pnl Manual}
\date{\today}
\author{}

\makeindex
\begin{document}
\maketitle
\tableofcontents

\section{Introduction}
\subsection{What is Pnl}

Pnl is a scientific library written in C and distributed under
the Gnu Lesser General Public Licence (LGPL). This manual is divided into four
parts.
\begin{itemize}
\item Mathematical functions: complex numbers, special functions, standard
  financial functions for the Black \& Scholes model.
\item Linear algebra : vectors, matrices (dense and sparse), hypermatrices, tridiagonal matrices,
  band matrices and the corresponding routines to manipulate them and solve linear systems.
\item Probabilistic functions: random number generators and  cumulative
  distribution functions.
\item Deterministic toolbox : FFT, Laplace inversion, numerical integration, zero searching,
  multivariate polynomial regression, $\dots$
\end{itemize}

\subsection{A few helpful conventions}

\begin{itemize}
  \item All header file names are prefixed by \verb!pnl_! and are surrounded by
    the preprocessor conditionals
\begin{lstlisting}
#ifndef _PNL_MATRIX_H
#define _PNL_MATRIX_H

...

#endif /* _PNL_MATRIX_H
\end{lstlisting}
All the header files are protected by an \verb!extern "C"! declaration for
possible use with a C++ compiler. The header files must be include using
\begin{lstlisting}
#include "pnl/pnl_xxx.h"
\end{lstlisting}

  \item All function names are prefixed by \verb!pnl_! except those implementing
    complex number arithmetic which are named following the \textit{C99}
    complex library but using a capitalised first letter \verb!C!. \\
    For example, the addition of two complex numbers is performed by the
    function \verb!Cadd!.

  \item Function containing \verb!_create! in their names always return a
    pointer to an object created by one or several calls to dynamic
    allocation. Once these objects are not used, they must be freed by calling
    the same function but ending in \verb!_free!.
    A function \verb!pnl_foo_create_yyy! returns a \verb!PnlFoo *! object (note
    the ``\ptr'') and a function \verb!pnl_foo_bar_create_yyy! returns a
    \verb!PnlFooBar *! object (note the ``\ptr'').
    These objects must be freed by calling respectively \verb!pnl_foo_free! or
    \verb!pnl_foo_bar_free!.

  \item Functions ending in \verb!_clone! take two arguments \verb!src! and
    \verb!dest! and modify \verb!dest! to make it identical to \verb!src!, ie.
    they have the same size and data. Note that no new object is allocated,
    \verb!dest! must exist before calling this function.

  \item Functions ending in \verb!_copy! create a new object identical (ie. with
    the same size and content) as its argument but independent (ie. modifying
    one of them does not alter the other). Calling \verb!A = pnl_xxx_copy(B)!
    is equivalent to first calling \verb!A = pnl_xxx_new()! function and then
    \verb!pnl_xxx_clone(A, B)!.
    
  \item Every object must implement a \verb!pnl_xxx_new! function which returns
    a pointer to an empty object with all its elements properly set to $0$. This
    means that the objects returned by the \verb!pnl_xxx_new! functions can be
    used as output arguments for functions ending in \verb!_inplace! for
    instance. They are suitable for being resized. 

  \item Functions containing \verb!_wrap_! in their names always return an
    object, not a pointer to an object, and do not make any use of dynamic
    allocation. The returned object must not be freed. 
    For instance, a function \verb!pnl_foo_wrap_xxx! returns an object
    \verb!PnlFoo! and a function \verb!pnl_foo_bar_wrap_xxx! returns an object
    \verb!PnlFooBar! 
    \begin{lstlisting}
    PnlVectComplex *v1;
    PnlVectComplex v2;
    v1 = pnl_vect_complex_create_from_scalar (5, Complex(0., 1.));
    v2 = pnl_vect_complex_wrap_subvect (v1, 1, 2);

    ...

    pnl_vect_complex_free (&v1);
    \end{lstlisting}
    The vector \verb!v1! is of size 5 and contains the pure imaginary number
    $i$. The vector \verb!v2! only provides a view to \verb!v1(1:1+2)!, which
    means that modifying \verb!v2! will also modify \verb!v1! and vice-versa
    because \verb!v1! shares part of its data with \verb!v2!. Note that only
    \verb!v1! must be freed and {\bf not} \verb!v2!.
  
  \item Functions ending in \verb!_init! do not create any object but only
    perform some internal initialisation.
  
  \item Hypermatrices, matrices and vectors are stored using a flat block of
    memory obtained by concatenating the matrix rows and C-style
    pointer-to-pointer arrays. Matrices are stored in row-major order, which
    means that the column index moves continuously.
    Note that this convention is not \textit{Blas \& Lapack} compliant since
    Fortran expects 2-dimensional arrays to be stored in a column-major order.

  \item Type names always begin with \verb!Pnl!, they do not contain underscores
    but instead we use capital letters to separate units in type names. \\
    Examples : \verb!PnlMat!, \verb!PnlMatComplex!.

  \item Object and function names are intimately linked : an object
    \verb!PnlFoo! is manipulated by functions starting in \verb!pnl_foo!, an
    object \verb!PnlFooBar! is manipulated by functions starting in
    \verb!pnl_foo_bar!. In table~\ref{pnltypes}, we summarise the types and their
    corresponding prefixes.

    \begin{figure}[h!]
      \centering\begin{tabular}{|l|l|}
      \hline
      Pnl types & Pnl prefix \\
      \hline
      PnlVect & pnl_vect \\
      PnlVectComplex & pnl_vect_complex \\
      PnlVectInt & pnl_vect_int \\
       &\\
      PnlMat & pnl_mat \\
      PnlMatComplex & pnl_mat_complex \\
      PnlMatInt & pnl_mat_int \\
      & \\
      PnlSpMat & pnl_sp_mat \\
      PnlSpMatComplex & pnl_sp_mat_complex \\
      PnlSpMatInt & pnl_sp_mat_int \\
      & \\
      PnlHmat & pnl_hmat \\
      PnlHmatComplex & pnl_hmat_complex \\
      PnlHmatInt & pnl_hmat_int \\
      & \\
      PnlTridiagMat & pnl_tridiag_mat \\
      PnlBandMat & pnl_band_mat \\
      & \\
      PnlList & pnl_list \\
      & \\
      PnlBasis & pnl_basis \\
      & \\
      PnlCgSolver & pnl_cg_solver \\
      PnlBicgSolver & pnl_bicg_solver \\
      PnlGmresSolver & pnl_gmres_solver \\
      \hline
    \end{tabular}
    \caption{Pnl types}
    \label{pnltypes}
  \end{figure}

  \item All macro names begin with \verb!PNL_! and are capitalised.

  \item Differences between \textbf{copy} and \textbf{clone} methods.
    The \verb!copy! methods take a single argument and return a pointer to an object
    of the same type which is an independent copy of its argument. 
    Example:
    \begin{lstlisting}
    PnlVect *v1, *v2;
    v1 = pnl_vect_create_from_scalar (5, 2.5);
    v2 = pnl_vect_copy (v1);
    \end{lstlisting}
    \verb!v1! and \verb!v2! are two vectors of size 5 with all their elements
    equal to 2.5. Note that \verb!v2! {\bf must not} have been created by a call
    to \verb!pnl_vect_create_xxx! because otherwise it will cause a memory leak.
    \verb!v1! and \verb!v2! are independent in the sense that a modification to
    one of them does not affect the other.

    The \verb!clone! methods take two arguments and fill the first one with the
    second one. 
    Example:
    \begin{lstlisting}
    PnlVect *v1, *v2;
    v1 = pnl_vect_create_from_scalar (5, 2.5);
    v2 = pnl_vect_new ();
    pnl_vect_clone (v2, v1);
    \end{lstlisting}
    \verb!v1! and \verb!v2! are two vectors of size 5 with all their elements
    equal to 2.5. Note that \verb!v2! {\bf must} have been created by a call to
    \verb!pnl_vect_new! because otherwise the function
    \verb!pnl_vect_clone!  will crash.  \verb!v1! and \verb!v2! are independent
    in the sense that a modification to one of them does not modify the other.


  \item All objects are measured using integers \verb!int! and not
    \verb!size_t!. Hence, iterations over vectors, matrices, \dots should use an
    index of type \verb!int!.

  \item In fonctions ending in \verb!inplace!, the output parameter must be different
    from any of the input parameters.
\end{itemize}


\subsection{Using Pnl}

In this section, we assume that the library is installed in the directory
\verb!$HOME/pnl-xxx!.

Once installed, the library can be found in the
\verb!$HOME/pnl-xxx/lib!  directory and the header files in the
\verb!$HOME/pnl-xxx/include! directory. 

\subsubsection{Compiling and Linking}

The header files of the library are installed in a root \verb!pnl! directory and
should always be included with this \verb!pnl/! prefix. So, for instance to use
random number generators you should include 
\begin{lstlisting}
#include <pnl/pnl_random.h>
\end{lstlisting}

\paragraph{Compiling and linking by hand.}

If \verb!gcc! or \verb!llvm! is used, you should pass the following options
\begin{itemize}
\item \verb!-I$HOME/pnl-xxx/include! for compiling
\item \verb!-L$HOME/pnl-xxx/lib -lpnl! for linking
\end{itemize}
This does not work straight away on all OS especially if the library is not
installed in a standard directory namely \verb+/usr/+ or \verb+/usr/local/+ for
which you need a privileged writing access.
On some systems, you may need to add to the linker flags the dependencies of the
library, which can become very tedious. Therefore, we provide a second automatic
mechanism which takes care of the dependencies on its own.

\paragraph{Compiling and linking using an automatic Makefile.}

This mechanism only works under Unix (it has been tested under various Linux
distributions and Mac OS X).

First, you need to create a new directory wherever you want, put in all your
code and create a Makefile as below

To define your target just add the executable name, say \verb!my-exec!, to the
\verb!BINS!  list and create an entry \verb!my_exec_SRC! carrying the list
of source files needed to create your executable.  Note that if dashes '-' may
appear in an executable name, the name of the associated variable holding the
list of source files is obtained by replacing dashes with underscores '_' and
adding the _SRC suffix.

Assume you want to create two binaries : \verb+my-exec+ based on mixed C and C++
code (\verb+file1.c+ and \verb+file2.cpp+) and \verb+mybinary+ based on
\verb+poo1.cxx+ and \verb+poo2.cpp+. You can use the following Makefile.
\begin{lstlisting}
## Flags passed to the linker
LDFLAGS=

## Flags passed to the compiler
CFLAGS=

## list of executables to create
BINS=my-exec mybinary

my_exec_SRC=file1.c file2.cpp
# optional flags for compiling and linking
my_exec_CFLAGS=
my_exec_CXXFLAGS=
my_exec_LDFLAGS=

mybinary_SRC=poo1.cxx poo2.cpp
# optional flags for compiling and linking
mybinary_CFLAGS=
mybinary_CXXFLAGS=
mybinary_LDFLAGS=


## This line must be the last one
include full_path_to_pnl_build/CMakeuser.incl
\end{lstlisting}
Let us comment a little the different variables
\begin{itemize}
  \item \verb+CFLAGS+: global flags used for creating objects based on C code
  \item \verb+CXXFLAGS+: global flags used for creating objects based on C++ code
  \item \verb+LDFLAGS+: gobal linker flags.
  \item \verb+binaryname_CFLAGS+: flags used when creating the objects based on
    C code and required by \verb+binaryname+
  \item \verb+binaryname_CXXFLAGS+: flags used when creating the objects based on
    C++ code and required by \verb+binaryname+
  \item \verb+binaryname_LDFLAGS+: flags used when linking objects for creating
    \verb+binaryname+
\end{itemize}
An example of such a Makefile can be found in \verb+pnl-xxx/perso+.

\textbf{Warning:} if a file appears in the source list of several binairies, the
flags used to compile this file are determined by the ones of the first binary
involving this file. In the following example \verb+main.cpp+ will always be compiled
with the flag \verb+-O3+ even for generating \verb+bin2+
\begin{lstlisting}
BINS=bin1 bin2

bin1_SRC=main.cpp poo1.c
my_exec_CXXFLAGS=-O3

bin2_SRC=main.cpp poo2.c
mybinary_CXXFLAGS=-g -O0

## This line must be the last one
include full_path_to_pnl_build/CMakeuser.incl
\end{lstlisting}

\paragraph{Compiling and linking using CMake.}

If you already use CMake for your new project, just add the following to your toplevel
\verb!CMakeLists.txt!
\begin{lstlisting}
find_package(Pnl REQUIRED)
set(LIBS ${LIBS} ${PNL_LIBRARIES})
include_directories(${PNL_INCLUDE_DIRS})
# Deactivate PNL debugging stuff on Release builds
if(${CMAKE_BUILD_TYPE} STREQUAL "Release")
    add_definitions(-DPNL_RANGE_CHECK_OFF)
endif()
\end{lstlisting}
Then, call cmake with the following extra flag 
\begin{lstlisting}
-DCMAKE_PREFIX_PATH=path/to/build-dir
\end{lstlisting}
or add the variable \verb!CMAKE_BUILD_TYPE! to the GUI. \\

Just in case, we give an example of a complete although elementary \verb!CMakeLists.txt!
\verbatiminput{../perso/CMakeLists-example.txt}

\subsubsection{Inline Functions and getters}
\label{sec:inline}

If it is supported by your compiler, getter and setter functions are declared
as inline functions. This is automatically detected when running CMake. By
default, setter and getter functions check that the required access is valid,
basically it boils down to checking whether the index of the access is within an
acceptable range. These extra tests can become very expensive when getter and
setter functions are intensively called.

Thus,  it is possible to alter this default behaviour by defining the macro
\texttt{PNL_RANGE_CHECK_OFF}. This macro is automatically defined when the
library is compiled in Release mode, ie. with \verb!-DCMAKE_BUILD_TYPE=Release!
passed to CMake.


\section{Objects}

\subsection{The top-level object}

The PnlObject structure is used to simulate some inheritance between the
ojbects of Pnl.  It must be the first element of all the objects existing in
Pnl so that casting any object to a PnlObject is legal

\describestruct{PnlObject}
\begin{lstlisting}
typedef unsigned int PnlType; 

typedef void (DestroyFunc) (void **);
typedef PnlObject* (CopyFunc) (PnlObject *);
typedef PnlObject* (NewFunc) (PnlObject *);
typedef void (CloneFunc) (PnlObject *dest, const PnlObject *src);
struct _PnlObject
{
  PnlType type; /*!< a unique integer id */
  const char *label; /*!< a string identifier (for the moment not useful) */
  PnlType parent_type; /*!< the identifier of the parent object is any,
                          otherwise parent_type=id */
  int nref; /*!< number of references on the object */ 
  DestroyFunc *destroy; /*!< frees an object */
  NewFunc     *constructor; /*!< New function */
  CopyFunc    *copy; /*!< Copy function */
  CloneFunc   *clone; /*!< Clone function */
};
\end{lstlisting}

Here is the list of all the types actually defined
\begin{table}
  \centering
  \begin{tabular}{l|l}
    \hline
    PnlType & Description \\
    \hline
    PNL_TYPE_VECTOR & general vectors  \\
    PNL_TYPE_VECTOR_DOUBLE & real vectors \\
    PNL_TYPE_VECTOR_INT & integer vectors \\
    PNL_TYPE_VECTOR_COMPLEX & complex vectors \\
    PNL_TYPE_MATRIX & general matrices  \\
    PNL_TYPE_MATRIX_DOUBLE & real matrices \\
    PNL_TYPE_MATRIX_INT & integer matrices \\
    PNL_TYPE_MATRIX_COMPLEX & complex matrices \\
    PNL_TYPE_TRIDIAG_MATRIX & general tridiagonal matrices \\
    PNL_TYPE_TRIDIAG_MATRIX_DOUBLE & real  tridiagonal matrices \\
    PNL_TYPE_BAND_MATRIX & general band matrices \\
    PNL_TYPE_BAND_MATRIX_DOUBLE & real band matrices \\
    PNL_TYPE_SP_MATRIX & sparse general matrices  \\
    PNL_TYPE_SP_MATRIX_DOUBLE & sparse real matrices \\
    PNL_TYPE_SP_MATRIX_INT & sparse integer matrices \\
    PNL_TYPE_SP_MATRIX_COMPLEX & sparse complex matrices \\
    PNL_TYPE_HMATRIX & general hyper matrices \\
    PNL_TYPE_HMATRIX_DOUBLE & real hyper matrices \\
    PNL_TYPE_HMATRIX_INT & integer hyper matrices \\
    PNL_TYPE_HMATRIX_COMPLEX & complex hyper matrices \\
    PNL_TYPE_BASIS & bases \\
    PNL_TYPE_RNG & random number generators \\
    PNL_TYPE_LIST & doubly linked list \\
    PNL_TYPE_ARRAY & array
  \end{tabular}
  \caption{PnlTypes}
  \label{types}
\end{table}

We provide several macros for manipulating PnlObejcts.
\begin{itemize}
\item \describemacro{PNL_OBJECT}{o}
  \sshortdescribe Cast any object into a PnlObject

\item \describemacro{PNL_VECT_OBJECT}{o}
  \sshortdescribe Cast any object into a PnlVectObject

\item \describemacro{PNL_MAT_OBJECT}{o}
  \sshortdescribe Cast any object into a PnlMatObject

\item \describemacro{PNL_SP_MAT_OBJECT}{o}
  \sshortdescribe Cast any object into a PnlSpMatObject

\item \describemacro{PNL_HMAT_OBJECT}{o}
  \sshortdescribe Cast any object into a PnlHmatObject

\item \describemacro{PNL_BAND_MAT_OBJECT}{o}
  \sshortdescribe Cast any object into a PnlBandMatObject

\item \describemacro{PNL_TRIDIAGMAT_OBJECT}{o}
  \sshortdescribe Cast any object into a PnlTridiagMatObject

\item \describemacro{PNL_BASIS_OBJECT}{o}
  \sshortdescribe Cast any object into a PnlBasis

\item \describemacro{PNL_RNG_OBJECT}{o}
  \sshortdescribe Cast any object into a PnlRng

\item \describemacro{PNL_LIST_OBJECT}{o}
  \sshortdescribe Cast any object into a PnlList

\item \describemacro{PNL_LIST_ARRAY}{o}
  \sshortdescribe Cast any object into a PnlArray

\item \describemacro{PNL_GET_TYPENAME}{o}
  \sshortdescribe Return the name of the type of any object inheriting from PnlObject

\item \describemacro{PNL_GET_TYPE}{o}
  \sshortdescribe Return the type of any object inheriting from PnlObject
  
\item \describemacro{PNL_GET_PARENT_TYPE}{o}
  \sshortdescribe Return the parent type of any object inheriting from PnlObject
\end{itemize}

\begin{itemize}
\item \describefun{\PnlObject\ptr }{pnl_object_create}{PnlType t}
  \sshortdescribe Create an empty PnlObject of type \var{t} which can any of
  the registered types, see Table~\ref{types}.
\end{itemize}

\subsection{List object}

This section describes functions for creating an manipulating lists. Lists are
internally stored as doubly linked lists.

The structures and functions related to lists are declared in
\verb!pnl/pnl_list.h!.

\describestruct{PnlList}\describestruct{PnlCell}
\begin{lstlisting}
typedef struct _PnlCell PnlCell;
struct _PnlCell
{
  struct _PnlCell *prev;  /*!< previous cell or 0 */
  struct _PnlCell *next;  /*!< next cell or 0 */
  PnlObject *self;       /*!< stored object */
};


typedef struct _PnlList PnlList;
struct _PnlList
{
  /**
   * Must be the first element in order for the object mechanism to work
   * properly. This allows any PnlList pointer to be cast to a PnlObject
   */
  PnlObject object; 
  PnlCell *first; /*!< first element of the list */
  PnlCell *last; /*!< last element of the list */
  PnlCell *curcell; /*!< last accessed element,
                         if never accessed is NULL */
  int icurcell; /*!< index of the last accessed element,
                     if never accessed is NULLINT */
  int len; /*!< length of the list */
};
\end{lstlisting}

\textbf{Important note}: Lists only store addresses of objects. So when an
object is inserted into a list, only its address is stored into the list. This
implies that you \textbf{must not} free any objects inserted into a list. The
deallocation is automatically handled by the function \reffun{pnl_list_free}.

\begin{itemize}
\item \describefun{\PnlList \ptr }{pnl_list_new}{}
  \sshortdescribe Create an empty list
\item \describefun{\PnlCell \ptr }{pnl_cell_new}{}
  \sshortdescribe Create an cell list
\item \describefun{\PnlList\ptr }{pnl_list_copy}{const \PnlList\ptr A}
  \sshortdescribe Create a copy of a \PnlList. Each element of the
  list \var{A} is copied by calling the its copy member.
\item \describefun{void}{pnl_list_clone}{\PnlList \ptr dest, const
  \PnlList\ptr src}
  \sshortdescribe Copy the content of \var{src} into the already existing
  list \var{dest}. The list \var{dest} is automatically resized. This is a
  hard copy, the contents of both lists are independent after cloning.
\item \describefun{void}{pnl_list_free}{\PnlList  \ptr \ptr L}
  \sshortdescribe Free a list
\item \describefun{void}{pnl_cell_free}{\PnlCell  \ptr \ptr c}
  \sshortdescribe Free a list
\item \describefun{\PnlObject\ptr}{pnl_list_get}{
    \PnlList \ptr L, int i}
  \sshortdescribe This function returns the content of the \var{i}--th cell of
  the list \var{L}. This function is optimized for linearly accessing all the
  elements, so it can be used inside a for loop for instance.
\item \describefun{void}{pnl_list_insert_first}{\PnlList  \ptr L,
    \PnlObject  \ptr o}
  \sshortdescribe Insert the object \var{o} on top of the list \var{L}. Note that
  \var{o} is not copied in \var{L}, so do  {\bf not} free \var{o} yourself, it
  will be done automatically when calling \reffun{pnl_list_free}
\item \describefun{void}{pnl_list_insert_last}{\PnlList  \ptr L,
    \PnlObject  \ptr o}
  \sshortdescribe Insert the object \var{o} at the bottom of the list \var{L}. Note that
  \var{o} is not copied in \var{L}, so do  {\bf not} free \var{o} yourself, it
  will be done automatically when calling \reffun{pnl_list_free}
\item \describefun{void}{pnl_list_remove_last}{\PnlList  \ptr L}
  \sshortdescribe Remove the last element of the list \var{L} and frees it.
\item \describefun{void}{pnl_list_remove_first}{\PnlList  \ptr L}
  \sshortdescribe Remove the first element of the list \var{L} and frees it.
\item \describefun{void}{pnl_list_remove_i}{\PnlList  \ptr L, int i}
  \sshortdescribe Remove the \var{i-th} element of the list \var{L} and frees it.
\item \describefun{void}{pnl_list_concat}{\PnlList  \ptr L1,
    \PnlList  \ptr L2}
  \sshortdescribe Concatenate the two lists \var{L1} and \var{L2}. The
  resulting list is store in \var{L1} on exit. Do {\bf not} free \var{L2}
  since concatenation does not actually copy objects but only manipulates
  addresses.
\item \describefun{void}{pnl_list_resize}{\PnlList \ptr L, int n}
  \sshortdescribe Change the length of \var{L} to become \var{n}. If the length
  of \var{L} id increased, the extra elements are set to NULL.
\item \describefun{void}{pnl_list_print}{const \PnlList  \ptr L}
  \sshortdescribe Only prints the types of each element. When  the
  \PnlObject object has a print member, we will use it.
\end{itemize}

\subsection{Array object}

This section describes functions for creating and manipulating arrays of
PnlObjects.

The structures and functions related to arrays are declared in
\verb!pnl/pnl_array.h!.

\describestruct{PnlArray}
\begin{lstlisting}
typedef struct _PnlArray PnlArray;
struct _PnlArray
{
  /**
   * Must be the first element in order for the object mechanism to work
   * properly. This allows any PnlArray pointer to be cast to a PnlObject
   */
  PnlObject object; 
  int size;
  PnlObject **array;
  int mem_size;
};
\end{lstlisting}

\textbf{Important note}: Arrays only store addresses of objects. So when an
object is inserted into an array, only its address is stored into the array. This
implies that you \textbf{must not} free any objects inserted into a array. The
deallocation is automatically handled by the function \reffun{pnl_array_free}.

\begin{itemize}
\item \describefun{\PnlArray \ptr }{pnl_array_new}{}
  \sshortdescribe Create an empty array
\item \describefun{\PnlArray \ptr }{pnl_array_create}{int n}
  \sshortdescribe Create an array of length \var{n}.
\item \describefun{\PnlArray\ptr }{pnl_array_copy}{const \PnlArray\ptr A}
  \sshortdescribe Create a copy of a \PnlArray. Each element of the
  array \var{A} is copied by calling the \var{A[i].object.copy}.
\item \describefun{void}{pnl_array_clone}{\PnlArray \ptr dest, const
  \PnlArray\ptr src}
  \sshortdescribe Copy the content of \var{src} into the already existing
  array \var{dest}. The array \var{dest} is automatically resized. This is a
  hard copy, the contents of both arrays are independent after cloning.
\item \describefun{void}{pnl_array_free}{\PnlArray  \ptr \ptr}
  \sshortdescribe Free an array and all the objects hold by the array.
\item \describefun{int}{pnl_array_resize}{\PnlArray \ptr  T, int size}
  \sshortdescribe Resize \var{T} to be \var{size} long. As much as possible of
  the original data is kept.
\item \describefun{\PnlObject\ptr}{pnl_array_get}{
    \PnlArray \ptr T, int i}
  \sshortdescribe This function returns the content of the \var{i}--th cell of
  the array \var{T}. No copy is made.
\item \describefun{\PnlObject\ptr}{pnl_array_set}{
    \PnlArray \ptr T, int i, \PnlObject\ptr O}
  \sshortdescribe \var{T[i] = O}. No copy is made, so the object \var{O} must
  not be freed manually.
\item \describefun{void}{pnl_array_print}{\PnlArray  \ptr}
  \sshortdescribe Not yet implemented because it would require that the
  structure \PnlObject has a field copy.
\end{itemize}


%%% Local Variables: 
%%% mode: latex
%%% TeX-master: "pnl-manual"
%%% End: 

\input{mathematical_functions.tex}
  \section{Linear Algebra}

% vector
\subsection{Vectors}
\subsubsection{Overview}

The structures and functions related to vectors are declared in
\verb!pnl/pnl_vector.h!.


Vectors are declared for several basic types : double, int, and
dcomplex. In the following declarations, {\tt BASE} must be replaced by one
the previous types and the corresponding vector structures are respectively
named PnlVect, PnlVectInt, PnlVectComplex
\describestruct{PnlVect}\describestruct{PnlVectInt}\describestruct{PnlVectComplex}
\begin{verbatim}
typedef struct _PnlVect {
  /**
   * Must be the first element in order for the object mechanism to work
   * properly. This allows any PnlVect pointer to be cast to a PnlObject
   */
  PnlObject object; 
  int size; /*!< size of the vector */
  int mem_size; /*!< size of the memory block allocated for array */
  double *array; /*!< pointer to store the data */
  int owner; /*!< 1 if the object owns its array member, 0 otherwise */
} PnlVect;

typedef struct _PnlVectInt {
  /**
   * Must be the first element in order for the object mechanism to work
   * properly. This allows any PnlVectInt pointer to be cast to a PnlObject
   */
  PnlObject object; 
  int size; /*!< size of the vector */ 
  int mem_size; /*!< size of the memory block allocated for array */
  int *array; /*!< pointer to store the data */
  int owner; /*!< 1 if the object owns its array member, 0 otherwise */
} PnlVectInt;

typedef struct _PnlVectComplex {
  /**
   * Must be the first element in order for the object mechanism to work
   * properly. This allows any PnlVectComplex pointer to be cast 
   * to a PnlObject
   */
  PnlObject object; 
  int size; /*!< size of the vector */ 
  int mem_size; /*!< size of the memory block allocated for array */
  dcomplex *array; /*!< pointer to store the data */
  int owner; /*!< 1 if the object owns its array member, 0 otherwise */
} PnlVectComplex;
\end{verbatim}
\var{size} is the size of the vector, \var{array} is a pointer containing the
data and \var{owner} is an integer to know if the vector owns its \var{array}
pointer (\var{owner=1}) or shares it with another structure (\var{owner=0}).
\var{mem_size} is the number of elements the vector can hold at most.

\subsubsection{Functions}

\paragraph{General functions}
These functions exist for all types of vector no matter what the basic type
is. The following conventions are used to name functions operating on vectors.
Here is the table of prefixes used for the different basic types.

\begin{center}
  \begin{tabular}[t]{lll}
    type & prefix & BASE\\
    \hline
    double & pnl_vect & double \\
    \hline
    int & pnl_vect_int & int \\
    \hline
    dcomplex & pnl_vect_complex & dcomplex
  \end{tabular}
\end{center}

In this paragraph, we present the functions operating on \PnlVect
which exist for all types. To deduce the prototypes of these functions for
other basic types, one must replace {\tt pnl_vect} and {\tt double} according
the above table. 
\subparagraph{Constructors and destructors}

There are no special functions to access the size of a vector, instead the field
\verb!size! should be accessed directly.

\begin{itemize}
  \item \describefun{\PnlVect \ptr }{pnl_vect_new}{}
  \sshortdescribe Create a new \PnlVect of size 0.  
\item \describefun{\PnlVect \ptr }{pnl_vect_create}{int size}
  \sshortdescribe Create a new \PnlVect pointer.  
\item \describefun{\PnlVect \ptr }{pnl_vect_create_from_zero}{int size}
  \sshortdescribe Create a new \PnlVect pointer and sets it to zero.  
\item \describefun{\PnlVect \ptr }{pnl_vect_create_from_scalar}
  {int size, double x}
  \sshortdescribe Create a new \PnlVect pointer and sets all
  elements t \var{x}.  
\item \describefun{\PnlVect \ptr }{pnl_vect_create_from_ptr}{int
    size, const double \ptr x}
  \sshortdescribe Create a new \PnlVect pointer and copies \var{x}
  to \var{array}.  
\item \describefun{\PnlVect \ptr }{pnl_vect_create_from_mat}{
  const PnlMat *M}
  \sshortdescribe Create a new \PnlVect pointer of size \var{M->mn}
  and copy the content of \var{M} row wise.
\item \describefun{\PnlVect \ptr }{pnl_vect_create_from_list}{int
    size, ...}
  \sshortdescribe Create a new \PnlVect pointer of length
  \var{size} filled with the extra arguments passed to the function. The
  number of extra arguments passed must be equal to \var{size} and they must be of the type BASE.
  Example: To create a vector \{1., 2.\}, you should enter pnl_vect_create_from_list(2, 1.0, 2.0) and NOT pnl_vect_create_from_list(2, 1.0, 2) or pnl_vect_create_from_list(2, 1, 2.0).
  Be aware that this cannot be checked inside the function.
\item \describefun{\PnlVect \ptr }{pnl_vect_create_from_file}
  {const char \ptr file}
  \sshortdescribe Read a vector from a file and creates the corresponding
  \PnlVect. The data might be stored as a row or column vector. Entries can be separated by spaces, tabs, commas or semicolons. Anything after a \verb!#! or \verb!%! is ignored up to the end of the line.

\item \describefun{\PnlVect \ptr }{pnl_vect_copy}{const
    \PnlVect \ptr v}
  \sshortdescribe This is a copying constructor. It creates a copy of a \PnlVect.
\item \describefun{void}{pnl_vect_clone}{\PnlVect \ptr clone, 
    const \PnlVect \ptr v} 
  \sshortdescribe Clone a \PnlVect. \var{clone} must be an
  already existing  \PnlVect. It is resized to match the size of
  \var{v} and the data are copied. Future modifications to \var{v} will not
  affect \var{clone}.

\item \describefun{\PnlVect \ptr }{pnl_vect_create_subvect_with_ind}{const
  \PnlVect \ptr V, const \PnlVectInt \ptr ind}
  \sshortdescribe Create a new vector containing \var{V(ind(:))}.

\item \describefun{void}{pnl_vect_extract_subvect_with_ind}{\PnlVect \ptr
  V_sub, const \PnlVect \ptr V, const \PnlVectInt \ptr ind}
  \sshortdescribe On exit, \var{V_sub = V(ind(:))}.

\item \describefun{\PnlVect \ptr }{pnl_vect_create_subvect}{const
    \PnlVect \ptr V, int i, int len}
  \sshortdescribe Create a new vector containing \var{V(i:i+len-1)}. The
  elements are copied.
  
\item \describefun{void}{pnl_vect_extract_subvect}{\PnlVect \ptr
  V_sub, const \PnlVect \ptr V, int i, int len}
  \sshortdescribe On exit, \var{V_sub = V(i:i+len-1)}.  The
  elements are copied.

\item \describefun{void}{pnl_vect_set_subblock}{\PnlVect \ptr dest, const \PnlVect \ptr src, int i}
  \sshortdescribe Set \var{dest[i:] = src}.

\item \describefun{void}{pnl_vect_free}{\PnlVect \ptr\ptr v}
  \sshortdescribe Free a \PnlVect pointer and set the data pointer to NULL  
\item \describefun{\PnlVect}{pnl_vect_wrap_array}{const double \ptr x, 
    int size}
    \sshortdescribe Create a \PnlVect containing the data 
  \var{x}. No copy is made. It is just a container.
  
\item \describefun{\PnlVect}{pnl_vect_wrap_subvect}{const
  \PnlVect \ptr x, int i, int s}
  \sshortdescribe Create a \PnlVect containing
  \var{x(i:i+s-1)}. No copy is made. It is just a container. The returned
  \PnlVect has \var{size=s} and \var{owner=0}.

\item \describefun{\PnlVect}{pnl_vect_wrap_subvect_with_last}{const \PnlVect \ptr x, int i, int j}
  \sshortdescribe Create a \PnlVect containing \var{x(i:j)}. No
  copy is made. It is just a container.

\item \describefun{\PnlVect}{pnl_vect_wrap_mat}
  {const \PnlMat \ptr M}
  \sshortdescribe Return a \PnlVect (not a pointer) whose array is
  the row wise array of \var{M}. The new vector shares its data with the
  matrix \var{M}, which means that any modification to one of them will affect
  the other.
\end{itemize}

\subparagraph{Resizing vectors}
\begin{itemize}
\item \describefun{int}{pnl_vect_resize}{\PnlVect \ptr v, int size}
  \sshortdescribe Resize a \PnlVect. It copies as much of the old
  data to fit in the resized object.
  \item \describefun{int}{pnl_vect_resize_from_scalar}{\PnlVect
  \ptr v, int size, double x} 
  \sshortdescribe Resize a \PnlVect. Copy as much of the old data as possible and fill the new cells with \var{x}.
\item \describefun{int}{pnl_vect_resize_from_ptr}{\PnlVect
    \ptr v, int size, double \ptr t} 
  \sshortdescribe Resize a \PnlVect and uses \var{t} to fill the
  vector. \var{t} must be of size \var{size}.
\end{itemize}  

\subparagraph{Accessing elements}

If it is supported by the compiler, the following functions are declared
inline. To speed up these functions, you can define the macro 
\texttt{PNL_RANGE_CHECK_OFF}, see Section~\ref{sec:inline} for an explanation. 

Accessing elements of a vector is faster using the following macros
\begin{itemize}
\item \describefun{}{GET}{\PnlVect \ptr v, int i}
  \sshortdescribe Return \var{v[i]} for reading, eg. \var{x=GET(v,i)}
\item \describefun{}{GET_INT}{\PnlVectInt \ptr v, int i}
  \sshortdescribe Same as \reffun{GET} but for an integer vector.
\item \describefun{}{GET_COMPLEX}{\PnlVectComplex \ptr v, int i}
  \sshortdescribe Same as \reffun{GET} but for a complex vector.
\item \describefun{}{LET}{\PnlVect \ptr v, int i}
  \sshortdescribe Return \var{v[i]} as a lvalue for writing, eg.
  \var{LET(v,i)=x}
\item \describefun{}{LET_INT}{\PnlVectInt \ptr v, int i}
  \sshortdescribe Same as \reffun{LET} but for an integer vector.
\item \describefun{}{LET_COMPLEX}{\PnlVectComplex \ptr v, int i}
  \sshortdescribe Same as \reffun{LET} but for a complex vector.
\end{itemize}

\begin{itemize}
\item \describefun{void}{pnl_vect_set}{\PnlVect \ptr v, int i, double x}
  \sshortdescribe Set v[i]=x.
\item \describefun{double}{pnl_vect_get}{const \PnlVect \ptr v, int i}
  \sshortdescribe Return the value of v[i].
\item \describefun{void}{pnl_vect_lget}{\PnlVect \ptr v, int i}
  \sshortdescribe Return the address of v[i].
\item \describefun{void}{pnl_vect_set_all}{\PnlVect \ptr v, double x}
  \sshortdescribe Set all elements to x.
\item \describefun{void}{pnl_vect_set_zero}{\PnlVect \ptr v}
  \sshortdescribe Set all elements to zero.
\end{itemize}


\subparagraph{Printing vector}
\begin{itemize}
\item \describefun{void}{pnl_vect_print}{const \PnlVect \ptr V}
  \sshortdescribe Print a \PnlVect as a column vector
\item \describefun{void}{pnl_vect_fprint}{FILE \ptr fic, const \PnlVect \ptr V}
  \sshortdescribe Print a \PnlVect in file \var{fic} as a column
  vector. The file can be read by \reffun{pnl_vect_create_from_file}.
\item \describefun{void}{pnl_vect_print_asrow}{const \PnlVect \ptr V}
  \sshortdescribe Print a \PnlVect as a row vector
\item \describefun{void}{pnl_vect_fprint_asrow}{FILE \ptr fic, const \PnlVect \ptr V}
  \sshortdescribe Print a \PnlVect in file \var{fic} as a row
  vector. The file can be read by \reffun{pnl_vect_create_from_file}.
\item \describefun{void}{pnl_vect_print_nsp}{const \PnlVect \ptr V}
  \sshortdescribe Print a vector to the standard output in a format
  compatible with Nsp.
\item \describefun{void}{pnl_vect_fprint_nsp}{FILE \ptr fic, const
    \PnlVect \ptr V}
  \sshortdescribe Print a vector to a file in a format compatible with Nsp.
\end{itemize}

\subparagraph{Applying external operation to vectors}

\begin{itemize}
\item \describefun{void}{pnl_vect_minus}{\PnlVect \ptr lhs}
  \sshortdescribe In-place unary minus
\item \describefun{void}{pnl_vect_plus_scalar}{\PnlVect \ptr lhs, double x}
  \sshortdescribe In-place vector scalar addition  
\item \describefun{void}{pnl_vect_minus_scalar}{\PnlVect \ptr lhs, double x}
  \sshortdescribe In-place vector scalar substraction  
\item \describefun{void}{pnl_vect_mult_scalar}{\PnlVect \ptr lhs, double x}
  \sshortdescribe In-place vector scalar multiplication  
\item \describefun{void}{pnl_vect_div_scalar}{\PnlVect \ptr lhs, double x}
  \sshortdescribe In-place vector scalar division  
\end{itemize}

\subparagraph{Element wise operations}

\begin{itemize}
\item \describefun{void}{pnl_vect_plus_vect}{\PnlVect \ptr lhs, 
    const \PnlVect \ptr rhs} 
  \sshortdescribe In-place vector vector addition  

\item \describefun{void}{pnl_vect_minus_vect}{\PnlVect \ptr lhs, 
    const \PnlVect \ptr rhs} 
  \sshortdescribe In-place vector vector substraction  

\item \describefun{void}{pnl_vect_inv_term}{\PnlVect \ptr lhs}
  \sshortdescribe In-place term by term vector inversion  

\item \describefun{void}{pnl_vect_div_vect_term}{\PnlVect
    \ptr lhs, const \PnlVect \ptr rhs} 
  \sshortdescribe In-place term by term vector division

\item \describefun{void}{pnl_vect_mult_vect_term}{\PnlVect
    \ptr lhs, const \PnlVect \ptr rhs} 
  \sshortdescribe In-place vector vector term by term multiplication  

\item \describefun{void}{pnl_vect_map}{\PnlVect \ptr lhs, const
    \PnlVect \ptr rhs, double(\ptr f)(double)} 
    \sshortdescribe \var{lhs = f(rhs)} 

\item \describefun{void}{pnl_vect_map_inplace}{\PnlVect \ptr lhs, double(\ptr f)(double)}
  \sshortdescribe \var{lhs = f(lhs)} 

\item \describefun{void}{pnl_vect_map_vect}{\PnlVect \ptr lhs, const
  \PnlVect \ptr rhs1, const \PnlVect \ptr rhs2,
  double(\ptr f)(double, double)} 
  \sshortdescribe \var{lhs = f(rhs1, rhs2)} 

\item \describefun{void}{pnl_vect_map_vect_inplace}{\PnlVect \ptr
  lhs, \PnlVect \ptr rhs, double(\ptr f)(double,double)}
  \sshortdescribe \var{lhs = f(lhs,rhs)} 

\item \describefun{void}{pnl_vect_axpby}{double a, const \PnlVect \ptr x, 
    double b, \PnlVect \ptr y} 
  \sshortdescribe Compute \var{y : = a x + b y}. When \var{b==0}, the content
  of \var{y} is not used on input and instead \var{y} is resized to match \var{x}.

\item \describefun{double}{pnl_vect_sum}{const \PnlVect \ptr lhs}
  \sshortdescribe Return the sum of all the elements of a vector  

\item \describefun{void}{pnl_vect_cumsum}{\PnlVect \ptr lhs}
  \sshortdescribe Compute the cumulative sum of all the elements of a
  vector. The original vector is modified

\item \describefun{double}{pnl_vect_prod}{const \PnlVect \ptr V}
  \sshortdescribe Return the product of all the elements of a vector  

\item \describefun{void}{pnl_vect_cumprod}{\PnlVect \ptr lhs}
  \sshortdescribe Compute the cumulative product of all the elements of a
  vector. The original vector is modified
\end{itemize}

\subparagraph{Scalar products and norms}
\begin{itemize}
\item \describefun{double}{pnl_vect_norm_two}{const \PnlVect \ptr V}
  \sshortdescribe Return the two norm of a vector  

\item \describefun{double}{pnl_vect_norm_one}{const \PnlVect \ptr V}
  \sshortdescribe Return the one norm of a vector  

\item \describefun{double}{pnl_vect_norm_infty}{const \PnlVect \ptr V}
  \sshortdescribe Return the infinity norm of a vector  

\item \describefun{double}{pnl_vect_scalar_prod}{const \PnlVect
    \ptr rhs1, const \PnlVect \ptr rhs2} 
  \sshortdescribe Compute the scalar product between 2 vectors  
\item \describefun{int}{pnl_vect_cross}{\PnlVect \ptr lhs, 
  const \PnlVect \ptr x, const \PnlVect \ptr y}
  \sshortdescribe Compute the cross product of \var{x} and \var{y} and store the
  result in \var{lhs}. The vectors \var{x} and \var{y} must be of size 3 and
  FAIL is returned otherwise.

\item \describefun{double}{pnl_vect_dist}{const \PnlVect \ptr x, const \PnlVect \ptr y}
  \sshortdescribe Compute the distance between \var{x} and \var{y}, ie
  $\sqrt{\sum_i |x_i - y_i|^2}$.

\end{itemize}

\subparagraph{Comparison functions}

\begin{itemize}
  \item \describefun{int}{pnl_vect_isequal}{const \PnlVect \ptr V1, const \PnlVect \ptr V2, double err}
    \sshortdescribe Test if two vectors are equal up to \var{err} component--wise. The error \var{err} is either relative or absolute depending on the magnitude of the components. Return \var{TRUE} or \var{FALSE}.
  \item \describefun{int}{pnl_vect_isequal_abs}{const \PnlVect \ptr V1, const \PnlVect \ptr V2, double abserr}
    \sshortdescribe Test if two vectors are equal up to an absolute error \var{abserr} component--wise. Return \var{TRUE} or \var{FALSE}.
  \item \describefun{int}{pnl_vect_isequal_rel}{const \PnlVect \ptr V1, const \PnlVect \ptr V2, double relerr}
    \sshortdescribe Test if two vectors are equal up to a relative error \var{relerr} component--wise. Return \var{TRUE} or \var{FALSE}.
  \item \describefun{int}{pnl_vect_eq_all}{const \PnlVect \ptr v, double x}
    \sshortdescribe Test if all the components of \var{v} are equal to
    \var{x}. Return \var{TRUE} or \var{FALSE}.
\end{itemize}

\subparagraph{Ordering functions}
The following functions are not defined for PnlVectComplex because there is
no total ordering on Complex numbers

\begin{itemize}
\item \describefun{double}{pnl_vect_max}{const \PnlVect \ptr V}
  \sshortdescribe Return the maximum of a a vector  

\item \describefun{double}{pnl_vect_min}{const \PnlVect \ptr V}
  \sshortdescribe Return the minimum of a vector  

\item \describefun{void}{pnl_vect_minmax}{double \ptr m, double \ptr M, const \PnlVect \ptr}
  \sshortdescribe Compute the minimum and maximum of a vector which are
  returned in  \var{m} and \var{M} respectively.
  
\item \describefun{void}{pnl_vect_min_index}{double \ptr m, int \ptr im, const \PnlVect \ptr}
  \sshortdescribe Compute the minimum of a vector and its index stored in 
  sets \var{m} and \var{im} respectively.

\item \describefun{void}{pnl_vect_max_index}{double \ptr M, int \ptr iM, const \PnlVect \ptr}
  \sshortdescribe Compute the maximum of a vector and its index stored in 
  sets \var{m} and \var{im} respectively.

\item \describefun{void}{pnl_vect_minmax_index}{double \ptr m, double \ptr M,
    int \ptr im, int \ptr iM, const \PnlVect \ptr}
  \sshortdescribe Compute the minimum and maximum of a vector and the
  corresponding indices stored respectively in \var{m}, \var{M}, \var{im} and
  \var{iM}.

\item \describefun{void}{pnl_vect_qsort}{\PnlVect \ptr , char order}
  \sshortdescribe Sort a vector using a quick sort algorithm according to
  \var{order} (\verb!'i'! for increasing or \verb!'d'! for decreasing).

\item \describefun{void}{pnl_vect_qsort_index}{\PnlVect \ptr ,
    \PnlVectInt *index, char order}
  \sshortdescribe Sort a vector using a quick sort algorithm according to
  \var{order} (\verb!'i'! for increasing or \verb!'d'! for decreasing ). On
  output, \var{index} contains the permutation used to sort the vector.

\item \describefun{int}{pnl_vect_find}{\PnlVectInt \ptr
  ind, char \ptr type, int(\ptr f)(double \ptr t), \ldots}
  \sshortdescribe \var{f} is a function taking a C array as argument and
  returning an integer. \var{type} is a string composed by the letters 'r' and
  'v' and is used to describe the types of the arguments appearing after \var{f}.
  This function aims at simulating Scilab's \var{find}
  function. Here are a few examples (capital letters are used for vectors and
  small letters for real values)
  \begin{itemize}
    \item \verb!ind = find ( a < X )!
      \begin{verbatim}
      int isless ( double *t ) { return t[0] < t[1]; }
      pnl_vect_find ( ind, "rv", isless, a, X );
      \end{verbatim}
    \item \verb!ind = find (X <= Y)!
      \begin{verbatim}
      int isless ( double *t ) { return t[0] <= t[1]; }
      pnl_vect_find ( ind, "vv", isless, X, Y );
      \end{verbatim}
    \item \verb!ind = find ((a < X) && (X <= Y))!
      \begin{verbatim}
      int cmp ( double *t ) 
      { 
        return (t[0] <= t[1]) && (t[1] <= t[2]); 
      }
      pnl_vect_find ( ind, "rvv", cmp, a, X, Y );
      \end{verbatim}
  \end{itemize}
  \var{ind} contains on exit the indices \var{i} for which the function \var{f}
  returned \var{1}. This function returns \var{OK} or \var{FAIL} when something
  went wrong (size mismatch between matrices, invalid string type).

\end{itemize}


\subparagraph{Misc}

\begin{itemize}
\item \describefun{void}{pnl_vect_swap_elements}{\PnlVect \ptr v,
    int i, int j}
  \sshortdescribe Exchange \var{v[i]} and \var{v[j]}.
\item \describefun{void}{pnl_vect_reverse}{\PnlVect \ptr v}
  \sshortdescribe Perform a mirror operation on v. On output \var{v[i]
    = v[n-1-i]} for \var{i=0,\ldots,n-1} where \var{n} is the length of the vector.
\end{itemize}


\paragraph{Complex vector functions}

\begin{itemize}
\item \describefun{void}{pnl_vect_complex_mult_double}
  {\PnlVectComplex \ptr lhs, double x}
  \sshortdescribe In-place multiplication by a double.

\item \describefun{PnlVectComplex\ptr }{pnl_vect_complex_create_from_array}{int
    size, const double \ptr re, const double \ptr im}
  \sshortdescribe Create a \PnlVectComplex given the arrays of the
  real parts \var{re} and imaginary parts \var{im}.
\item \describefun{void}{pnl_vect_complex_split_in_array}{const \PnlVectComplex
    \ptr v, double \ptr re, double \ptr im}
    \sshortdescribe Split a complex vector into two C arrays : the
    real parts of the elements of \var{v} are stored into \var{re} and the
    imaginary parts into \var{im}.
\item \describefun{void}{pnl_vect_complex_split_in_vect}{const \PnlVectComplex
    \ptr v, \PnlVect \ptr re, \PnlVect \ptr im}
  \sshortdescribe Split a complex vector into two \PnlVect's : the
  real parts of the elements of \var{v} are stored into \var{re} and the
imaginary parts into \var{im}.
\end{itemize}

There exist functions to directly access the real or imaginary parts of an
element of a complex vector. These functions also have inlined versions that
are used if the variable \var{HAVE_INLINE} was declared at compilation time.

\begin{itemize}
\item \describefun{double}{pnl_vect_complex_get_real}
  {const \PnlVectComplex \ptr v, int i}
  \sshortdescribe Return the real part of \var{v[i]}.
  
\item \describefun{double}{pnl_vect_complex_get_imag}
  {const \PnlVectComplex \ptr v, int i}
  \sshortdescribe Return the imaginary part of \var{v[i]}.

\item \describefun{double\ptr }{pnl_vect_complex_lget_real}
  {const \PnlVectComplex \ptr v, int i}
  \sshortdescribe Return the real part of \var{v[i]} as a lvalue.

\item \describefun{double\ptr }{pnl_vect_complex_lget_imag}
  {const \PnlVectComplex \ptr v, int i}
  \sshortdescribe Return the imaginary part of \var{v[i]} as a lvalue.

\item \describefun{void}{pnl_vect_complex_set_real}
  {const \PnlVectComplex \ptr v, int i, double re}
  \sshortdescribe Set the real part of \var{v[i]} to \var{re}.

\item \describefun{void}{pnl_vect_complex_set_imag}
  {const \PnlVectComplex \ptr v, int i, double im}
  \sshortdescribe Set the imaginary part of \var{v[i]} to \var{im}.
\end{itemize}

Equivalently to these functions, there exist macros. When the compiler is able
to handle inline code, there is no gain in using macros instead of inlined
functions at least in principle.
\begin{itemize}
\item \describefun{}{GET_REAL}{v, i}
  \sshortdescribe Return the real part of \var{v[i]}.
  
\item \describefun{}{GET_IMAG}{v, i}
  \sshortdescribe Return the imaginary part of \var{v[i]}.
  
\item \describefun{}{LET_REAL}{v, i}
  \sshortdescribe Return the real part of \var{v[i]} as a lvalue.
  
\item \describefun{}{LET_IMAG}{v, i}
  \sshortdescribe Return the imaginary part of \var{v[i]} as a lvalue.
\end{itemize}

\subsection{Compact Vectors}
\subsubsection{Short description}

\describestruct{PnlVectCompact}
\begin{verbatim}
typedef struct PnlVectCompact {
  /**
   * Must be the first element in order for the object mechanism to work
   * properly. This allows any PnlVectCompact pointer to be cast to a PnlObject
   */
  PnlObject object; 
  int size; /* size of the vector */
  double val; /* single value */
  double *array; /* Pointer to double values */
  char convert; /* 'a', 'd' : array, double */
} PnlVectCompact;
\end{verbatim}

\subsubsection{Functions}

\begin{itemize}
  \item \describefun{\PnlVectCompact \ptr }{pnl_vect_compact_new}{}
  \sshortdescribe Create a \PnlVectCompact of size 0.  

\item \describefun{\PnlVectCompact \ptr }{pnl_vect_compact_create}{int n, double x}
  \sshortdescribe Create a \PnlVectCompact filled in with \var{x}
\item \describefun{\PnlVectCompact \ptr }{pnl_vect_compact_create_from_ptr}{int n, double *x}
  \sshortdescribe Create a \PnlVectCompact filled in with the content of
  \var{x}. Note that \var{x} must have at least \var{n} elements.

\item \describefun{int}{pnl_vect_compact_resize}{\PnlVectCompact
    \ptr v, int size, double x} 
  \sshortdescribe Resize a \PnlVectCompact.  

\item \describefun{\PnlVectCompact
    \ptr }{pnl_vect_compact_copy} {const \PnlVectCompact\ptr v}
  \sshortdescribe Copy a \PnlVectCompact  

\item \describefun{void}{pnl_vect_compact_free}{\PnlVectCompact \ptr \ptr v}
  \sshortdescribe Free a \PnlVectCompact  

\item \describefun{\PnlVect \ptr }{pnl_vect_compact_to_pnl_vect}
  {const \PnlVectCompact \ptr C} 
  \sshortdescribe Convert a \PnlVectCompact pointer to a \PnlVect pointer.  

\item \describefun{double}{pnl_vect_compact_get}{const \PnlVectCompact \ptr C, int i}
  \sshortdescribe Access function

\item \describefun{void}{pnl_vect_compact_set_all}{\PnlVectCompact
    \ptr C, double x}
  \sshortdescribe Set all elements of \var{C} to \var{x}. \var{C} is
  converted to a compact storage.
  
\item \describefun{void}{pnl_vect_compact_set_ptr}{\PnlVectCompact
    \ptr C, double \ptr ptr}
  \sshortdescribe Copy the array \var{ptr} into \var{C}. We assume that the
  sizes match. \var{C} is converted to a non compact storage.
\end{itemize}

%% matrix

\subsection{Matrices}
\subsubsection{Overview}

The structures and functions related to matrices are declared in
\verb!pnl/pnl_matrix.h!.

\describestruct{PnlMat}\describestruct{PnlMatInt}\describestruct{PnlMatComplex}
\begin{verbatim}
typedef struct _PnlMat{
  /**
   * Must be the first element in order for the object mechanism to work
   * properly. This allows any PnlMat pointer to be cast to a PnlObject
   */
  PnlObject object; 
  int m; /*!< nb rows */ 
  int n; /*!< nb columns */ 
  int mn; /*!< product m*n */
  int mem_size; /*!< size of the memory block allocated for array */
  double *array; /*!< pointer to store the data row-wise */
  int owner; /*!< 1 if the object owns its array member, 0 otherwise */
} PnlMat;

typedef struct _PnlMatInt{
  /**
   * Must be the first element in order for the object mechanism to work
   * properly. This allows any PnlMatInt pointer to be cast to a PnlObject
   */
  PnlObject object; 
  int m; /*!< nb rows */ 
  int n; /*!< nb columns */ 
  int mn; /*!< product m*n */
  int mem_size; /*!< size of the memory block allocated for array */
  int *array; /*!< pointer to store the data row-wise */
  int owner; /*!< 1 if the object owns its array member, 0 otherwise */
} PnlMatInt;

typedef struct _PnlMatComplex{
  /**
   * Must be the first element in order for the object mechanism to work
   * properly. This allows any PnlMatComplex pointer to be cast 
   * to a PnlObject
   */
  PnlObject object; 
  int m; /*!< nb rows */ 
  int n; /*!< nb columns */ 
  int mn; /*!< product m*n */
  int mem_size; /*!< size of the memory block allocated for array */
  dcomplex *array; /*!< pointer to store the data row-wise */
  int owner; /*!< 1 if the object owns its array member, 0 otherwise */
} PnlMatComplex;
\end{verbatim}
\var{m} is the number of rows, \var{n} is the number of columns. \var{array}
is a pointer containing the data of the matrix stored line wise, The element
\verb!(i, j)! of the matrix is \verb!array[i*m+j]!. \var{owner} is an integer to
know if the matrix owns its \var{array} pointer (\var{owner=1}) or shares it
with another structure (\var{owner=0}). \var{mem_size} is the number of
elements the matrix can hold at most.

The following operations are implemented on matrices and vectors. \var{alpha}
and \var{beta} are numbers, \var{A} and \var{B} are matrices and \var{x}
and \var{y} are vectors.
\begin{tabular}{ll}
  \reffun{pnl_mat_axpy} & \var{B := alpha * A + B} \\
  \reffun{pnl_mat_scalar_prod} & \var{x' A y} \\
  \reffun{pnl_mat_dgemm} & \var{C := alpha * op (A) * op (B) + beta * C}\\
  \reffun{pnl_mat_mult_vect_transpose_inplace} & \var{y = A' * x}\\
  \reffun{pnl_mat_mult_vect_inplace} & \var{y = A * x}\\
  \reffun{pnl_mat_lAxpby} & \var{y := lambda * A * x + beta * y}\\
  \reffun{pnl_mat_dgemv} & \var{y := alpha * op (A) * x + beta * y}\\
  \reffun{pnl_mat_dger} & \var{A := alpha x * y' + A}
\end{tabular}


\subsubsection{Generic Functions}
These functions exist for all types of matrices no matter what the basic type
is. The following conventions are used to name functions operating on matrices.
Here is the table of prefixes used for the different basic types.

\begin{center}
  \begin{tabular}[t]{lll}
    type & prefix & BASE\\
    \hline
    double & pnl_mat & double \\
    \hline
    int & pnl_mat_int & int \\
    \hline
    dcomplex & pnl_mat_complex & dcomplex
  \end{tabular}
\end{center}

In this paragraph we present the functions operating on \PnlMat
which exist for all types. To deduce the prototypes of these functions for
other basic types, one must replace {\tt pnl_mat} and {\tt double} according
the above table.

\paragraph{Constructors and destructors}


There are no special functions to access the sizes of a matrix, instead the fields
\verb!m!, \verb!n! and \verb!mn! give direct access to the number of rows, columns
and the size of the matrix.

\begin{itemize}
  \item \describefun{\PnlMat \ptr }{pnl_mat_new}{}
  \sshortdescribe Create a \PnlMat of size 0

\item \describefun{\PnlMat \ptr }{pnl_mat_create}{int m, int n}
  \sshortdescribe Create a \PnlMat  with \var{m} rows and \var{n} columns.

\item \describefun{\PnlMat \ptr }{pnl_mat_create_from_scalar}{int m, int n, double x}
  \sshortdescribe Create a \PnlMat with \var{m} rows and \var{n}
  columns and sets all the elements to \var{x}.

\item \describefun{\PnlMat \ptr }{pnl_mat_create_from_zero}{int m, int n}
  \sshortdescribe Create a \PnlMat with \var{m} rows and \var{n}
  columns and sets all elements to 0.

\item \describefun{\PnlMat \ptr }{pnl_mat_create_from_ptr}{int m, int n, const double \ptr x}
  \sshortdescribe Create a \PnlMat with \var{m} rows and \var{n}
  columns and copies the array \var{x} to the new vector. Be sure that \var{x}
  is long enough to fill all the vector because it cannot be checked inside the function.

\item \describefun{\PnlMat \ptr }{pnl_mat_create_from_list}{int
    m, int n, ...}
  \sshortdescribe Create a new \PnlMat pointer of size \var{m
    x n} filled with the extra arguments passed to the function. The
  number of extra arguments passed must be equal to \var{m x n}, be
  aware that this cannot be checked inside the function.

\item \describefun{\PnlMat \ptr }{pnl_mat_copy}{const \PnlMat \ptr M}
  \sshortdescribe Create a new \PnlMat which is a copy of \var{M}.
  
\item \describefun{\PnlMat \ptr }{pnl_mat_create_diag_from_ptr}
  {const double \ptr x, int d}
  \sshortdescribe Create a new squared \PnlMat by specifying its size and
  diagonal terms as an array.

\item \describefun{\PnlMat \ptr }{pnl_mat_create_diag}
  {const \PnlVect \ptr V}
  \sshortdescribe Create a new squared \PnlMat by specifying its diagonal
  terms in a \PnlVect.

\item \describefun{\PnlMat \ptr }{pnl_mat_create_from_file}{const char \ptr file}
  \sshortdescribe Read a matrix from a file and creates the corresponding
  \PnlMat. One row of the matrix corresponds to one line of the file and the elements of a row can be separated by spaces, tabs, commas or semicolons. Anything after a \verb!#! or \verb!%! is ignored up to the end of the line.

\item \describefun{void}{pnl_mat_free}{\PnlMat \ptr \ptr M}
  \sshortdescribe Free a \PnlMat and sets \var{\ptr M} to \var{NULL} 
\item \describefun{\PnlMat}{pnl_mat_wrap_array}{const double \ptr x, 
    int m, int n}
    \sshortdescribe Create a \PnlMat of size \var{m x n} 
    which contains \var{x}. No copy is made. It is just a container.
\item \describefun{\PnlMat}{pnl_mat_wrap_vect}
  {const \PnlVect \ptr V}
  \sshortdescribe Return a \PnlMat (not a pointer) whose array is
  the array of \var{V}. The new matrix shares its data with the
  vector \var{V}, which means that any modification to one of them will affect
  the other.


\item \describefun{void}{pnl_mat_clone}{\PnlMat \ptr clone, const
    \PnlMat \ptr M}
  \sshortdescribe Clone \var{M} into \var{clone}. No no new
  \PnlMat is created.

\item \describefun{int}{pnl_mat_resize}{\PnlMat \ptr M, int m, int n}
  \sshortdescribe Resize a \PnlMat. The new matrix is of size
  \var{m x n}. The old data are lost.
\item \describefun{\PnlVect \ptr }{pnl_vect_create_submat}{const
  \PnlMat \ptr M, const \PnlVectInt \ptr indi, const
  \PnlVectInt \ptr indj}
  \sshortdescribe Create a new vector containing the values \var{M(indi(:),
  indj(:))}. \var{indi} and \var{indj} must be of the same size.

\item \describefun{void}{pnl_vect_extract_submat}{\PnlVect \ptr
  V_sub, const \PnlMat \ptr M, const \PnlVectInt \ptr indi,
  const \PnlVectInt \ptr indj}
  \sshortdescribe On exit, \var{V_sub = M(indi(:), indj(:))}. \var{indi} and
  \var{indj} must be of the same size.

\item \describefun{void}{pnl_mat_extract_subblock}{\PnlMat \ptr
    M_sub, const \PnlMat \ptr M, int i, int len_i, int j, int
    len_j}
  \sshortdescribe \var{M_sub = M(i:i+len_i-1, j:j+len_j-1)}. \var{len_i}
  (resp. \var{len_j}) is the number of rows (resp. columns) to be extracted.
  
\item \describefun{void}{pnl_mat_set_subblock}{\PnlMat \ptr
    M, const \PnlMat \ptr block, int i, int j}
    \sshortdescribe If \var{block} is a matrix of size \var{m_block x n_block},
    the dimensions of \var{M} must satisfy that \var{M->m >= i + m_block} and 
    \var{M->n >= j + n_block}. On output \var{M(i:i+m_block-1, j:j+n_block-1) =
    block}. 
\end{itemize}  


\paragraph{Accessing elements.}

If it is supported by the compiler, the following functions are declared
inline. To speed up these functions, you can define the macro 
\texttt{PNL_RANGE_CHECK_OFF}, see Section~\ref{sec:inline} for an explanation. 

Accessing elements of a matrix is faster using the following macros
\begin{itemize}
  \item \describefun{}{MGET}{\PnlMat \ptr M, int i, int j}
  \sshortdescribe Return \var{M[i,j]} for reading, eg. \var{x=MGET(M,i,j)}
\item \describefun{}{MGET_INT}{\PnlMatInt \ptr M, int i, int j}
  \sshortdescribe Same as \reffun{MGET} but for an integer matrix.
\item \describefun{}{MGET_COMPLEX}{\PnlMatComplex \ptr M, int i, int j}
  \sshortdescribe Same as \reffun{MGET} but for a complex matrix.
\item \describefun{}{MLET}{\PnlMat \ptr M, int i, int j}
  \sshortdescribe Return \var{M[i,j]} as a lvalue for writing, eg.
  \var{MLET(M,i,j)=x}
\item \describefun{}{MLET_INT}{\PnlMatInt \ptr M, int i, int j}
  \sshortdescribe Same as \reffun{MLET} but for an integer matrix.
\item \describefun{}{MLET_COMPLEX}{\PnlMatComplex \ptr M, int i, int j}
  \sshortdescribe Same as \reffun{MLET} but for a complex matrix.
\end{itemize}

\begin{itemize}
\item \describefun{void}{pnl_mat_set}{\PnlMat \ptr M, int i, int j, double x}
  \sshortdescribe Set the value of M[i, j]=x  

\item \describefun{double}{pnl_mat_get}{const \PnlMat \ptr M, int i, int j}
  \sshortdescribe Get the value of M[i, j]  

\item \describefun{double \ptr }{pnl_mat_lget}{\PnlMat \ptr M, int i, int j}
  \sshortdescribe Return the address of M[i, j] for use as a lvalue.

\item \describefun{void}{pnl_mat_set_all}{\PnlMat \ptr M, double x}
  \sshortdescribe Set all elements of \var{M} to \var{x}.
\item \describefun{void}{pnl_mat_set_zero}{\PnlMat \ptr M}
  \sshortdescribe Set all elements of \var{M} to \var{0}.
  
\item \describefun{void}{pnl_mat_set_id}{\PnlMat \ptr M}
  \sshortdescribe Set the matrix \var{M} to the identity
  matrix. \var{M} must be a square matrix.

\item \describefun{void}{pnl_mat_set_diag}{\PnlMat \ptr M,
    double x, int d}
  \sshortdescribe Set the $\var{d}^{\text{th}}$ diagonal terms of the matrix
  \var{M} to the value \var{x}. \var{M} must be a square matrix.
\item \describefun{void}{pnl_mat_set_from_ptr}{\PnlMat \ptr M,
  const double \ptr x}
  \sshortdescribe Set \var{M} row--wise with the values given by \var{x}. The
  array \var{x} must be at least M->mn long.
\item \describefun{void}{pnl_mat_get_row}{\PnlVect
    \ptr V, const \PnlMat \ptr M, int i}
  \sshortdescribe Extract and copies the \var{i}-th row of \var{M} into
  \var{V}.

\item \describefun{void}{pnl_mat_get_col}{\PnlVect \ptr V, 
    const \PnlMat \ptr M, int j}
  \sshortdescribe Extract and copies the \var{j}-th column of \var{M} into \var{V}.
  
\item \describefun{\PnlVect}{pnl_vect_wrap_mat_row}
  {const \PnlMat \ptr M, int i}
  \sshortdescribe Return a \PnlVect (not a pointer) whose array is
  the \var{i}-th row of \var{M}. The new vector shares its data with the
  matrix \var{M}, which means that any modification to one of them will affect
  the other.
\item \describefun{\PnlMat}{pnl_mat_wrap_mat_rows}
  {const \PnlMat \ptr M, int i_start, int i_end}
  \sshortdescribe Return a \PnlMat (not a pointer) holding rows from
  \var{i_start} to \var{i_end} (included) of \var{M}.
  The new matrix shares its data with the
  matrix \var{M}, which means that any modification to one of them will affect
  the other.
  
\item \describefun{void}{pnl_mat_swap_rows}{\PnlMat \ptr M, int i, int j}
  \sshortdescribe Swap two rows of a matrix.  

\item \describefun{void}{pnl_mat_set_col}{\PnlMat \ptr M, 
    const \PnlVect \ptr V, int j}
    \sshortdescribe Set the \var{i}-th column of a matrix \var{M} with the content of \var{V}
\item \describefun{void}{pnl_mat_set_col_from_ptr}{\PnlMat \ptr M, 
    const double \ptr x, int j}
    \sshortdescribe Set the \var{i}-th column of \var{M} with the content of \var{x}.

\item \describefun{void}{pnl_mat_set_row}{\PnlMat \ptr M, 
    const \PnlVect \ptr V, int i}
    \sshortdescribe Set the \var{i}-th row of \var{M} with the content of \var{V}
  \item \describefun{void}{pnl_mat_set_row_from_ptr}{\PnlMat \ptr M, 
    const double \ptr x, int i}
    \sshortdescribe Set the \var{i}-th row of \var{M} with the content of \var{x}

\item \describefun{void}{pnl_mat_add_row}{\PnlMat \ptr M, int i, const \PnlVect \ptr r}
  \sshortdescribe Add a row in matrix \var{M} before position \var{i} and fill
  it with the content of \var{r}. If \var{r == NULL}, row \var{i} is left
  uninitialized. The index \var{i} may vary between \var{0} --- add a row at the
  top of the matrix --- and \var{M->m} --- add a row after all rows.

\item \describefun{void}{pnl_mat_del_row}{\PnlMat \ptr M, int i}
  \sshortdescribe Delete the row with index \var{i} (between \var{0} and
  \var{M->m-1}) of the matrix \var{M}.

\end{itemize}

\paragraph{Printing Matrices}

\begin{itemize}
\item \describefun{void}{pnl_mat_print}{const \PnlMat \ptr M}
  \sshortdescribe Print a matrix to the standard output.  

\item \describefun{void}{pnl_mat_fprint}{FILE \ptr fic, const \PnlMat \ptr M}
  \sshortdescribe Print a matrix to a file. The
  saved matrix can be reloaded by the function
  \reffun{pnl_mat_create_from_file}.

\item \describefun{void}{pnl_mat_print_nsp}{const \PnlMat \ptr M}
  \sshortdescribe Print a matrix to the standard output in a format
  compatible with Nsp.  

\item \describefun{void}{pnl_mat_fprint_nsp}{FILE \ptr fic, const
    \PnlMat \ptr M}
  \sshortdescribe Print a matrix to a file in a format compatible with Nsp.
\end{itemize}

\paragraph{Applying external operations}
\begin{itemize}
  \item \describefun{void}{pnl_mat_plus_scalar}{\PnlMat \ptr lhs, double x}
  \sshortdescribe In-place matrix scalar addition  

\item \describefun{void}{pnl_mat_minus_scalar}{\PnlMat \ptr lhs, double x}
  \sshortdescribe In-place matrix scalar substraction  

\item \describefun{void}{pnl_mat_mult_scalar}{\PnlMat \ptr lhs, double x}
  \sshortdescribe In-place matrix scalar multiplication  

\item \describefun{void}{pnl_mat_div_scalar}{\PnlMat \ptr lhs, double x}
  \sshortdescribe In-place matrix scalar division  

\end{itemize}

\paragraph{Element wise operations}

\begin{itemize}
\item \describefun{void}{pnl_mat_mult_mat_term}{\PnlMat \ptr lhs, 
    const \PnlMat \ptr rhs} 
  \sshortdescribe In-place matrix matrix term by term product  

\item \describefun{void}{pnl_mat_div_mat_term}{\PnlMat \ptr lhs, 
    const \PnlMat \ptr rhs} 
  \sshortdescribe In-place matrix matrix term by term division

\item \describefun{void}{pnl_mat_kron_mat_inplace}{\PnlMat \ptr res, const \PnlMat \ptr B, const \PnlMat \ptr B}
  \sshortdescribe In-place Kroenecker product of \var{A} and \var{B}

\item \describefun{\PnlMat \ptr}{pnl_mat_kron_mat}{const \PnlMat \ptr B, const \PnlMat \ptr B}
  \sshortdescribe Return the Kroenecker product of \var{A} and \var{B}

\item \describefun{void}{pnl_mat_map_inplace}{\PnlMat \ptr lhs, 
    double(\ptr f)(double)} 
  \sshortdescribe \var{lhs = f(lhs)}.


\item \describefun{void}{pnl_mat_map}{\PnlMat \ptr lhs, const
    \PnlMat \ptr rhs, double(\ptr f)(double)} 
    \sshortdescribe \var{lhs = f(rhs)}.

  \item \describefun{void}{pnl_mat_map_mat_inplace}{\PnlMat \ptr lhs, const
  \PnlMat \ptr rhs, double(\ptr f)(double, double)} 
  \sshortdescribe \var{lhs = f(lhs, rhs)}.

\item \describefun{void}{pnl_mat_map_mat}{\PnlMat \ptr lhs, const
  \PnlMat \ptr rhs1, const
  \PnlMat \ptr rhs2, double(\ptr f)(double, double)} 
  \sshortdescribe \var{lhs = f(rhs1, rhs2)}.


\item \describefun{double}{pnl_mat_sum}{const \PnlMat \ptr lhs}
  \sshortdescribe Sum matrix component-wise  

\item \describefun{void}{pnl_mat_sum_vect}{\PnlVect \ptr y, const
    \PnlMat \ptr A, char c}
  \sshortdescribe Sum matrix column or row wise. Argument \var{c} can be
  either 'r' (to get a row vector) or 'c' (to get a column vector). When
  \var{c='r'}, $y(j) = \sum_i A_{ij}$ and when \var{c='rc}, $y(i) = \sum_j
  A_{ij}$.

\item \describefun{void}{pnl_mat_cumsum}{\PnlMat \ptr A, char c} 
  \sshortdescribe Cumulative sum over the rows or columns. Argument \var{c}
  can be either 'r' to sum over the rows or 'c' to sum over the columns. When
  \var{c='r'}, $A_{ij} = \sum_{1 \le k \le i} A_{kj}$ and when \var{c='rc}, 
  $A_{ij} = \sum_{1 \le k \le j} A_{ik}$.

\item \describefun{double}{pnl_mat_prod}{const \PnlMat \ptr lhs}
  \sshortdescribe Product matrix component-wise

\item \describefun{void}{pnl_mat_prod_vect}{\PnlVect \ptr y, const
    \PnlMat \ptr A, char c}
  \sshortdescribe Prod matrix column or row wise. Argument \var{c} can be
  either 'r' (to get a row vector) or 'c' (to get a column vector). When
  \var{c='r'}, $y(j) = \prod_i A_{ij}$ and when \var{c='rc}, $y(i) = \prod_j
  A_{ij}$.

\item \describefun{void}{pnl_mat_cumprod}{\PnlMat \ptr A, char c} 
  \sshortdescribe Cumulative prod over the rows or columns. Argument \var{c}
  can be either 'r' to prod over the rows or 'c' to prod over the columns. When
  \var{c='r'}, $A_{ij} = \prod_{1 \le k \le i} A_{kj}$ and when \var{c='rc}, 
  $A_{ij} = \prod_{1 \le k \le j} A_{ik}$.
\end{itemize}

\subparagraph{Comparison functions}

\begin{itemize}
  \item \describefun{int}{pnl_mat_isequal}{const \PnlMat \ptr A, const \PnlMat \ptr B, double err}
    \sshortdescribe Test if two matrices are equal up to \var{err} component--wise. The error \var{err} is either relative or absolute depending on the magnitude of the components. Return \var{TRUE} or \var{FALSE}.
  \item \describefun{int}{pnl_mat_isequal_abs}{const \PnlMat \ptr A, const \PnlMat \ptr B, double abserr}
    \sshortdescribe Test if two matrices are equal up to an absolute error \var{abserr} component--wise. Return \var{TRUE} or \var{FALSE}.
  \item \describefun{int}{pnl_mat_isequal_rel}{const \PnlMat \ptr A, const \PnlMat \ptr B, double relerr}
    \sshortdescribe Test if two matrices are equal up to a relative error \var{relerr} component--wise. Return \var{TRUE} or \var{FALSE}.
\end{itemize}

\paragraph{Ordering operations}

\begin{itemize}
\item \describefun{void}{pnl_mat_max}{ \PnlVect \ptr M, const
    \PnlMat \ptr A, char d}
  \sshortdescribe On exit, $\var{M}(i) = \max_{j}(\var{A}(i, j))$ when \var{d='c'}
  and $\var{M}(i) = \max_{j}(\var{A}(j, i))$ when \var{d='r'} and $\var{M(0)} =
  \max_{i,j} = \var{A}(i, j)$ when \var{d='*'}.

\item \describefun{void}{pnl_mat_min}{ \PnlVect \ptr m,const
    \PnlMat \ptr A, char d}
  \sshortdescribe On exit, $\var{m}(i) = \min_{j}(\var{A}(i, j))$ when \var{d='c'}
  and $\var{m}(i) = \min_{j}(\var{A}(j, i))$ when \var{d='r'} and $\var{M(0)} =
  \min_{i,j} = \var{A}(i, j)$ when \var{d='*'}.

\item \describefun{void}{pnl_mat_minmax}{ \PnlVect \ptr m,
    \PnlVect \ptr M, const \PnlMat \ptr A, char d}
  \sshortdescribe On exit, $\var{m}(i) = \min_{j}(\var{A}(i, j))$ and $\var{M}(i) =
  \max_{j}(\var{A}(i, j))$ when \var{d='c'} and $\var{m}(i) = \min_{j}(\var{A}(j, i))$
  and $\var{M}(i) = \min_{j}(\var{A}(j, i))$ when \var{d='r'} and $\var{M(0)} =
  \max_{i,j} = \var{A}(i, j)$ and  $\var{m(0)} =\min_{i,j} = \var{A}(i, j)$ when \var{d='*'}.
  
\item \describefun{void}{pnl_mat_min_index}{ \PnlVect \ptr m,
    \PnlVectInt \ptr im, const \PnlMat \ptr  A, char d}
  \sshortdescribe Idem as \reffun{pnl_mat_min} and \var{index} contains the
  indices of the minima. If \var{index==NULL}, the indices are not computed.

\item \describefun{void}{pnl_mat_max_index}{ \PnlVect \ptr M,
    \PnlVectInt \ptr iM, const \PnlMat \ptr  A, char d}
  \sshortdescribe Idem as \reffun{pnl_mat_max} and \var{index} contains the
  indices of the maxima. If \var{index==NULL}, the indices are not computed.

\item \describefun{void}{pnl_mat_minmax_index}{ \PnlVect \ptr m,
    \PnlVect \ptr M, \PnlVectInt \ptr im,
    \PnlVectInt \ptr iM, const \PnlMat \ptr A, char d}
  \sshortdescribe Idem as \reffun{pnl_mat_minmax} and \var{im} contains the
  indices of the minima and \var{iM} contains the indices of the minima. If
  \var{im==NULL} (resp. \var{iM==NULL}, the indices of the minima
  (resp. maxima) are not computed.

\item \describefun{void}{pnl_mat_qsort}{\PnlMat \ptr , char dir, char order}
  \sshortdescribe Sort a matrix using a quick sort algorithm according to
  \var{order} (\verb!'i'! for increasing or \verb!'d'! for decreasing). The parameter \var{dir} determines
  whether the matrix is sorted by rows or columns. If \var{dir='c'}, each row
  is sorted independently of the others whereas if \var{dir='r'}, each column
  is sorted independently of the others.

\item \describefun{void}{pnl_mat_qsort_index}{\PnlMat \ptr ,
    \PnlMatInt *index, char dir, char order}
  \sshortdescribe Sort a matrix using a quick sort algorithm according to
  \var{order} (\verb!'i'! for increasing or \verb!'d'! for decreasing). The
  parameter \var{dir} determines whether the matrix is sorted by rows or
  columns. If \var{dir='c'}, each row is sorted independently of the others
  whereas if \var{dir='r'}, each column is sorted independently of the
  others. In addition to the function \reffun{pnl_mat_qsort}, the permutation
  index is computed and stored into \var{index}.

\item \describefun{int}{pnl_mat_find}{\PnlVectInt \ptr
  indi, \PnlVectInt indj, char \ptr type, int(\ptr f)(double \ptr t), \ldots}
  \sshortdescribe \var{f} is a function taking a C array as argument and
  returning an integer. \var{type} is a string composed by the letters 'r' and
  'm' and is used to describe the types of the arguments appearing after \var{f}
  : 'r' for real numbers and 'm' for matrices.
  This function aims at simulating Scilab's \var{find}
  function. Here are a few examples (capital letters are used for matrices and
  small letters for real values)
  \begin{itemize}
    \item \verb![indi, indj] = find ( a < X )!
      \begin{verbatim}
      int isless ( double *t ) { return t[0] < t[1]; }
      pnl_mat_find ( indi, indj, "rm", isless, a, X );
      \end{verbatim}
    \item \verb!ind = find (X <= Y)!
      \begin{verbatim}
      int isless ( double *t ) { return t[0] <= t[1]; }
      pnl_mat_find ( ind, "mm", isless, X, Y );
      \end{verbatim}
    \item \verb![indi, indj] = find ((a < X) && (X <= Y))!
      \begin{verbatim}
      int cmp ( double *t ) 
      { 
        return (t[0] <= t[1]) && (t[1] <= t[2]); 
      }
      pnl_mat_find ( indi, indj, "rmm", cmp, a, X, Y );
      \end{verbatim}
  \end{itemize}
  \var{(indi, indj)} contains on exit the indices \var{(i,j)} for which the function \var{f}
  returned \var{1}. Note that if \var{indj == NULL} on entry, a linear indexing
  is used for matrices, which means that matrices are seen as large vectors
  built up be stacking rows. This function returns \var{OK} or \var{FAIL} if
  something went wrong (size mismatch between matrices, invalid string type).
\end{itemize}


\paragraph{Standard matrix operations}

  
\begin{itemize}
\item \describefun{void}{pnl_mat_plus_mat}{\PnlMat \ptr lhs, const
    \PnlMat \ptr rhs} 
  \sshortdescribe In-place matrix matrix addition  

\item \describefun{void}{pnl_mat_minus_mat}{\PnlMat \ptr lhs, 
    const \PnlMat \ptr rhs} 
  \sshortdescribe In-place matrix matrix substraction  
  
\item \describefun{void}{pnl_mat_sq_transpose}{\PnlMat \ptr M}
  \sshortdescribe On exit, \var{M} is transposed

\item \describefun{\PnlMat \ptr}{pnl_mat_transpose}{const
    \PnlMat \ptr M} 
    \sshortdescribe Create a new matrix which is the transposition of \var{M}

\item \describefun{void}{pnl_mat_tr}{
  \PnlMat \ptr tM, const \PnlMat \ptr M} 
  \sshortdescribe On exit, \var{tM = M'}

\item \describefun{double}{pnl_mat_trace}{const \PnlMat \ptr M}
  \sshortdescribe Return the trace of a square matrix.

\item \describefun{void}{pnl_mat_axpy}{double alpha, const \PnlMat
    \ptr A, \PnlMat \ptr B}
  \sshortdescribe Compute \var{B := alpha * A + B}

\item \describefun{void}{pnl_mat_dger}{double alpha, const \PnlVect
    \ptr x, const \PnlVect \ptr y, \PnlMat \ptr A}
  \sshortdescribe Compute \var{A := alpha x * y' + A}

\item \describefun{\PnlVect \ptr }{pnl_mat_mult_vect}{const
    \PnlMat \ptr A, const \PnlVect \ptr x} 
  \sshortdescribe Matrix vector multiplication  \var{A * x}

\item \describefun{void}{pnl_mat_mult_vect_inplace}{\PnlVect
    \ptr y, const \PnlMat \ptr A, const \PnlVect
    \ptr x} 
    \sshortdescribe In place matrix vector multiplication  \var{y = A * x}. You
    cannot use the same vector for \var{x} and \var{y}.

\item \describefun{\PnlVect \ptr }{pnl_mat_mult_vect_transpose}{const
    \PnlMat \ptr A, const \PnlVect \ptr x} 
  \sshortdescribe Matrix vector multiplication  \var{A' * x}

\item \describefun{void}{pnl_mat_mult_vect_transpose_inplace}{\PnlVect
    \ptr y, const \PnlMat \ptr A, const \PnlVect
    \ptr x} 
  \sshortdescribe In place matrix vector multiplication  \var{y = A' * x}.  You
  cannot use the same vector for \var{x} and \var{y}. The vectors \var{x} and \var{y}
  must be different.

\item \describefun{int}{pnl_mat_cross}{\PnlMat \ptr lhs, const
  \PnlMat \ptr A, const \PnlMat \ptr B}
  \sshortdescribe Compute the cross products of the vectors given in matrices
  \var{A} and \var{B} which must have either 3 rows or 3 columns. A row wise
  computation is first tried, then a column wise approach is tested.
  \var{FAIL} is returned in case no dimension equals 3.

  
\item \describefun{void}{pnl_mat_lAxpby}{double lambda, const \PnlMat
    \ptr A, const \PnlVect \ptr x, double b, \PnlVect \ptr y} 
  \sshortdescribe Compute \var{y := lambda A x + b y}. When \var{b=0}, the
  content of \var{y} is not used on input and instead \var{y} is resized to
  match \var{A*x}. The vectors \var{x} and \var{y} must be different.


\item \describefun{void}{pnl_mat_dgemv}{char trans, double lambda, const
    \PnlMat \ptr A, const \PnlVect \ptr x, double mu, 
    \PnlVect \ptr b}
  \sshortdescribe Compute \var{b := lambda op(A) x + mu b}, where \var{op (X) =
    X} or \var{op (X) = X'}. If \var{trans='N'} or \var{trans='n'}, \var{op (A)
    = A}, whereas if \var{trans='T'} or \var{trans='t'}, \var{op (A) = A'}.When
  \var{mu==0}, the content of \var{b} is not used and instead \var{b} is resized
  to match \var{op(A)*x}. The vectors \var{x} and \var{b} must be different.

\item \describefun{void}{pnl_mat_dgemm}{char transA, char transB, double
    alpha, const \PnlMat \ptr A, const \PnlMat \ptr B, 
    double beta, \PnlMat \ptr C}
  \sshortdescribe Compute \var{C := alpha * op(A) * op (B) + beta *
    C}. When beta=0, the content of \var{C} is unused and instead \var{C}
  is resized to store \var{alpha A \ptr B}. If \var{transA='N'} or
  \var{transA='n'}, \var{op (A) = A}, whereas if \var{transA='T'} or
  \var{transA='t'}, \var{op (A) = A'}. The same holds for \var{transB}. The matrix
  \var{C} must be different from \var{A} and \var{B}.
  
\item \describefun{\PnlMat \ptr }{pnl_mat_mult_mat}{const
    \PnlMat \ptr rhs1, const \PnlMat \ptr rhs2} 
  \sshortdescribe Matrix multiplication  \var{rhs1 * rhs2}

\item \describefun{void}{pnl_mat_mult_mat_inplace}{\PnlMat
    \ptr lhs, const \PnlMat \ptr rhs1, const \PnlMat
    \ptr rhs2} 
    \sshortdescribe In-place matrix multiplication  \var{lhs = rhs1 * rhs2}. The
    matrix \var{lhs} must be different from \var{rhs1} and \var{rhs2}.

\item \describefun{double}{pnl_mat_scalar_prod}{const \PnlMat
    \ptr A, const \PnlVect \ptr x, const \PnlVect \ptr y}
  \sshortdescribe Compute \var{x' * A * y}
\item \describefun{void}{pnl_mat_exp}{\PnlMat \ptr B, 
    const \PnlMat \ptr A}
  \sshortdescribe Compute the matrix exponential \var{B = exp(A)}.

\item \describefun{void}{pnl_mat_log}{\PnlMat \ptr B, 
    const \PnlMat \ptr A}
  \sshortdescribe Compute the matrix logarithm \var{B = log(A)}. For the
  moment, this function only works if \var{A} is diagonalizable.

\item \describefun{void}{pnl_mat_eigen}{\PnlVect *v, \PnlMat \ptr P, 
    const \PnlMat \ptr A, int with_eigenvector}
  \sshortdescribe Compute the eigenvalues (stored in \var{v}) and optionally
  the eigenvectors stored column wise in \var{P} when
  \var{with_eigenvector==TRUE}. If \var{A} is symmetric or Hermitian in the
  complex case, \var{P} is orthonormal. When \var{with_eigenvector=FALSE},
  \var{P} can be \var{NULL}.
\end{itemize}


\paragraph{Linear systems and matrix decompositions}

The following functions are designed to solve linear system of the from \var{A x
= b} where \var{A} is a matrix and \var{b} is a vector except in the functions
\reffun{pnl_mat_syslin_mat}, \reffun{pnl_mat_lu_syslin_mat} and
\reffun{pnl_mat_chol_syslin_mat} which expect the right hand side member to be a
matrix too. Whenever the vector \var{b} is not needed once the system is solved,
you should consider using ``inplace'' functions.


All the functions described in this paragraph return \var{OK} if the
computations have been carried out successfully and \var{FAIL} otherwise.

\begin{itemize}
\item \describefun{int}{pnl_mat_chol}{\PnlMat \ptr M}
  \sshortdescribe Compute the Cholesky decomposition of \var{M}. \var{M} must
  be symmetric, the positivity is tested in the algorithm.  \var{M = L * L'}.
  On exit, the lower part of \var{M} contains the Cholesky decomposition L and
  the upper part is set to zero. 

\item \describefun{int}{pnl_mat_pchol}{\PnlMat \ptr M, double tol,
  int \ptr rank, \PnlVectInt \ptr p}
    \sshortdescribe Compute the
    Cholesky decomposition of \var{M} with complete pivoting. 
    \var{P' * A * P = L * L'}.
    \var{M} must be
    symmetric positive semi-definite. On exit, the lower part of \var{M}
    contains the Cholesky decomposition \var{L} and the upper part is set to zero. The
    permutation matrix is stored in an integer vector \var{p} : the only non
    zero elements of \var{P} are \var{P(p(k),k) = 1}

\item \describefun{int}{pnl_mat_lu}{\PnlMat \ptr A, 
    \PnlPermutation \ptr p} 
  \sshortdescribe Compute a P A = LU factorization. \var{P} must be an
  already allocated  \PnlPermutation. On exit the decomposition is
  stored in \var{A}, the lower part of \var{A} contains L while the upper part
  (including the diagonal terms) contains U. Remember that the diagonal
  elements of \var{L} are all 1. Row \var{i} of \var{A} was interchanged with
  row \var{p(i)}.
  
\item \describefun{int}{pnl_mat_upper_syslin}{\PnlVect
    \ptr x, const \PnlMat \ptr U, const \PnlVect\ptr b}
  \sshortdescribe Solve an upper triangular linear system \var{U x = b}

\item \describefun{int}{pnl_mat_lower_syslin}{\PnlVect
    \ptr x, const \PnlMat \ptr L, const \PnlVect\ptr b}
  \sshortdescribe Solve a lower triangular linear system  \var{L x = b}
  
\item \describefun{int}{pnl_mat_chol_syslin}{\PnlVect \ptr x, 
    const \PnlMat \ptr chol, const \PnlVect \ptr b} 
  \sshortdescribe Solve a symmetric definite positive linear system A x = b, 
  in which \var{chol} is assumed to be the Cholesky decomposition of A
  computed by \reffun{pnl_mat_chol}

\item \describefun{int}{pnl_mat_chol_syslin_inplace}{
    const \PnlMat \ptr chol, \PnlVect \ptr b} 
  \sshortdescribe Solve a symmetric definite positive linear system A x = b, 
  in which \var{chol} is assumed to be the Cholesky decomposition of A
  computed by \reffun{pnl_mat_chol}. The solution of the system is stored in
  \var{b} on exit.

\item \describefun{int}{pnl_mat_lu_syslin}{\PnlVect \ptr x, const
    \PnlMat \ptr LU, const \PnlPermutation \ptr p, 
    const \PnlVect \ptr b} 
  \sshortdescribe Solve a linear system A x = b using a LU decomposition.
  \var{LU} and \var{P} are assumed to be the PA = LU decomposition as computed
  by \reffun{pnl_mat_lu}. In particular, the structure of the matrix \var{LU}
  is the following : the lower part of \var{A} contains L while the upper part
  (including the diagonal terms) contains U. Remember that the diagonal
  elements of \var{L} are all 1.

\item \describefun{int}{pnl_mat_lu_syslin_inplace}{const
    \PnlMat \ptr LU, const \PnlPermutation \ptr p, 
    \PnlVect \ptr b} 
  \sshortdescribe Solve a linear system A x = b using a LU decomposition.
  \var{LU} and \var{P} are assumed to be the PA = LU decomposition as computed
  by \reffun{pnl_mat_lu}. In particular, the structure of the matrix \var{LU}
  is the following : the lower part of \var{A} contains L while the upper part
  (including the diagonal terms) contains U. Remember that the diagonal
  elements of \var{L} are all 1. The solution of the system is stored in \var{b}
  on exit.
  
\item \describefun{int}{pnl_mat_syslin}{\PnlVect \ptr x, const
    \PnlMat \ptr A, const \PnlVect \ptr b} 
  \sshortdescribe Solve a linear system A x = b using a LU factorization
  which is computed inside this function.

\item \describefun{int}{pnl_mat_syslin_inplace}{\PnlMat \ptr A, 
    \PnlVect \ptr b} 
  \sshortdescribe Solve a linear system A x = b using a LU factorization
  which is computed inside this function. The solution of the system is stored
  in \var{b} and \var{A} is overwritten by its LU decomposition.

\item \describefun{int}{pnl_mat_syslin_mat}{\PnlMat\ptr A, 
    \PnlMat \ptr B} 
  \sshortdescribe Solve a linear system A X = B using a LU factorization
  which is computed inside this function. \var{A} and  \var{B} are
  matrices. \var{A} must be square. The solution of the system is stored in
  \var{B} on exit. On exit, \var{A} contains the LU decomposition of the input
  matrix which is lost.

\item \describefun{int}{pnl_mat_chol_syslin_mat}{const \PnlMat\ptr A, \PnlMat \ptr B}
  \sshortdescribe Solve a linear system A X = B using a Cholesky factorization
  of the symmetric positive defnite matrix \var{A}.  \var{A} contains the
  Cholesky decomposition as computed by \reffun{pnl_mat_chol}. \var{B} is matrix
  with the same number of rows as \var{A}. The solution of
  the system is stored in \var{B} on exit. 

\item \describefun{int}{pnl_mat_lu_syslin_mat}{const \PnlMat\ptr A,
  const \PnlPermutation \ptr p, \PnlMat \ptr B}
  \sshortdescribe Solve a linear system A X = B using a \var{P A = L U} factorization.
  \var{A} contains the \var{L U} factors and \var{p} the associated permutation.
  \var{A} and \var{p} must have been computed by \reffun{pnl_mat_lu}. \var{B} is matrix
  with the same number of rows as \var{A}.
  The solution of the system is stored in \var{B} on exit. 

\end{itemize}


The following functions are designed to invert matrices. The authors provide
these functions although they cannot find good reasons to use them. Note that
to solve a linear system, one must used the \var{syslin} functions and not
invert the system matrix because it is much longer.
\begin{itemize}
\item \describefun{int}{pnl_mat_upper_inverse}{\PnlMat \ptr A, 
    const \PnlMat \ptr B}
  \sshortdescribe Inversion of an upper triangular matrix  

\item \describefun{int}{pnl_mat_lower_inverse}{\PnlMat \ptr A, 
    const \PnlMat \ptr B}
  \sshortdescribe Inversion of a lower triangular matrix  

\item \describefun{int}{pnl_mat_inverse}{\PnlMat
  \ptr inverse, const \PnlMat \ptr A}
  \sshortdescribe Compute the inverse of a matrix A and stores the result
  into \var{inverse}. A LU factorisation of the matrix \var{A} is computed
  inside this function.
\item \describefun{int}{pnl_mat_inverse_with_chol}{\PnlMat
  \ptr inverse, const \PnlMat \ptr A}
  \sshortdescribe Compute the inverse of a symmetric positive definite matrix
  A and stores the result into \var{inverse}. The Cholesky factorisation of
  the matrix \var{A} is computed inside this function.
\end{itemize}

\subsubsection{Functions specific to base type {\tt double}}


\paragraph{Linear systems and matrix decompositions}

The following functions are designed to solve linear system of the from \var{A x
= b} where \var{A} is a matrix and \var{b} is a vector except in the functions
\reffun{pnl_mat_syslin_mat}, \reffun{pnl_mat_lu_syslin_mat} and
\reffun{pnl_mat_chol_syslin_mat} which expect the right hand side member to be a
matrix too. Whenever the vector \var{b} is not needed once the system is solved,
you should consider using ``inplace'' functions.


All the functions described in this paragraph return \var{OK} if the
computations have been carried out successfully and \var{FAIL} otherwise.

\begin{itemize}
\item \describefun{int}{pnl_mat_qr}{\PnlMat \ptr Q,
  \PnlMat \ptr R, \PnlPermutation \ptr p,
  const \PnlMat \ptr A} 
  \sshortdescribe Compute a \var{A P = QR} decomposition. If on entry
  \var{P=NULL}, then the decomposition is computed without pivoting, i.e
  \var{A = QR}. When $P \ne NULL$, \var{P} must be an already allocated
  \PnlPermutation. \var{Q} is an orthogonal matrix, i.e
  $\var{Q}^{-1} = \var{Q}^{T}$ and \var{R} is an upper triangular matrix. The
  use of pivoting improves the numerical stability when \var{A} is almost rank
  deficient, i.e when the smallest eigenvalue of \var{A} is very close to $0$.

\item \describefun{int}{pnl_mat_qr_syslin}{\PnlVect \ptr x,
    const \PnlMat \ptr Q, const \PnlMat \ptr R,
    const \PnlVectInt \ptr p, const \PnlVect \ptr b}
  \sshortdescribe Solve a linear system \var{A x = b} where \var{A} is given by
  its QR decomposition with column pivoting as computed by the function
  \reffun{pnl_mat_qr}.
\item \describefun{int}{pnl_mat_ls}{const \PnlMat\ptr A, \PnlVect \ptr b}
  \sshortdescribe Solve a linear system A x = b in the least square sense,
  i.e. $\var{x} = \arg\min_U \| A * u - b\|^2$. The solution is stored into
  \var{b} on exit. It internally uses a \var{AP = QR} decomposition.

\item \describefun{int}{pnl_mat_ls_mat}{const \PnlMat\ptr A,
    \PnlMat \ptr B}
  \sshortdescribe Solve a linear system A X = B with \var{A} and \var{B} two
  matrices in the least square sense, i.e. $\var{X} = \arg\min_U \| A * U -
  B\|^2$. The solution is stored into \var{B} on exit. It internally uses a
  \var{AP = QR} decomposition. Same function as \reffun{pnl_mat_ls} but handles
  several r.h.s.

\end{itemize}

\subsubsection{Functions specific to base type {\tt dcomplex}}

\begin{itemize}
  \item \describefun{\PnlMatComplex\ptr }{pnl_mat_complex_create_from_mat}{const \PnlMat \ptr R}
    \sshortdescribe Create a complex matrix using a real one. The complex parts
    of the entries of the returned matrix are all set to zero.
\end{itemize}

\subsubsection{Permutations}

\describestruct{PnlPermutation}
\begin{verbatim}
typedef PnlVectInt PnlPermutation;
\end{verbatim}

The \verb!PnlPermutation! type is actually nothing else than a vector of
integers, i.e. a \verb!PnlVectInt!. It is used to store the partial pivoting
with row interchanges transformation needed in the LU decomposition.  We use the
{\it Blas} convention for storing permutations. Consider a \verb!PnlPermutation p!
generated by a LU decomposition of a matrix \verb!A! : to compute the
decomposition, row \verb!i! of \verb!A! was interchanged with row \verb!p(i)!.


\begin{itemize}
  \item \describefun{\PnlPermutation \ptr }{pnl_permutation_new}{}
    \sshortdescribe Create an empty \PnlPermutation.  

\item \describefun{\PnlPermutation \ptr }{pnl_permutation_create}{int n}
  \sshortdescribe Create a \PnlPermutation of size \var{n}.  

\item \describefun{void}{pnl_permutation_free}{\PnlPermutation \ptr \ptr p}
  \sshortdescribe Free a \PnlPermutation.

\item \describefun{void}{pnl_permutation_inverse}{\PnlPermutation\ptr
    inv, const \PnlPermutation\ptr p}
  \sshortdescribe Compute in \var{inv} the inverse of the permutation \var{p}.
\item \describefun{void}{pnl_vect_permute}{\PnlVect \ptr px, const
    \PnlVect \ptr x, const \PnlPermutation \ptr p} 
  \sshortdescribe Apply a \PnlPermutation to a \PnlVect.  

\item \describefun{void}{pnl_vect_permute_inplace}{\PnlVect \ptr x, 
    const \PnlPermutation \ptr p} 
  \sshortdescribe Apply a \PnlPermutation to a
  \PnlVect in-place.  

\item \describefun{void}{pnl_vect_permute_inverse}{\PnlVect \ptr px, const
    \PnlVect \ptr x, const \PnlPermutation \ptr p} 
  \sshortdescribe Apply the inverse of \PnlPermutation to a \PnlVect.  

\item \describefun{void}{pnl_vect_permute_inverse_inplace}{\PnlVect \ptr x, 
    const \PnlPermutation \ptr p} 
  \sshortdescribe Apply the inverse of a \PnlPermutation to a
  \PnlVect in-place.  

\item \describefun{void}{pnl_mat_col_permute}{\PnlMat \ptr pX, const
    \PnlMat \ptr X, const
    \PnlPermutation \ptr p}
  \sshortdescribe Apply a \PnlPermutation to the columns of a
  matrix. \var{pX} contains the result of the permutation applied to \var{X}. 
\item \describefun{void}{pnl_mat_row_permute}{\PnlMat \ptr pX, const
    \PnlMat \ptr X, const \PnlPermutation \ptr p}
  \sshortdescribe Apply a \PnlPermutation to the rows of a
  matrix. \var{pX} contains the result of the permutation applied to \var{X}. 
  
\item \describefun{void}{pnl_permutation_fprint}{FILE \ptr fic, const \PnlPermutation \ptr p}
  \sshortdescribe Print a permutation to a file.  

\item \describefun{void}{pnl_permutation_print}{const \PnlPermutation \ptr p}
  \sshortdescribe Print a permutation to the standard output.  
\end{itemize}


%% tridiag

\subsection{Tridiagonal Matrices}
\subsubsection{Overview}

The structures and functions related to tridiagonal matrices are declared in
\verb!pnl/pnl_tridiag_matrix.h!. 

We only store the three main diagonals as three vectors.

\describestruct{PnlTridiagMat}
\begin{verbatim}
typedef struct PnlTridiagMat{
  /**
   * Must be the first element in order for the object mechanism to work
   * properly. This allows any PnlTridiagMat pointer to be cast to a PnlObject
   */
  PnlObject object; 
  int size; /*!< number of rows, the matrix must be square */
  double *D; /*!< diagonal elements */
  double *DU; /*!< upper diagonal elements */
  double *DL; /*!< lower diagonal elements */
} PnlTridiagMat;
\end{verbatim}
\var{size} is the size of the matrix, \var{D} is an array of size \var{size}
containing the diagonal terms. \var{DU},
\var{DL} are two arrays of size \var{size-1} containing respectively the upper
diagonal ($M_{i, i+1}$) and the lower diagonal ($M_{i-1, i}$). 

\describestruct{PnlTridiagMatLU}
\begin{verbatim}
typedef struct PnlTridiagMatLU{
  /** 
   * Must be the first element in order for the object mechanism to work
   * properly. This allows any PnlTridiagMatLU pointer to be cast to a PnlObject
   */
  PnlObject object; 
  int size; /*!< number of rows, the matrix must be square */
  double *D; /*!< diagonal elements */
  double *DU; /*!< upper diagonal elements */
  double *DU2; /*!< second upper diagonal elements */
  double *DL; /*!< lower diagonal elements */
  int *ipiv; /*!< Permutation: row i has been interchanged with row ipiv(i) */
};
\end{verbatim}
This type is used to store the LU decomposition of a tridiagonal matrix. 

\subsubsection{Functions}
\paragraph{Constructors and destructors}
\begin{itemize}
  \item \describefun{\PnlTridiagMat \ptr }{pnl_tridiag_mat_new}{}
    \sshortdescribe Create a \PnlTridiagMat with size 0
  \item \describefun{\PnlTridiagMat \ptr }{pnl_tridiag_mat_create}{int size}
    \sshortdescribe Create a \PnlTridiagMat with size \var{size}
  \item \describefun{\PnlTridiagMat \ptr }{pnl_tridiag_mat_create_from_scalar}{int size, double x}
    \sshortdescribe Create a \PnlTridiagMat with the 3 diagonals
    filled with \var{x}
  \item \describefun{\PnlTridiagMat \ptr }{pnl_tridiag_mat_create_from_two_scalar}{int size, double x, double y}
    \sshortdescribe Create a \PnlTridiagMat  with the diagonal
    filled with \var{x} and the upper and lower diagonals filled with \var{y}
  \item \describefun{\PnlTridiagMat
      \ptr }{pnl_tridiag_mat_create_from_ptr}{int size, const double
      \ptr lower_D, const double \ptr D, const double \ptr upper_D}
    \sshortdescribe Create a \PnlTridiagMat  
  \item \describefun{\PnlTridiagMat \ptr }{pnl_tridiag_mat_create_from_mat}
    {const \PnlMat \ptr mat}
    \sshortdescribe Create a tridiagonal matrix from a full matrix (all the
    elements but the 3 diagonal ones are ignored).
  \item \describefun{\PnlMat \ptr }{pnl_tridiag_mat_to_mat}
    {const \PnlTridiagMat \ptr T}
    \sshortdescribe Create a full matrix from a tridiagonal one.
  \item \describefun{\PnlTridiagMat \ptr }{pnl_tridiag_mat_copy}
    {const \PnlTridiagMat \ptr T}
    \sshortdescribe Copy a tridiagonal matrix.
  \item \describefun{void}{pnl_tridiag_mat_clone}
    {\PnlTridiagMat \ptr clone, const \PnlTridiagMat \ptr T}
    \sshortdescribe Copy the content of \var{T} into \var{clone}
  \item \describefun{void }{pnl_tridiag_mat_free}{\PnlTridiagMat \ptr \ptr v}
    \sshortdescribe Free a \PnlTridiagMat  
  \item \describefun{int}{pnl_tridiag_mat_resize}{\PnlTridiagMat \ptr v, int size}
    \sshortdescribe Resize a \PnlTridiagMat.  
\end{itemize}


\paragraph{Accessing elements.}

If it is supported by the compiler, the following functions are declared
inline. To speed up these functions, you can use the macro constant
\texttt{PNL_RANGE_CHECK_OFF}, see Section~\ref{sec:inline} for an explanation. 
\begin{itemize}
  \item \describefun{void}{pnl_tridiag_mat_set}{\PnlTridiagMat \ptr self, int d, int up, double x}
    \sshortdescribe Set \var{self[d, d+up] = x}, \var{up} can be $\{-1, 0, 1\}$.  
  \item \describefun{double}{pnl_tridiag_mat_get}{const \PnlTridiagMat \ptr self, int d, int up}
    \sshortdescribe Get \var{self[d, d+up]}, \var{up} can be $\{-1, 0, 1\}$.  
  \item \describefun{double \ptr }{pnl_tridiag_mat_lget}{\PnlTridiagMat \ptr self, int d, int up}
    \sshortdescribe Return the address \var{self[d, d+up] = x}, \var{up} can be $\{-1, 0, 1\}$.  
\end{itemize}

\paragraph{Printing Matrix}
\begin{itemize}
  \item \describefun{void}{pnl_tridiag_mat_fprint}{FILE \ptr fic, const \PnlTridiagMat \ptr M}
    \sshortdescribe Print a tri-diagonal matrix to a file.  
  \item \describefun{void}{pnl_tridiag_mat_print}{const \PnlTridiagMat \ptr M}
    \sshortdescribe Print a tridiagonal matrix to the standard output.  
\end{itemize}

\paragraph{Algebra operations}
\begin{itemize}
  \item \describefun{void}{pnl_tridiag_mat_plus_tridiag_mat}{\PnlTridiagMat \ptr lhs, const \PnlTridiagMat \ptr rhs}
    \sshortdescribe In-place matrix matrix addition  
  \item \describefun{void}{pnl_tridiag_mat_minus_tridiag_mat}{\PnlTridiagMat \ptr lhs, const \PnlTridiagMat \ptr rhs}
    \sshortdescribe In-place matrix matrix substraction  
  \item \describefun{void}{pnl_tridiag_mat_plus_scalar}{\PnlTridiagMat \ptr lhs, double x}
    \sshortdescribe In-place matrix scalar addition  
  \item \describefun{void}{pnl_tridiag_mat_minus_scalar}{\PnlTridiagMat \ptr lhs, double x}
    \sshortdescribe In-place matrix scalar substraction  
  \item \describefun{void}{pnl_tridiag_mat_mult_scalar}{\PnlTridiagMat \ptr lhs, double x}
    \sshortdescribe In-place matrix scalar multiplication  
  \item \describefun{void}{pnl_tridiag_mat_div_scalar}{\PnlTridiagMat \ptr lhs, double x}
    \sshortdescribe In-place matrix scalar division
\end{itemize}

\paragraph{Element-wise operations}
\begin{itemize}
  \item \describefun{void}{pnl_tridiag_mat_mult_tridiag_mat_term}{\PnlTridiagMat \ptr lhs, const \PnlTridiagMat \ptr rhs}
    \sshortdescribe In-place matrix matrix term by term product  
  \item \describefun{void}{pnl_tridiag_mat_div_tridiag_mat_term}{\PnlTridiagMat \ptr lhs, const \PnlTridiagMat \ptr rhs}
    \sshortdescribe In-place matrix matrix term by term division  
  \item \describefun{void}{pnl_tridiag_mat_map_inplace}{\PnlTridiagMat \ptr lhs, 
    double(\ptr f)(double)} 
  \sshortdescribe \var{lhs = f(lhs)}.


\item \describefun{void}{pnl_tridiag_mat_map_tridiag_mat_inplace}{\PnlTridiagMat \ptr lhs, const
  \PnlTridiagMat \ptr rhs, double(\ptr f)(double, double)} 
  \sshortdescribe \var{lhs = f(lhs, rhs)}.
\end{itemize}

\paragraph{Standard matrix operations \& Linear systems}
\begin{itemize}
  \item \describefun{void}{pnl_tridiag_mat_mult_vect_inplace}{\PnlVect
    \ptr lhs, const \PnlTridiagMat \ptr mat, const \PnlVect
    \ptr rhs}
    \sshortdescribe In place matrix multiplication. The vector \var{lhs} must be
    different from \var{rhs}.
  \item \describefun{\PnlVect \ptr }{pnl_tridiag_mat_mult_vect}{const
    \PnlTridiagMat \ptr mat, const \PnlVect \ptr vec}
    \sshortdescribe Matrix multiplication  
  \item \describefun{void}{pnl_tridiag_mat_lAxpby}{double lambda, const \PnlTridiagMat
      \ptr A, const \PnlVect \ptr x, double mu, \PnlVect \ptr b} 
    \sshortdescribe Compute \var{b := lambda A x + mu b}. When \var{mu==0}, the
    content of \var{b} is not used on input and instead \var{b} is resized to
    match \var{A*x}. Note that the vectors \var{x} and \var{b} must be different.
  \item \describefun{double}{pnl_tridiag_mat_scalar_prod}{const \PnlVect
    \ptr x,const \PnlTridiagMat \ptr A, const \PnlVect \ptr y}
    \sshortdescribe Compute \var{x' * A * y}
  \item \describefun{void}{pnl_tridiag_mat_syslin_inplace}{
      \PnlTridiagMat \ptr M, \PnlVect \ptr b}
    \sshortdescribe Solve the linear system M x = b. The solution is written into
    \var{b} on exit. On exit, \var{M} is modified and becomes unusable.
  \item \describefun{void}{pnl_tridiag_mat_syslin}{\PnlVect
      \ptr x, \PnlTridiagMat \ptr M, const \PnlVect \ptr b}
    \sshortdescribe Solve the linear system M x = b. On exit, \var{M} is modified and becomes unusable.
  \item \describefun{\PnlTridiagMatLU\ptr }{pnl_tridiag_mat_lu_new}{}
    \sshortdescribe Create an empty \PnlTridiagMatLU
  \item \describefun{\PnlTridiagMatLU\ptr }{pnl_tridiag_mat_lu_create}{int size}
    \sshortdescribe Create a \PnlTridiagMatLU with size \var{size}
  \item \describefun{\PnlTridiagMatLU\ptr }{pnl_tridiag_mat_lu_copy}{const \PnlTridiagMatLU  \ptr mat}
    \sshortdescribe Create a new \PnlTridiagMatLU which is a copy of
    \var{mat}.
  \item \describefun{void}{pnl_tridiag_mat_lu_clone}{\PnlTridiagMatLU \ptr clone, const \PnlTridiagMatLU \ptr mat}
    \sshortdescribe Clone a \PnlTridiagMatLU. \var{clone} must
    already exist, no memory is allocated for the envelope. 
  \item \describefun{void}{pnl_tridiag_mat_lu_free}{\PnlTridiagMatLU \ptr \ptr m}
    \sshortdescribe Free a \PnlTridiagMatLU
  \item \describefun{int}{pnl_tridiag_mat_lu_resize}{\PnlTridiagMatLU \ptr v, int size}
    \sshortdescribe Resize a \PnlTridiagMatLU
  \item \describefun{int}{pnl_tridiag_mat_lu_compute}{\PnlTridiagMatLU \ptr LU, const \PnlTridiagMat \ptr A}
    \sshortdescribe Compute the LU factorisation of a tridiagonal matrix
    \var{A}. \var{LU} must have already been created using
    \reffun{pnl_tridiag_mat_lu_new}. On exit, \var{LU} contains the
    decomposition which is suitable for use in \reffun{pnl_tridiag_mat_lu_syslin}.
  \item \describefun{int}{pnl_tridiag_mat_lu_syslin_inplace}{\PnlTridiagMatLU \ptr LU, \PnlVect \ptr b}
    \sshortdescribe Solve a linear system \var{A x = b} where the matrix \var{LU}
    is given the LU decomposition of A previously computed by
    \reffun{pnl_tridiag_mat_lu_compute}. On exit, \var{b} is overwritten by the
    solution \var{x}. 
  \item \describefun{int}{pnl_tridiag_mat_lu_syslin}{\PnlVect \ptr x, \PnlTridiagMatLU \ptr LU, const \PnlVect \ptr b}
    \sshortdescribe Solve a linear system \var{A x = b} where the matrix \var{LU}
    is given the LU decomposition of A previously computed by
    \reffun{pnl_tridiag_mat_lu_compute}. 
\end{itemize}



\subsection{Band Matrices}
\subsubsection{Overview}

\describestruct{PnlBandMat}
\begin{verbatim}
typedef struct
{
  /**
   * Must be the first element in order for the object mechanism to work
   * properly. This allows any PnlBandMat pointer to be cast to a PnlObject
   */
  PnlObject object; 
  int m; /*!< nb rows */ 
  int n; /*!< nb columns */ 
  int nu; /*!< nb of upperdiagonals */
  int nl; /*!< nb of lowerdiagonals */
  int m_band; /*!< nb rows of the band storage */
  int n_band; /*!< nb columns of the band storage */
  double *array;  /*!< a block to store the bands */  
} PnlBandMat;
\end{verbatim}


The structures and functions related to band matrices are declared in
\verb!pnl/pnl_band_matrix.h!. 


\subsubsection{Functions}
\paragraph{Constructors and destructors}
\begin{itemize}
  \item \describefun{\PnlBandMat\ptr }{pnl_band_mat_new}{}
  \sshortdescribe Create a band matrix of size 0.

\item \describefun{\PnlBandMat\ptr }{pnl_band_mat_create}{int m, int n, int
    nl, int nu}
  \sshortdescribe Create a band matrix of size \var{m x n} with \var{nl} lower
  diagonals and \var{nu} upper diagonals.

\item \describefun{\PnlBandMat\ptr }{pnl_band_mat_create_from_mat}{const
    \PnlMat \ptr BM, int nl, int nu}
  \sshortdescribe Extract a band matrix from a \PnlMat.

\item \describefun{void}{pnl_band_mat_free}{\PnlBandMat\ptr\ptr}
  \sshortdescribe Free a band matrix.

\item \describefun{void}{pnl_band_mat_clone}{\PnlBandMat \ptr clone, 
  const \PnlBandMat \ptr M}
  \sshortdescribe Copy the band matrix \var{M} into \var{clone}. No new
  \PnlBandMat is created.

\item \describefun{\PnlBandMat\ptr }{pnl_band_mat_copy}{\PnlBandMat \ptr BM}
  \sshortdescribe Create a new band matrix which is a copy of \var{BM}. Each
  band matrix owns its data array.

\item \describefun{\PnlMat\ptr }{pnl_band_mat_to_mat}{\PnlBandMat \ptr BM}
  \sshortdescribe Create a full matrix from a band matrix.

\item \describefun{int}{pnl_band_mat_resize}{\PnlBandMat \ptr BM, int
  m, int n, int nl, int nu}
  \sshortdescribe Resize \var{BM} to store a \var{m x n} band matrix with
  \var{nu} upper diagonals and \var{nl} lower diagonals.
\end{itemize}

\paragraph{Accessing elements.}

If it is supported by the compiler, the following functions are declared
inline. To speed up these functions, you can use the macro constant
\texttt{PNL_RANGE_CHECK_OFF}, see Section~\ref{sec:inline} for an explanation. 
\begin{itemize}
\item \describefun{void}{pnl_band_mat_set}{\PnlBandMat
    \ptr M, int i, int j, double x}
  \sshortdescribe $M_{i, j}=x$.

\item \describefun{void}{pnl_band_mat_get}{\PnlBandMat
    \ptr M, int i, int j}
    \sshortdescribe Return $M_{i, j}$.

\item \describefun{void}{pnl_band_mat_lget}{\PnlBandMat
    \ptr M, int i, int j}
    \sshortdescribe Return the address $\&(M_{i, j})$.
\item \describefun{void}{pnl_band_mat_set_all}{\PnlBandMat
    \ptr M, double x}
    \sshortdescribe Set all the elements of \var{M} to \var{x}.

  \item \describefun{void}{pnl_band_mat_print_as_full}{\PnlBandMat
    \ptr M}
    \sshortdescribe Print a band matrix in a full format.
\end{itemize}

\subparagraph{Element wise operations}

\begin{itemize}
  \item \describefun{void}{pnl_band_mat_plus_scalar}{\PnlBandMat \ptr lhs, 
  double x} 
    \sshortdescribe In-place addition, \var{lhs += x} 

  \item \describefun{void}{pnl_band_mat_minus_scalar}{\PnlBandMat \ptr lhs, 
  double x} 
  \sshortdescribe In-place substraction \var{lhs -= x} 

\item \describefun{void}{pnl_band_mat_div_scalar}{\PnlBandMat
  \ptr lhs, double x} 
  \sshortdescribe \var{lhs = lhs ./ x}

\item \describefun{void}{pnl_band_mat_mult_scalar}{\PnlBandMat
  \ptr lhs, double x} 
  \sshortdescribe \var{lhs = lhs * x}

\item \describefun{void}{pnl_band_mat_plus_band_mat}{\PnlBandMat \ptr lhs, 
    const \PnlBandMat \ptr rhs} 
    \sshortdescribe In-place addition, \var{lhs += rhs} 

\item \describefun{void}{pnl_band_mat_minus_band_mat}{\PnlBandMat \ptr lhs, 
    const \PnlBandMat \ptr rhs} 
  \sshortdescribe In-place substraction \var{lhs -= rhs} 

\item \describefun{void}{pnl_band_mat_inv_term}{\PnlBandMat \ptr lhs}
  \sshortdescribe In-place term by term  inversion \var{lhs = 1 ./ rhs} 

\item \describefun{void}{pnl_band_mat_div_band_mat_term}{\PnlBandMat
    \ptr lhs, const \PnlBandMat \ptr rhs} 
  \sshortdescribe In-place term by term  division \var{lhs = lhs ./ rhs}

\item \describefun{void}{pnl_band_mat_mult_band_mat_term}{\PnlBandMat
    \ptr lhs, const \PnlBandMat \ptr rhs} 
  \sshortdescribe In-place term by term multiplication  \var{lhs = lhs .* rhs}

\item \describefun{void}{pnl_band_mat_map}{\PnlBandMat \ptr lhs, const
    \PnlBandMat \ptr rhs, double(\ptr f)(double)} 
  \sshortdescribe \var{lhs = f(rhs)}

\item \describefun{void}{pnl_band_mat_map_inplace}{\PnlBandMat \ptr lhs, double(\ptr f)(double)}
  \sshortdescribe  \var{lhs = f(lhs)}

\item \describefun{void}{pnl_band_mat_map_band_mat_inplace}{\PnlBandMat \ptr lhs,
  const \PnlBandMat \ptr rhs, double(\ptr f)(double,double)} 
  \sshortdescribe \var{lhs = f(lhs,rhs)}
\end{itemize}


\paragraph{Standard matrix operations \& Linear system}
\begin{itemize}
\item \describefun{void}{pnl_band_mat_lAxpby}{double lambda, const \PnlBandMat
    \ptr A, const \PnlVect \ptr x, double mu, \PnlVect \ptr b} 
  \sshortdescribe Compute \var{b := lambda A x + mu b}. When \var{mu==0}, the
  content of \var{b} is not used on input and instead \var{b} is resized to
  match the size of \var{A*x}.
\item \describefun{void}{pnl_band_mat_mult_vect_inplace}{\PnlVect \ptr
  y, const \PnlBandMat \ptr BM, const \PnlVect \ptr x}
  \sshortdescribe \var{y = BM * x}
\item 
  \describefun{void}{pnl_band_mat_syslin_inplace}{\PnlBandMat
    \ptr M, \PnlVect \ptr b}
    \sshortdescribe Solve the linear system \var{M x = b} with \var{M} a \PnlBandMat.
  {\bf Note} that M is modified on output and becomes unusable. On exit, the
  solution \var{x} is stored in \var{b}.
\item 
  \describefun{void}{pnl_band_mat_syslin}{\PnlVect \ptr x,\PnlBandMat
    \ptr M, \PnlVect \ptr b}
    \sshortdescribe Solve the linear system \var{M x = b} with \var{M} a \PnlBandMat.
  {\bf Note} that M is modified on output and becomes unusable. 
\item \describefun{void}{pnl_band_mat_lu}{\PnlBandMat \ptr BM,
  \PnlVectInt \ptr p}
  \sshortdescribe Compute the LU decomposition with partial pivoting with row
  interchanges. On exit, \var{BM} is enlarged to store the LU decomposition. On
  exit, \var{p} stores the permutation applied to the rows. Note that the Lapack format
  is used to store \var{p}, this format differs from the one used by
  \PnlPermutation.
\item  \describefun{void}{pnl_band_mat_lu_syslin_inplace}{const \PnlBandMat \ptr M, 
  \PnlVectInt \ptr p, \PnlVect \ptr b} 
  \sshortdescribe Solve the band linear system \var{M x = b} where \var{M} is
  the LU decomposition computed by \reffun{pnl_band_mat_lu}  and \var{p} the
  associated permutation. On exit, the solution \var{x} is stored in \var{b}.
\item  \describefun{void}{pnl_band_mat_lu_syslin}{\PnlVect \ptr x,
  const \PnlBandMat \ptr M, \PnlVectInt \ptr p, const \PnlVect \ptr b} 
  \sshortdescribe Solve the band linear system \var{M x = b} where \var{M} is the LU
  decomposition computed by \reffun{pnl_band_mat_lu} and \var{p} the associated permutation. 
\end{itemize}


\subsection{Sparse Matrices}
\subsubsection{Short description}
The structures and functions related to matrices are declared in
\verb!pnl/pnl_sp_matrix.h!.

\describestruct{PnlSpMat}\describestruct{PnlSpMatInt}\describestruct{PnlSpMatComplex}
\begin{verbatim}
typedef struct _PnlSpMat
{
  /** 
   * Must be the first element in order for the object mechanism to work
   * properly. This allows a PnlSpMat pointer to be cast to a PnlObject
   */
  PnlObject object; 
  int m; /*!< number of rows */
  int n; /*!< number of columns */
  int nz; /*!< number of non-zero elements */
  int *J; /*!< column indices, vector of size nzmax */
  int *I; /*!< row offset integer vector, 
            array[I[i]] is the first element of row i.
            Vector of size (m+1) */ 
  double *array; /*!< pointer to store the data of size nzmax*/
  int nzmax; /*!< size of the memory block allocated for array */
} PnlSpMat;

typedef struct _PnlSpMatInt
{
  /** 
   * Must be the first element in order for the object mechanism to work
   * properly. This allows a PnlSpMat pointer to be cast to a PnlObject
   */
  PnlObject object; 
  int m; /*!< number of rows */
  int n; /*!< number of columns */
  int nz; /*!< number of non-zero elements */
  int *J; /*!< column indices, vector of size nzmax */
  int *I; /*!< row offset integer vector, 
            array[I[i]] is the first element of row i.
            Vector of size (m+1) */ 
  int *array; /*!< pointer to store the data of size nzmax */
  int nzmax; /*!< size of the memory block allocated for array */
} PnlSpMatInt;

typedef struct _PnlSpMatComplex
{
  /** 
   * Must be the first element in order for the object mechanism to work
   * properly. This allows a PnlSpMat pointer to be cast to a PnlObject
   */
  PnlObject object; 
  int m; /*!< number of rows */
  int n; /*!< number of columns */
  int nz; /*!< number of non-zero elements */
  int *J; /*!< column indices, vector of size nzmax */
  int *I; /*!< row offset integer vector, 
            array[I[i]] is the first element of row i.
            Vector of size (m+1) */ 
  dcomplex *array; /*!< pointer to store the data of size nzmax */
  int nzmax; /*!< size of the memory block allocated for array */
} PnlSpMatComplex;
\end{verbatim}

The non zero elements of row \var{i} are stored in \var{array} between the
indices \var{I[i]} and \var{I[i+1]-1}. The array \var{J} contains the column
indices of every element of \var{array}.
\\

Sparse matrices are defined using the internal template approach and can be used
for integer, float or complex base data according to the following table
\begin{center}
  \begin{tabular}[t]{lll}
    base type & prefix & type \\
    \hline
    double & pnl_sp_mat & PnlSpMat \\
    \hline
    int & pnl_sp_mat_int & PnlSpMatInt \\
    \hline
    dcomplex & pnl_sp_mat_complex & PnlSpMatComplex
  \end{tabular}
\end{center}

\subsubsection{Functions}

\paragraph{Constructors and destructors}

\begin{itemize}
\item \describefun{\PnlSpMat \ptr}{pnl_sp_mat_new}{}
  \sshortdescribe Create an empty sparse matrix.
\item \describefun{\PnlSpMat \ptr}{pnl_sp_mat_create}{int m, int n, int nzmax}
  \sshortdescribe Create a sparse matrix with size \var{m x n} designed to
  hold at most \var{nzmax} non zero elements.
\item \describefun{void}{pnl_sp_mat_clone}{\PnlSpMat \ptr dest, const \PnlSpMat \ptr src}
  \sshortdescribe Clone \var{src} into \var{dest}, which is automatically
  resized. On output, \var{dest} and \var{src} are equal but independent.
\item \describefun{\PnlSpMat \ptr}{pnl_sp_mat_copy}{\PnlSpMat \ptr src}
  \sshortdescribe Create an independent copy of \var{src}.
\item \describefun{void}{pnl_sp_mat_free}{\PnlSpMat \ptr\ptr}
  \sshortdescribe Delete a sparse matrix.
\item \describefun{int}{pnl_sp_mat_resize}{\PnlSpMat \ptr M, int m, int n, int nzmax}
  \sshortdescribe Resize an existing \PnlSpMat to become a \var{m x n} sparse
  matrices holding at most \var{nzmax}. Note that no old data are kept except if
  \var{M->m} is left unchanged and we only call this function to increase
  \var{M->nzmax}. Return \var{OK} or \var{FAIL}.
\item \describefun{\PnlMat\ptr }{pnl_mat_create_from_sp_mat}{const \PnlSpMat \ptr M}
  \sshortdescribe Create a dense \PnlMat from a spare one.
\item \describefun{\PnlSpMat \ptr }{pnl_sp_mat_create_from_mat}{const \PnlMat \ptr M}
  \sshortdescribe Create a sparse matrix from a dense one.
\item \describefun{void}{pnl_sp_mat_create_from_file}{char *file}
  \sshortdescribe Read a sparse matrix from the file with name \var{file}. We use the Matrix Market Exchange Format
  \begin{verbatim}
M  N  L                                   | <--- rows, columns, entries
I1  J1  A(I1, J1)                         | <--+
I2  J2  A(I2, J2)                         |    |
I3  J3  A(I3, J3)                         |    |-- L lines
    . . .                                 |    |
IL JL  A(IL, JL)                          | <--+
  \end{verbatim}
  The format \verb!(I1, J1) A(I1, J1)! is also accepted. Anything after a \verb!#! or \verb!%! is ignored up to the end of the line.
\end{itemize}

\paragraph{Accessing elements}

\begin{itemize}
\item \describefun{void}{pnl_sp_mat_set}{\PnlSpMat \ptr M, int i, int j, double x}
  \sshortdescribe Set \var{M[i,j] = x}. This function increases \var{M->nzmax}
  if necessary.
\item \describefun{double}{pnl_sp_mat_get}{const \PnlSpMat \ptr M, int i, int j}
  \sshortdescribe Return \var{M[i,j]}. If \var{M} has no entry with such an
  index, zero is returned.
\end{itemize}

\paragraph{Applying external operations}

\begin{itemize}
\item \describefun{void}{pnl_sp_mat_plus_scalar}{\PnlSpMat \ptr M, double x}
  \sshortdescribe Add \var{x} to all non zero entries of \var{M}. To apply the
  operation to all entries including the zero ones, first convert \var{M} to
  a dense matrix and use \reffun{pnl_mat_plus_scalar}.
\item \describefun{void}{pnl_sp_mat_minus_scalar}{\PnlSpMat \ptr M, double x}
  \sshortdescribe Substract \var{x} to all non zero entries of \var{M}. To apply the
  operation to all entries including the zero ones, first convert \var{M} to
  a dense matrix and use \reffun{pnl_mat_minus_scalar}.
\item \describefun{void}{pnl_sp_mat_mult_scalar}{\PnlSpMat \ptr M, double x}
  \sshortdescribe In-place matrix scalar multiplication
\item \describefun{void}{pnl_sp_mat_div_scalar}{\PnlSpMat \ptr M, double x}
  \sshortdescribe In-place matrix scalar division
\end{itemize}


\paragraph{Standard matrix operations}

\begin{itemize}
\item \describefun{void}{pnl_sp_mat_fprint}{FILE \ptr fic, const \PnlSpMat\ptr  M}
  \sshortdescribe Print a sparse matrix to a file descriptor using the format
  \var{(row, col) --> val}. The file can be read by \reffun{pnl_sp_mat_create_from_file}.
\item \describefun{void}{pnl_sp_mat_print}{const \PnlSpMat\ptr  M}
  \sshortdescribe Same as \reffun{pnl_sp_mat_fprint} but print to standard
  output.
\item \describefun{void}{pnl_sp_mat_mult_vect}{\PnlVect \ptr y, const
  \PnlSpMat \ptr A, const \PnlVect \ptr x}
  \sshortdescribe \var{y = A x}.
\item \describefun{void}{pnl_sp_mat_lAxpby}{double lambda, const \PnlSpMat
  \ptr A, const \PnlVect \ptr x, double b, \PnlVect \ptr y} 
  \sshortdescribe Compute \var{y := lambda A x + b y}. When \var{b=0}, the
  content of \var{y} is not used on input and instead \var{y} is resized to
  match \var{A*x}. The vectors \var{x} and \var{y} must be different.
\item \describefun{void }{pnl_sp_mat_plus_sp_mat_inplace}{\PnlSpMat \ptr res, const \PnlSpMat \ptr A, const \PnlSpMat \ptr B}
  \sshortdescribe In-place addition: \var{res = A + B}.
\item \describefun{\PnlSpMat \ptr}{pnl_sp_mat_sp_mat}{const \PnlSpMat \ptr A, const \PnlSpMat \ptr B}
  \sshortdescribe Return the sum of \var{A} and \var{B}.
\item \describefun{void }{pnl_sp_mat_kron_inplace}{\PnlSpMat \ptr result, const \PnlSpMat \ptr A, const \PnlSpMat \ptr B}
  \sshortdescribe In-place Kroenecker product of \var{A} and \var{B}.
\item \describefun{\PnlSpMat \ptr}{pnl_sp_mat_kron}{const \PnlSpMat \ptr A, const \PnlSpMat \ptr B}
  \sshortdescribe Return the Kroenecker product of \var{A} and \var{B}.
\end{itemize}

\paragraph{Comparison functions}

\begin{itemize}
  \item \describefun{int}{pnl_sp_mat_isequal}{const \PnlSpMat \ptr x, const \PnlSpMat \ptr y, double abserr}
    \sshortdescribe Test if two sparse matrices are equal up to \var{err} component--wise. The error \var{err} is either relative or absolute depending on the magnitude of the components. Return \var{TRUE} or \var{FALSE}.
  \item \describefun{int}{pnl_sp_mat_isequal_abs}{const \PnlSpMat \ptr x, const \PnlSpMat \ptr y, double relerr}
    \sshortdescribe Test if two sparse matrices are equal up to an absolute error \var{abserr} component--wise. Return \var{TRUE} or \var{FALSE}.
  \item \describefun{int}{pnl_sp_mat_isequal_rel}{const \PnlSpMat \ptr x, const \PnlSpMat \ptr y, double err}
    \sshortdescribe Test if two sparse matrices are equal up to a relative error \var{relerr} component--wise. Return \var{TRUE} or \var{FALSE}.
\end{itemize}


\subsection{Hyper Matrices}
\subsubsection{Short description}

The Hyper matrix types and related functions are defined in the header \verb!pnl/pnl_matrix.h!.

\describestruct{PnlHmat}\describestruct{PnlHmatInt}\describestruct{PnlHmatComplex}
\begin{verbatim}
typedef struct PnlHmat{
  /**
   * Must be the first element in order for the object mechanism to work
   * properly. This allows any PnlHmat pointer to be cast to a PnlObject
   */
  PnlObject object; 
  int ndim; /*!< nb dimensions */ 
  int *dims; /*!< pointer to store the values of the ndim dimensions */ 
  int mn; /*!< product dim_1 *...*dim_ndim */
  int *pdims; /*!< array of size ndim, s.t. pdims[i] = dims[ndim-1] x ... dims[i+1]
                with pdims[ndim - 1] = 1 */
  double *array; /*!< pointer to store */
} PnlHmat;

typedef struct PnlHmatInt{
  /**
   * Must be the first element in order for the object mechanism to work
   * properly. This allows any PnlHmatInt pointer to be cast to a PnlObject
   */
  PnlObject object; 
  int ndim; /*!< nb dimensions */ 
  int *dims; /*!< pointer to store the value of the ndim dimensions */ 
  int mn; /*!< product dim_1 *...*dim_ndim */
  int *pdims; /*!< array of size ndim, s.t. pdims[i] = dims[ndim-1] x ... dims[i+1]
                with pdims[ndim - 1] = 1 */
  int *array; /*!< pointer to store */
} PnlHmatInt;

typedef struct PnlHmatComplex{
  /**
   * Must be the first element in order for the object mechanism to work
   * properly. This allows any PnlHmatComplex pointer to be cast to a PnlObject
   */
  PnlObject object; 
  int ndim; /*!< nb dimensions */ 
  int *dims; /*!< pointer to store the value of the ndim dimensions */ 
  int mn; /*!< product dim_1 *...*dim_ndim */
  int *pdims; /*!< array of size ndim, s.t. pdims[i] = dims[ndim-1] x ... dims[i+1]
                with pdims[ndim - 1] = 1 */
  dcomplex *array; /*!< pointer to store */
} PnlHmatComplex;
\end{verbatim}
\var{ndim} is the number of dimensions, \var{dim} is an array to store the
size of each dimension and \var{nm} contains the product of the sizes of each
dimension. \var{array} is an array of size \var{mn} containing the data. The
integer array \var{pdims} is used to create the one--to--one map between the
natural indexing and the linear indexing used in \var{array}.


\subsubsection{Functions}
These functions exist for all types of hypermatrices no matter what the basic type
is. The following conventions are used to name functions operating on hypermatrices.
Here is the table of prefixes used for the different basic types.

\begin{center}
  \begin{tabular}[t]{lll}
    base type & prefix & type\\
    \hline
    double & pnl_hmat & PnlHmat \\
    \hline
    int & pnl_hmat_int & PnlHmatInt \\
    \hline
    dcomplex & pnl_hmat_complex & PnlHmatComplex
  \end{tabular}
\end{center}

In this paragraph, we present the functions operating on \PnlHmat
which exist for all types. To deduce the prototypes of these functions for
other basic types, one must replace {\tt pnl_hmat} and {\tt double} according
the above table.


\paragraph{Constructors and destructors}
\begin{itemize}
  \item \describefun{\PnlHmat \ptr }{pnl_hmat_new}{}
    \sshortdescribe Create an empty \PnlHmat.

\item \describefun{\PnlHmat \ptr }{pnl_hmat_create}{int ndim, const int \ptr dims}
  \sshortdescribe Create a \PnlHmat with \var{ndim} dimensions and
  the size of each dimension is given by the entries of the integer array
  \var{dims}
  
\item 
  \describefun{\PnlHmat \ptr }{pnl_hmat_create_from_scalar}{int ndim, const int \ptr dims, double x}
  \sshortdescribe Create a \PnlHmat with \var{ndim} dimensions given
  by $\prod_i \var{dims[i]}$ filled with \var{x}.
  
\item 
  \describefun{\PnlHmat \ptr }{pnl_hmat_create_from_ptr}{int ndim, const int \ptr dims, const double \ptr x}
  
\item \describefun{void}{pnl_hmat_free}{\PnlHmat \ptr \ptr H}
  \sshortdescribe Free a \PnlHmat
  
\item \describefun{\PnlHmat \ptr }{pnl_hmat_copy}{const \PnlHmat \ptr H}
  \sshortdescribe Copy a \PnlHmat.
  
\item \describefun{void}{pnl_hmat_clone}{\PnlHmat \ptr clone, const \PnlHmat \ptr H}
  \sshortdescribe Clone a \PnlHmat.
  
\item \describefun{int}{pnl_hmat_resize}{\PnlHmat \ptr H, int ndim, const int \ptr dims}
  \sshortdescribe Resize a \PnlHmat.
\end{itemize}  

\paragraph{Accessing elements}

\begin{itemize}
\item   \describefun{void}{pnl_hmat_set}{\PnlHmat \ptr self, int \ptr tab, double x}
  \sshortdescribe Set the element of index \var{tab} to \var{x}.
  
\item \describefun{double}{pnl_hmat_get}{const \PnlHmat \ptr self, int \ptr tab}
  \sshortdescribe Return the value of the element of index \var{tab} 
  
\item \describefun{double\ptr }{pnl_hmat_lget}{\PnlHmat \ptr self, int \ptr tab}
  \sshortdescribe Return the address of self[tab] for use as a lvalue.  

\item \describefun{\PnlMat}{pnl_mat_wrap_hmat}{\PnlHmat \ptr H, int \ptr t}
  \sshortdescribe Return a true \PnlMat not a pointer holding the data
  \var{H(t,:,:)}. Note that \var{t} must be of size \var{ndim-2} and that it
  cannot be checked within the function. The returned matrix shares its data
  with \var{H}, it is only a view not a true copy.

\item \describefun{\PnlVect}{pnl_vect_wrap_hmat}{\PnlHmat \ptr H, int \ptr t}
  \sshortdescribe Return a true \PnlVect not a pointer holding the data
  \var{H(t,:)}. Note that \var{t} must be of size \var{ndim-1} and that it
  cannot be checked within the function. The returned vector shares its data
  with \var{H}, it is only a view not a true copy.


\end{itemize}  

\paragraph{Printing hypermatrices}

\begin{itemize}
\item \describefun{void}{pnl_hmat_print}{const \PnlHmat \ptr H}
  \sshortdescribe Print an hypermatrix.
\end{itemize}

\paragraph{Term by term operations}

\begin{itemize}
\item \describefun{void}{pnl_hmat_plus_hmat}{\PnlHmat \ptr lhs, const \PnlHmat \ptr rhs}
  \sshortdescribe Compute \var{lhs += rhs}.
  
\item \describefun{void}{pnl_hmat_mult_scalar}{\PnlHmat \ptr lhs, double x}
  \sshortdescribe Compute \var{lhs *= x} where x is a real number.
\end{itemize}

% \subsection{Morse Matrix}
% \subsubsection{Overview}

% A system of linear equation is called sparse if only a relatively small number
% of its matrix elements $M_{i, j}$ are nonzero. It is wasteful to use full
% structure to solve the linear system because most of the operations devoted to
% solving the system use elements with values zero. Furthermore, for some 
% high dimensional problems, storing the full matrix with its zero elements is not
% possible because of memory limitations.


% In the following, we propose two structures for Sparse Matrices.  Must of the
% algorithms which use sparse matrices can be divided in two steps.  The first
% step is the construction of the matrix. For this, \PnlMorseMat should
% be used. The second step is the resolution of a sparse linear system. We
% have two ways of doing that. The first one is to use a direct method based on
% matrix-decomposition, like the LU decomposition. The \PnlSparseMat is
% implemented to do that. The second one is to use iterative methods like
% Conjugate Gradient, BICGstab or GMRES. These methods are discussed in the next
% section. If we use iterative methods, we can use \PnlMorseMat. 

% \begin{verbatim}
% typedef struct SpRow{
%   int size;  /*!< size of a row */
%   int Max_size; /*!< max size allocation of a row */
%   int    *Index; /*!< pointer to an int array giving the columns or row i */
%   double *Value; /*!< Pointer on values */
% }SpRow;
% \end{verbatim}
% \var{size} is the number of elements, 
% \var{Max_size} is the size of memory allocation.
% \var{Index}, is the pointer containing the index of row or column, 
% \var{Value}, is the pointer containing the value of row or column.
% So for a \refstruct{SpRow} which contains row $i$ of $M$.
% If $k \leq size $ then
% $$M_{i, Index[k]}=Value[k].$$  

% \begin{verbatim}
% typedef struct PnlMorseMat{
%   int m; /*!< nb rows */ 
%   int n; /*!< nb columns */ 
%   SpRow * array; /*!< pointer in each row or col to store no nul coefficients */
%   int RC; /*!< 0 if we use row-wise storage, 1 if we use column-wise storage */ 
% } PnlMorseMat;
% \end{verbatim}
% \var{m} is the number of rows, \var{n} is the number of columns.
% \var{array} is the pointer containing on SpRow array of size n or m (depend of
% RC).
% \var{RC} is an integer to know if the matrix is stored by row or columns.

% \subsubsection{Functions}
% \paragraph{Constructors and destructors}
% \begin{itemize}
% \item \describefun{\PnlMorseMat\ptr }{pnl_morse_mat_create}{int m, 
%     int n, int Max_size_row, int RC}
%   \sshortdescribe Create an empty \PnlMorseMat with memory
%   allocated for each component of the array. 
% \item
%   \describefun{\PnlMorseMat\ptr }{pnl_morse_mat_create_fromfull}
%   {\PnlMat \ptr FM, int RC}
%   \sshortdescribe Create a \PnlMorseMat from  a \PnlMat
%   storing only its nonzero elements.

% \item \describefun{void}{pnl_morse_mat_free}{\PnlMorseMat\ptr \ptr  M}
%   \sshortdescribe Free a \PnlMorseMat

% \item \describefun{int}{pnl_morse_mat_freeze}{PnlMorseMat\ptr  M}
%   \sshortdescribe Set Max size equal to size for each SpRow and frees the extra
%   memory.

% \item \describefun{\PnlMat \ptr }{pnl_morse_mat_full}
%   {\PnlMorseMat\ptr  M}
%   \sshortdescribe Create a full matrix from a morse matrix.
% \end{itemize}


% \paragraph{Accessing elements}
% \begin{itemize}
% \item \describefun{double}{ pnl_morse_mat_get}{PnlMorseMat\ptr  M, int i, int j}
%   \sshortdescribe Return $M_{i, j}$. 
% \item \describefun{int}{ pnl_morse_mat_set}{PnlMorseMat\ptr  M, int i, int
%     j, double Val}
%   \sshortdescribe Do $M_{i, j} = Val$. For example, if $RC=1$ and $(i, j)$ is a valid index, replace
%   $array[i]\rightarrow Value[k]$ with $k$ such that $array[i]\rightarrow Index[k]=j$.
%   If $(i, j)$ is not a valid index, add $j$ to $array[i]\rightarrow Index$ and $Val$ to
%   $array[i] \rightarrow Value$ with memory allocation if needed. 
% \item \describefun{double\ptr }{pnl_morse_mat_lget}{PnlMorseMat\ptr  M, int
%     i, int j}
%   \sshortdescribe Return the address of $M_{i, j}$. For example, 
%   if $RC=1$ and $(i, j)$ is a valid index, replace return address of
%   $array[i]\rightarrow Value[k]$ with $k$ such that $array[i]\rightarrow
%   Index[k]=j$.  If $(i, j)$ is not a valid index, add $j$ to
%   $array[i]\rightarrow Index$ and add element to $array[i] \rightarrow Value$
%   (with memory allocation if needed), returns address of this element. In
%   practice this function is used to do $M_{i, j} += a$.
% \end{itemize}

% \paragraph{Printing Matrix}
% \begin{itemize}
% \item \describefun{void}{pnl_morse_mat_print}{const \PnlMorseMat\ptr M}
% \end{itemize}

% \paragraph{Standard matrix operations}
% \begin{itemize}
% \item \describefun{void}{pnl_morse_mat_mult_vect_inplace}{\PnlVect
%     \ptr lhs, const \PnlMorseMat\ptr M, const \PnlVect
%     \ptr rhs}
%   \sshortdescribe Compute $ lhs=M \ rhs$.
% \item \describefun{\PnlVect\ptr }{pnl_morse_mat_mult_vect}{const
%     \PnlMorseMat\ptr M, const \PnlVect \ptr vec}
%   \sshortdescribe Compute $ vec=M \ vec$.
% \end{itemize}


% \subsection{Sparse Matrix}

% \PnlSparseMat is the cs structure of the Csparse library written by
% Timothy A.Davis.  For the sake of convenience, we have renamed some functions
% and structures. We have also reduced the number of function parameters for non
% expert users in sparse matrices.  In the following, we only use the LU
% factorisation for sparse systems. If the same operator is used at each time
% step, direct methods relying on factorisations are faster than iterative
% methods. When the PDE coefficients are time dependent, the answer is not so
% clear.

% \subsubsection{Functions}
% \paragraph{Constructors and destructors}
% \begin{itemize}
% \item \describefun{\PnlSparseMat
%     \ptr }{pnl_sparse_mat_create_fromfull}{\PnlMat \ptr M}
%   \sshortdescribe Create a \PnlSparseMat from  a
%   \PnlMat storing only nonzero elements.
% \item \describefun{\PnlSparseMat
%     \ptr }{pnl_sparse_mat_create_frommorse}{\PnlMorseMat\ptr  M}
%   \sshortdescribe Create a \PnlSparseMat from  a
%   \PnlMorseMat with $M\rightarrow M->RC =1$.
% \item \describefun{void}{pnl_sparse_mat_free}{\PnlSparseMat
%     \ptr \ptr M}
%   \sshortdescribe Free a \PnlSparseMat.
% \end{itemize}

% \paragraph{Printing Matrix}
% \begin{itemize}
% \item \describefun{void}{pnl_sparse_mat_print}{\PnlSparseMat
%     \ptr A}
%   \sshortdescribe Print a \PnlSparseMat.
% \end{itemize}

% \subparagraph{Element wise operations}

% \begin{itemize}
% \item \describefun{void}{pnl_sparse_mat_plus_sparse_mat}{\PnlSparseMat \ptr lhs, 
%     const \PnlSparseMat \ptr rhs} 
%   \sshortdescribe In-place addition  

% \item \describefun{void}{pnl_sparse_mat_minus_sparse_mat}{\PnlSparseMat \ptr lhs, 
%     const \PnlSparseMat \ptr rhs} 
%   \sshortdescribe In-place substraction  

% \item \describefun{void}{pnl_sparse_mat_inv_term}{\PnlSparseMat \ptr lhs}
%   \sshortdescribe In-place term by term inversion  

% \item \describefun{void}{pnl_sparse_mat_div_mat_term}{\PnlSparseMat
%     \ptr lhs, const \PnlSparseMat \ptr rhs} 
%   \sshortdescribe In-place term by term division

% \item \describefun{void}{pnl_sparse_mat_mult_mat_term}{\PnlSparseMat
%     \ptr lhs, const \PnlSparseMat \ptr rhs} 
%   \sshortdescribe In-place term by term multiplication  



% \item \describefun{void}{pnl_sparse_mat_map_inplace}{\PnlSparseMat \ptr M, double(\ptr f)(double)}
%   \sshortdescribe Apply function \var{f} to each entry of \var{M}, which
%   is modified on exit.
% \end{itemize}


% \paragraph{Standard matrix operations}
% \begin{itemize}
% \item \describefun{int}{pnl_sparse_mat_gaxpby}{\PnlVect \ptr lhs, 
%     const \PnlSparseMat \ptr M, const \PnlVect
%     \ptr rhs}
%   \sshortdescribe Compute $lhs=lhs+ M * rhs$.
% \item \describefun{int}{pnl_sparse_mat_mult_vect_inplace}{\PnlVect
%     \ptr lhs, const \PnlSparseMat \ptr M, const
%     \PnlVect \ptr rhs}
%   \sshortdescribe Compute $lhs= M * rhs$.
% \end{itemize}

% \subsubsection{LU structure}

% From the sparse matrix, we extract the LU decomposition stored in \PnlSparseFactorization.
% \paragraph{Constructors and desctructors}
% \begin{itemize}
% \item \describefun{\PnlSparseFactorization
%     \ptr }{pnl_sparse_factorization_lu_create}{const \PnlSparseMat \ptr A, double tol}
%   \sshortdescribe Compute the LU factorisation of \var{A}

% \item \describefun{void}{pnl_sparse_factorization_free}{\PnlSparseFactorization \ptr \ptr  F}
%   \sshortdescribe Free a \PnlSparseFactorization.
% \end{itemize}

% \paragraph{Solving linear systems}

% \begin{itemize}
% \item \describefun{void}{pnl_sparse_factorization_lu_syslin}{const
%     \PnlSparseFactorization \ptr N, PnlVect \ptr b}
%   \sshortdescribe Solve the linear system \var{Nx = b} and stores the solution \var{x}
%   into \var{b} which means that the r.h.s member of the system is overwritten
%   during the resolution of the system. \var{N} is the decomposition computed by
%   \reffun{pnl_sparse_factorization_lu_create}.
% \end{itemize}

%% solver

\subsection{Iterative Solvers}
\subsubsection{Overview}

The structures and functions related to solvers are declared in
\verb!pnl/pnl_linalgsolver.h!. 

\describestruct{PnlIterationBase}
\describestruct{PnlCgSolver}
\describestruct{PnlBicgSolver}
\describestruct{PnlGmresSolver}
\begin{verbatim}
typedef struct _PnlIterationBase PnlIterationBase;
typedef struct _PnlCgSolver PnlCgSolver;
typedef struct _PnlBicgSolver PnlBicgSolver;
typedef struct _PnlGmresSolver PnlGmresSolver;

struct _PnlIterationBase
{
  /** 
   * Must be the first element in order for the object mechanism to work
   * properly. This allows any PnlVectXXX pointer to be cast to a PnlObject
   */
  PnlObject object; 
  int iteration;
  int max_iter;
  double normb;
  double tol_;
  double resid;
  int error;
  /* char *  err_msg; */
};

/* When you repeatedly use iterative solvers, do not malloc each time */
struct _PnlCgSolver 
{
  /** 
   * Must be the first element in order for the object mechanism to work
   * properly. This allows any PnlCgSolver  pointer to be cast to a PnlObject
   */
  PnlObject object; 
  PnlVect * r;
  PnlVect * z;
  PnlVect * p;
  PnlVect * q;
  double rho;
  double oldrho;
  double beta;
  double alpha;
  PnlIterationBase * iter;
} ;

struct _PnlBicgSolver 
{ 
  /** 
   * Must be the first element in order for the object mechanism to work
   * properly. This allows any PnlBicgSolver pointer to be cast to a PnlObject
   */
  PnlObject object; 
  double rho_1, rho_2, alpha, beta, omega;
  PnlVect * p;
  PnlVect * phat;
  PnlVect * s;
  PnlVect * shat;
  PnlVect * t;
  PnlVect * v;
  PnlVect * r;
  PnlVect *  rtilde;
  PnlIterationBase * iter;
} ;

struct _PnlGmresSolver
{ 
  /** 
   * Must be the first element in order for the object mechanism to work
   * properly. This allows any PnlGmresSolver pointer to be cast to a PnlObject
   */
  PnlObject object; 
  int restart;
  double beta;
  PnlVect * s;
  PnlVect * cs;
  PnlVect * sn;
  PnlVect * w;
  PnlVect * r;
  PnlMat * H;
  PnlVect * v[MAX_RESTART];
  PnlIterationBase *iter;
  PnlIterationBase *iter_inner;
} ;
\end{verbatim}

A Left preconditioner solves the problem :
$$ P M x = P b, $$
and whereas right preconditioner solves
$$ M P y  = b, \quad \quad P y = x.$$

%% With some simplifications, the number of algorithm iterations depends on
%% conditioning. Conditioning is ratio of maximum eigenvalue over minimum
%% eigenvalue of $M$. For GMRES algorithm is depend of conditioning of $M^{T}
%% M$. So if we can find $P_L$ and $P_R$ such that $P_L M P_R$ is closed to
%% identity matrix, then preconditioning problem converge faster than initial
%% problem. We have also to solve $P_R y = x$ so $P_R$ has to be constructed to
%% do that fast.

More information is given in {\em Saad, Yousef (2003). Iterative methods for
  sparse linear systems (2nd ed. ed.). SIAM. ISBN 0898715342. OCLC 51266114}.
The reader will find in this book some discussion about right or/and left
preconditioner and a description of the following algorithms.

These algorithms, we implemented with a left preconditioner. Right preconditioner
can be easily computed changing matrix vector multiplication operator from $M \
x $ to $ M \ P_R \ x$ and solving $P_R y = x$ at the end of algorithm.


\subsubsection{Functions}

Three methods are implemented : Conjugate Gradient, BICGstab and GMRES with
restart. For each of them a structure is created to store temporary vectors
used in the algorithm. In some cases, we have to apply iterative methods more
than once : for example to solve at each time step a discrete form of an
elliptic problem come from parabolic problem. In the cases, do not call the constructor and
destructor at each time, but instead use the initialization and solve procedures.

Formally we have, 
\begin{verbatim}
Create iterative method
For each time step
  Initialisation of iterative method
  Solve linear system link to elliptic problem
end for
free iterative method
\end{verbatim}

In these functions, we don't use any particular matrix structure. We give the
matrix vector multiplication as a parameter of the solver. 

\paragraph{Conjugate Gradient method}

Only available for symmetric and positive matrices.
\begin{itemize}
  \item \describefun{\refstruct{PnlCgSolver} \ptr }{pnl_cg_solver_new}{}
    \sshortdescribe Create an empty \refstruct{PnlCgSolver}  
\item   \describefun{\refstruct{PnlCgSolver} \ptr }{pnl_cg_solver_create}{int Size, int max-iter, double tolerance}
  \sshortdescribe Create a new \refstruct{PnlCgSolver} pointer.  
\item \describefun{void}{pnl_cg_solver_initialisation}{\refstruct{PnlCgSolver} \ptr Solver, const \PnlVect \ptr b}
  \sshortdescribe Initialisation of the solver at the beginning of iterative method.  
\item \describefun{void}{pnl_cg_solver_free}{\refstruct{PnlCgSolver} \ptr \ptr Solver}
  \sshortdescribe Destructor of iterative solver  
\item \describefun{int}{pnl_cg_solver_solve}{void(\ptr matrix vector-product)(const void \ptr , const \PnlVect \ptr , const double, const double, \PnlVect \ptr ), const void \ptr Matrix-Data, void(\ptr matrix vector-product-PC)(const void \ptr , const \PnlVect \ptr , const double, const double, \PnlVect \ptr ), const void \ptr PC-Data, \PnlVect \ptr x, const \PnlVect \ptr b, \refstruct{PnlCgSolver} \ptr Solver}
  \sshortdescribe Solve the linear system matrix vector-product is the matrix vector multiplication function matrix vector-product-PC is the preconditionner function Matrix-Data \& PC-Data is data to compute matrix vector multiplication.  
\end{itemize}
\paragraph{BICG stab}
\begin{itemize}
  \item \describefun{\refstruct{PnlBicgSolver} \ptr }{pnl_bicg_solver_new}{}
    \sshortdescribe Create an empty \refstruct{PnlBicgSolver}.  
\item \describefun{\refstruct{PnlBicgSolver} \ptr }{pnl_bicg_solver_create}{int Size, int max-iter, double tolerance}
  \sshortdescribe Create a new \refstruct{PnlBicgSolver} pointer.  
\item \describefun{void}{pnl_bicg_solver_initialisation}{\refstruct{PnlBicgSolver} \ptr Solver, const \PnlVect \ptr b}
  \sshortdescribe Initialisation of the solver at the beginning of iterative method.  
\item \describefun{void}{pnl_bicg_solver_free}{\refstruct{PnlBicgSolver} \ptr \ptr Solver}
  \sshortdescribe Destructor of iterative solver  
\item \describefun{int}{pnl_bicg_solver_solve}{void(\ptr matrix vector-product)(const void \ptr , const \PnlVect \ptr , const double, const double, \PnlVect \ptr ), const void \ptr Matrix-Data, void(\ptr matrix vector-product-PC)(const void \ptr , const \PnlVect \ptr , const double, const double, \PnlVect \ptr ), const void \ptr PC-Data, \PnlVect \ptr x, const \PnlVect \ptr b, \refstruct{PnlBicgSolver} \ptr Solver}
  \sshortdescribe Solve the linear system matrix vector-product is the matrix vector multiplication function matrix vector-product-PC is the preconditioner function Matrix-Data \& PC-Data is data to compute matrix vector multiplication.  
\end{itemize}

\paragraph{GMRES with restart} See {\em Saad, Yousef (2003)} for a discussion
about the restart parameter. For GMRES we need to store at the p-th iteration
$p$ vectors of the same size of the right and side. It could be very expensive
in term of memory allocation. So GMRES with restart algorithm stop if
$p=restart$ and restarts the algorithm with the previously computed solution
as initial guess.

Note that if restart equals $m$, we have a classical GMRES algorithm.

\begin{itemize}
  \item \describefun{\refstruct{PnlGmresSolver} \ptr }{pnl_gmres_solver_new}{}
    \sshortdescribe Create an empty \refstruct{PnlGmresSolver}  
\item \describefun{\refstruct{PnlGmresSolver} \ptr }{pnl_gmres_solver_create}{int Size, int max-iter, int restart, double tolerance}
  \sshortdescribe Create a new \refstruct{PnlGmresSolver} pointer.  
\item \describefun{void}{pnl_gmres_solver_initialisation}{\refstruct{PnlGmresSolver} \ptr Solver, const \PnlVect \ptr b}
  \sshortdescribe Initialisation of the solver at the beginning of iterative method.  
\item \describefun{void}{pnl_gmres_solver_free}{\refstruct{PnlGmresSolver} \ptr \ptr Solver}
  \sshortdescribe Destructor of iterative solver  
\item \describefun{int}{pnl_gmres_solver_solve}{void(\ptr matrix vector-product)(const void \ptr , const \PnlVect \ptr , const double, const double, \PnlVect \ptr ), const void \ptr Matrix-Data, void(\ptr matrix vector-product-PC)(const void \ptr , const \PnlVect \ptr , const double, const double, \PnlVect \ptr ), const void \ptr PC-Data, \PnlVect \ptr x, const \PnlVect \ptr b, \refstruct{PnlGmresSolver} \ptr Solver}
  \sshortdescribe Solve the linear system matrix vector-product is the matrix vector multiplication function matrix vector-product-PC is the preconditionner function Matrix-Data \& PC-Data is data to compute matrix vector multiplication.  
\end{itemize}


In the next paragraph, we write all the solvers for \PnlMat. This
will be done as follows: construct an application matrix vector.
\begin{verbatim}
static void pnl_mat_mult_vect_applied(const void *mat, const PnlVect *vec, 
                                      const double a , const double b, 
                                      PnlVect *lhs)
{pnl_mat_lAxpby(a, (PnlMat*)mat, vec, b, lhs);}
\end{verbatim}
and give it as the parameter of the iterative method
\begin{verbatim}
int pnl_mat_cg_solver_solve(const PnlMat * Matrix, const PnlMat * PC, 
                            PnlVect * x, const PnlVect *b, PnlCgSolver * Solver)
{ return pnl_cg_solver_solve(pnl_mat_mult_vect_applied, 
                             Matrix, pnl_mat_mult_vect_applied, 
                             PC, x, b, Solver);}
\end{verbatim}

In practice, we cannot define all iterative methods for all structures.
With this implementation, the user can easily :
\begin{itemize}
\item implement right precondioner, 
\item implement method with sparse matrix and diagonal preconditioner, or
  special combination of this form $\dots$
\end{itemize}


\paragraph{Iterative algorithms for \PnlMat}


\begin{itemize}
\item \describefun{int}{pnl_mat_cg_solver_solve}{const \PnlMat \ptr M, const \PnlMat \ptr PC, \PnlVect \ptr x, const \PnlVect \ptr b, \refstruct{PnlCgSolver} \ptr Solver}
  \sshortdescribe Solve the linear system \var{M x = b} with preconditionner PC.  
\item \describefun{int}{pnl_mat_bicg_solver_solve}{const \PnlMat \ptr M, const \PnlMat \ptr PC, \PnlVect \ptr x, const \PnlVect \ptr b, \refstruct{PnlBicgSolver} \ptr Solver}
  \sshortdescribe Solve the linear system \var{M x = b} with preconditionner PC.  
\item \describefun{int}{pnl_mat_gmres_solver_solve}{const \PnlMat \ptr M, const \PnlMat \ptr PC, \PnlVect \ptr x, \PnlVect \ptr b, \refstruct{PnlGmresSolver} \ptr Solver}
  \sshortdescribe Solve the linear system \var{M x = b} with preconditionner PC.
\end{itemize}

% vim:spelllang=en:spell:



\input{cdf.tex}
\section{Random Number Generators}

The functionalities described in this chapter are declared in
\verb!pnl/pnl_random.h!.

Random number generators should be called through the new {\em rng} interface
based on the \PnlRng object. This interface uses reentrant functions
and is suitable for multi-threaded applications.

The older {\em rand} interface is kept for compatibility
purposes only and should not be used in new code.
\begin{table}[h!]
  \begin{tabular}{l|l|l|l}
    Random generator & index & Type & Info\\
    \hline
    KNUTH & PNL_RNG_KNUTH & pseudo &\\
    MRGK3 & PNL_RNG_MRGK3 & pseudo &\\
    MRGK5 & PNL_RNG_MRGK5 & pseudo& \\
    SHUFL & PNL_RNG_SHUFL & pseudo &\\
    L'ECUYER & PNL_RNG_L_ECUYER & pseudo &\\
    TAUSWORTHE & PNL_RNG_TAUSWORTHE & pseudo& \\
    MERSENNE & PNL_RNG_MERSENNE & pseudo &\\
    SQRT & PNL_RNG_SQRT & quasi &\\
    HALTON & PNL_RNG_HALTON & quasi &\\
    FAURE & PNL_RNG_FAURE & quasi& \\
    SOBOL_I4 & PNL_RNG_SOBOL_I4 & quasi & uses 32 bit intergers\\
    SOBOL_I8 & PNL_RNG_SOBOL2_I8 & quasi & uses 64 bit intergers\\
    NIEDERREITER & PNL_RNG_NIEDERREITER & quasi &
  \end{tabular}
  \caption{Indices of the random generators}
  \label{rng-indices}
\end{table}


\subsection{The rng interface}
\label{rng-int}

It is possible to create several random number generators each with its own state
variable so that they can evolve independently in a shared memory environment.
These generators are suitable for use in multi-threaded programs. 

\describestruct{PnlRng}
\begin{lstlisting}
typedef struct _PnlRng PnlRng;
struct _PnlRng
{
  PnlObject object;
  int type; /*!< generator type *
  void (*Compute)(PnlRng *g, double *sample); /*!< the function to compute the
                                                next number in the sequence */
  int rand_or_quasi; /*!< can be PNL_MC or PNL_QMC */
  int dimension; /*!< dimension of the space in which we draw the samples */
  int counter; /*!< counter = number of samples already drawn */
  int has_gauss; /*!< Is a gaussian deviate available? */
  double gauss; /*!< If has_gauss==1, gauss a gaussian sample */
  int size_state; /*!< size in bytes of the state variable */
  void *state; /*!< state of the random generator */
};
\end{lstlisting}


\begin{itemize}
\item \describefun{void}{pnl_rng_free}{\PnlRng \ptr \ptr }
  \sshortdescribe Free a \PnlRng.
\item \describefun{\PnlRng\ptr }{pnl_rng_create}{int type}
  \sshortdescribe Create a \PnlRng corresponding to \var{type}
  which can be any of the values \var{PNL_RNG_XXX} listed in
  Table~\ref{rng-indices} which correspond to {\bf pseudo} random number generators.
  Once a generator has been created, you {\bf must} call
  \reffun{pnl_rng_sseed} before using it. 
\item \describefun{void}{pnl_rng_sseed}{\PnlRng \ptr rng, unsigned
    long int s}
  \sshortdescribe Set the seed of the genrator \var{rng} using \var{s}. If
  \var{s=0}, then a default seed (depending on the generator) is used.
\item \describefun{int}{pnl_rng_sdim}{\PnlRng \ptr rng, int dim}
  \sshortdescribe Set the dimension of the state space for a QMC generator and
  initializes it accordingly.  Returns OK if the generator has been initialized
  properly and FAIL otherwise.
\item \describefun{\PnlRng\ptr }{pnl_rng_copy}{const
  \PnlRng \ptr rng}
  \sshortdescribe Create a copy of \var{rng}.
\item \describefun{void}{pnl_rng_clone}{\PnlRng \ptr dest, const
  \PnlRng \ptr src}
  \sshortdescribe Copy the content of \var{src} into the already existing
  basis \var{dest}. On exit, \var{src} and \var{dest} are identical but
  independent. 
\item \describefun{\PnlRng\ptr}{pnl_rng_dcmt_create_id}{int id, ulong
    seed}
  \sshortdescribe Create a generator with type \var{PNL_RNG_DCMT} and identifier
  \var{id}. Two generators with different \var{id}s are independent. Note that
  the returned generator must be initialized with \reffun{pnl_rng_sseed} before
  usage. The identifier \var{id} can for instance correspond to the thread
  number or the processor rank in parallel computing.
\item \describefun{\PnlRng\ptr \ptr
  }{pnl_rng_dcmt_create_array_id}{int start_id, int max_id, ulong seed, int \ptr
    count}
  \sshortdescribe Create an array of generators with types \var{PNL_RNG_DCMT}
  and identifiers linearly varying between \var{start_id} and \var{max_id}. The
  number of generators created is \var{max_id - start_id + 1}. All the
  generators are independent. Note that each generator of the returned array
  must be initialized with \reffun{pnl_rng_sseed} before usage.
\item \describefun{\PnlRng \ptr \ptr}{pnl_rng_dcmt_create_array}
  {int n, ulong seed, int \ptr count}
  \sshortdescribe Create an array of \var{n} independent DCMT. \var{seed} is
  the seed used to initialize the Mersenne Twister generator internally used to
  find new DCMT. On exit, \var{count} contains the number of generators actually
  created. Before using the generators, you must initialize each of them by calling the
  function \reffun{pnl_rng_sseed} \var{count} times.
\end{itemize}

Some auxiliary functions internally used (to be used with caution)
\begin{itemize}
\item \describefun{\PnlRng\ptr }{pnl_rng_new}{}
  \sshortdescribe Create an empty \PnlRng.
\item \describefun{void}{pnl_rng_init}{\PnlRng \ptr rng, int type}
  \sshortdescribe Initialize an empty \PnlRng as returned by
  \reffun{pnl_rng_new} to become a generator of type \var{type} which can be
  any of the values \var{PNL_RNG_XXX} listed in Table~\ref{rng-indices} which
  correspond to {\bf pseudo} random number generators.
  Calling \reffun{pnl_rng_create} is equivalent to calling first
  \reffun{pnl_rng_new} and then \reffun{pnl_rng_init}. 
\item \describefun{\PnlRng\ptr }{pnl_rng_get_from_id} {int id}
  \sshortdescribe Return the global generator described by its macro name.
  The variable \var{id} can be any of the values \var{PNL_RNG_XXX} listed in
  Table~\ref{rng-indices}.
\end{itemize}


The following functions return one sample from the specified distribution.
\begin{itemize}
\item \describefun{int}{pnl_rng_bernoulli}{double p, \PnlRng \ptr rng}
  \sshortdescribe Generate a sample from the Bernouilli law on $\{0, 1\}$ with
  parameter \var{p}.

\item \describefun{long}{pnl_rng_poisson}{double lambda, \PnlRng \ptr rng}
  \sshortdescribe Generate a sample from the Poisson law with
  parameter \var{lambda}.

\item \describefun{double}{pnl_rng_exp}{double lambda, \PnlRng \ptr rng}
  \sshortdescribe Generate a sample from the Exponential law with
  parameter \var{lambda}.

\item \describefun{double}{pnl_rng_dblexp}{double lambda_p, double lambda_m, double p, \PnlRng \ptr rng}
  \sshortdescribe Generate a sample from the asymmetric exponential distribution
  with density 
  \begin{equation*}
    p \lambda_p \expp{-\lambda_p y} \ind{y > 0} + (1-p) \lambda_m
    \expp{-\lambda_m |y|} \ind{y < 0}
  \end{equation*}
  where $\lambda_p >0, \lambda_m >0$ and $p \in [0, 1]$.

\item \describefun{double}{pnl_rng_uni} {\PnlRng \ptr rng}
  \sshortdescribe Generate a sample from the Uniform law on $]0, 1]$.

\item \describefun{double}{pnl_rng_uni_ab} {double a, double b,
    \PnlRng \ptr rng}
  \sshortdescribe Generate a sample from the Uniform law on $[a, b]$.

\item \describefun{double}{pnl_rng_normal} {\PnlRng \ptr rng}
  \sshortdescribe Generate a sample from the standard normal distribution.

\item \describefun{double}{pnl_rng_lognormal}{double m, double sigma2, \PnlRng \ptr rng}
  \sshortdescribe Generate a sample from the log-normal distribution. The
  underlying normal distribution has mean \var{m} and variance \var{sigma2}.

\item \describefun{double}{pnl_rng_invgauss}{double mu, double lambda, \PnlRng \ptr rng}
  \sshortdescribe Generate a sample from the inverse Gaussian distribution with
  mean \var{mu} and shape parameter \var{lambda}.

\item \describefun{long}{pnl_rng_poisson1}{double lambda, double
    t,\PnlRng \ptr rng}
  \sshortdescribe Generate a sample from a Poisson process with intensity
  \var{lambda} at time \var{t}.

\item \describefun{double}{pnl_rng_gamma} {double a, double b, \PnlRng \ptr rng}
  \sshortdescribe Generate a sample from the $\Gamma(a, b)$ distribution.

\item \describefun{double}{pnl_rng_chi2} {double df, \PnlRng \ptr rng}
  \sshortdescribe Generate a sample from the centered $\chi^2(df)$ distribution.
\item \describefun{double}{pnl_rng_ncchi2} {double df, double xnonc, \PnlRng \ptr rng}
  \sshortdescribe Generate a sample from the non central $\chi^2$ distribution
  with \var{df} degrees of freedom and non central parameter \var{xnonc}.
\item \describefun{double}{pnl_rng_bessel} {double v, double a,\PnlRng \ptr rng}
  \sshortdescribe Generate a sample from the Bessel distribution with parameters
  \var{v > -1} and \var{a > 0}.
\item \describefun{double}{pnl_rng_gauss}{int d, int
    create_or_retrieve, int index, PnlRng \ptr rng}
  \sshortdescribe The second argument can be either \var{CREATE} (to actually
  draw the sample) or \var{RETRIEVE} (to retrieve that element of index
  \var{index}). With \var{CREATE}, it draws \var{d} random normal variables
  and stores them for future usage. They can be withdrawn using \var{RETRIEVE}
  with the index of the number to be retrieved.
\end{itemize}

The following functions take an already existing \PnlVect\ptr  as
first argument and fill each entry of the vector with a sample from the
specified distribution. All the entries are independent. The difference between
$n-$samples from a distribution in dimension $1$, and one sample from the same
distribution in dimension $n$ only matters when using a {\bf quasi} random
number generator.
\begin{itemize}
\item \describefun{void}{pnl_vect_rng_bernoulli}{\PnlVect \ptr V, int samples,
  double a, double b, double p, \PnlRng \ptr rng}
  \sshortdescribe Simulate an i.i.d. sample from the Bernoulli
  distribution with values in \var{a,b} and parameter \var{p}. The result is
  stored in \var{V}.

\item \describefun{void}{pnl_vect_rng_bernoulli_d}{\PnlVect \ptr V, int
  dimension, const \PnlVect \ptr a, const \PnlVect \ptr b, const \PnlVect \ptr p, \PnlRng \ptr rng}
  \sshortdescribe Simulate a random vector according to the Bernoulli
  distribution with values in \var{\{a,b\}} and parameter \var{p}. The result is
  stored in \var{V}, ie. \var{V(i)} follows a Bernoulli distribution on
  \var{\{a(i), b(i)\}} with parameter \var{p(i)}.

\item \describefun{void}{pnl_vect_rng_poisson}{\PnlVect \ptr V, int samples,
  double lambda,  \PnlRng \ptr rng}
  \sshortdescribe Simulate an i.i.d. sample from the Poisson distribution with
  parameter \var{lambda}. The result is stored in \var{V}. Note that, we are
  using double based vectors and not integer based vectors.

\item \describefun{void}{pnl_vect_rng_poisson_d}{\PnlVect \ptr V, int
  dimension, const \PnlVect \ptr lambda, \PnlRng \ptr rng}
  \sshortdescribe Simulate a random vector according to the Poisson distribution
  with \textbf{vector} parameter \var{lambda}. The result is stored in \var{V},
  ie. \var{V(i)} follows a Poisson distribution with parameter \var{lambda(i)}.
  Note that, we are using double based vectors and not integer based vectors.
  
\item \describefun{void}{pnl_vect_rng_uni}{\PnlVect \ptr G, int
    samples, double a, double b, \PnlRng \ptr rng}
  \sshortdescribe \var{G} is a vector of independent and identically distributed
  samples from the uniform distribution on $[a, b]$.

\item \describefun{void}{pnl_vect_rng_normal}{\PnlVect \ptr G,
    int samples, \PnlRng \ptr rng}
  \sshortdescribe \var{G} is a vector of independent and identically distributed
  samples from the standard normal distribution.

\item \describefun{void}{pnl_vect_rng_uni_d}{\PnlVect \ptr G, int
    d, double a, double b, \PnlRng \ptr rng}
  \sshortdescribe \var{G} is a sample from the uniform distribution on $[a,
  b]^{\text{d}}$.

\item \describefun{void}{pnl_vect_rng_normal_d}{\PnlVect \ptr G,
    int d, \PnlRng \ptr rng}
  \sshortdescribe \var{G} is a sample from the \var{d}-dimensional
  standard normal distribution.

\end{itemize}

The following functions take an already existing \PnlMat\ptr  as
first argument and fill each entry of the matrix with a sample from the
specified distribution. All the entries are independent. On return, the matrix \var{M}
is of size \verb!samples x dimension!. The rows of \var{M} are independent
and identically distributed. Each row is a sample from the given law in
dimension \var{dimension}.
\begin{itemize}
\item \describefun{void}{pnl_mat_rng_uni}{\PnlMat \ptr M, int
    samples, int d, const PnlVect \ptr a, const PnlVect \ptr b,
    \PnlRng \ptr rng}
  \sshortdescribe \var{M} contains \var{samples} samples from the uniform
  distribution on $\prod_{i=1}^d [a_i, b_i]$.

\item \describefun{void}{pnl_mat_rng_uni2}{\PnlMat \ptr M, int
    samples, int d, double a, double b, \PnlRng \ptr rng}
  \sshortdescribe \var{M} contains \var{samples} samples from the uniform
  distribution on $[a, b]^{\text{d}}$.

\item \describefun{void}{pnl_mat_rng_normal}{\PnlMat \ptr M, int
    samples, int d, \PnlRng \ptr rng}
  \sshortdescribe \var{M} contains \var{samples} samples from the
  \var{d}-dimensional standard normal distribution.

\item \describefun{void}{pnl_mat_rng_bernoulli}{\PnlMat \ptr M, int samples, int dimension,
  const \PnlVect \ptr a, const \PnlVect \ptr b, const \PnlVect \ptr p, \PnlRng \ptr rng}
  \sshortdescribe Compute a random matrix with independent rows, each of them
  having a vector Bernoulli distribution, ie. \var{M(i, j)} follows a
  Bernoulli distribution on \var{\{a(j), b(j)\}} with parameter \var{p(j)}.

\item \describefun{void}{pnl_mat_rng_poisson}{\PnlMat \ptr M, int samples, int dimension,
  const \PnlVect \ptr lambda, \PnlRng \ptr rng}
  \sshortdescribe Compute a random matrix with independent rows, each of them
  having a vector Poisson distribution, ie. \var{M(i, j)} follows a
  Poisson distribution with parameter \var{p(j)}.
\end{itemize}

Some examples
\begin{lstlisting}
#include <stdlib.h>
#include "pnl/pnl_random.h"

int main ()
{
  int i, M;
  PnlRng *rng = pnl_rng_create(PNL_RNG_MERSENNE);
  PnlVect *v = pnl_vect_new();
  M = 10000;

  /* rng must be initialized. When sseed=0, a default 
     value depending on the generator is used */
     pnl_rng_sseed(rng, 0);

  for (i=0 ; i<M ; i++)
  {
    /* Simulates a normal random vector in R^{10} */
    pnl_vect_rng_normal(v, 10, rng);
    /* Do something with v */
  }

  pnl_vect_free(&v);
  pnl_rng_free(&rng); /* Frees the generator */
  exit(0);
}
\end{lstlisting}

\begin{lstlisting}
#include <stdlib.h>
#include <time.h>
#include "pnl/pnl_random.h"

int main ()
{
  int i, M;
  double E;
  PnlRng *rng = pnl_rng_create(PNL_RNG_MERSENNE);
  M = 10000;

  /* rng must be initialized. */
  pnl_rng_sseed(rng, time (NULL));

  for (i=0 ; i<M ; i++)
  {
    /* Simulates an exponential random variable */
    E = pnl_rng_exp(1, rng);
    /* Do something with E */
  }

  pnl_rng_free(&rng); /* Frees the generator */
  exit(0);
}
\end{lstlisting}

\subsection{The {\em rand} interface (deprecated)}
\label{rand-int}

{\itshape 
\textbf{Note}:
For backward compatibility with older versions of the PNL, we still provide the old
{\em rand} interface to random number generation although we strongly encourage users
to use the new {\em rng} interface (see section~\ref{rng-int}).
}

Every generator is identified by an integer valued macro. One must {\bf NOT} refer
to a generator using directly the value of the macro \var{PNL_RNG_XXX} because there
is no warranty that the order used to store the generators will remain the same in
future releases.  Instead, one should call generators directly using their macro
names.

The initial seeds of all the generators are fixed by the function
\reffun{pnl_rand_init} but you can change it by calling \reffun{pnl_rand_sseed}.

Before starting to use random number generators, you {\bf must} initialize them by
calling
\begin{itemize}
\item \describefun{int}{pnl_rand_init}{int type_generator, int
    simulation_dim, long samples}
  \sshortdescribe It resets the sample counter to $0$ and checks that the
  generator described by \var{type_generator} can actually generate
  \var{samples} in dimension \var{simulation_dim} and fixes the seed.
\end{itemize}

\begin{itemize}

\item \describefun{int}{pnl_rand_or_quasi}{int type_generator}
  \sshortdescribe Return the type the generator of index \var{type_generator},
  \var{PNL_MC} or \var{PNL_QMC}
\item \describefun{void}{pnl_rand_sseed}{(int type_generator, unsigned long int
  seed)}
  \sshortdescribe It sets the seed of the generator \var{type_generator} with
  \var{seed}.
\item \describefun{const char \ptr }{pnl_rand_name}{int type_generator}
  \sshortdescribe Return the name of the generator of index \var{type_generator}
\end{itemize}

Once a generator is chosen, there are several functions available in the
library to draw samples according to a given law.

The following functions return one sample from a specified law.
\begin{itemize}
\item \describefun{int}{pnl_rand_bernoulli}{double p, int type_generator}
  \sshortdescribe Generate a sample from the Bernouilli law on $\{0, 1\}$ with
  parameter \var{p}.

\item \describefun{long}{pnl_rand_poisson}{double lambda, int type_generator}
  \sshortdescribe Generate a sample from the Poisson law with
  parameter \var{lambda}.

\item \describefun{double}{pnl_rand_exp}{double lambda, int type_generator}
  \sshortdescribe Generate a sample from the Exponential law with
  parameter \var{lambda}.

\item \describefun{double}{pnl_rand_uni} {int type_generator}
  \sshortdescribe Generate a sample from the Uniform law on $[0, 1]$.

\item \describefun{double}{pnl_rand_uni_ab} {double a, double b, int
    type_generator}
  \sshortdescribe Generate a sample from the Uniform law on $[a, b]$.

\item \describefun{double}{pnl_rand_normal} {int type_generator}
  \sshortdescribe Generate a sample from the standard normal distribution.

\item \describefun{long}{pnl_rand_poisson1}{double lambda, double t, int
    type_generator}
  \sshortdescribe Generate a sample from a Poisson process with intensity
  \var{lambda} at time \var{t}.

\item \describefun{double}{pnl_rand_gamma} {double a, double b, int type_generator}
  \sshortdescribe Generate a sample from the $\Gamma(a, b)$ distribution.

\item \describefun{double}{pnl_rand_chi2} {double n, int type_generator}
  \sshortdescribe Generate a sample from the centered $\chi^2(n)$ distribution.
\item \describefun{double}{pnl_rand_bessel} {double v, double a, int generator}
  \sshortdescribe Generate a sample from the Bessel distribution with parameters
  \var{v > -1} and \var{a > 0}.
\end{itemize}

The following functions take an already existing \PnlVect\ptr\  as
its first argument and fill each entry of the vector with a sample from the
specified law. All the entries are independent. The difference between
$n-$samples from a distribution in dimension $1$, and one sample from the same
distribution in dimension $n$ only matters when using a {\bf Quasi} random
number generator.
\begin{itemize}
\item \describefun{void}{pnl_vect_rand_uni}{\PnlVect \ptr G, int
    samples, double a, double b, int type_generator}
  \sshortdescribe \var{G} is a vector of independent and identically distributed
  samples from the uniform distribution on $[a, b]$.

\item \describefun{void}{pnl_vect_rand_normal}{\PnlVect \ptr G,
    int samples, int generator}
  \sshortdescribe \var{G} is a vector of independent and identically distributed
  samples from the standard normal distribution.

\item \describefun{void}{pnl_vect_rand_uni_d}{\PnlVect \ptr G, int
    d, double a, double b, int type_generator}
  \sshortdescribe \var{G} is a sample from the uniform distribution on $[a,
  b]^{\text{d}}$.

\item \describefun{void}{pnl_vect_rand_normal_d}{\PnlVect \ptr G,
    int d, int generator}
  \sshortdescribe \var{G} is a sample from the \var{d}-dimensional
  standard normal distribution.

\end{itemize}

The following functions take an already existing \PnlMat\ptr\  as
first argument and fill each entry of the vector with a sample from the
specified law. All the entries are in-dependant. On return, the matrix \var{M}
is of size \verb!samples x dimension!. The rows of \var{M} are independently
and identically distributed. Each row is a sample from the given law in
dimension \var{dimension}.
\begin{itemize}
\item \describefun{void}{pnl_mat_rand_uni}{\PnlMat \ptr M, int
    samples, int d, const PnlVect \ptr a, const PnlVect \ptr b, int
    type_generator}
  \sshortdescribe \var{M} contains \var{samples} samples from the uniform
  distribution on $\prod_{i=1}^d [a_i, b_i]$.

\item \describefun{void}{pnl_mat_rand_uni2}{\PnlMat \ptr M, int
    samples, int d, double a, double b, int type_generator}
  \sshortdescribe \var{M} contains \var{samples} samples from the uniform
  distribution on $[a, b]^{\text{d}}$.

\item \describefun{void}{pnl_mat_rand_normal}{\PnlMat \ptr M, int
    samples, int d, int type_generator}
  \sshortdescribe \var{M} contains \var{samples} samples from the
  \var{d}-dimensional standard normal distribution.
\end{itemize}

Because of the use of {\bf Quasi} random number generators, you may need to
draw a set of samples at once because they represent one sample from a
multi-dimensional distribution. The following function enables to draw one
sample from the \var{dimension}-dimensional standard normal distribution and
store it so that you can access the elements individually afterwards.
\begin{itemize}
\item \describefun{double}{pnl_rand_gauss}{int d, int
    create_or_retrieve, int index, int type_generator}
  \sshortdescribe The second argument can be either \var{CREATE} (to actually
  draw the sample) or \var{RETRIEVE} (to retrieve that element of index
  \var{index}). With \var{CREATE}, it draws \var{d} random normal variables
  and stores them for future usage. They can be withdrawn using \var{RETRIEVE}
  with the index of the number to be retrieved.
\end{itemize}


% \subsubsection{Advanced usage}
%
% We also provide functions for directly manipulating Mersenne Twister and
% ``Dynamically created Mersenne Twister'' random number generators, although we
% believe one should rather use the new {\em rng} interface.
%
% \paragraph{Mersenne Twister}
%
% It is possible to create Mersenne Twister random number generators each with
% its state variable.
% \begin{lstlisting}
% typedef struct
% {
%   unsigned long mt[624];
%   int mti;
% } mt_state;
% typedef unsigned long ulong;
% \end{lstlisting}
%
% \begin{itemize}
% \item \describefun{void}{pnl_mt_sseed}{mt_state \ptr state, unsigned long int
%     s}
%   \sshortdescribe Set the initial value of variable \var{state} using \var{s}
% \item \describefun{ulong}{pnl_mt_genrand}{mt_state \ptr state}
%   \sshortdescribe Return the following number in the sequence as an unsigned
%   long variable. A mask is applied so that only the lowest 32-bits are used.
% \item \describefun{double}{pnl_mt_genrand_double}{mt_state \ptr state}
%   \sshortdescribe Return the following number in the sequence as a double.
% \end{itemize}
%
%
% \paragraph{Dynamically created Mersenne Twister}
%
%
% These are Mersenne Twister type generators with Mersenne exponent fixed to
% \var{p=521} and word length \var{w=32} bits. These choices are hard coded and
% cannot be changed without altering the code directly.
%
% \begin{lstlisting}
% typedef struct
% {
%   ulong aaa;
%   int mm,nn,rr,ww;
%   ulong wmask,umask,lmask;
%   int shift0, shift1, shiftB, shiftC;
%   ulong maskB, maskC;
%   int i;
%   ulong state[17];
% } dcmt_state;
% \end{lstlisting}
%
% Some functions to use ``Dynamically Created Mersenne Twister'' random number
% generators (DCMT).
% \begin{itemize}
% \item \describefun{dcmt_state\ptr}{pnl_dcmt_get_parameter}{ulong seed}
%   \sshortdescribe Create a DCMT. \var{seed} is the seed used to initialize
%   the Mersenne Twister generator internally used to find new DCMT.
% \item \describefun{dcmt_state \ptr \ptr}{pnl_dcmt_create_array}{int n, ulong seed, int \ptr count}
%   \sshortdescribe Create an array of \var{n} independent DCMT. \var{seed} is
%   the seed used to initialize the Mersenne Twister generator internally used to
%   find new DCMT. On exit, \var{count} contains the number of generators actually
%   created.
% \item \describefun{double}{pnl_dcmt_genrand_double}{dcmt_state \ptr mts}
%   \sshortdescribe Generate a uniformly distributed random variable on \var{[0,1]}.
% \item \describefun{void}{pnl_dcmt_free}{dcmt_state \ptr \ptr mts}
%   \sshortdescribe Free a dcmt.
% \item \describefun{void}{pnl_dcmt_free_array}{dcmt_state \ptr \ptr mts, int count}
%   \sshortdescribe Free an array of dcmt as returned by \reffun{pnl_dcmt_create_array}
% \end{itemize}

% vim:spelllang=en:spell:


\section{Function bases and regression}
\subsection{Overview}

To use these functionalities, you should include \verb!pnl/pnl_basis.h!.

\describestruct{PnlBasis}
\begin{verbatim}
struct _PnlBasis
{
  /**
   * Must be the first element in order for the object mechanism to work
   * properly. This allows any PnlBasis pointer to be cast to a PnlObject
   */
  PnlObject     object;
  /** The basis type */
  int           id;
  /** The string to label the basis */
  const char   *label;
  /** The number of variates */
  int           nb_variates;
  /** The total number of elements in the basis */
  int           nb_func;
  /** The tensor matrix */
  PnlMatInt    *T;
  /** The sparse Tensor matrix */
  PnlSpMatInt  *SpT;
  /** The number of functions in the tensor #T */
  int           len_T;
  /** Compute the i-th element of the one dimensional basis.  As a convention, (*f)(x, 0) MUST be equal to 1 */
  double      (*f)(double x, int i, int dim, void *params);
  /** Compute the first derivative of i-th element of the one dimensional basis */
  double      (*Df)(double x, int i, int dim, void *params);
  /** Compute the second derivative of the i-th element of the one dimensional basis */
  double      (*D2f)(double x, int i, int dim, void *params);
  /** PNL_TRUE if the basis is reduced */
  int           isreduced;
  /** The center of the domain */
  double       *center;
  /** The inverse of the scaling factor to map the domain to [-1, 1]^nb_variates */
  double       *scale;
  /** An array of additional functions */
  PnlRnFuncR   *func_list;
  /** The number of functions in #func_list */
  int           len_func_list;
  /** Extra parameters to pass to basis functions */
  void         *params;
  /** Size of params in bytes to be passed to malloc */
  size_t        params_size;
};
\end{verbatim}

A \refstruct{PnlBasis} is a family of multivariate functions with real values. Tow different kinds of functions can be stored in these families: tensor functions --- originally, this was the only possibility --- and standard multivariate function typed as \refstruct{PnlRnFuncR}.

\paragraph{Tensor functions.}  Tensors functions are built as a tensor product of one
dimensional elements. Hence, we only need a tensor matrix \var{T} to describe a
multi-dimensional basis in terms of the one dimensional one. These tensors functions can
be easily evaluated and differentiated twice, see \reffun{pnl_basis_eval},
\reffun{pnl_basis_eval_vect}, \reffun{pnl_basis_eval_D}, \reffun{pnl_basis_eval_D_vect},
\reffun{pnl_basis_eval_D2}, \reffun{pnl_basis_eval_D2_vect},
\reffun{pnl_basis_eval_derivs}, \reffun{pnl_basis_eval_derivs_vect}.

The two tensors \var{T} and \var{SpT} do actually store the same information ---
\var{T(i,j)} is the degree w.r.t the \var{j}-th variable in the \var{i}-th
function. Originally, we were only using the dense representation \var{T}, which
is far more convenient to use when building the basis but it slows down the
evaluation of the basis by a great deal. To overcome this lack of efficiency, a
sparse storage has been added. 

\begin{table}[h!]
  \begin{describeconst}
    \constentry{PNL_BASIS_CANONICAL}{for the Canonical polynomials}
    \constentry{PNL_BASIS_HERMITE}{for the Hermite polynomials}
    \constentry{PNL_BASIS_TCHEBYCHEV}{for the Tchebychev polynomials}
  \end{describeconst}
  \caption{Names of the bases. See also function
  \reffun{pnl_basis_type_register} to register more basis types.}
  \label{basis_index}
\end{table}

The Hermite polynomials are defined by
\begin{equation*}
  H_n(x) = (-1)^n \expp{\frac{x^2}{2}} \frac{d^n}{dx^n} \expp{-\frac{x^2}{2}}.
\end{equation*}
If $G$ is a real valued standard normal random variable, ${\mathbb E}[H_n(G) H_m(G)] = n!
\ind{n = m}$. \\

\paragraph{Standard multivariate functions.} These functions are supposed to be \refstruct{PnlRnFuncR}.

  To make this toolbox more complete, it is now possible to add some extra functions, which are not tensor product. They are stored using an independent mechanism in \var{func_list}.  These additional functions are only taken into account by the methods \reffun{pnl_basis_i}, \reffun{pnl_basis_i_vect}, \reffun{pnl_basis_eval} and \reffun{pnl_basis_eval_vect}. Note in particular that it is not possible to differentiate these functions and that they are not sensitive to the \var{isreduced} attribute. To add an extra function to an existing \refstruct{PnlBasis}, call the function \reffun{pnl_basis_add_function}.

\subsection{Functions}

\begin{itemize}
\item \describefun{int}{pnl_basis_type_register}{const char *name, double (*f)(double, int, int, void), double (*Df)(double, int, int, void), double (*D2f)(double, int, int, void), int is_orthogonal}
\sshortdescribe Register a new basis type and return the index to be passed to
\reffun{pnl_basis_create} . The variable \var{name} is a unique
string identifier of the family. The variables \var{f}, \var{Df}, \var{D2f} are
the one dimensional basis functions, its first and second order derivatives.
Each of these functions must return a \var{double} and take two arguments : the
first one is the point at which evaluating the basis functions, the second one
is the index of function. The var{is_orthogonal} if the elements of the basis are orthogonal for the $L^2$ scalar product. Here is a toy example to show how the canonical basis
is registered (this family is actually already available with the id
PNL_BASIS_CANONICAL, so the following example may look a little fake)
\begin{verbatim}
  double f(double x, int n, int dim, void *params) { return pnl_pow_i(x, n); }
  double Df(double x, int n, int dim, void *params) { return n * pnl_pow_i(x, n-1); }
  double f(double x, int n, int dim, void *params) { return n * (n-1) * pnl_pow_i(x, n-2); }

  int id = pnl_basis_register ("Canonic", f, Df, D2f, PNL_FALSE);
  /*
   * B is the Canonical basis of polynomials with degree less or equal than 2 in
   * dimension 5.
   */
  PnlBasis *B = pnl_basis_create_from_degree (id, 2, 5);
  \end{verbatim}

\item \describefun{\PnlBasis *}{pnl_basis_new}{}
  \sshortdescribe Create an empty \PnlBasis.

\item \describefun{void}{pnl_basis_print}{const \PnlBasis \ptr B}
  \sshortdescribe Print the characteristics of a basis.

\item \describefun{\PnlBasis *}{pnl_basis_create}{int index, int
    nb_func, int nb_variates}
  \sshortdescribe Create a \PnlBasis for the family
  defined by \var{index} (see Table~\ref{basis_index} and
  \reffun{pnl_basis_type_register}) with \var{nb_variates}
  variates. The basis will contain \var{nb_func}.

\item \describefun{\PnlBasis *}{pnl_basis_create_from_degree}{int
    index, int degree, int nb_variates}
  \sshortdescribe Create a \PnlBasis for the family
  defined by \var{index} (see Table~\ref{basis_index} and \reffun{pnl_basis_type_register}) with total degree less
  or equal than \var{degree} and \var{nb_variates} variates. The total degree is
  the sum of the partial degrees.\\
  For instance, calling \verb!pnl_basis_create_from_degree(index, 2, 4)! is
  equivalent to calling \verb!pnl_basis_create_from_tensor(index, T)! where
  \var{T} is given by
  \[ \left(
    \begin{array}{cccc}
      0 & 0 & 0 & 0\\
      1 & 0 & 0 & 0\\
      0 & 1 & 0 & 0\\
      0 & 0 & 1 & 0\\
      0 & 0 & 0 & 1\\
      1 & 1 & 0 & 0\\
      1 & 0 & 1 & 0\\
      1 & 0 & 0 & 1\\
      0 & 1 & 1 & 0\\
      0 & 1 & 0 & 1\\
      0 & 0 & 1 & 1\\
      2 & 0 & 0 & 0\\
      0 & 2 & 0 & 0\\
      0 & 0 & 2 & 0\\
      0 & 0 & 0 & 2\\
    \end{array}
  \right) \]
\item \describefun{\PnlBasis *}{pnl_basis_create_from_prod_degree}{int
    index, int degree, int nb_variates}
  \sshortdescribe Create a \PnlBasis for the family
  defined by \var{index} (see Table~\ref{basis_index} and \reffun{pnl_basis_type_register}) with total degree less
  or equal than \var{degree} and \var{nb_variates} variates. The total degree is
  the product of \var{MAX(1, d_i)} where the \var{d_i} are the partial degrees.


\item \describefun{\PnlBasis *}{pnl_basis_create_from_tensor}{int
    index, PnlMatInt \ptr T}
  \sshortdescribe Create a \PnlBasis for the polynomial family
  defined by \var{index} (see Table~\ref{basis_index}) using the basis
  described by the tensor matrix \var{T}. The number of lines of \var{T} is
  the number of functions of the basis whereas the numbers of columns of
  \var{T} is the number of variates of the functions.
  Note that \var{T} is not copied inside this function but only its address is
  stored, so {\bf never} free \var{T}. It will be freed when calling
  \reffun{pnl_basis_free} on the returned object. i\\
  Here is an example of a tensor matrix. Assume you are working with three
  variate functions, the basis \verb!{ 1, x, y, z, x^2, xy, yz, z^3}! is
  decomposed in the one dimensional canonical basis using the following tensor
  matrix
  \[ \left(
    \begin{array}{ccc}
      0 & 0 & 0 \\
      1 & 0 & 0 \\
      0 & 1 & 0 \\
      0 & 0 & 1 \\
      2 & 0 & 0 \\
      1 & 1 & 0 \\
      0 & 1 & 1\\
      0 & 0 & 3
    \end{array}
  \right) \]

\item \describefun{void}{pnl_basis_add_function}{\PnlBasis \ptr b, \PnlRnFuncR \ptr f}
  \sshortdescribe Add the function \var{f} to the already existing basis \var{b}.
\item \describefun{void}{pnl_basis_clone}{\PnlBasis \ptr dest, const \PnlBasis \ptr src}
  \sshortdescribe Clone \var{src} into \var{dest}. The basis \var{dest} must
  already exist before calling this function. On exit, \var{dest} and \var{src}
  are identical and independent.
\item \describefun{\PnlBasis\ptr }{pnl_basis_copy}{const \PnlBasis \ptr B}
  \sshortdescribe Create a copy of \var{B}.
\item \describefun{void }{pnl_basis_set_from_tensor}{\PnlBasis \ptr
    b, int index, const \PnlMatInt \ptr T}
  \sshortdescribe Set an alredy existing basis \var{b} to a polynomial family
  defined by \var{index} (see Table~\ref{basis_index}) using the basis
  described by the tensor matrix \var{T}. The number of lines of \var{T} is
  the number of functions of the basis whereas the numbers of columns of
  \var{T} is the number of variates of the functions. \\
  Same function as \reffun{pnl_basis_create_from_tensor} except that it
  operates on an already existing basis.

\item  \describefun{\PnlBasis\ptr}{pnl_basis_create_from_hyperbolic_degree}
  {int index, double degree, double q, int n}
  \sshortdescribe Create a sparse basis of polynomial with \var{n}
  variates. We give the example of the Canonical basis. A canonical polynomial
  with \var{n} variates writes $X_1^{\alpha_1} X_2^{\alpha_2} \dots
  X_n^{\alpha_n}$. To be a member of the basis, it must satisfy $\left(\sum_{i=1}^n
    {\alpha_i}^q \right)^{1/q} \leq degree$. This kind of basis based on an
  hyperbolic set of indices gives priority to polynomials associated to low
  order interaction.

\item \describefun{\PnlBasis\ptr}{pnl_basis_create_local}{int \ptr n_intervals, int space_dim}
  \sshortdescribe Create a local bases on $(-1, 1)^{\var{space_dim}}$ with \var{n_intervals[i]} along dimension \var{i} for $i \in \{1,\dots,\var{space_dim}\}$. This basis is orthogonal. Note that this basis is not differentiable.

\item  \describefun{void}{pnl_basis_free}{\PnlBasis \ptr\ptr basis}
  \sshortdescribe Free a \PnlBasis created by
  \reffun{pnl_basis_create}. Beware that \var{basis} is the address of a
  \PnlBasis\ptr.


\item \describefun{void}{pnl_basis_del_elt}{\PnlBasis \ptr B, const \PnlVectInt \ptr d}
  \sshortdescribe Remove the function defined by the tensor product \var{d} from
  an existing basis \var{B}.

\item \describefun{void}{pnl_basis_del_elt_i}{\PnlBasis \ptr B, int i}
  \sshortdescribe Remove the \var{i-th} element of basis \var{B}.

\item \describefun{void}{pnl_basis_add_elt}{\PnlBasis \ptr B, const \PnlVectInt \ptr d}
  \sshortdescribe Add the function defined by the tensor \var{d} to the Basis \var{B}.


\end{itemize}


Functional regression based on a least square approach often leads to ill conditioned linear systems. One way of improving the stability of the system is to use centered and renormalised polynomials so that the original domain of interest $\cD$ (a subset of $\R^d$) is mapped to $[-1,1]^d$. If the domain $\cD$ is rectangular and writes $[a, b]$ where $a,b \in \R^d$, the mapping is done by 
\begin{equation}
  \label{basis_reduced}
  x \in \cD \longmapsto \left(\frac{x_i - (b_i+a_i)/2}{(b_i - a_i)/2}
  \right)_{i=1,\cdots,d}
\end{equation}
\begin{itemize}
\item \describefun{void}{pnl_basis_set_domain}{\PnlBasis \ptr B, 
  const \PnlVect \ptr a, const \PnlVect \ptr b}
  \sshortdescribe This function declares \var{B} as a centered and normalised basis
  as defined by Equation~\ref{basis_reduced}. Calling this function is equivalent to
  calling \reffun{pnl_basis_set_reduced} with \var{center=(b+a)/2} and
  \var{scale=(b-a)/2}.
\item \describefun{void}{pnl_basis_set_reduced}{\PnlBasis \ptr B,
  const \PnlVect \ptr center, const \PnlVect \ptr scale}
  \sshortdescribe This function declares \var{B} as a centered and normalised basis
  using the mapping
  \begin{equation*}
    x \in \cD \longmapsto \left(\frac{x_i - \var{center}_i }{\var{scale}_i}
    \right)_{i=1,\cdots,d}
  \end{equation*}
\end{itemize}
Note that this renormlization does not apply to the extra functions by \reffun{pnl_basis_add_function} but only to the functions defined by the tensor \var{T}.


\begin{itemize}
\item \describefun{int}{pnl_basis_fit_ls}{\PnlBasis \ptr
    P, \PnlVect \ptr  coef, \PnlMat \ptr  x,
    \PnlVect \ptr  y}
  \sshortdescribe Compute the coefficients \var{coef} defined by
  \begin{equation*}
    \var{coef} = \arg\min_\alpha \sum_{i=1}^n
    \left( \sum_{j=0}^{\var{N}} \alpha_j  P_j(x_i) - y_i\right)^2
  \end{equation*}
  where \var{N} is the number of functions to regress upon and $n$ is the number
  of points at which the values of the original function are known. $P_j$ is the
  $j-th$ basis function. Each row of the matrix \var{x} defines the coordinates
  of one point $x_i$. The function to be approximated is defined by the vector
  \var{y} of the values of the function at the points \var{x}.

\item \describefun{double}{pnl_basis_ik_vect}{const \PnlBasis \ptr b, const \PnlVect \ptr x, int i, int k}
  \sshortdescribe An element of a basis writes $\prod_{l=0}^{\var{nb_variates}}
  \phi_l(x_l)$ where the $\phi$'s are one dimensional polynomials. This
  functions computes the therm $\phi_k$ of the \var{i-th} basis function at the
  point \var{x}.
\item \describefun{double}{pnl_basis_i_vect}{const \PnlBasis \ptr b, const \PnlVect \ptr x, int i}
  \sshortdescribe If \var{b} is composed of $f_0, \dots, f_{\var{nb\_func}-1}$,
  then this function returns $f_i(x)$. 

\item \describefun{double}{pnl_basis_i_D_vect}{const \PnlBasis \ptr b, const \PnlVect \ptr x, int i, int j}
  \sshortdescribe If \var{b} is composed of $f_0, \dots, f_{\var{nb\_func}-1}$,
  then this function returns $\partial_{x_{\var{j}}} f_i(x)$.

  
\item \describefun{double}{pnl_basis_i_D2_vect}{const \PnlBasis \ptr b, const \PnlVect \ptr x, int i, int j1, int j2}
  \sshortdescribe If \var{b} is composed of $f_0, \dots, f_{\var{nb\_func}-1}$,
  then this function returns $\partial^2_{x_{\var{j1}}, x_{\var{j2}}}
  f_i(x)$.


\item \describefun{void}{pnl_basis_eval_derivs_vect}{const \PnlBasis \ptr b, const \PnlVect \ptr coef, const \PnlVect \ptr x, double \ptr fx, \PnlVect \ptr Dfx, \PnlMat \ptr D2fx}
  \sshortdescribe Compute the function, the gradient and the Hessian matrix
  of $\sum_{k=0}^n \var{coef}_k  P_k(\cdot)$ at the point \var{x}.
  On output, \var{fx} contains the value of the function, \var{Dfx} its
  gradient and \var{D2fx} its Hessian matrix. This function is optimized and
  performs much better than calling \reffun{pnl_basis_eval},
  \reffun{pnl_basis_eval_D} and \reffun{pnl_basis_eval_D2} sequentially.

\item \describefun{double}{pnl_basis_eval_vect}{const \PnlBasis \ptr basis, const \PnlVect \ptr coef, const \PnlVect \ptr x}
  \sshortdescribe Compute the linear combination of \var{P_k(x)} defined by
  \var{coef}. Given the coefficients computed by the function
  \reffun{pnl_basis_fit_ls}, this function returns $\sum_{k=0}^n
  \var{coef}_k  P_k(\var{x})$.

\item \describefun{double}{pnl_basis_eval_D_vect}{const \PnlBasis \ptr basis, const \PnlVect \ptr coef, const \PnlVect \ptr x, int i}
  \sshortdescribe Compute the derivative with respect to \var{x_i} of the
  linear combination of \var{P_k(x)} defined by \var{coef}. Given the
  coefficients computed by the function \reffun{pnl_basis_fit_ls}, this function
  returns $\partial_{x_i} \sum_{k=0}^n \var{coef}_k  P_k(\var{x})$ The index
  \var{i} may vary between \var{0} and \var{P->nb_variates - 1}.


\item \describefun{double}{pnl_basis_eval_D2_vect}{const \PnlBasis \ptr basis, const \PnlVect \ptr coef, const \PnlVect \ptr x, int i, int j}
  \sshortdescribe Compute the derivative with respect to \var{x_i} of the
  linear combination of \var{P_k(x)} defined by \var{coef}. Given the
  coefficients computed by the function \reffun{pnl_basis_fit_ls}, this function
  returns $\partial_{x_i} \partial_{x_j} \sum_{k=0}^n \var{coef}_k
  P_k(\var{x})$.  The indices \var{i} and \var{j} may vary between \var{0} and
  \var{P->nb_variates - 1}.

\end{itemize}
The following functions are provided for compatibility purposes but are marked as
deprecated. Use the functions with the \verb!_vect! extension.
\begin{itemize}
\item \describefun{double}{pnl_basis_ik}{const \PnlBasis \ptr b, const double \ptr x, int i, int k}
  \sshortdescribe Same as function \reffun{pnl_basis_ik_vect} but takes a
  C array as the point of evaluation.
\item  \describefun{double}{pnl_basis_i}{\PnlBasis \ptr b, double \ptr x, int i}
  \sshortdescribe Same as function \reffun{pnl_basis_i_vect} but takes a
  C array as the point of evaluation.
\item \describefun{double}{pnl_basis_i_D}{ const \PnlBasis \ptr b, const double \ptr x, int i, int j }
  \sshortdescribe Same as function \reffun{pnl_basis_i_D_vect} but takes a
  C array as the point of evaluation.
\item \describefun{double}{pnl_basis_i_D2}{const \PnlBasis \ptr b, const double \ptr x, int i, int j1, int j2}
  \sshortdescribe Same as function \reffun{pnl_basis_i_D2_vect} but takes a
  C array as the point of evaluation.
\item \describefun{double}{pnl_basis_eval}{\PnlBasis \ptr P, \PnlVect\ptr  coef, double \ptr x}
  \sshortdescribe Same as function \reffun{pnl_basis_eval_vect} but takes a
  C array as the point of evaluation.
\item \describefun{double}{pnl_basis_eval_D}{\PnlBasis \ptr P, \PnlVect \ptr  coef, double \ptr x, int i}
  \sshortdescribe Same as function \reffun{pnl_basis_eval_D_vect} but takes a
  C array as the point of evaluation.
\item \describefun{double}{pnl_basis_eval_D2}{\PnlBasis \ptr  P, \PnlVect \ptr  coef, double \ptr x,  int i, int j}
  \sshortdescribe Same as function \reffun{pnl_basis_eval_D2_vect} but takes a 
  C array as the point of evaluation.
\item \describefun{void}{pnl_basis_eval_derivs}{\PnlBasis \ptr P, \PnlVect\ptr coef, double \ptr x, double \ptr fx, \PnlVect \ptr Dfx, \PnlMat \ptr D2fx}
  \sshortdescribe Same as function \reffun{pnl_basis_eval_derivs_vect} but takes a
  C array as the point of evaluation.
\end{itemize}


% vim:spelllang=en:spell:


\section{Numerical integration}
\subsection{Overview}

To use these functionalities, you should include \verb!pnl/pnl_integration.h!.

Numerical integration methods are designed to numerically evaluate the integral
over a finite or non finite interval (resp. over a square) of real valued
functions defined on $\R$ (resp. on $\R^2$).

\begin{lstlisting}
typedef struct {
  double (*function) (double x, void *params);
  void *params;
} PnlFunc;

typedef struct {
  double (*function) (double x, double y, void *params);
  void *params;
} PnlFunc2D;
\end{lstlisting}

We provide the following two macros to evaluate a \refstruct{PnlFunc} or
\refstruct{PnlFunc2D} at a given point
\begin{lstlisting}
#define PNL_EVAL_FUNC(F, x) (*((F)->function))(x, (F)->params)
#define PNL_EVAL_FUNC2D(F, x, y) (*((F)->function))(x, y, (F)->params)
\end{lstlisting}



\subsection{Functions}

\begin{itemize}
\item \describefun{double}{pnl_integration}{\refstruct{PnlFunc} \ptr F,
    double x0, double x1, int n, char \ptr meth}
  \sshortdescribe Evaluate $\int_{x_0}^{x_1} F$ using \var{n} discretization
  steps. The method used to discretize the integral is defined by \var{meth}
  which can be \var{"rect"} (rectangle rule), \var{"trap"} (trapezoidal rule),
  \var{"simpson"} (Simpson's rule).

\item \describefun{double}{pnl_integration_2d}{\refstruct{PnlFunc2D} \ptr F,
    double x0, double x1, double y0, double y1, int nx, int ny, char \ptr meth}
  \sshortdescribe Evaluate $\int_{[x_0, x_1] \times [y_0, y_1]} F$ using
  \var{nx} (resp. \var{ny}) discretization steps for \var{[x0, x1]}
  (resp. \var{[y0, y1]}). The method used to discretize the integral is
  defined by \var{meth} which can be \var{"rect"} (rectangle rule),
  \var{"trap"} (trapezoidal rule),   \var{"simpson"} (Simpson's rule).


\item \describefun{int}{pnl_integration_qng}{\refstruct{PnlFunc} \ptr F,
    double x0, double x1, double epsabs, double epsrel, double \ptr result,
    double \ptr abserr,  int \ptr neval}
  \sshortdescribe Evaluate $\int_{x_0}^{x_1} F$ with an absolute error less
  than \var{espabs} and a relative error less than \var{esprel}. The value of
  the integral is stored in \var{result}, while the variables \var{abserr} and
  \var{neval} respectively contain the absolute error and the number of function
  evaluations. This function returns \var{OK} if the required accuracy has been
  reached and \var{FAIL} otherwise. This function uses a non-adaptive Gauss
  Konrod procedure (qng routine from {\it QuadPack}).

\item \describefun{int}{pnl_integration_GK}{\refstruct{PnlFunc} \ptr F,
    double x0, double x1, double epsabs, double epsrel, double \ptr result,
    double \ptr abserr,  int \ptr neval}
  \sshortdescribe This function is a synonymous of
  \reffun{pnl_integration_qng} and is only available for backward
  compatibility. It is deprecated, please use \reffun{pnl_integration_qng}
  instead.

\item \describefun{int}{pnl_integration_qng_2d}{\refstruct{PnlFunc2D} \ptr F,
    double x0, double x1, double y0, double y1, double epsabs, double epsrel,
    double \ptr result, double \ptr abserr, int \ptr neval}
  \sshortdescribe Evaluate $\int_{[x_0, x_1] \times [y_0, y_1]} F$ with an
  absolute error less than \var{espabs} and a relative error less than
  \var{esprel}. The value of the integral is stored in \var{result}, while the
  variables \var{abserr} and \var{neval} respectively contain the absolute error
  and the number of function evaluations. This function returns \var{OK} if the
  required accuracy has been reached and \var{FAIL} otherwise.

\item \describefun{int}{pnl_integration_GK2D}{\refstruct{PnlFunc} \ptr F,
    double x0, double x1, double epsabs, double epsrel, double \ptr result,
    double \ptr abserr,  int \ptr neval}
  \sshortdescribe This function is a synonymous of
  \reffun{pnl_integration_qng_2d} and is only available for backward
  compatibility. It is deprecated, please use \reffun{pnl_integration_qng_2d}
  instead.

\item \describefun{int}{pnl_integration_qag}{\refstruct{PnlFunc} \ptr F,
    double x0, double x1, double epsabs, int limit, double epsrel, double \ptr
    result, double \ptr abserr,  int \ptr neval}
  \sshortdescribe Evaluate $\int_{x_0}^{x_1} F$ with an absolute error less
  than \var{espabs} and a relative error less than \var{esprel}. \var{x0} and
  \var{x1} can be non finite (i.e. \var{PNL_NEGINF} or \var{PNL_POSINF}). The
  value of the integral is stored in \var{result}, while the variables
  \var{abserr} and \var{neval} respectively contain the absolute error and the
  number of iterations. \var{limit} is the maximum number of subdivisions of the
  interval \var{(x0,x1)} used during the integration. If on input, \var{limit
    0}, then 750 subdivisions are used.  This function returns \var{OK} if the
  required accuracy has been reached and \var{FAIL} otherwise. This function
  uses some adaptive procedures (qags and qagi routines from {\it QuadPack}).
  This function is able to handle functions \var{F} with integrable
  singularities on the interval \var{[x0,x1]}.

\item \describefun{int}{pnl_integration_qagp}{\refstruct{PnlFunc} \ptr F,
    double x0, double x1, const PnlVect \ptr singularities, double epsabs,
    int limit, double epsrel, double \ptr result, double \ptr abserr,  int \ptr neval}
  \sshortdescribe Evaluate $\int_{x_0}^{x_1} F$  for a function
  \var{F} with known singularities listed in \var{singularities}.
  \var{singularities} must be a sorted vector which does not contain \var{x0}
  nor \var{x1}.  \var{x0} and \var{x1} must be  finite. The value of the
  integral is stored in \var{result}, while the variables \var{abserr} and
  \var{neval} respectively contain the absolute error and the number of
  iterations. \var{limit} is the maximum number of subdivisions of the interval
  \var{(x0,x1)} used during the integration. If on input, \var{limit = 0}, then
  750 subdivisions are used.  This function returns \var{OK} if the required
  accuracy has been reached and \var{FAIL} otherwise. This function uses some
  adaptive procedures (qagp routine from {\it QuadPack}).  This function is
  able to handle functions \var{F} with integrable singularities on the interval
  \var{[x0,x1]}.
\end{itemize}

% vim:spelllang=en:spell:


\input{fft.tex}

\section{Inverse Laplace Transform}

For a real valued function $f$ such that $t \longmapsto f(t) \expp{- \sigma_c
  t}$ is integrable over $\R^+$, we can define its Laplace transform
\begin{equation*}
  \hat{f}(\lambda) = \int_0^\infty f(t) \expp{- \lambda t} dt \qquad
  \mbox{for $\lambda \in \C$ with $\real{\lambda} \ge \sigma_c$}.
\end{equation*}

To use the following functions, you should include \verb!pnl/pnl_laplace.h!.

\describestruct{PnlCmplxFunc}
\begin{lstlisting}
typedef struct
{
  dcomplex (*F) (dcomplex x, void *params);
  void *params;
} PnlCmplxFunc;
 \end{lstlisting}

\begin{itemize}
\item \describefun{double}{pnl_ilap_euler}{\PnlCmplxFunc
    \ptr fhat, double t, int N, int M}
  \sshortdescribe Compute $f(\var{t})$ where $f$ is given by its Laplace
  transform \var{fhat} by numerically inverting the Laplace transform using
  Euler's summation. The values \var{N = M = 15} usually give a very good
  accuracy. For more details on the accuracy of the method.

\item \describefun{double}{pnl_ilap_cdf_euler}{\PnlCmplxFunc
    \ptr fhat, double t, double h, int N, int M}
  \sshortdescribe Compute the cumulative distribution function $F(\var{t})$
  where $F(x) = \int_0^x f(t) dt$ and $f$ is a density function with values on
  the positive real linegiven by its Laplace transform \var{fhat}. The
  computation is carried out by numerical inversion of the Laplace transform
  using Euler's summation. The values \var{N = M = 15} usually give a very
  good accuracy. The parameter \var{h} is the discretization step, the
  algorithm is very sensitive to the choice of \var{h}.

\item \describefun{double}{pnl_ilap_fft}{\PnlVect \ptr res,
    \PnlCmplxFunc \ptr fhat, double T, double eps}
  \sshortdescribe Compute $f(t)$ for $t \in [h, \var{T}]$ on a regular grid
  and stores the values in \var{res}, where $h = T / {\mathrm size}(res)$. The
  function $f$ is defined by its Laplace transform \var{fhat}, which is
  numerically inverted using a Fast Fourier Transform algorithm. The size of
  \var{res} is related to the choice of the relative precision \var{eps}
  required on the value of $f(t)$ for all $t \le T$.

\item \describefun{double}{pnl_ilap_gs}{\refstruct{PnlFunc} \ptr fhat, double
    t, int n}
  \sshortdescribe Compute $f(\var{t})$ where $f$ is given by its Laplace
  transform \var{fhat} by numerically inverting the Laplace transform using a
  weighted combination of different Gaver Stehfest's algorithms. Note that
  this function does not need the complex valued Laplace transform but only the
  real valued one. \var{n} is the number of terms used in the weighted combination.

\item \describefun{double}{pnl_ilap_gs_basic}{\refstruct{PnlFunc}
    \ptr fhat, double t, int n}
  \sshortdescribe Compute $f(\var{t})$ where $f$ is given by its Laplace
  transform \var{fhat} by numerically inverting the Laplace transform using
  Gaver Stehfest's method. Note that this function does not
  need the complex valued Laplace transform but only the real valued
  one. \var{n} is the number of iterations of the algorithm.
  {\bf Note : }~This function is provided for test purposes only. The 
  function \reffun{pnl_ilap_gs} gives far more accurate results.
\end{itemize}

% vim:spelllang=en:spell:

\section{Ordinary differential equations}
\subsection{Overview}

To use these functionalities, you should include \verb!pnl/pnl_integration.h!.

These functions are designed for numerically solving $n-$dimensional first order
ordinary differential equation of the general form
\begin{equation*}
  \frac{dy_i}{dt}(t) = f_i(t, y_1(t), \cdots, y_n(t)) 
\end{equation*}
The system of equations is defined by the following structure
\describestruct{PnlODEFunc}
\begin{lstlisting}
typedef struct
{
  void (*F) (int neqn, double t, const double *y, double *yp, void *params);
  int neqn; 
  void *params;
} PnlODEFunc ;
\end{lstlisting}

\begin{itemize}
\item \describevar{int}{neqn} 
  \sshortdescribe Number of equations
\item \describevar{void \ptr}{params} 
  \sshortdescribe An untyped structure used to pass extra
  arguments to the function \var{f} defining the system
\item \describefun*{void}{(\ptr\ F)}{int neqn, double t, const double \ptr
    y, double \ptr yp, void \ptr params}
  \sshortdescribe After calling the fuction, \var{yp} should be defined as
  follows \var{yp_i = f_i(neqn, t, y, params)}. \var{y} and \var{yp} should be
  both of size \var{neqn}
\end{itemize}
We provide the following macro to evaluate a \refstruct{PnlODEFunc}
at a given point
\begin{lstlisting}
#define PNL_EVAL_ODEFUNC(Fstruct, t, y, yp) \
    (*((Fstruct)->F))((Fstruct)->neqn, t, y, yp, (Fstruct)->params)
\end{lstlisting}

\subsection{Functions}

\begin{itemize}
\item \describefun{int}{pnl_ode_rkf45}{\refstruct{PnlODEFunc} \ptr f, double
    \ptr y, double t, double t_out, double relerr, double abserr, int \ptr flag}
  \sshortdescribe This function computes the solution of the system defined by
  the \refstruct{PnlODEFunc} \var{f} at the point \var{t_out}. On input,
  $\var{(t,y)}$ should be the initial condition, \var{abserr,relerr} are the
  maximum absolute and relative errors for local error tests (at each step,
    \var{abs(local error)} should be less that \var{relerr * abs(y) + abserr}).
  Note that if \var{abserr = 0} or \var{relerr = 0}  on input, an optimal value
  for these variables is computed inside the function The function returns an
  error \var{OK} or \var{FAIL}. In case of an \var{OK} code, the \var{y}
  contains the solution computed at \var{t_out}, in case of a \var{FAIL} code,
  \var{flag} should be examined to determine the reason of the error. Here are
  the different possible values for \var{flag}
  \begin{itemize}
  \item \var{flag = 2} : integration reached \var{t_out}, it indicates
    successful return and is the normal mode for continuing integration.
 \item \var{flag = 3} : integration was not completed because relative error
   tolerance was too small. relerr has been increased appropriately for
   continuing.
 \item \var{flag = 4} : integration was not completed because more than 3000
   derivative evaluations were needed. this is approximately 500 steps.
 \item \var{flag = 5} : integration was not completed because solution vanished
   making a pure relative error test impossible. must use non-zero abserr to
   continue.  using the one-step integration mode for one step is a good way to
   proceed.
 \item \var{flag = 6} : integration was not completed because requested accuracy
   could not be achieved using smallest allowable stepsize. user must increase
   the error tolerance before continued integration can be attempted.
 \item \var{flag = 7} : it is likely that rkf45 is inefficient for solving this
   problem. too much output is restricting the natural stepsize choice. use the
   one-step integrator mode. see \reffun{pnl_ode_rkf45_step}.
 \item \var{flag = 8} : invalid input parameters this indicator occurs if any of
   the following is satisfied -   neqn <= 0, t=tout,  relerr or abserr <= 0.
  \end{itemize}
\item \describefun{int}{pnl_ode_rkf45_step}{\refstruct{PnlODEFunc} \ptr f,
    double \ptr y, double \ptr t, double t_out, double \ptr relerr, double
    abserr, double \ptr work, int \ptr iwork, int \ptr flag} 
  \sshortdescribe Same as \reffun{pnl_ode_rkf45} but it only computes one step
  of integration in the direction of \var{t_out}. \var{work} and \var{iwork} are
  working arrays of size \var{3 + 6 * neqn} and \var{5} respectively and should
  remain untouched between successive calls to the function. 
  On output \var{t} holds the point at which integration stopped and \var{y} the
  value of the solution at that point.
\end{itemize}

% vim:spelllang=en:spell:


\input{optim.tex}
\section{Root finding}
\subsection{Overview}
\label{sec:PnlFunc}

To provide a uniformed framework to root finding functions, we use several
structures for storing different kind of functions. The pointer
\var{params} is used to store the extra parameters. These new types come
with dedicated macros starting in \verb!PNL_EVAL!  to evaluate the function
and their Jacobian.
\describestruct{PnlFunc}
\begin{lstlisting}
/*
 * f: R --> R
 * The function  pointer returns f(x)
 *
typedef struct {
  double (*F) (double x, void *params);
  void *params;
} PnlFunc ;
#define PNL_EVAL_FUNC(Fstruct, x) (*((Fstruct)->F))(x, (Fstruct)->params)
\end{lstlisting}

\describestruct{PnlFunc2D}
\begin{lstlisting}
/*
 * f: R^2 --> R
 * The function pointer returns f(x)
 *
typedef struct {
  double (*F) (double x, double y, void *params);
  void *params;
} PnlFunc2D ;
#define PNL_EVAL_FUNC2D(Fstruct, x, y) (*((Fstruct)->F))(x, y, (Fstruct)->params)
\end{lstlisting}

\describestruct{PnlFuncDFunc}
\begin{lstlisting}
/*
 * f: R --> R
 * The function pointer computes f(x) and Df(x) and stores them in fx
 * and dfx respectively
 *
typedef struct {
  void (*F) (double x, double *fx, double *dfx, void *params);
  void *params;
} PnlFuncDFunc ;
#define PNL_EVAL_FUNC_FDF(Fstruct, x, fx, dfx) (*((Fstruct)->F))(x, fx, dfx, (Fstruct)->params)
\end{lstlisting}

\describestruct{PnlRnFuncR}
\begin{lstlisting}
/*
 * f: R^n --> R
 * The function pointer returns f(x)
 *
typedef struct {
  double (*F) (const PnlVect *x, void *params);
  void *params;
} PnlRnFuncR ;
#define PNL_EVAL_RNFUNCR(Fstruct, x) (*((Fstruct)->F))(x, (Fstruct)->params)
\end{lstlisting}

\describestruct{PnlRnFuncRm}
\describestruct{PnlRnFuncRn}
\begin{lstlisting}
/*
 * f: R^n --> R^m
 * The function pointer computes the vector f(x) and stores it in
 * fx (vector of size m)
 *
typedef struct {
  void (*F) (const PnlVect *x, PnlVect *fx, void *params);
  void *params;
} PnlRnFuncRm ;
#define PNL_EVAL_RNFUNCRM(Fstruct, x, fx) (*((Fstruct)->F))(x, fx, (Fstruct)->params)

/*
 * Synonymous of PnlRnFuncRm for f:R^n --> R^n
 *
typedef PnlRnFuncRm PnlRnFuncRn;
#define PNL_EVAL_RNFUNCRN  PNL_EVAL_RNFUNCRM
\end{lstlisting}

\describestruct{PnlRnFuncRmDFunc}
\describestruct{PnlRnFuncRnDFunc}
\begin{lstlisting}
/*
 * f: R^n --> R^m
 * The function pointer computes the vector f(x) and stores it in fx
 * (vector of size m)
 * The Dfunction pointer computes the matrix Df(x) and stores it in dfx
 * (matrix of size m x n)
 *
typedef struct {
  void (*F) (const PnlVect *x, PnlVect *fx, void *params);
  void (*DF) (const PnlVect *x, PnlMat *dfx, void *params);
  void (*FDF) (const PnlVect *x, PnlVect *fx, PnlMat *dfx, void *params);
  void *params;
} PnlRnFuncRmDFunc ;
#define PNL_EVAL_RNFUNCRM_DF(Fstruct, x, dfx) \
    (*((Fstruct)->Dfunction))(x, dfx, (Fstruct)->params)
#define PNL_EVAL_RNFUNCRM_FDF(Fstruct, x, fx, dfx) \
    (*((Fstruct)->F))(x, fx, dfx, (Fstruct)->params)
#define PNL_EVAL_RNFUNCRM_F_DF(Fstruct, x, fx, dfx)    \
      if ( (Fstruct)->FDF != NULL )                    \
        {                                              \
          PNL_EVAL_RNFUNCRN_FDF (Fstruct, x, fx, dfx); \
        }                                              \
      else                                             \
        {                                              \
          PNL_EVAL_RNFUNCRN (Fstruct, x, fx);          \
          PNL_EVAL_RNFUNCRN_DF (Fstruct, x, dfx);      \
        }
/*
 * Synonymous of PnlRnFuncRmDFunc for f:R^n --> R^m
 *
typedef PnlRnFuncRmDFunc PnlRnFuncRnDFunc;
#define PNL_EVAL_RNFUNCRN_DF PNL_EVAL_RNFUNCRM_DF
#define PNL_EVAL_RNFUNCRN_FDF PNL_EVAL_RNFUNCRM_FDF
#define PNL_EVAL_RNFUNCRN_F_DF PNL_EVAL_RNFUNCRM_F_DF
\end{lstlisting}

\subsection{Functions}

To use the following functions, you should include \verb!pnl/pnl_root.h!.

\paragraph{Real valued functions of a real argument}
\begin{itemize}
\item \describefun{double}{pnl_root_brent}{\refstruct{PnlFunc} \ptr F, double
    x1, double  x2, double \ptr tol}
  \sshortdescribe Find the root of \var{F} between \var{x1} and \var{x2} with
  an accuracy of order \var{tol}. On exit \var{tol} is an upper bound of the
  error.

\item \describefun{int}{pnl_root_newton_bisection}{\refstruct{PnlFuncDFunc} \ptr  Func,
    double x_min, double x_max, double tol, int N_Max, double \ptr res}
  \sshortdescribe Find the root of \var{F} between \var{x1} and \var{x2} with
  an accuracy of order \var{tol} and a maximum of \var{N_max} iterations. On
  exit, the root is stored in \var{res}. Note that the function \var{Func} must
  also compute the first derivative of the function. This function relies on
  combining Newton's approach with a bisection technique.


\item \describefun{int}{pnl_root_newton}{\refstruct{PnlFuncDFunc} \ptr Func,
  double x0, double x_eps, double fx_eps, int max_iter, double \ptr res}
  \sshortdescribe Find the root of \var{f} starting from \var{x0} using Newton's
  method with descent direction given by the inverse of the derivative, ie.
  $d_k = f(x_k) / f'(x_k)$. Armijo's line search is used to make sure \var{|f|}
  decreases along the iterations. $\alpha_k = \max\{ \gamma^j \; ; \; j \ge 0\}$
  such that
  \begin{align*}
    |f(x_k + \alpha_k d_k)| & \le |f(x_k)| (1 - \omega \alpha_k).
  \end{align*}
  In this implementation, $\omega = 10^{-4}$ and $\gamma = 1/2$.
  The algorithm stops when one of the three following
  conditions is met:
  \begin{itemize}
    \item the maximum number of iterations \var{max_iter} is reached;
    \item the last improvement over \var{x} is smaller that \var{x * x_eps};
    \item at the current position \var{|f(x)| < fx_eps}
  \end{itemize}
  On exit, the root is stored in \var{res}. 
\item \describefun{int}{pnl_root_bisection}{\refstruct{PnlFunc} \ptr Func,
    double xmin, double xmax, double epsrel, double espabs, int N_max, double
    \ptr res}
  \sshortdescribe Find the root of \var{F} between \var{x1} and \var{x2} with
  the accuracy \var{|x2 - x1| < epsrel * x1 + epsabs} or with the maximum number
  of iterations \var{N_max}. On exit, \var{res = (x2 + x1) / 2}.
\end{itemize}

\paragraph{Vector valued functions with several arguments}

\begin{itemize}
\item \describefun{int}{pnl_multiroot_newton}{\refstruct{PnlRnFuncRnDFunc} \ptr
  func, const \PnlVect \ptr x0, double x_eps, double fx_eps, int
  max_iter, int verbose, \PnlVect \ptr res}
  \sshortdescribe Find the root of \var{func} starting from \var{x0} using
  Newton's method with descent direction given by the inverse of the Jacobian
  matrix, ie. $d_k = (\nabla f(x_k))^{-1} f(x_k)$. Armijo's line search is used to make sure \var{|f|}
  decreases along the iterations. $\alpha_k = \max\{ \gamma^j \; ; \; j \ge 0\}$
  such that
  \begin{align*}
    |f(x_k + \alpha_k d_k)| & \le |f(x_k)| (1 - \omega \alpha_k).
  \end{align*}
  In this implementation, $\omega = 10^{-4}$ and $\gamma = 1/2$.  The algorithm
  stops when one of the three following conditions is met:
  \begin{itemize}
    \item the maximum number of iterations \var{max_iter} is reached;
    \item the norm of the last improvement over \var{x} is smaller that \var{|x| * x_eps};
    \item at the current position \var{|f(x)| < fx_eps}
  \end{itemize}
  On exit, the root is stored in \var{res}.  Note that the function \var{F} must
  also compute the first derivative of the function. When defining \var{Func},
  you must either define \var{Func->F} and \var{Func->DF} or \var{Func->FDF}.
\end{itemize}

We provide two wrappers for calling minpack routines.
\begin{itemize}
\item \describefun
  {int}{pnl_root_fsolve}{\refstruct{PnlRnFuncRnDFunc} \ptr f,
    \PnlVect \ptr x, \PnlVect \ptr fx, double xtol,
    int maxfev, int \ptr nfev, \PnlVect \ptr scale, int
    error_msg}
  \sshortdescribe Compute the root of a function $f:\R^n \longmapsto
  \R^n$. Note that the number of components of \var{f} must be equal to the
  number of variates of \var{f}. This function returns \var{OK} or
  \var{FAIL} if something went wrong.
  \parameters
  \begin{itemize}
  \item \var{f} is a pointer to a \refstruct{PnlRnFuncRnDFunc} used to
    store the function whose root is to be found. \var{f} can also
    store the Jacobian of the function, if not it is computed using
    finite differences (see the file \url{examples/minpack_test.c} for
    a usage example). \var{f->FDF} can be NULL because it is
    not used in this function.
  \item  \var{x} contains on input the starting point of the search and
    an approximation of the root of \var{f} on output,
  \item \var{xtol} is the precision required on \var{x}, if set to 0 a
    default value is used.
  \item \var{maxfev} is the maximum number of evaluations of the function
    \var{f} before the algorithm returns, if set to 0, a coherent
    number is determined internally.
  \item \var{nfev} contains on output the number of evaluations of
    \var{f} during the algorithm,
  \item \var{scale} is a vector used to rescale \var{x} in a way that
    each coordinate of the solution is approximately of order 1 after
    rescaling. If on input \var{scale=NULL}, a scaling vector is
    computed internally by the algorithm.
  \item \var{error_msg} is a boolean
    (\var{TRUE} or \var{FALSE}) to specify if an error message should be
    printed when the algorithm stops before having converged.
  \item On output, \var{fx} contains \var{f(x)}.
  \end{itemize}

\item \describefun {int}{pnl_root_fsolve_lsq}{\refstruct{PnlRnFuncRmDFunc}
    \ptr f, \PnlVect \ptr x, int m, \PnlVect \ptr fx,
    double xtol, double ftol, double gtol, int maxfev, int \ptr nfev,
    \PnlVect \ptr scale, int error_msg}
  \sshortdescribe Compute the root of $x \in \R^n \longmapsto
  \sum_{i=1}^m f_i(x)^2$, note that there is no reason why \var{m} should
  be equal to \var{n}.
  \parameters
  \begin{itemize}
  \item \var{f} is a pointer to a \refstruct{PnlRnFuncRmDFunc} used to
    store the function whose root is to be found. \var{f} can also
    store the Jacobian of the function, if not it is computed using
    finite differences (see the file \url{examples/minpack_test.c} for
    a usage example). \var{f->FDF} can be NULL because it is
    not used in this function.
  \item  \var{x} contains on input the starting
    point of the search and an approximation of the root of \var{f} on
    output,
  \item \var{m} is the number of components of \var{f},
  \item \var{xtol} is the precision required on \var{x}, if set to 0 a
    default value is used.
  \item \var{ftol} is the precision required on \var{f}, if set to 0 a
    default value is used.
  \item \var{gtol} is the precision required on the Jacobian of
    \var{f}, if set to 0 a default value is used.
  \item \var{maxfev} is the maximum number of evaluations of the function
    \var{f} before the algorithm returns, if set to 0, a coherent
    number is determined internally.
  \item \var{nfev} contains on output the number of evaluations of
    \var{f} during the algorithm,
  \item \var{scale} is a vector used to rescale \var{x} in a way that
    each coordinate of the solution is approximately of order 1 after
    rescaling.  If on input \var{scale=NULL}, a scaling vector is
    computed internally by the algorithm.
  \item \var{error_msg} is a boolean (\var{TRUE} or \var{FALSE}) to
    specify if an error message should be printed when the algorithm
    stops before having converged.
  \item On output, \var{fx} contains \var{f(x)}.
  \end{itemize}
\end{itemize}

% vim:spelllang=en:spell:


\input{specfun.tex}

\input{some_bindings.tex}

\input{finance.tex}

\clearpage
\addcontentsline{toc}{section}{Index}
\printindex


\end{document}
