
\documentclass[a4paper,11pt,twoside]{article}
\usepackage{a4wide}
\usepackage{t1enc,lmodern}
\usepackage[dvips=true]{hyperref}
\usepackage{textcomp}
\usepackage{verbatim,moreverb,alltt}
\usepackage{makeidx}
\usepackage[dvips]{color}
\usepackage[latin1]{inputenc}
\usepackage[english]{babel}
\usepackage[strings]{underscore}
\usepackage{amsmath,amsfonts}



%% --------------------------------------------
\hbadness=10000
\emergencystretch=\hsize
\tolerance=9999
\parindent=0pt
%% --------------------------------------------

\newcommand{\R}{{\mathbb R}}
\newcommand{\C}{{\mathbb C}}
\newcommand{\N}{{\mathbb N}}
\newcommand{\ptr}{\textasteriskcentered}

%%\renewcommand{\exp}[1]{\operatorname{e}^{ #1 } }
\newcommand{\cotan}{\operatorname{cotan}}
\newcommand{\cotanh}{\operatorname{cotanh}}
\newcommand{\expp}[1]{\operatorname{e}^{ #1 } }
\newcommand{\paren}[1]{\left( #1 \right)}
\newcommand{\recaco}[1]{\left [ #1 \right ]}
\newcommand{\crochet}[1]{\left \langle #1 \right \rangle}
\newcommand{\acolade}[1]{\left\{ #1 \right\}}
\newcommand{\real}[1]{\operatorname{Re}(#1)}

\makeatletter

% For dvips + ps2pdf generation
\ifx\HCode\undefined%

\def\var#1{{\tt #1}}

\newcommand{\describefun}[3]{%
  \index{#2}\label{#2} {#1~{\bf #2}~(#3)}}

\newcommand{\describemacro}[2]{%
  \index{#1}\label{#1} {{\bf #1}~(#2)}}

\newcommand{\constentry}[2]{%
  \index{#1}\label{#1}{\bf #1} &&  #2 \\ }

\def\refstruct#1{{\color[named]{Red} #1}}
\def\reffun#1{\hyperref[#1]{#1}}
\def\refmacro{\reffun}
\newenvironment{describeconst}{%
  \noindent\begin{tabular}{lp{1cm}l}}{\end{tabular}}
\def\shortdescribe{\unskip\vskip1ex{\color[named]{Blue} Description~}}
\def\sshortdescribe{\unskip\newline\hskip1em {\color[named]{Blue} Description~}}
\def\parameters{\unskip\newline\hskip1em {\color[named]{Blue} Parameters~}}
\def\example{\unskip\newline\hskip1em {\color[named]{Blue} Example~}}

% Html related stuff
\else%

\def\var#1{\HCode{<span class='var'>}#1\HCode{</span>}}

\newcommand{\describefun}[3]{%
\index{#2}\label{#2} {\HCode{<span class='ret'>}#1\HCode{</span>}~{\bf \HCode{<span class='fun'>}#2\HCode{</span>}}~(\HCode{<span class='args'>}#3\HCode{</span>})}}

\newcommand{\describemacro}[2]{%
  \index{#1}\label{#1} {{\bf #1}~(#2)}}

\newcommand{\constentry}[2]{%
  \index{#1}\label{#1}\HCode{<span class='struct'>}#1\HCode{</span>} && #2 \\ }

\def\refstruct#1{\HCode{<span class='struct'>}#1\HCode{</span>}}
\def\reffun#1{\hyperref[#1]{#1}}
\def\refmacro{\reffun}
\newenvironment{describeconst}{%
  \noindent\begin{tabular}{lp{1cm}l}}{\end{tabular}}
\def\shortdescribe{\unskip\vskip1ex{\HCode{<span class='description'>}Description~\HCode{</span>}}}
\def\sshortdescribe{\unskip\newline\hskip1em \HCode{<span class='description'>}Description~\HCode{</span>}}

\fi
\makeatother


\title{Pnl Manual}
\date{\today}
\author{}

\makeindex
\begin{document}
\maketitle
\tableofcontents

\section{What is Pnl}

Pnl is a scientific library written in C and distributed under
the Gnu Lesser General Public License (LGPL). This manual is divided into four
parts.
\begin{itemize}
\item Mathematical functions: complex numbers, special functions, standard
  financial functions for the Black \& Scholes model.
\item Linear algebra : vectors, matrices, hypermatrices, tridiagonal matrices,
  band matrices and the corresponding routines to manipulate them and solve linear systems.
\item Probabilistic functions: random number generators and  cumulative
  distribution functions.
\item Deterministic toolbox : FFT, Laplace inversion, numerical integration, zero searching,
  multivariate polynomial regression, $\dots$
\end{itemize}

\section{A few helpful conventions}

\begin{itemize}
  \item All header file names are prefixed by \verb!pnl_! and are surrounded by
    the preprocessor conditionals
\begin{verbatim}
#ifndef _PNL_MATRIX_H
#define _PNL_MATRIX_H

...

#endif /* _PNL_MATRIX_H
\end{verbatim}
All the header files are protected by an \verb!extern "C"! declaration for
possible use with a C++ compiler.

  \item All function names are prefixed by \verb!pnl_! except those implementing
    complex number arithmetic which are named following the \textit{C99}
    complex library but using a capitalised first letter \verb!C!. \\
    For example, the addition of two complex numbers is performed by the
    function \verb!Cadd!.

  \item Function containing \verb!_create! in their names always return a
    pointer to an object created by one or several calls to dynamic
    allocation. Once these objects are not used, they must be freed by calling
    the same function but ending in \verb!_free!.
    A function \verb!pnl_foo_create_yyy! returns a \verb!PnlFoo *! object (note
    the ``\ptr'') and a function \verb!pnl_foo_bar_create_yyy! returns a
    \verb!PnlFooBar *! object (note the ``\ptr'').
    These objects must be freed by calling respectively \verb!pnl_foo_free! or
    \verb!pnl_foo_bar_free!.
    
  \item Every object must implement a \verb!pnl_xxx_new! function which returns
    a pointer to an empty object with all its elements properly set to $0$. This
    means that the objects returned by the \verb!pnl_xxx_new! functions can be
    used as output arguments for functions ending in \verb!_inplace! for
    instance. They are suitable for being resized. 

  \item Functions containing \verb!_wrap_! in their names always return an
    object, not a pointer to an object, and do not make any use of dynamic
    allocation. The returned object must not be freed. 
    For instance, a function \verb!pnl_foo_wrap_xxx! returns an object
    \verb!PnlFoo! and a function \verb!pnl_foo_bar_wrap_xxx! returns an object
    \verb!PnlFooBar! 
    \begin{verbatim}
    PnlVectComplex *v1;
    PnlVectComplex v2;
    v1 = pnl_vect_complex_create_from_complex (5, Complex(0., 1.));
    v2 = pnl_vect_complex_wrap_subvect (v1, 1, 2);

    ...

    pnl_vect_complex_free (&v1);
    \end{verbatim}
    The vector \verb!v1! is of size 5 and contains the pure imaginary number
    $i$. The vector \verb!v2! only provides a view to \verb!v1(1:1+2)!, which
    means that modifying \verb!v2! will also modify \verb!v1! and vice-versa
    because \verb!v1! shares part of its data with \verb!v2!. Note that only
    \verb!v1! must be freed and {\bf not} \verb!v2!.
  
  \item Functions ending in \verb!_init! do not create any object but only
    perform some internal initialisation.
  
  \item Hypermatrices, matrices and vectors are stored using a flat block of
    memory obtained by concatenating the matrix rows and C-style
    pointer-to-pointer arrays. Matrices are stored in row-major order, which
    means that the column index moves continuously.
    Note that this convention is not \textit{Blas \& Lapack} compliant since
    Fortran expects 2-dimensional arrays to be stored in a column-major order.

  \item Type names always begin with \verb!Pnl!, they do not contain underscores
    but instead we use capital letters to separate units in type names. \\
    Examples : \verb!PnlMat!, \verb!PnlMatComplex!.

  \item Object and function names are intimately linked : an object
    \verb!PnlFoo! is manipulated by functions starting in \verb!pnl_foo!, an
    object \verb!PnlFooBar! is manipulated by functions starting in
    \verb!pnl_foo_bar!. In table~\ref{pnltypes}, we summarise the types and their
    corresponding prefixes.

    \begin{figure}[h!]
      \centering\begin{tabular}{|l|l|}
      \hline
      Pnl types & Pnl prefix \\
      \hline
      PnlVect & pnl_vect \\
      PnlVectComplex & pnl_vect_complex \\
      PnlVectInt & pnl_vect_int \\
       &\\
      PnlMat & pnl_mat \\
      PnlMatComplex & pnl_mat_complex \\
      PnlMatInt & pnl_mat_int \\
      & \\
      PnlHmat & pnl_hmat \\
      PnlHmatComplex & pnl_hmat_complex \\
      PnlHmatInt & pnl_hmat_int \\
      & \\
      PnlTridiagMat & pnl_tridiag_mat \\
      PnlBandMat & pnl_band_mat \\
      & \\
      PnlBasis & pnl_basis \\
      & \\
      PnlCgSolver & pnl_cg_solver \\
      PnlBicgSolver & pnl_bicg_solver \\
      PnlGmresSolver & pnl_gmres_solver \\
      \hline
    \end{tabular}
    \caption{Pnl types}
    \label{pnltypes}
  \end{figure}

  \item All macro names begin with \verb!PNL_! and are capitalised.

  \item Differences between \textbf{copy} and \textbf{clone} methods.
    The \verb!copy! methods take a single argument and return a pointer to an object
    of the same type which is an independent copy of its argument. 
    Example:
    \begin{verbatim}
    PnlVect *v1, *v2;
    v1 = pnl_vect_create_from_double (5, 2.5);
    v2 = pnl_vect_copy (v1);
    \end{verbatim}
    \verb!v1! and \verb!v2! are two vectors of size 5 with all their elements
    equal to 2.5. Note that \verb!v2! {\bf must not} have been created by a call
    to \verb!pnl_vect_create_xxx! because otherwise it will cause a memory leak.
    \verb!v1! and \verb!v2! are independent in the sense that a modification to
    one of them does not affect the other.

    The \verb!clone! methods take two arguments and fill in the first one with the
    second one. 
    Example:
    \begin{verbatim}
    PnlVect *v1, *v2;
    v1 = pnl_vect_create_from_double (5, 2.5);
    v2 = pnl_vect_new ();
    pnl_vect_clone (v2, v1);
    \end{verbatim}
    \verb!v1! and \verb!v2! are two vectors of size 5 with all their elements
    equal to 2.5. Note that \verb!v2! {\bf must} have been created by a call to
    \verb!pnl_vect_new! because otherwise the function
    \verb!pnl_vect_clone!  will crash.  \verb!v1! and \verb!v2! are independent
    in the sense that a modification to one of them does not modify the other.


  \item All objects are measured using integers \verb!int! and not
    \verb!size_t!. Hence, iterations over vectors, matrices, \dots should use an
    index of type \verb!int!.
\end{itemize}




\section{Mathematical framework}


\subsection{General tools}

The macros and functions of this paragraph are defined in \verb!pnl/pnl_mathtools.h!.

\subsubsection{Constants} A few mathematical constants are provided by the
library. Most of them are actually already defined in {\tt math.h}, {\tt
  values.h} or {\tt limits.h} and a few others have been added.
\begin{describeconst}
  \constentry{M_E}{$e^1$}
  \constentry{M_LOG2E}{$\log_2 e$}
  \constentry{M_LOG10E}{$\log_{10} e$}
  \constentry{M_LN2}{$\log_e 2$}
  \constentry{M_LN10}{$\log_e 10$}
  \constentry{M_PI}{$\pi$}
  \constentry{M_2PI}{$2 \pi$}
  \constentry{M_PI_2}{$\pi/2$}
  \constentry{M_PI_4}{$\pi/4$}
  \constentry{M_1_PI}{$1/\pi$}
  \constentry{M_2_PI}{$2/\pi$}
  \constentry{M_2_SQRTPI}{$2/\sqrt{\pi}$}
  \constentry{M_SQRT2PI}{$sqrt{2\pi}$}
  \constentry{M_SQRT2}{$\sqrt{2}$}
  \constentry{M_EULER}{$\gamma = \lim_{n \rightarrow \infty } \left( \sum_{k=1}^{n} \frac {1}{k} - \ln(n) \right)$}
  \constentry{M_SQRT1_2}{$1/\sqrt{2}$}
  \constentry{M_1_SQRT2PI}{$1/\sqrt{2 \pi}$}
  \constentry{M_SQRT2_PI}{$\sqrt{2/\pi}$}
  \constentry{INT_MAX}{$2147483647$}
  \constentry{MAX_INT}{INT_MAX}
  \constentry{DBL_MAX}{$1.79769313486231470e+308$}
  \constentry{DOUBLE_MAX}{DBL_MAX}
  \constentry{DBL_EPSILON}{$2.2204460492503131e-16$}
  \constentry{PNL_NEGINF}{$-\infty$}
  \constentry{PNL_POSINF}{$+\infty$}
  \constentry{PNL_INF}{$+\infty$}
  \constentry{NAN}{Not a Number}
\end{describeconst}

\subsubsection{A few macros}
\begin{itemize}
\item \describemacro{PNL_IS_ODD}{int n}
  \sshortdescribe Returns $1$ if \var{n} is odd and $0$ otherwise.
\item \describemacro{PNL_IS_EVEN}{int n}
  \sshortdescribe Returns $1$ if \var{n} is even and $0$ otherwise.
\item \describemacro{PNL_ALTERNATE}{int n}
  \sshortdescribe Returns $(-1)^{\var{n}}$.
\item \describemacro{MIN}{x,y}
  \sshortdescribe Returns the minimum of \var{x} and \var{y}.
\item \describemacro{MAX}{x,y}
  \sshortdescribe Returns the maximum of \var{x} and \var{y}.
\item \describemacro{ABS}{x}
  \sshortdescribe Returns the absolute value of \var{x}.
\item \describemacro{PNL_SIGN}{x}
  \sshortdescribe Returns the sign of \var{x} (-1 if x < 0, 0 otheriwse).
\item \describemacro{SQR}{x}
  \sshortdescribe Returns \var{$x^2$}.
\item \describemacro{CUB}{x}
  \sshortdescribe Returns \var{$x^3$}.
\end{itemize}

\subsubsection{Functions}
\begin{itemize}
\item \describefun{int}{intapprox}{double s}
  \sshortdescribe Returns the nearest integer with the convention ({\tt
  intapprox(1.5)=1}). This function is similar to the \var{round} function
  (provided by the C library) but the result is typed as an integer instead of a
  double.

\item \describefun{double}{trunc}{double s}
  \sshortdescribe Returns the nearest integer not greater than the absolute
  value of \var{s}. This function is part of C99.

\item \describefun{double}{Cnp}{int n, int p}
  \sshortdescribe Computes the binomial coefficient $\binom{n}{p}$ in double
  precision.

\item \describefun{double}{pnl_fact}{int n}
  \sshortdescribe Computes $n! = \Gamma(n+1)$ in double precision.

\item \describefun{double}{pnl_pow_i}{double x, int n}
  \sshortdescribe Computes $x^n$ for $n \in \N$ using a squaring method.


\item \describefun{void}{pnl_qsort}{void *a, int n, int es, int lda, int *t,
  int ldt, int use_index, int (*cmp)(void const *, void const *)}
  \sshortdescribe Sorts the array \var{a} using the comparison function
  \var{cmp}. \var{n} is the number of elements in \var{a}, each element being of
  size \var{es}. \var{t} is an array of integers of length \var{n} used to store
  the permutation when \var{use_index=TRUE}. \var{lda} and \var{ldt} are the
  leading dimensions of the arrays \var{a} and \var{t} and are used to sort
  matrices column-wise. 

\item \describefun{double}{pnl_nan}{}
  \sshortdescribe Returns \var{Nan}

\item \describefun{double}{pnl_posinf}{}
  \sshortdescribe Returns \var{+ infinity} 

\item \describefun{double}{pnl_neginf}{}
  \sshortdescribe Returns \var{- infinity} 

\item \describefun{int}{pnl_isnan}{double x}
  \sshortdescribe Returns \var{1} if \var{x = Nan}

\item \describefun{int}{pnl_isfinite}{}
  \sshortdescribe Returns \var{1} if \var{x != Inf}

\item \describefun{int}{pnl_isinf}{}
  \sshortdescribe Returns \var{+1} if \var{x = +Inf}, \var{-1} if \var{x = -Inf},
  \var{0} otherwise.
\end{itemize}

We provide a few functions which are part of C99 as \verb!pnl_funcname!.
\begin{itemize}
\item \describefun{double}{lgamma}{double x}
  \sshortdescribe Computes $\log(\Gamma(x))$.
\item \describefun{double}{tgamma}{double x}
  \sshortdescribe Computes $\Gamma(x)$.
\item \describefun{double}{pnl_acosh}{double x}
  \sshortdescribe Computes \var{acosh(x)}.
\item \describefun{double}{pnl_asinh}{double x}
\sshortdescribe Computes \var{asinh(x)}.
\item \describefun{double}{pnl_atanh}{double x}
\sshortdescribe Computes \var{atanh(x)}.
\end{itemize}<++>

% vim:spelllang=en:spell:


\subsection{Complex numbers}
\subsubsection{Overview}

The complex type and related functions are defined in the header
\verb!pnl/pnl_complex.h!.\\

The first native implementation of complex numbers in the C language appeared in
C99, which is unfortunately not available on all platforms. For this reason, we
provide here an implementation of complex numbers.

\label{dcomplex}
\begin{verbatim}
typedef struct {
    double r; /*!< real part */
    double i; /*!< imaginary part */
} dcomplex;
\end{verbatim}


\subsubsection{Constants}

\begin{describeconst}
  \constentry{CZERO}{$0$ as a complex number}
  \constentry{CONE}{$1$ as a complex number}
  \constentry{CI}{$I$ the unit complex number}
\end{describeconst}

\subsubsection{Functions}
\begin{itemize}
\item \describefun{double}{Creal}{\refstruct{dcomplex} z}
  \sshortdescribe $ \mathrm{R}(z) $ 

\item \describefun{double}{Cimag}{\refstruct{dcomplex} z}
  \sshortdescribe  $ \mathrm{Im}(z) $  

\item \describefun{dcomplex}{Cadd}{\refstruct{dcomplex} z, \refstruct{dcomplex} b}
  \sshortdescribe \var{ z+b }  

\item \describefun{dcomplex}{CRadd}{\refstruct{dcomplex} z, double b}
  \sshortdescribe \var{ z+b }  

\item \describefun{dcomplex}{RCadd}{double b, \refstruct{dcomplex} z}
  \sshortdescribe \var{ b+z }  

\item \describefun{dcomplex}{Csub}{\refstruct{dcomplex} z, \refstruct{dcomplex} b}
  \sshortdescribe \var{ z-b }

\item \describefun{dcomplex}{CRsub}{\refstruct{dcomplex} z, double b}
  \sshortdescribe \var{ z-b }

\item \describefun{dcomplex}{RCsub}{double b, \refstruct{dcomplex} z}
  \sshortdescribe \var{ b-z }

\item \describefun{dcomplex}{Cminus}{\refstruct{dcomplex} z}
  \sshortdescribe \var{ -z }  

\item \describefun{dcomplex}{Cmul}{\refstruct{dcomplex} z, \refstruct{dcomplex} b}
  \sshortdescribe \var{ z*b }  

\item \describefun{dcomplex}{RCmul}{double x, \refstruct{dcomplex} z}
  \sshortdescribe \var{ x*z }

\item \describefun{dcomplex}{CRmul}{\refstruct{dcomplex} z, double x}
  \sshortdescribe \var{ z * x }

\item \describefun{dcomplex}{CRdiv}{\refstruct{dcomplex} z, double x}
  \sshortdescribe \var{ z/x }

\item \describefun{dcomplex}{RCdiv}{double x, \refstruct{dcomplex} z}
  \sshortdescribe \var{ x/z }

\item \describefun{dcomplex}{Complex}{double x, double y}
  \sshortdescribe \var{x + i y}  

\item \describefun{dcomplex}{Complex_polar}{double r, double theta}
  \sshortdescribe  \var{ r exp(i theta) }  

\item \describefun{dcomplex}{Conj}{\refstruct{dcomplex} z}
  \sshortdescribe $\overline{z}$  

\item \describefun{dcomplex}{Cinv}{\refstruct{dcomplex} z}
  \sshortdescribe \var{ 1/z }  

\item \describefun{dcomplex}{Cdiv}{\refstruct{dcomplex} z, \refstruct{dcomplex} w}
  \sshortdescribe \var{ z/w }  

\item \describefun{double}{Csqr_norm}{\refstruct{dcomplex} z}
  \sshortdescribe $ \mathrm{Re}(z)^2 + \mathrm{im}(z)^2 $  

\item \describefun{double}{Cabs}{\refstruct{dcomplex} z}
  \sshortdescribe \var{|z|}  

\item \describefun{dcomplex}{Csqrt}{\refstruct{dcomplex} z}
  \sshortdescribe \var{ sqrt(z) },  square root (with positive real part)  

\item \describefun{dcomplex}{Clog}{\refstruct{dcomplex} z}
  \sshortdescribe \var{log(z)}  

\item \describefun{dcomplex}{Cexp}{\refstruct{dcomplex} z}
  \sshortdescribe \var{ exp(z) }  

\item \describefun{dcomplex}{CIexp}{double t}
  \sshortdescribe \var{ exp( it ) }

\item \describefun{dcomplex}{Cpow}{\refstruct{dcomplex} z, \refstruct{dcomplex} w}
  \sshortdescribe $ z^w$, power function  

\item \describefun{dcomplex}{Cpow_real}{\refstruct{dcomplex} z, double x}
  \sshortdescribe $ z^x$, power function  

\item \describefun{dcomplex}{Ccos}{\refstruct{dcomplex} z}
  \sshortdescribe \var{ cos(g)}  

\item \describefun{dcomplex}{Csin}{\refstruct{dcomplex} z}
  \sshortdescribe \var{sin(g)}  

\item \describefun{dcomplex}{Ctan}{\refstruct{dcomplex} z}
  \sshortdescribe \var{tan(z)}

\item \describefun{dcomplex}{Ccotan}{\refstruct{dcomplex} z}
  \sshortdescribe \var{cotan(z)}

\item \describefun{dcomplex}{Ccosh}{\refstruct{dcomplex} z}
  \sshortdescribe \var{ cosh(g)}  

\item \describefun{dcomplex}{Csinh}{\refstruct{dcomplex} z}
  \sshortdescribe \var{sinh(g)}  

\item \describefun{dcomplex}{Ctanh}{\refstruct{dcomplex} z}
  \sshortdescribe $\tanh(z) = \frac{1 - \expp{-2z} }{1 + \expp{-2z} }$  

\item \describefun{dcomplex}{Ccotanh}{\refstruct{dcomplex} z}
  \sshortdescribe $\cotanh(z) = \frac{1 + \expp{-2z} }{1 - \expp{-2z} }$  

\item \describefun{double}{Carg}{\refstruct{dcomplex} z}
  \sshortdescribe \var{arg(z) }

\item \describefun{dcomplex}{Ctgamma}{\refstruct{dcomplex} z}
  \sshortdescribe \var{ Gamma(z)}, the Gamma function  

\item \describefun{dcomplex}{Clgamma}{\refstruct{dcomplex} z}
  \sshortdescribe \var{ log(Gamma (z))}, the logarithm of the Gamma function

\item \describefun{void}{Cprintf}{\refstruct{dcomplex} z}
  \sshortdescribe Print a complex number on the standard output
\end{itemize}

Most algebraic operations on complex numbers are implemented using the
following naming for the functions
\begin{itemize}
\item All these function names begin in {\tt C_op_}, 
\item The small letters {\tt a, b} denote two complex numbers whereas {\tt d} is a real number, 
\item The letter {\tt i} denotes the multiplication by the pure imagniary
  number $\imath$, 
\item The letter {\tt c} indicates that the next coming number is conjugated.
\item The letters {\tt p, m} denote the two standard operations {\it plus} and
  {\it minus} respectively.
\end{itemize}
For example C_op_idamcb is $\imath d \left( a - \overline{b} \right)$. So
functions are :
\begin{itemize}
\item \describefun{dcomplex}{C_op_apib}{\refstruct{dcomplex} a, \refstruct{dcomplex} b}
  \sshortdescribe $ a+\imath b  $.
\item \describefun{dcomplex}{C_op_apcb}{\refstruct{dcomplex} a, \refstruct{dcomplex} b}
  \sshortdescribe $ a+\overline{ b}  $.
\item \describefun{dcomplex}{C_op_amcb}{\refstruct{dcomplex} a, \refstruct{dcomplex} b}
  \sshortdescribe $ a-\overline{b}  $.
\item \describefun{dcomplex}{C_op_amib}{\refstruct{dcomplex} a, 
    \refstruct{dcomplex} b}
    \sshortdescribe \var{a - i b}
\item \describefun{dcomplex}{C_op_dapb}{double d, \refstruct{dcomplex} a, 
    \refstruct{dcomplex} b}
  \sshortdescribe $ d(a+ b)  $.
\item \describefun{dcomplex}{C_op_damb}{double d, \refstruct{dcomplex} a, 
    \refstruct{dcomplex} b}
  \sshortdescribe $ d(a-b)  $.
\item \describefun{dcomplex}{C_op_dapib}{double d, \refstruct{dcomplex} a, 
    \refstruct{dcomplex} b}
  \sshortdescribe $ d(a+\imath b)  $.
\item \describefun{dcomplex}{C_op_damib}{double d, \refstruct{dcomplex} a, 
    \refstruct{dcomplex} b}
  \sshortdescribe $ d(a-\imath b)  $.
\item \describefun{dcomplex}{C_op_dapcb}{double d, \refstruct{dcomplex} a, 
    \refstruct{dcomplex} b}
  \sshortdescribe $ d\left(a+\overline{b}\right)  $.
\item \describefun{dcomplex}{C_op_damcb}{double d, \refstruct{dcomplex} a, 
    \refstruct{dcomplex} b}
  \sshortdescribe $ d\left(a-\overline{b}\right)  $.
\item \describefun{dcomplex}{C_op_idapb}{double d, \refstruct{dcomplex} a, 
    \refstruct{dcomplex} b}
  \sshortdescribe $\imath d\left(a+b\right) $.
\item \describefun{dcomplex}{C_op_idamb}{double d, \refstruct{dcomplex} a, 
    \refstruct{dcomplex} b}
  \sshortdescribe $\imath  d\left(a-b\right) $.
\item \describefun{dcomplex}{C_op_idapcb}{double d, \refstruct{dcomplex} a, 
    \refstruct{dcomplex} b}
  \sshortdescribe $ \imath d\left(a+\overline{b}\right) $.
\item \describefun{dcomplex}{C_op_idamcb}{double d, \refstruct{dcomplex} a, 
    \refstruct{dcomplex} b}
  \sshortdescribe $ \imath  d\left(a-\overline{b}\right) $.
\end{itemize}

% vim:spelllang=en:spell:





\section{Linear Algebra}

% vector
\subsection{Vectors}
\subsubsection{Short Description}

The structures and functions related to vectors are declared in
\verb!pnl_vector.h!.


Vectors are declared for several basic types : double, int, uint and
dcomplex. In the following declarations, {\tt BASE} must be replaced by one
the previous types and the corresponding vector structures are respectively
named PnlVect, PnlVectInt, PnlVectUint, PnlVectComplex
\begin{verbatim}
typedef struct PnlVect {
  int size; /*!< size of the vector */
  int mem_size; /*!< size of the memory block allocated for array */
  double *array; /*!< pointer to store the data */
  int owner; /*!< 1 if the structure owns its array pointer */
} PnlVect;

typedef struct PnlVectUint {
  int size; /*!< size of the vector */ 
  int mem_size; /*!< size of the memory block allocated for array */
  uint *array; /*!< pointer to store the data */
  int owner; /*!< 1 if the structure owns its array pointer */
} PnlVect;

typedef struct PnlVectInt {
  int size; /*!< size of the vector */ 
  int mem_size; /*!< size of the memory block allocated for array */
  int *array; /*!< pointer to store the data */
  int owner; /*!< 1 if the structure owns its array pointer */
} PnlVect;

typedef struct PnlVectComplex {
  int size; /*!< size of the vector */ 
  int mem_size; /*!< size of the memory block allocated for array */
  dcomplex *array; /*!< pointer to store the data */
  int owner; /*!< 1 if the structure owns its array pointer */
} PnlVect;
\end{verbatim}
\var{size} is the size of the vector, \var{array} is a pointer containing the
data and \var{owner} is an integer to know if the vector owns its \var{array}
pointer (\var{owner=1}) or shares it with another structure (\var{owner=0}).
\var{mem_size} is the number of elements the vector can hold at most.

\subsubsection{Functions}

\paragraph{General functions}
These functions exist for all types of vector no matter what the basic type
is. The following conventions are used to name functions operating on vectors.
Here is the table of prefixes used for the different basic types.

\begin{center}
  \begin{tabular}[t]{lll}
    type & prefix & BASE\\
    \hline
    double & pnl_vect & double \\
    \hline
    int & pnl_vect_int & int \\
    \hline
    uint & pnl_vect_uint & uint\\
    \hline
    dcomplex & pnl_vect_complex & dcomplex
  \end{tabular}
\end{center}

In this paragraph, we present the functions operating on \refstruct{PnlVect}
which exist for all types. To deduce the prototypes of these functions for
other basic types, one must replace {\tt pnl_vect} and {\tt double} according
the above table. 
\subparagraph{Constructors and destructors}
\begin{itemize}
\item \describefun{\refstruct{PnlVect} \ptr }{pnl_vect_create}{int size}
  \sshortdescribe Creates a new \refstruct{PnlVect} pointer.  
\item \describefun{\refstruct{PnlVect} \ptr }{pnl_vect_create_from_zero}{int size}
  \sshortdescribe Creates a new \refstruct{PnlVect} pointer and sets it to zero.  
\item \describefun{\refstruct{PnlVect} \ptr }{pnl_vect_create_from_double}
  {int size, double x}
  \sshortdescribe Creates a new \refstruct{PnlVect} pointer and sets all
  elements t \var{x}.  
\item \describefun{\refstruct{PnlVect} \ptr }{pnl_vect_create_from_ptr}{int
    size, const double \ptr x}
  \sshortdescribe Creates a new \refstruct{PnlVect} pointer and copies \var{x}
  to \var{array}.  
\item \describefun{\refstruct{PnlVect} \ptr }{pnl_vect_create_from_list}{int
    size, ...}
  \sshortdescribe Creates a new \refstruct{PnlVect} pointer of length
  \var{size} filled with the extra arguments passed to the function. The
  number of extra arguments passed must be equal to \var{size}, be aware that
  this cannot be checked inside the function.
\item \describefun{\refstruct{PnlVect}}{pnl_vect_create_wrap_array}{const double \ptr x, 
    int size}
    \sshortdescribe Creates a \refstruct{PnlVect} containing the data 
  \var{x}. No copy is made. It is just a container.
  
\item \describefun{\refstruct{PnlVect}}{pnl_vect_wrap_subvect}{const
  \refstruct{PnlVect} \ptr x, int i, int s}
  \sshortdescribe Creates a \refstruct{PnlVect} containing \var{x(i:i+s)}. No
  copy is made. It is just a container.

\item \describefun{\refstruct{PnlVect}}{pnl_vect_wrap_subvect_with_last}{const \refstruct{PnlVect} \ptr x, int i, int j}
  \sshortdescribe Creates a \refstruct{PnlVect} containing \var{x(i:j)}. No
  copy is made. It is just a container.

\item \describefun{\refstruct{PnlVect} \ptr }{pnl_vect_create_from_file}
  {const char \ptr file}
  \sshortdescribe Reads a vector from a file and creates the corresponding
  \refstruct{PnlVect}.  

\item \describefun{\refstruct{PnlVect} \ptr }{pnl_vect_copy}{const
    \refstruct{PnlVect} \ptr \refstruct{v}}
  \sshortdescribe This is a copying constructor. It creates a copy of a \refstruct{PnlVect}.
\item \describefun{void}{pnl_vect_clone}{\refstruct{PnlVect} \ptr clone, 
    const \refstruct{PnlVect} \ptr \refstruct{v}} 
  \sshortdescribe Clones a \refstruct{PnlVect}. \var{clone} must be an
  already existing  \refstruct{PnlVect}. It is resized to match the size of
  \var{v} and the data are copied. Future modifications to \var{v} will not
  affect \var{clone}.

\item \describefun{void}{pnl_vect_free}{\refstruct{PnlVect} \ptr\ptr v}
  \sshortdescribe Frees a \refstruct{PnlVect} pointer and set the data pointer to NULL  
\end{itemize}

\subparagraph{Resizing vectors}
\begin{itemize}
\item \describefun{int}{pnl_vect_resize}{\refstruct{PnlVect} \ptr \refstruct{v}, int size}
  \sshortdescribe Resizes a \refstruct{PnlVect}. The old data are kept up to
  the new size.
\item \describefun{int}{pnl_vect_resize_from_double}{\refstruct{PnlVect}
    \ptr \refstruct{v}, int size, double x} 
  \sshortdescribe Resizes a \refstruct{PnlVect}.  The old data are kept. If
  the new size is larger than the old one, the new cells are set to \var{x}.
\item \describefun{int}{pnl_vect_resize_from_ptr}{\refstruct{PnlVect}
    \ptr \refstruct{v}, int size, double \ptr t} 
  \sshortdescribe Resizes a \refstruct{PnlVect} and uses \var{t} to fill the
  vector. \var{t} must be of size \var{size}.
\end{itemize}  

\subparagraph{Accessing elements}

If it is supported by the compiler, the following functions are declared
inline. You just need to define the macro \verb!HAVE_INLINE! for by passing
\verb!-DHAVE_INLINE! to gcc to use the inline versions of the following
functions.
\begin{itemize}
\item \describefun{void}{pnl_vect_set}{\refstruct{PnlVect} \ptr \refstruct{v}, int i, double x}
  \sshortdescribe Sets v[i]=x  
\item \describefun{double}{pnl_vect_get}{const \refstruct{PnlVect} \ptr \refstruct{v}, int i}
  \sshortdescribe Returns the value of v[i].  
\item \describefun{void}{pnl_vect_lget}{\refstruct{PnlVect} \ptr \refstruct{v}, int i}
  \sshortdescribe Returns the address of v[i].  
\item \describefun{void}{pnl_vect_set_double}{\refstruct{PnlVect} \ptr \refstruct{v}, double x}
  \sshortdescribe Sets all elements to x.  
\item \describefun{void}{pnl_vect_set_zero}{\refstruct{PnlVect} \ptr \refstruct{v}}
  \sshortdescribe Sets all elements to zero.  
\end{itemize}
Equivalently to these functions, there exist macros for {\bf \refstruct{PnlVect} only}.
\begin{itemize}
\item \describefun{}{GET}{v, i}
  \sshortdescribe Returns \var{v[i]}.
  
\item \describefun{}{LET}{v, i}
  \sshortdescribe Returns \var{v[i]} as a lvalue.
\end{itemize}


\subparagraph{Printing vector}
\begin{itemize}
\item \describefun{void}{pnl_vect_print}{const \refstruct{PnlVect} \ptr V}
  \sshortdescribe Prints a \refstruct{PnlVect}.  
\item \describefun{void}{pnl_vect_fprint}{FILE \ptr fic, const \refstruct{PnlVect} \ptr V}
  \sshortdescribe Prints a \refstruct{PnlVect} in file \var{fic}.  
\item \describefun{void}{pnl_vect_print_nsp}{const \refstruct{PnlVect} \ptr V}
  \sshortdescribe Prints a vector to the standard output in a format
  compatible with Nsp.  

\item \describefun{void}{pnl_vect_fprint_nsp}{FILE \ptr fic, const
    \refstruct{PnlVect} \ptr V}
  \sshortdescribe Prints a vector to a file in a format compatible with Nsp.
\end{itemize}

\subparagraph{Applying external operation to vectors}

\begin{itemize}
\item \describefun{void}{pnl_vect_minus}{\refstruct{PnlVect} \ptr lhs}
  \sshortdescribe In-place unary minus
\item \describefun{void}{pnl_vect_plus_double}{\refstruct{PnlVect} \ptr lhs, double x}
  \sshortdescribe In-place vector scalar addition  
\item \describefun{void}{pnl_vect_minus_double}{\refstruct{PnlVect} \ptr lhs, double x}
  \sshortdescribe In-place vector scalar substraction  
\item \describefun{void}{pnl_vect_mult_double}{\refstruct{PnlVect} \ptr lhs, double x}
  \sshortdescribe In-place vector scalar multiplication  
\item \describefun{void}{pnl_vect_div_double}{\refstruct{PnlVect} \ptr lhs, double x}
  \sshortdescribe In-place vector scalar division  
\end{itemize}

\subparagraph{Element wise operations}

\begin{itemize}
\item \describefun{void}{pnl_vect_plus_vect}{\refstruct{PnlVect} \ptr lhs, 
    const \refstruct{PnlVect} \ptr rhs} 
  \sshortdescribe In-place vector vector addition  

\item \describefun{void}{pnl_vect_minus_vect}{\refstruct{PnlVect} \ptr lhs, 
    const \refstruct{PnlVect} \ptr rhs} 
  \sshortdescribe In-place vector vector substraction  

\item \describefun{void}{pnl_vect_inv_term}{\refstruct{PnlVect} \ptr lhs}
  \sshortdescribe In-place term by term vector inversion  

\item \describefun{void}{pnl_vect_div_vect_term}{\refstruct{PnlVect}
    \ptr lhs, const \refstruct{PnlVect} \ptr rhs} 
  \sshortdescribe In-place term by term vector division

\item \describefun{void}{pnl_vect_mult_vect_term}{\refstruct{PnlVect}
    \ptr lhs, const \refstruct{PnlVect} \ptr rhs} 
  \sshortdescribe In-place vector vector term by term multiplication  

\item \describefun{void}{pnl_vect_map}{\refstruct{PnlVect} \ptr lhs, const
    \refstruct{PnlVect} \ptr rhs, double(\ptr f)(double)} 
    \sshortdescribe \var{lhs = f(rhs)} 

\item \describefun{void}{pnl_vect_map_inplace}{\refstruct{PnlVect} \ptr lhs, double(\ptr f)(double)}
  \sshortdescribe \var{lhs = f(lhs)} 

\item \describefun{void}{pnl_vect_map_vect}{\refstruct{PnlVect} \ptr lhs, const
  \refstruct{PnlVect} \ptr rhs, double(\ptr f)(double, double)} 
    \sshortdescribe \var{lhs = f(lhs,rhs)} 

\item \describefun{void}{pnl_vect_axpby}{double a, const \refstruct{PnlVect} \ptr x, 
    double b, \refstruct{PnlVect} \ptr y} 
  \sshortdescribe Computes \var{y : = a x + b y}. When \var{b==0}, the content
  of \var{y} is not used on input and instead \var{y} is resized to match \var{x}.

\item \describefun{double}{pnl_vect_sum}{const \refstruct{PnlVect} \ptr lhs}
  \sshortdescribe Returns the sum of all the elements of a vector  

\item \describefun{void}{pnl_vect_cumsum}{\refstruct{PnlVect} \ptr lhs}
  \sshortdescribe Computes the cumulative sum of all the elements of a
  vector. The original vector is modified

\item \describefun{double}{pnl_vect_prod}{const \refstruct{PnlVect} \ptr V}
  \sshortdescribe Returns the product of all the elements of a vector  

\item \describefun{void}{pnl_vect_cumprod}{\refstruct{PnlVect} \ptr lhs}
  \sshortdescribe Computes the cumulative product of all the elements of a
  vector. The original vector is modified
\end{itemize}

\subparagraph{Ordering functions}
The following functions are not defined for PnlVectComplex because there is
no total ordering on Complex numbers

\begin{itemize}
\item \describefun{double}{pnl_vect_max}{const \refstruct{PnlVect} \ptr V}
  \sshortdescribe Returns the maximum of a a vector  

\item \describefun{double}{pnl_vect_min}{const \refstruct{PnlVect} \ptr V}
  \sshortdescribe Returns the minimum of a vector  

\item \describefun{void}{pnl_vect_minmax}{const \refstruct{PnlVect} \ptr , 
    double \ptr m, double \ptr M}
  \sshortdescribe Computes the minimum and maximum of a vector which are
  returned in  \var{m} and \var{M} respectively.
  
\item \describefun{void}{pnl_vect_min_index}{const \refstruct{PnlVect} \ptr , 
    double \ptr m, int \ptr im}
  \sshortdescribe Computes the minimum of a vector and its index stored in 
  sets \var{m} and \var{im} respectively.

\item \describefun{void}{pnl_vect_max_index}{const \refstruct{PnlVect} \ptr , 
    double \ptr M, int \ptr iM}
  \sshortdescribe Computes the maximum of a vector and its index stored in 
  sets \var{m} and \var{im} respectively.

\item \describefun{void}{pnl_vect_minmax_index}{const \refstruct{PnlVect}
    \ptr , double \ptr m, double \ptr M, int \ptr im, int \ptr iM}
  \sshortdescribe Computes the minimum and maximum of a vector and the
  corresponding indices stored respectively in \var{m}, \var{M}, \var{im} and
  \var{iM}.

\item \describefun{void}{pnl_vect_qsort}{\refstruct{PnlVect} \ptr , char order}
  \sshortdescribe Sorts a vector using a quick sort algorithm according to
  \var{order} (\verb!'i'! for increasing or \verb!'d'! for decreasing).

\item \describefun{void}{pnl_vect_qsort_index}{\refstruct{PnlVect} \ptr ,
    \refstruct{PnlVectInt} *index, char order}
  \sshortdescribe Sorts a vector using a quick sort algorithm according to
  \var{order} (\verb!'i'! for increasing or \verb!'d'! for decreasing ). On
  output, \var{index} contains the permutation used to sort the vector.
\end{itemize}

\subparagraph{Scalar products and norms}
\begin{itemize}
\item \describefun{double}{pnl_vect_norm_two}{const \refstruct{PnlVect} \ptr V}
  \sshortdescribe Returns the two norm of a vector  

\item \describefun{double}{pnl_vect_norm_one}{const \refstruct{PnlVect} \ptr V}
  \sshortdescribe Returns the one norm of a vector  

\item \describefun{double}{pnl_vect_norm_infty}{const \refstruct{PnlVect} \ptr V}
  \sshortdescribe Returns the infinity norm of a vector  

\item \describefun{double}{pnl_vect_scalar_prod}{const \refstruct{PnlVect}
    \ptr rhs1, const \refstruct{PnlVect} \ptr rhs2} 
  \sshortdescribe Computes the scalar product between 2 vectors  
\end{itemize}

\subparagraph{Misc}

\begin{itemize}
\item \describefun{void}{pnl_vect_swap_elements}{\refstruct{PnlVect} \ptr v,
    int i, int j}
  \sshortdescribe Exchanges \var{v[i]} and \var{v[j]}.
\item \describefun{void}{pnl_vect_reverse}{\refstruct{PnlVect} \ptr v}
  \sshortdescribe Performs a mirror operation on v. On output \var{v[i]
    = v[n-i]} where \var{n} is the length of the vector.
\end{itemize}


\paragraph{Complex vector functions}

\begin{itemize}
\item \describefun{void}{pnl_vect_complex_mult_double}
  {\refstruct{PnlVectComplex} \ptr lhs, double x}
  \sshortdescribe In-place multiplication by a double.

\item \describefun{PnlVectComplex\ptr }{pnl_vect_complex_create_from_array}{int
    size, const double \ptr re, const double \ptr im}
  \sshortdescribe Creates a \refstruct{PnlVectComplex} given the arrays of the
  real parts \var{re} and imaginary parts \var{im}.
\item \describefun{void}{pnl_vect_complex_split_in_array}{\refstruct{PnlVectComplex}
    \ptr v, double \ptr re, double \ptr im}
  \sshortdescribe Splits a complex vector into two arrays : the array of the
  real parts of the elements of \var{v} and the array of the imaginary parts
  of the elements of \var{v}.
\item \describefun{void}{pnl_vect_complex_split_in_vect}{\refstruct{PnlVectComplex}
    \ptr v, \refstruct{PnlVect} \ptr re, \refstruct{PnlVect} \ptr im}
  \sshortdescribe Splits a complex vector into two \refstruct{PnlVect}s : the
  \refstruct{PnlVect} of the real parts of the elements of \var{v} and the
  \refstruct{PnlVect} of the imaginary parts of the elements of \var{v}.
\end{itemize}

There exist functions to directly access the real or imaginary parts of an
element of a complex vector. These functions also have inlined versions that
are used if the variable \var{HAVE_INLINE} was declared at compilation time.

\begin{itemize}
\item \describefun{double}{pnl_vect_complex_get_real}
  {const \refstruct{PnlVectComplex} \ptr v, int i}
  \sshortdescribe Returns the real part of \var{v[i]}.
  
\item \describefun{double}{pnl_vect_complex_get_imag}
  {const \refstruct{PnlVectComplex} \ptr v, int i}
  \sshortdescribe Returns the imaginary part of \var{v[i]}.

\item \describefun{double\ptr }{pnl_vect_complex_lget_real}
  {const \refstruct{PnlVectComplex} \ptr v, int i}
  \sshortdescribe Returns the real part of \var{v[i]} as a lvalue.

\item \describefun{double\ptr }{pnl_vect_complex_lget_imag}
  {const \refstruct{PnlVectComplex} \ptr v, int i}
  \sshortdescribe Returns the imaginary part of \var{v[i]} as a lvalue.

\item \describefun{void}{pnl_vect_complex_set_real}
  {const \refstruct{PnlVectComplex} \ptr v, int i, double re}
  \sshortdescribe Sets the real part of \var{v[i]} to \var{re}.

\item \describefun{void}{pnl_vect_complex_set_imag}
  {const \refstruct{PnlVectComplex} \ptr v, int i, double im}
  \sshortdescribe Sets the imaginary part of \var{v[i]} to \var{im}.
\end{itemize}

Equivalently to these functions, there exist macros. When the compiler is able
to handle inline code, there is no gain in using macros instead of inlined
functions at least in principle.
\begin{itemize}
\item \describefun{}{GET_REAL}{v, i}
  \sshortdescribe Returns the real part of \var{v[i]}.
  
\item \describefun{}{GET_IMAG}{v, i}
  \sshortdescribe Returns the imaginary part of \var{v[i]}.
  
\item \describefun{}{LET_REAL}{v, i}
  \sshortdescribe Returns the real part of \var{v[i]} as a lvalue.
  
\item \describefun{}{LET_IMAG}{v, i}
  \sshortdescribe Returns the imaginary part of \var{v[i]} as a lvalue.
\end{itemize}

\subsection{Compact Vectors}
\subsubsection{Short description}

\begin{verbatim}
typedef struct PnlVectCompact {
  int size; /* size of the vector */
  union {
    double val; /* single value */
    double *array; /* Pointer to double values */
  };
  char convert; /* 'a', 'd' : array, double */
} PnlVectCompact;
\end{verbatim}

\subsubsection{Functions}

\begin{itemize}
\item \describefun{\refstruct{PnlVectCompact} \ptr }{pnl_vect_compact_create}{int n, double x}
  \sshortdescribe Allocates a \refstruct{PnlVectCompact}.  

\item \describefun{int}{pnl_vect_compact_resize}{\refstruct{PnlVectCompact}
    \ptr \refstruct{v}, int size, double x} 
  \sshortdescribe Resizes a \refstruct{PnlVectCompact}.  

\item \describefun{\refstruct{PnlVectCompact}
    \ptr }{pnl_vect_compact_copy} {const \refstruct{PnlVectCompact}\ptr \refstruct{v}}
  \sshortdescribe Copies a \refstruct{PnlVectCompact}  

\item \describefun{void}{pnl_vect_compact_free}{\refstruct{PnlVectCompact} \ptr \ptr \refstruct{v}}
  \sshortdescribe Free a \refstruct{PnlVectCompact}  

\item \describefun{\refstruct{PnlVect} \ptr }{pnl_vect_compact_to_pnl_vect}
  {const \refstruct{PnlVectCompact} \ptr C} 
  \sshortdescribe Converts a \refstruct{PnlVectCompact} pointer to a \refstruct{PnlVect} pointer.  

\item \describefun{double}{pnl_vect_compact_get}{const \refstruct{PnlVectCompact} \ptr C, int i}
  \sshortdescribe Access function  
\end{itemize}

%% matrix

\subsection{Matrices}
\subsubsection{Short Description}

The structures and functions related to matrices are declared in
\verb!pnl_matrix.h!.

\begin{verbatim}
typedef struct PnlMat{
  int m; /*!< nb rows */ 
  int n; /*!< nb columns */ 
  int mn; /*!< product m*n */
  int mem_size; /*!< size of the memory block allocated for array */
  double *array; /*!< pointer to store the data row-wise */
  int owner; /*!< 1 if the structure owns its array pointer */
} PnlMat;

typedef struct PnlMatUint{
  int m; /*!< nb rows */ 
  int n; /*!< nb columns */ 
  int mn; /*!< product m*n */
  int mem_size; /*!< size of the memory block allocated for array */
  uint *array; /*!< pointer to store the data row-wise */
  int owner; /*!< 1 if the structure owns its array pointer */
} PnlMatUint;

typedef struct PnlMatInt{
  int m; /*!< nb rows */ 
  int n; /*!< nb columns */ 
  int mn; /*!< product m*n */
  int mem_size; /*!< size of the memory block allocated for array */
  int *array; /*!< pointer to store the data row-wise */
  int owner; /*!< 1 if the structure owns its array pointer */
} PnlMatInt;

typedef struct PnlMatComplex{
  int m; /*!< nb rows */ 
  int n; /*!< nb columns */ 
  int mn; /*!< product m*n */
  int mem_size; /*!< size of the memory block allocated for array */
  dcomplex *array; /*!< pointer to store the data row-wise */
  int owner; /*!< 1 if the structure owns its array pointer */
} PnlMatComplex;
\end{verbatim}
\var{m} is the number of rows, \var{n} is the number of columns. \var{array}
is a pointer containing the data of the matrix stored linewise, The element
\verb!(i, j)! of the matrix is \verb!array[i*m+j]!. \var{owner} is an integer to
know if the matrix owns its \var{array} pointer (\var{owner=1}) or shares it
with another structure (\var{owner=0}). \var{mem_size} is the number of
elements the matrix can hold at most.

The following operations are implemented on matrices and vectors. \var{alpha}
and \var{beta} are real numbers, \var{A} and \var{B} are matrices and \var{x}
and \var{y} are vectors.
\begin{tabular}{ll}
  \reffun{pnl_mat_axpy} & \var{B := alpha * A + B} \\
  \reffun{pnl_mat_scalar_prod_A} & \var{y' A x} \\
  \reffun{pnl_mat_dgemm} & \var{C := alpha * op (A) * op (B) + beta * C}\\
  \reffun{pnl_mat_mult_vect_transpose_inplace} & \var{y = A' * x}\\
  \reffun{pnl_mat_mult_vect_inplace} & \var{y = A * x}\\
  \reffun{pnl_mat_lAxpby} & \var{y := alpha * A * x + beta * y}\\
  \reffun{pnl_mat_dgemv} & \var{y := alpha * op (A) * x + beta * y}\\
  \reffun{pnl_mat_dger} & \var{A := alpha x * y' + A}
\end{tabular}


\subsubsection{Generic Functions}
\paragraph{General functions}
These functions exist for all types of matrices no matter what the basic type
is. The following conventions are used to name functions operating on matrices.
Here is the table of prefixes used for the different basic types.

\begin{center}
  \begin{tabular}[t]{lll}
    type & prefix & BASE\\
    \hline
    double & pnl_mat & double \\
    \hline
    int & pnl_mat_int & int \\
    \hline
    uint & pnl_mat_uint & uint\\
    \hline
    dcomplex & pnl_mat_complex & dcomplex
  \end{tabular}
\end{center}

In this paragraph we present the functions operating on \refstruct{PnlMat}
which exist for all types. To deduce the prototypes of these functions for
other basic types, one must replace {\tt pnl_mat} and {\tt double} according
the above table.

\paragraph{Constructors and destructors}
\begin{itemize}
\item \describefun{\refstruct{PnlMat} \ptr }{pnl_mat_create}{int m, int n}
  \sshortdescribe Creates a \refstruct{PnlMat}  with \var{m} rows and \var{n} columns.

\item \describefun{\refstruct{PnlMat} \ptr }{pnl_mat_create_from_double}{int m, int n, double x}
  \sshortdescribe Creates a \refstruct{PnlMat} with \var{m} rows and \var{n}
  columns and sets all the elements to \var{x}

\item \describefun{\refstruct{PnlMat} \ptr }{pnl_mat_create_from_ptr}{int m, int n, const double \ptr x}
  \sshortdescribe Creates a \refstruct{PnlMat} with \var{m} rows and \var{n}
  columns and copies the array \var{x} to the new vector. Be sure that \var{x}
  is long enough to fill all the vector because it cannot be checked inside the function.

\item \describefun{\refstruct{PnlMat} \ptr }{pnl_mat_create_from_list}{int
    m, int n, ...}
  \sshortdescribe Creates a new \refstruct{PnlMat} pointer of size \var{m
    x n} filled with the extra arguments passed to the function. The
  number of extra arguments passed must be equal to \var{m x n}, be
  aware that this cannot be checked inside the function.

\item \describefun{PnlMat}{pnl_mat_create_wrap_array}{const double \ptr x, 
    int m, int n}
    \sshortdescribe Creates a \refstruct{PnlMat} of size \var{m x n} 
    which contains \var{x}. No copy is made. It is just a container.
  
\item \describefun{\refstruct{PnlMat} \ptr }{pnl_mat_create_diag_from_ptr}
  {const double \ptr x, int d}
  \sshortdescribe Creates a new squared \refstruct{PnlMat} by specifying its size and
  diagonal terms as an array.

\item \describefun{\refstruct{PnlMat} \ptr }{pnl_mat_create_diag}
  {const \refstruct{PnlVect} \ptr V}
  \sshortdescribe Creates a new squared \refstruct{PnlMat} by specifying its diagonal
  terms in a \refstruct{PnlVect}.

\item \describefun{\refstruct{PnlMat} \ptr }{pnl_mat_create_from_file}{const char \ptr file}
  \sshortdescribe Reads a matrix from a file and creates the corresponding \refstruct{PnlMat}.  

\item \describefun{void}{pnl_mat_free}{\refstruct{PnlMat} \ptr \ptr \refstruct{v}}
  \sshortdescribe Frees a \refstruct{PnlMat} and sets \var{\ptr v} to \var{NULL} 

\item \describefun{\refstruct{PnlMat} \ptr }{pnl_mat_copy}{const \refstruct{PnlMat} \ptr \refstruct{v}}
  \sshortdescribe Creates a new \refstruct{PnlMat} which is a copy of
  \var{v}.
  
\item \describefun{void}{pnl_mat_clone}{\refstruct{PnlMat} \ptr clone, const \refstruct{PnlMat} \ptr M}
  \sshortdescribe Clones \var{M} into \var{clone}. No no new
  \refstruct{PnlMat} is created.

\item \describefun{int}{pnl_mat_resize}{\refstruct{PnlMat} \ptr \refstruct{v}, int m, int n}
  \sshortdescribe Resizes a \refstruct{PnlMat}. The new matrix is of size
  \var{m x n}.  
\end{itemize}  


\paragraph{Accessing elements}

\begin{itemize}
\item \describefun{void}{pnl_mat_set}{\refstruct{PnlMat} \ptr self, int i, int j, double x}
  \sshortdescribe Sets the value of self[i, j]=x  

\item \describefun{double}{pnl_mat_get}{const \refstruct{PnlMat} \ptr self, int i, int j}
  \sshortdescribe Gets the value of self[i, j]  

\item \describefun{double \ptr }{pnl_mat_lget}{\refstruct{PnlMat} \ptr self, int i, int j}
  \sshortdescribe Returns the address of self[i, j] for use as a lvalue.

\item \describefun{void}{pnl_mat_set_double}{\refstruct{PnlMat} \ptr self, double x}
  \sshortdescribe Sets all elements of \var{self} to \var{x}.
  
\item \describefun{void}{pnl_mat_set_id}{\refstruct{PnlMat} \ptr self}
  \sshortdescribe Sets the matrix \var{self} to the identity
  matrix. \var{self} must be a square matrix.

\item \describefun{void}{pnl_mat_set_diag}{\refstruct{PnlMat} \ptr self,
    double x, int d}
  \sshortdescribe Sets the $\var{d}^{\text{th}}$ diagonal terms of the matrix
  \var{self} to the value \var{x}. \var{self} must be a square matrix.

\item \describefun{\refstruct{PnlVect}}{pnl_mat_wrap_row}
  {const \refstruct{PnlMat} \ptr M, int i}
  \sshortdescribe Returns a \refstruct{PnlVect} (not a pointer) whose array is
  the \var{i}-th row of \var{M}. The new vector shares its data with the
  matrix \var{M}, which means that any modification to one of them will affect
  the other.
  
\item \describefun{\refstruct{PnlVect}}{pnl_mat_wrap_vect}
  {const \refstruct{PnlMat} \ptr M}
  \sshortdescribe Returns a \refstruct{PnlVect} (not a pointer) whose array is
  the row-wise array of \var{M}. The new vector shares its data with the
  matrix \var{M}, which means that any modification to one of them will affect
  the other.

\item \describefun{void}{pnl_mat_get_row}{\refstruct{PnlVect}
    \ptr V, const \refstruct{PnlMat} \ptr M, int i}
  \sshortdescribe Extracts and copies the \var{i}-th row of \var{M} into
  \var{V}.

\item \describefun{void}{pnl_mat_get_col}{\refstruct{PnlVect} \ptr V, 
    const \refstruct{PnlMat} \ptr M, int j}
  \sshortdescribe Extracts and copies the \var{j}-th column of \var{M} into \var{V}.
  
\item \describefun{void}{pnl_mat_swap_rows}{\refstruct{PnlMat} \ptr M, int i, int j}
  \sshortdescribe Swaps two rows of a matrix.  

\item \describefun{void}{pnl_mat_set_col}{\refstruct{PnlMat} \ptr M, 
    const \refstruct{PnlVect} \ptr V, int j}
  \sshortdescribe Replaces the \var{i}-th column of a matrix M by a vector V 

\item \describefun{void}{pnl_mat_set_row}{\refstruct{PnlMat} \ptr M, 
    const \refstruct{PnlVect} \ptr V, int i}
  \sshortdescribe Replaces the \var{i}-th row of a matrix M by a vector V  
\end{itemize}

Equivalently to the functions \reffun{pnl_mat_get} and \reffun{pnl_mat_set},
there exist macros for {\bf \refstruct{PnlMat} only}.
\begin{itemize}
\item \describefun{}{MGET}{M, i, j}
  \sshortdescribe Returns \var{M[i,j]}.
  
\item \describefun{}{MLET}{M, i, j}
  \sshortdescribe Returns \var{M[i,j]} as a lvalue for assignment.
\end{itemize}


\paragraph{Printing Matrices}

\begin{itemize}
\item \describefun{void}{pnl_mat_print}{const \refstruct{PnlMat} \ptr M}
  \sshortdescribe Prints a matrix to the standard output.  

\item \describefun{void}{pnl_mat_fprint}{FILE \ptr fic, const \refstruct{PnlMat} \ptr M}
  \sshortdescribe Prints a matrix to a file.

\item \describefun{void}{pnl_mat_print_nsp}{const \refstruct{PnlMat} \ptr M}
  \sshortdescribe Prints a matrix to the standard output in a format
  compatible with Nsp.  

\item \describefun{void}{pnl_mat_fprint_nsp}{FILE \ptr fic, const
    \refstruct{PnlMat} \ptr M}
  \sshortdescribe Prints a matrix to a file in a format compatible with Nsp.
\end{itemize}

\paragraph{Applying external operations}
\begin{itemize}
\item \describefun{void}{pnl_mat_plus_double}{\refstruct{PnlMat} \ptr lhs, double x}
  \sshortdescribe In-place matrix scalar addition  

\item \describefun{void}{pnl_mat_minus_double}{\refstruct{PnlMat} \ptr lhs, double x}
  \sshortdescribe In-place matrix scalar substraction  

\item \describefun{void}{pnl_mat_mult_double}{\refstruct{PnlMat} \ptr lhs, double x}
  \sshortdescribe In-place matrix scalar multiplication  

\item \describefun{void}{pnl_mat_div_double}{\refstruct{PnlMat} \ptr lhs, double x}
  \sshortdescribe In-place matrix scalar division  

\end{itemize}

\paragraph{Element wise operations}

\begin{itemize}
\item \describefun{void}{pnl_mat_mult_mat_term}{\refstruct{PnlMat} \ptr lhs, 
    const \refstruct{PnlMat} \ptr rhs} 
  \sshortdescribe In-place matrix matrix term by term product  

\item \describefun{void}{pnl_mat_div_mat_term}{\refstruct{PnlMat} \ptr lhs, 
    const \refstruct{PnlMat} \ptr rhs} 
  \sshortdescribe In-place matrix matrix term by term division

\item \describefun{void}{pnl_mat_map_inplace}{\refstruct{PnlMat} \ptr lhs, 
    double(\ptr f)(double)} 
  \sshortdescribe \var{lhs = f(lhs)}.


\item \describefun{void}{pnl_mat_map}{\refstruct{PnlMat} \ptr lhs, const
    \refstruct{PnlMat} \ptr rhs, double(\ptr f)(double)} 
    \sshortdescribe \var{lhs = f(rhs)}.

\item \describefun{void}{pnl_mat_map_mat}{\refstruct{PnlMat} \ptr lhs, const
  \refstruct{PnlMat} \ptr rhs, double(\ptr f)(double, double)} 
  \sshortdescribe \var{lhs = f(lhs, rhs)}.

\item \describefun{double}{pnl_mat_sum}{const \refstruct{PnlMat} \ptr lhs}
  \sshortdescribe Sums matrix component-wise  

\item \describefun{void}{pnl_mat_sum_vect}{\refstruct{PnlVect} \ptr y, const
    \refstruct{PnlMat} \ptr A, char c}
  \sshortdescribe Sums matrix column or row wise. Argument \var{c} can be
  either 'r' (to get a row vector) or 'c' (to get a column vector). When
  \var{c='r'}, $y(j) = \sum_i A_{ij}$ and when \var{c='rc}, $y(i) = \sum_j
  A_{ij}$.

\item \describefun{void}{pnl_mat_cumsum}{\refstruct{PnlMat} \ptr A, char c} 
  \sshortdescribe Cumulative sum over the rows or columns. Argument \var{c}
  can be either 'r' to sum over the rows or 'c' to sum over the columns. When
  \var{c='r'}, $A_{ij} = \sum_{1 \le k \le i} A_{kj}$ and when \var{c='rc}, 
  $A_{ij} = \sum_{1 \le k \le j} A_{ik}$.

\item \describefun{double}{pnl_mat_prod}{const \refstruct{PnlMat} \ptr lhs}
  \sshortdescribe Products matrix component-wise

\item \describefun{void}{pnl_mat_prod_vect}{\refstruct{PnlVect} \ptr y, const
    \refstruct{PnlMat} \ptr A, char c}
  \sshortdescribe Prods matrix column or row wise. Argument \var{c} can be
  either 'r' (to get a row vector) or 'c' (to get a column vector). When
  \var{c='r'}, $y(j) = \prod_i A_{ij}$ and when \var{c='rc}, $y(i) = \prod_j
  A_{ij}$.

\item \describefun{void}{pnl_mat_cumprod}{\refstruct{PnlMat} \ptr A, char c} 
  \sshortdescribe Cumulative prod over the rows or columns. Argument \var{c}
  can be either 'r' to prod over the rows or 'c' to prod over the columns. When
  \var{c='r'}, $A_{ij} = \prod_{1 \le k \le i} A_{kj}$ and when \var{c='rc}, 
  $A_{ij} = \prod_{1 \le k \le j} A_{ik}$.
\end{itemize}

\paragraph{Ordering operations}

\begin{itemize}
\item \describefun{void}{pnl_mat_max}{const \refstruct{PnlMat} \ptr A,
    \refstruct{PnlVect} \ptr M, char d}
  \sshortdescribe On exit, $\var{M}(i) = \max_{j}(\var{A}(i, j))$ when \var{d='c'}
  and $\var{M}(i) = \max_{j}(\var{A}(j, i))$ when \var{d='r'}.

\item \describefun{void}{pnl_mat_min}{const \refstruct{PnlMat} \ptr A,
    \refstruct{PnlVect} \ptr m, char d}
  \sshortdescribe On exit, $\var{m}(i) = \min_{j}(\var{A}(i, j))$ when \var{d='c'}
  and $\var{m}(i) = \min_{j}(\var{A}(j, i))$ when \var{d='r'}.

\item \describefun{void}{pnl_mat_minmax}{const \refstruct{PnlMat} \ptr A, 
    \refstruct{PnlVect} \ptr m, \refstruct{PnlVect} \ptr M, char d}
  \sshortdescribe On exit, $\var{m}(i) = \min_{j}(\var{A}(i, j))$ and $\var{M}(i) =
  \max_{j}(\var{A}(i, j))$ when \var{d='c'} and $\var{m}(i) = \min_{j}(\var{A}(j, i))$
  and $\var{M}(i) = \min_{j}(\var{A}(j, i))$ when \var{d='r'}.
  
\item \describefun{void}{pnl_mat_min_index}{const \refstruct{PnlMat} \ptr  A, 
    \refstruct{PnlVect} \ptr m, \refstruct{PnlVectInt} \ptr im, char d}
  \sshortdescribe Idem as \reffun{pnl_mat_min} and \var{index} contains the
  indices of the minima. If \var{index==NULL}, the indices are not computed.

\item \describefun{void}{pnl_mat_max_index}{const \refstruct{PnlMat} \ptr  A, 
    \refstruct{PnlVect} \ptr M, \refstruct{PnlVectInt} \ptr iM, char d}
  \sshortdescribe Idem as \reffun{pnl_mat_max} and \var{index} contains the
  indices of the maxima. If \var{index==NULL}, the indices are not computed.

\item \describefun{void}{pnl_mat_minmax_index}{const \refstruct{PnlMat} \ptr 
    A, \refstruct{PnlVect} \ptr m, \refstruct{PnlVect} \ptr M,
    \refstruct{PnlVectInt} \ptr im, \refstruct{PnlVectInt} \ptr iM, char d}
  \sshortdescribe Idem as \reffun{pnl_mat_minmax} and \var{im} contains the
  indices of the minima and \var{iM} contains the indices of the minima. If
  \var{im==NULL} (resp. \var{iM==NULL}, the indices of the minima
  (resp. maxima) are not computed.

\item \describefun{void}{pnl_mat_qsort}{\refstruct{PnlMat} \ptr , char dir, char order}
  \sshortdescribe Sorts a matrix using a quick sort algorithm according to
  \var{order} (\verb!'i'! for increasing or \verb!'d'! for decreasing). The parameter \var{dir} determines
  whether the matrix is sorted by rows or columns. If \var{dir='c'}, each row
  is sorted independtly of the others whereas if \var{dir='r'}, each column
  is sorted independtly of the others.

\item \describefun{void}{pnl_mat_qsort_index}{\refstruct{PnlMat} \ptr ,
    \refstruct{PnlMatInt} *index, char dir, char order}
  \sshortdescribe Sorts a matrix using a quick sort algorithm according to
  \var{order} (\verb!'i'! for increasing or \verb!'d'! for decreasing). The
  parameter \var{dir} determines whether the matrix is sorted by rows or
  columns. If \var{dir='c'}, each row is sorted independently of the others
  whereas if \var{dir='r'}, each column is sorted independently of the
  others. In addition to the function \reffun{pnl_mat_qsort}, the permutation
  index is computed and stored into \var{index}.
\end{itemize}


\paragraph{Standard matrix operations}
\begin{itemize}
\item \describefun{void}{pnl_mat_plus_mat}{\refstruct{PnlMat} \ptr lhs, const
    \refstruct{PnlMat} \ptr rhs} 
  \sshortdescribe In-place matrix matrix addition  

\item \describefun{void}{pnl_mat_minus_mat}{\refstruct{PnlMat} \ptr lhs, 
    const \refstruct{PnlMat} \ptr rhs} 
  \sshortdescribe In-place matrix matrix substraction  
  
\item \describefun{void}{pnl_mat_sq_transpose}{\refstruct{PnlMat} \ptr M}
  \sshortdescribe In-place transposition of square matrices  

\item \describefun{\refstruct{PnlMat} \ptr }{pnl_mat_transpose}{const
    \refstruct{PnlMat} \ptr M} 
  \sshortdescribe Transposition of matrices

\item \describefun{void}{pnl_mat_axpy}{double alpha, const \refstruct{PnlMat}
    \ptr A, \refstruct{PnlMat} \ptr B}
  \sshortdescribe Computes \var{B := alpha * A + B}

\item \describefun{void}{pnl_mat_dger}{double alpha, const \refstruct{PnlVect}
    \ptr x, const \refstruct{PnlVect} \ptr y, \refstruct{PnlMat} \ptr A}
  \sshortdescribe Computes \var{A := alpha x * y' + A}

\item \describefun{\refstruct{PnlVect} \ptr }{pnl_mat_mult_vect}{const
    \refstruct{PnlMat} \ptr A, const \refstruct{PnlVect} \ptr x} 
  \sshortdescribe Matrix vector multiplication  \var{A * x}

\item \describefun{void}{pnl_mat_mult_vect_inplace}{\refstruct{PnlVect}
    \ptr y, const \refstruct{PnlMat} \ptr A, const \refstruct{PnlVect}
    \ptr x} 
  \sshortdescribe In place matrix vector multiplication  \var{y = A * x}

\item \describefun{\refstruct{PnlVect} \ptr }{pnl_mat_mult_vect_transpose}{const
    \refstruct{PnlMat} \ptr A, const \refstruct{PnlVect} \ptr x} 
  \sshortdescribe Matrix vector multiplication  \var{A' * x}

\item \describefun{void}{pnl_mat_mult_vect_transpose_inplace}{\refstruct{PnlVect}
    \ptr y, const \refstruct{PnlMat} \ptr A, const \refstruct{PnlVect}
    \ptr x} 
  \sshortdescribe In place matrix vector multiplication  \var{y = A' * x}
  
\item \describefun{void}{pnl_mat_lAxpby}{double lambda, const \refstruct{PnlMat}
    \ptr A, const \refstruct{PnlVect} \ptr x, double mu, \refstruct{PnlVect} \ptr b} 
  \sshortdescribe Computes \var{b := lambda A x + mu b}. When \var{mu==0}, the
  content of \var{b} is not used on input and instead \var{b} is resized to
  match \var{A*x}

\item \describefun{void}{pnl_mat_dgemv}{char trans, double lambda, const
    \refstruct{PnlMat} \ptr A, const \refstruct{PnlVect} \ptr x, double mu, 
    \refstruct{PnlVect} \ptr b} \sshortdescribe Computes \var{b := lambda
    op(A) x + mu b}, where \var{op (X) = X} or \var{op (X) = X'}. When
  \var{mu==0}, the content of \var{b} is not used and instead \var{b} is resized
  to match \var{op(A)*x}

\item \describefun{void}{pnl_mat_dgemm}{char transA, char transB, double
    alpha, const \refstruct{PnlMat} \ptr A, const \refstruct{PnlMat} \ptr B, 
    double beta, \refstruct{PnlMat} \ptr C}
  \sshortdescribe Computes \var{C := alpha * op(A) * op (B) + beta *
    C}. When beta=0, the content of \var{C} is unused and instead \var{C}
  is resized to store \var{alpha A \ptr B}. If \var{transA='N'} or
  \var{transA='n'}, \var{op (A) = A}, whereas If \var{transA='T'} or
  \var{transA='t'}, \var{op (A) = A'}. The same holds for \var{transB}.
  
\item \describefun{\refstruct{PnlMat} \ptr }{pnl_mat_mult_mat}{const
    \refstruct{PnlMat} \ptr rhs1, const \refstruct{PnlMat} \ptr rhs2} 
  \sshortdescribe Matrix multiplication  \var{rhs1 * rhs2}

\item \describefun{void}{pnl_mat_mult_mat_inplace}{\refstruct{PnlMat}
    \ptr lhs, const \refstruct{PnlMat} \ptr rhs1, const \refstruct{PnlMat}
    \ptr rhs2} 
  \sshortdescribe In-place matrix multiplication  \var{lhs = rhs1 * rhs2}
\end{itemize}

\subsubsection{Functions specific to base type {\tt double}}

\paragraph{Standard matrix operations}
\begin{itemize}

\item \describefun{double}{pnl_mat_scalar_prod_A}{const \refstruct{PnlMat}
    \ptr A, const \refstruct{PnlVect} \ptr x, const \refstruct{PnlVect} \ptr y}
  \sshortdescribe Computes \var{y' * A * y}

  
\item \describefun{void}{pnl_mat_exp}{\refstruct{PnlMat} \ptr B, 
    const \refstruct{PnlMat} \ptr A}
  \sshortdescribe Computes the matrix exponential \var{B = exp(A)}.

\item \describefun{void}{pnl_mat_log}{\refstruct{PnlMat} \ptr B, 
    const \refstruct{PnlMat} \ptr A}
  \sshortdescribe Computes the matrix logarithm \var{B = log(A)}. For the
  moment, this function only works if \var{A} is diagonalizable.

\item \describefun{void}{pnl_mat_eigen}{\refstruct{PnlVect} *v, \refstruct{PnlMat} \ptr P, 
    const \refstruct{PnlMat} \ptr A, int with_eigenvector}
  \sshortdescribe Computes the eigenvalues (storred in \var{v}) and optionally
  the eigenvectors storred columnwise in \var{P} when
  \var{with_eigenvector==TRUE}. If \var{A} is symmetric, \var{P} is orthonormalized.
\end{itemize}

\paragraph{Linear systems and matrix decompositions}

The following functions are designed to solve linear system of the from \var{A
  x = b} where \var{A} is a matrix and \var{b} is a vector except in the
functions \reffun{pnl_mat_syslin_mat} and \reffun{pnl_mat_chol_syslin_mat}
which expect the right hand side member to be a matrix too. Whenever the
vector \var{b} is not needed once the system is solved, you should consider
using ``inplace'' functions.

\begin{itemize}
\item \describefun{void}{pnl_mat_chol}{\refstruct{PnlMat} \ptr M}
  \sshortdescribe Computes the Cholesky decomposition of \var{M}. \var{M} must
  be symmetric, the positivity is tested in the algorithm. On exist, the lower
  part of \var{M} contains the Cholesky decomposition and the upper part is
  set to zero.

\item \describefun{void}{pnl_mat_chol_robust}{\refstruct{PnlMat} \ptr M}
  \sshortdescribe Same function as \reffun{pnl_mat_chol} except that if a negative
  eigen value greater than a given threshold is found, this eigenvalue is
  considered to be that positive.

\item \describefun{void}{pnl_mat_lu}{\refstruct{PnlMat} \ptr A, 
    \refstruct{PnlPermutation} \ptr p} 
  \sshortdescribe Computes a P A = LU factorization. \var{P} must be an
  already allocated  \refstruct{PnlPermutation}. On exit the decomposition is
  stored in \var{A}, the lower part of \var{A} contains L while the upper part
  (including the diagonal terms) contains U. Remember that the diagonal
  elements of \var{L} are all 1.
  
\item \describefun{void}{pnl_mat_qr}{\refstruct{PnlMat} \ptr Q,
  \refstruct{PnlMat} \ptr R, \refstruct{PnlPermutation} \ptr p,
  \refstruct{PnlMat} \ptr A} 
  \sshortdescribe Computes a \var{A P = QR} decomposition. If on entry
  \var{P=NULL}, then the decomposition is computed without pivoting, i.e
  \var{A = QR}. When $P \ne NULL$, \var{P} must be an already allocated
  \refstruct{PnlPermutation}. \var{Q} is an orthogonal matrix, i.e
  $\var{Q}^{-1} = \var{Q}^{T}$ and \var{R} is an upper triangualr matrix. The
  use os pivoting improves the numerical stability when \var{A} is almost rank
  deficient, i.e when the smallest eigenvalue of \var{A} is very close to $0$.

\item \describefun{void}{pnl_mat_upper_syslin}{\refstruct{PnlVect}
    \ptr x, const \refstruct{PnlMat} \ptr U, const \refstruct{PnlVect}\ptr b}
  \sshortdescribe Solves an upper triangular linear system \var{U x = b}

\item \describefun{void}{pnl_mat_lower_syslin}{\refstruct{PnlVect}
    \ptr x, const \refstruct{PnlMat} \ptr L, const \refstruct{PnlVect}\ptr b}
  \sshortdescribe Solves a lower triangular linear system  \var{L x = b}
  
\item \describefun{void}{pnl_mat_chol_syslin}{\refstruct{PnlVect} \ptr x, 
    const \refstruct{PnlMat} \ptr chol, const \refstruct{PnlVect} \ptr b} 
  \sshortdescribe Solves a symmetric definite positive linear system A x = b, 
  in which \var{chol} is assumed to be the Cholesky decomposition of A
  computed by \reffun{pnl_mat_chol}

\item \describefun{void}{pnl_mat_chol_syslin_inplace}{
    const \refstruct{PnlMat} \ptr chol, \refstruct{PnlVect} \ptr b} 
  \sshortdescribe Solves a symmetric definite positive linear system A x = b, 
  in which \var{chol} is assumed to be the Cholesky decomposition of A
  computed by \reffun{pnl_mat_chol}. The solution of the system is stored in
  \var{b} on exit.

\item \describefun{void}{pnl_mat_lu_syslin}{\refstruct{PnlVect} \ptr x, const
    \refstruct{PnlMat} \ptr LU, const \refstruct{PnlPermutation} \ptr p, 
    const \refstruct{PnlVect} \ptr b} 
  \sshortdescribe Solves a linear system A x = b using a LU decomposition.
  \var{LU} and \var{P} are assumed to be the PA = LU decomposition as computed
  by \reffun{pnl_mat_lu}. In particular, the structure of the matrix \var{LU}
  is the following : the lower part of \var{A} contains L while the upper part
  (including the diagonal terms) contains U. Remember that the diagonal
  elements of \var{L} are all 1.

\item \describefun{void}{pnl_mat_lu_syslin_inplace}{const
    \refstruct{PnlMat} \ptr LU, const \refstruct{PnlPermutation} \ptr p, 
    \refstruct{PnlVect} \ptr b} 
  \sshortdescribe Solves a linear system A x = b using a LU decomposition.
  \var{LU} and \var{P} are assumed to be the PA = LU decomposition as computed
  by \reffun{pnl_mat_lu}. In particular, the structure of the matrix \var{LU}
  is the following : the lower part of \var{A} contains L while the upper part
  (including the diagonal terms) contains U. Remember that the diagonal
  elements of \var{L} are all 1. The solution of the system is stored in \var{b}
  on exit.
  
\item \describefun{void}{pnl_mat_syslin}{\refstruct{PnlVect} \ptr x, const
    \refstruct{PnlMat} \ptr A, const \refstruct{PnlVect} \ptr b} 
  \sshortdescribe Solves a linear system A x = b using a LU factorization
  which is computed inside this function.

\item \describefun{void}{pnl_mat_syslin_inplace}{\refstruct{PnlMat} \ptr A, 
    \refstruct{PnlVect} \ptr b} 
  \sshortdescribe Solves a linear system A x = b using a LU factorization
  which is computed inside this function. The solution of the system is stored
  in \var{b} and \var{A} is overwritten by its LU decomposition.

\item \describefun{void}{pnl_mat_syslin_mat}{\refstruct{PnlMat}\ptr A, 
    \refstruct{PnlMat} \ptr B} 
  \sshortdescribe Solves a linear system A X = B using a LU factorization
  which is computed inside this function. \var{A} and  \var{B} are
  matrices. \var{A} must be square. The solution of the system is stored in
  \var{B} on exit. On exit, \var{A} contains the LU decomposition of the input
  matrix which is lost.

\item \describefun{void}{pnl_mat_chol_syslin_mat}{\refstruct{PnlMat}\ptr A, 
    \refstruct{PnlMat} \ptr B}
  \sshortdescribe Solves a linear system A X = B
  using a Cholesky factorization which is computed inside this
  function. \var{A} and \var{B} are matrices. \var{A} must be symmetric
  positive definite. The solution of the system is stored in \var{B} on
  exit. On exit, \var{A} contains the Cholesky decomposition of the input
  matrix which is lost.

\item \describefun{void}{pnl_mat_ls}{\refstruct{PnlMat}\ptr A, 
    \refstruct{PnlVect} \ptr b}
  \sshortdescribe Solves a linear system A x = b in the least square sense,
  i.e. $\var{x} = \arg\min_U \| A * u - b\|^2$. The solution is stored into
  \var{b} on exit. It internally uses a \var{AP = QR} decomposition.

\item \describefun{void}{pnl_mat_ls_mat}{\refstruct{PnlMat}\ptr A,
    \refstruct{PnlMat} \ptr B}
  \sshortdescribe Solves a linear system A X = B with \var{A} and \var{B} two
  matrices in the least square sense, i.e. $\var{X} = \arg\min_U \| A * U -
  B\|^2$. The solution is stored into \var{B} on exit. It internally uses a
  \var{AP = QR} decomposition. Same function as \reffun{pnl_mat_ls} but handles
  several r.h.s.

\end{itemize}


The following functions are designed to invert matrices. The authors provide
these functions although they cannot find good reasons to use them. Note that
to solve a linear system, one must used the \var{syslin} functions and not
invert the system matrix because it is much longer.
\begin{itemize}
\item \describefun{void}{pnl_mat_upper_inverse}{\refstruct{PnlMat} \ptr A, 
    const \refstruct{PnlMat} \ptr B}
  \sshortdescribe Inversion of an upper triangular matrix  

\item \describefun{void}{pnl_mat_lower_inverse}{\refstruct{PnlMat} \ptr A, 
    const \refstruct{PnlMat} \ptr B}
  \sshortdescribe Inversion of a lower triangular matrix  

\item \describefun{void}{pnl_mat_chol_inverse}{\refstruct{PnlMat}
    \ptr inverse, const \refstruct{PnlMat} \ptr A}
  \sshortdescribe Computes the inverse of a symmetric definite positive matrix
  A and stores the result into \var{inverse}. The inverse is computed using the
  Cholesky factorization of \var{A}.

\item \describefun{void}{pnl_mat_inverse}{\refstruct{PnlMat}
    \ptr inverse, const \refstruct{PnlMat} \ptr A}
  \sshortdescribe Computes the inverse of a matrix A and stores the result
  into \var{inverse}. The inverse is computed using a LU factorization of
  \var{A}.
\end{itemize}

\subsubsection{Permutations}

\begin{verbatim}
typedef PnlVectInt PnlPermutation;
\end{verbatim}

\begin{itemize}
\item \describefun{\refstruct{PnlPermutation} \ptr }{pnl_permutation_create}{int n}
  \sshortdescribe Creates of a \refstruct{PnlPermutation} of size \var{n}.  

\item \describefun{void}{pnl_permutation_init}{\refstruct{PnlPermutation} \ptr p}
  \sshortdescribe Initializes an existing permutation to the identity permutation.  

\item \describefun{void}{pnl_permutation_swap}{\refstruct{PnlPermutation} \ptr p, int i, int j}
  \sshortdescribe Swaps two elements of a permutation.  

\item \describefun{void}{pnl_permutation_free}{\refstruct{PnlPermutation} \ptr \ptr p}
  \sshortdescribe Frees a \refstruct{PnlPermutation}.

\item \describefun{void}{pnl_vect_permute}{\refstruct{PnlVect} \ptr px, const
    \refstruct{PnlVect} \ptr x, const \refstruct{PnlPermutation} \ptr p} 
  \sshortdescribe Applies a \refstruct{PnlPermutation} to a \refstruct{PnlVect}.  

\item \describefun{void}{pnl_vect_permute_inplace}{\refstruct{PnlVect} \ptr x, 
    const \refstruct{PnlPermutation} \ptr p} 
  \sshortdescribe Applies a \refstruct{PnlPermutation} to a
  \refstruct{PnlVect} in-place.  
  
\item \describefun{void}{pnl_permutation_fprint}{FILE \ptr fic, const \refstruct{PnlPermutation} \ptr p}
  \sshortdescribe Prints a permutation to a file.  

\item \describefun{void}{pnl_permutation_print}{const \refstruct{PnlPermutation} \ptr p}
  \sshortdescribe Prints a permutation to the standard output.  
\end{itemize}


%% tridiag

\subsection{Tridigonal matrix}
\subsubsection{Short Description}

The structures and functions related to tridigonal matrices are declared in
\verb!pnl_tridiag_matrix.h!. 

We only store the three main digonals as three vectors.

\begin{verbatim}
typedef struct PnlTriDiagMat{
  int size; /*!< number of rows, the matrix must be square */
  double *D; /*!< diagonal elements */
  double *DU; /*!< upper diagonal elements */
  double *DL; /*!< lower diagonal elements */
} PnlTriDiagMat;
\end{verbatim}

\var{size} is the size of the matrix, \var{D} is an array of size \var{size}
containing the diagonal terms. \var{DU},
\var{DL} are two arrays of size \var{size-1} containing respectively the upper
diagonal ($M_{i, i+1}$) and the lower diagonal ($M_{i-1, i}$). 
\subsubsection{Functions}
\paragraph{Constructors and destructors}
\begin{itemize}
  \item 
    \describefun{\refstruct{PnlTriDiagMat} \ptr }{pnl_tridiagmat_create}{int size}
    \sshortdescribe Creates a \refstruct{PnlTriDiagMat}
  \item \describefun{\refstruct{PnlTriDiagMat} \ptr }{pnl_tridiagmat_create_from_double}{int size, double x}
    \sshortdescribe Creates a \refstruct{PnlTriDiagMat} with the 3 diagonals
    filled with \var{x}
  \item \describefun{\refstruct{PnlTriDiagMat} \ptr }{pnl_tridiagmat_create_from_two_double}{int size, double x, double y}
    \sshortdescribe Creates a \refstruct{PnlTriDiagMat}  with the diagonal
    filled with \var{x} and the upper and lower diagonals filled with \var{y}
  \item \describefun{\refstruct{PnlTriDiagMat}
      \ptr }{pnl_tridiagmat_create_from_ptr}{int size, const double
      \ptr lower_D, const double \ptr D, const double \ptr upper_D}
    \sshortdescribe Creates a \refstruct{PnlTriDiagMat}  
  \item \describefun{\refstruct{PnlTriDiagMat} \ptr }{pnl_tridiagmat_create_from_mat}
    {const \refstruct{PnlMat} \ptr mat}
    \sshortdescribe Creates a tridiagonal matrix from a full matrix (all the
    elements but the 3 diagonal ones are ignored).
  \item \describefun{\refstruct{PnlMat} \ptr }{pnl_tridiagmat_to_mat}
    {const \refstruct{PnlTriDiagMat} \ptr T}
    \sshortdescribe Creates a full matrix from a tridiagonal one.
  \item \describefun{\refstruct{PnlTriDiagMat} \ptr }{pnl_tridiagmat_copy}
    {const \refstruct{PnlTriDiagMat} \ptr T}
    \sshortdescribe Copies a tridiagonal matrix.
  \item \describefun{void}{pnl_tridiagmat_clone}
    {\refstruct{PnlTriDiagMat} \ptr clone, const \refstruct{PnlTriDiagMat} \ptr T}
    \sshortdescribe Copies the content of \var{T} into \var{clone}
  \item \describefun{void }{pnl_tridiagmat_free}{\refstruct{PnlTriDiagMat} \ptr \ptr \refstruct{v}}
    \sshortdescribe Frees a \refstruct{PnlTriDiagMat}  
  \item \describefun{int}{pnl_tridiagmat_resize}{\refstruct{PnlTriDiagMat} \ptr \refstruct{v}, int size}
    \sshortdescribe Resizes a \refstruct{PnlTriDiagMat}.  
\end{itemize}
\paragraph{Accessing elements}
\begin{itemize}
  \item \describefun{void}{pnl_tridiagmat_set}{\refstruct{PnlTriDiagMat} \ptr self, int d, int up, double x}
    \sshortdescribe Sets \var{self[d, d+up] = x}, \var{up} can be $\{-1, 0, 1\}$.  
  \item \describefun{double}{pnl_tridiagmat_get}{const \refstruct{PnlTriDiagMat} \ptr self, int d, int up}
    \sshortdescribe Gets \var{self[d, d+up]}, \var{up} can be $\{-1, 0, 1\}$.  
  \item \describefun{double \ptr }{pnl_tridiagmat_lget}{\refstruct{PnlTriDiagMat} \ptr self, int d, int up}
    \sshortdescribe Returns the address \var{self[d, d+up] = x}, \var{up} can be $\{-1, 0, 1\}$.  
\end{itemize}

\paragraph{Printing Matrix}
\begin{itemize}
  \item \describefun{void}{pnl_tridiagmat_fprint}{FILE \ptr fic, const \refstruct{PnlTriDiagMat} \ptr M}
    \sshortdescribe Prints a tri-diagonal matrix to a file.  
  \item \describefun{void}{pnl_tridiagmat_print}{const \refstruct{PnlTriDiagMat} \ptr M}
    \sshortdescribe Prints a tri-diagonal matrix to the standard output.  
\end{itemize}

\paragraph{Algebra operations}
\begin{itemize}
  \item \describefun{void}{pnl_tridiagmat_plus_tridiagmat}{\refstruct{PnlTriDiagMat} \ptr lhs, const \refstruct{PnlTriDiagMat} \ptr rhs}
    \sshortdescribe In-place matrix matrix addition  
  \item \describefun{void}{pnl_tridiagmat_minus_tridiagmat}{\refstruct{PnlTriDiagMat} \ptr lhs, const \refstruct{PnlTriDiagMat} \ptr rhs}
    \sshortdescribe In-place matrix matrix substraction  
  \item \describefun{void}{pnl_tridiagmat_plus_double}{\refstruct{PnlTriDiagMat} \ptr lhs, double x}
    \sshortdescribe In-place matrix scalar addition  
  \item \describefun{void}{pnl_tridiagmat_minus_double}{\refstruct{PnlTriDiagMat} \ptr lhs, double x}
    \sshortdescribe In-place matrix scalar substraction  
  \item \describefun{void}{pnl_tridiagmat_mult_double}{\refstruct{PnlTriDiagMat} \ptr lhs, double x}
    \sshortdescribe In-place matrix scalar multiplication  
  \item \describefun{void}{pnl_tridiagmat_div_double}{\refstruct{PnlTriDiagMat} \ptr lhs, double x}
    \sshortdescribe In-place matrix scalar division
\end{itemize}

\paragraph{Element-wise operations}
\begin{itemize}
  \item \describefun{void}{pnl_tridiagmat_mult_tridiagmat_term}{\refstruct{PnlTriDiagMat} \ptr lhs, const \refstruct{PnlTriDiagMat} \ptr rhs}
    \sshortdescribe In-place matrix matrix term by term product  
  \item \describefun{void}{pnl_tridiagmat_div_tridiagmat_term}{\refstruct{PnlTriDiagMat} \ptr lhs, const \refstruct{PnlTriDiagMat} \ptr rhs}
    \sshortdescribe In-place matrix matrix term by term division  
  \item \describefun{void}{pnl_tridiagmat_map_inplace}{\refstruct{PnlTriDiagMat} \ptr lhs, 
    double(\ptr f)(double)} 
  \sshortdescribe \var{lhs = f(lhs)}.


\item \describefun{void}{pnl_tridiagmat_map_tridiagmat}{\refstruct{PnlTriDiagMat} \ptr lhs, const
  \refstruct{PnlTriDiagMat} \ptr rhs, double(\ptr f)(double, double)} 
  \sshortdescribe \var{lhs = f(lhs, rhs)}.
\end{itemize}

\paragraph{Standard matrix operations \& Linear systems}
\begin{itemize}
  \item \describefun{void}{pnl_tridiagmat_mult_vect_inplace}{\refstruct{PnlVect} \ptr lhs, const \refstruct{PnlTriDiagMat} \ptr mat, const \refstruct{PnlVect} \ptr rhs}
    \sshortdescribe In place matrix multiplication  
  \item \describefun{\refstruct{PnlVect} \ptr }{pnl_tridiagmat_mult_vect}{const \refstruct{PnlTriDiagMat} \ptr mat, const \refstruct{PnlVect} \ptr vec}
    \sshortdescribe Matrix multiplication  
  \item \describefun{void}{pnl_tridiagmat_lAxpby}{double lambda, const \refstruct{PnlTriDiagMat}
      \ptr A, const \refstruct{PnlVect} \ptr x, double mu, \refstruct{PnlVect} \ptr b} 
    \sshortdescribe Computes \var{b := lambda A x + mu b}. When \var{mu==0}, the
    content of \var{b} is not used on input and instead \var{b} is resized to
    match \var{A*x}
  \item \describefun{double}{pnl_tridiagmat_scalar_prod}{const \refstruct{PnlVect} \ptr x,const \refstruct{PnlTriDiagMat} \ptr A, const \refstruct{PnlVect} \ptr y}
    \sshortdescribe Computes \var{x' * A * y}
  \item \describefun{void}{pnl_tridiagmat_syslin_inplace}{const
      \refstruct{PnlTriDiagMat} \ptr M, \refstruct{PnlVect} \ptr b}
    \sshortdescribe Solves the linear system M x = b. The solution is written into
    \var{b} on exit.
  \item \describefun{void}{pnl_tridiagmat_syslin}{\refstruct{PnlVect}
      \ptr x, const \refstruct{PnlTriDiagMat} \ptr M, const \refstruct{PnlVect} \ptr b}
    \sshortdescribe Solves the linear system M x = b. 
\end{itemize}



\subsection{Band Matrix structure}
\subsubsection{Short Description}

\begin{verbatim}
typedef struct
{
  int m; /*!< nb rows */ 
  int n; /*!< nb columns */ 
  int nu; /*!< nb of upperdiagonals */
  int nl; /*!< nb of lowerdiagonals */
  int m_band; /*!< nb rows of the band storage */
  int n_band; /*!< nb columns of the band storage */
  double *array;  /*!< a block to store the bands */  
} PnlBandMat;
\end{verbatim}


The structures and functions related to band matrices are declared in
\verb!pnl_band_matrix.h!. 


\subsubsection{Functions}
\paragraph{Constructors and destructors}
\begin{itemize}
\item \describefun{\refstruct{PnlBandMat}\ptr }{pnl_bandmat_create}{int m, int n, int
    nl, int nu}
  \sshortdescribe Creates a band matrix of size \var{m x n} with \var{nl} lower
  diagonals and \var{nu} upper diagonals.

\item \describefun{\refstruct{PnlBandMat}\ptr }{pnl_bandmat_create_from_mat}{const
    \refstruct{PnlMat} \ptr BM, int nl, int nu}
  \sshortdescribe Extracts a band matrix from a \refstruct{PnlMat}.

\item \describefun{void}{pnl_bandmat_free}{\refstruct{PnlBandMat}\ptr\ptr}
  \sshortdescribe Frees a band matrix.

\item \describefun{void}{pnl_bandmat_clone}{\refstruct{PnlBandMat} \ptr clone, 
  const \refstruct{PnlBandMat} \ptr M}
  \sshortdescribe Copies the band matrix \var{M} into \var{clone}. No new
  \refstruct{PnlBandMat} is created.

\item \describefun{\refstruct{PnlBandMat}\ptr }{pnl_bandmat_copy}{\refstruct{PnlBandMat} \ptr BM}
  \sshortdescribe Creates a new band matrix which is a copy of \var{BM}. Each
  band matrix owns its data array.

\item \describefun{\refstruct{PnlMat}\ptr }{pnl_bandmat_to_mat}{\refstruct{PnlBandMat} \ptr BM}
  \sshortdescribe Creates a full matrix from a band matrix.

\item \describefun{int}{pnl_bandmat_resize}{\refstruct{PnlBandMat} \ptr BM, int
  m, int n, int nl, int nu}
  \sshortdescribe Resizes \var{BM} to store a \var{m x n} band matrix with
  \var{nu} upper diagonals and \var{nl} lower diagonals.
\end{itemize}
\paragraph{Accessing elements}
\begin{itemize}
\item \describefun{void}{pnl_bandmat_set}{\refstruct{PnlBandMat}
    \ptr M, int i, int j, double x}
  \sshortdescribe $M_{i, j}=x$.

\item \describefun{void}{pnl_bandmat_get}{\refstruct{PnlBandMat}
    \ptr M, int i, int j}
    \sshortdescribe Returns $M_{i, j}$.

\item \describefun{void}{pnl_bandmat_lget}{\refstruct{PnlBandMat}
    \ptr M, int i, int j}
    \sshortdescribe Returns the address $\&(M_{i, j})$.
\item \describefun{void}{pnl_bandmat_set_double}{\refstruct{PnlBandMat}
    \ptr M, double x}
    \sshortdescribe Sets all the elements of \var{M} to \var{x}.

  \item \describefun{void}{pnl_bandmat_print_as_full}{\refstruct{PnlBandMat}
    \ptr M}
    \sshortdescribe Prints a band matrix in a full format.
\end{itemize}

\subparagraph{Element wise operations}

\begin{itemize}
  \item \describefun{void}{pnl_bandmat_plus_double}{\refstruct{PnlBandMat} \ptr lhs, 
  double x} 
    \sshortdescribe In-place addition, \var{lhs += x} 

  \item \describefun{void}{pnl_bandmat_minus_double}{\refstruct{PnlBandMat} \ptr lhs, 
  double x} 
  \sshortdescribe In-place substraction \var{lhs -= x} 

\item \describefun{void}{pnl_bandmat_div_double}{\refstruct{PnlBandMat}
  \ptr lhs, double x} 
  \sshortdescribe \var{lhs = lhs ./ x}

\item \describefun{void}{pnl_bandmat_mult_double}{\refstruct{PnlBandMat}
  \ptr lhs, double x} 
  \sshortdescribe \var{lhs = lhs * x}

\item \describefun{void}{pnl_bandmat_plus_bandmat}{\refstruct{PnlBandMat} \ptr lhs, 
    const \refstruct{PnlBandMat} \ptr rhs} 
    \sshortdescribe In-place addition, \var{lhs += rhs} 

\item \describefun{void}{pnl_bandmat_minus_bandmat}{\refstruct{PnlBandMat} \ptr lhs, 
    const \refstruct{PnlBandMat} \ptr rhs} 
  \sshortdescribe In-place substraction \var{lhs -= rhs} 

\item \describefun{void}{pnl_bandmat_inv_term}{\refstruct{PnlBandMat} \ptr lhs}
  \sshortdescribe In-place term by term  inversion \var{lhs = 1 ./ rhs} 

\item \describefun{void}{pnl_bandmat_div_bandmat_term}{\refstruct{PnlBandMat}
    \ptr lhs, const \refstruct{PnlBandMat} \ptr rhs} 
  \sshortdescribe In-place term by term  division \var{lhs = lhs ./ rhs}

\item \describefun{void}{pnl_bandmat_mult_bandmat_term}{\refstruct{PnlBandMat}
    \ptr lhs, const \refstruct{PnlBandMat} \ptr rhs} 
  \sshortdescribe In-place term by term multiplication  \var{lhs = lhs .* rhs}

\item \describefun{void}{pnl_bandmat_map}{\refstruct{PnlBandMat} \ptr lhs, const
    \refstruct{PnlBandMat} \ptr rhs, double(\ptr f)(double)} 
  \sshortdescribe \var{lhs = f(rhs)}

\item \describefun{void}{pnl_bandmat_map_inplace}{\refstruct{PnlBandMat} \ptr lhs, double(\ptr f)(double)}
  \sshortdescribe  \var{lhs = f(lhs)}

\item \describefun{void}{pnl_bandmat_map_bandmat}{\refstruct{PnlBandMat} \ptr lhs, const
  \refstruct{PnlBandMat} \ptr rhs, double(\ptr f)(double,double)} 
  \sshortdescribe \var{lhs = f(lhs,rhs)}
\end{itemize}


\paragraph{Standard matrix operations \& Linear system}
\begin{itemize}
\item \describefun{void}{pnl_bandmat_lAxpby}{double lambda, const \refstruct{PnlBandMat}
    \ptr A, const \refstruct{PnlVect} \ptr x, double mu, \refstruct{PnlVect} \ptr b} 
  \sshortdescribe Computes \var{b := lambda A x + mu b}. When \var{mu==0}, the
  content of \var{b} is not used on input and instead \var{b} is resized to
  match the size of \var{A*x}.
\item \describefun{void}{pnl_bandmat_mult_vect_inplace}{\refstruct{PnlVect} \ptr
  y, const \refstruct{PnlBandMat} \ptr BM, const \refstruct{PnlVect} \ptr x}
  \sshortdescribe \var{y = BM * x}
\item 
  \describefun{void}{pnl_bandmat_syslin_inplace}{\refstruct{PnlBandMat}
    \ptr M, \refstruct{PnlVect} \ptr b}
    \sshortdescribe Solves the linear system \var{M x = b} with \var{M} a \refstruct{PnlBandMat}.
  {\bf Note} that M is modified on output and becomes unusable. On exit, the
  solution \var{x} is stored in \var{b}.
\item 
  \describefun{void}{pnl_bandmat_syslin}{\refstruct{PnlVect} \ptr x,\refstruct{PnlBandMat}
    \ptr M, \refstruct{PnlVect} \ptr b}
    \sshortdescribe Solves the linear system \var{M x = b} with \var{M} a \refstruct{PnlBandMat}.
  {\bf Note} that M is modified on output and becomes unusable. 
\item \describefun{void}{pnl_bandmat_lu}{\refstruct{PnlBandMat} \ptr BM,
  \refstruct{PnlVectInt} \ptr p}
  \sshortdescribe Computes the LU decomposition with partial pivoting with row
  interchanges. On exit, \var{BM} is enlarged to store the LU decomposition. On
  exit, \var{p} stores the permutation applied to the rows. Note that the Lapack format
  is used to store \var{p}, this format differs from the one used by
  \refstruct{PnlPermutation}.
\item  \describefun{void}{pnl_bandmat_lu_syslin_inplace}{const \refstruct{PnlBandMat} \ptr M, 
  \refstruct{PnlVectInt} \ptr p, \refstruct{PnlVect} \ptr b} 
  \sshortdescribe Solves the band linear system \var{M x = b} where \var{M} is
  the LU decomposition computed by \reffun{pnl_bandmat_lu}  and \var{p} the
  associated permutation. On exit, the solution \var{x} is stored in \var{b}.
\item  \describefun{void}{pnl_bandmat_lu_syslin}{\refstruct{PnlVect} \ptr x,
  const \refstruct{PnlBandMat} \ptr M, \refstruct{PnlVectInt} \ptr p, const \refstruct{PnlVect} \ptr b} 
  \sshortdescribe Solves the band linear system \var{M x = b} where \var{M} is the LU
  decomposition computed by \reffun{pnl_bandmat_lu} and \var{p} the associated permutation. 
\end{itemize}

\subsection{Hyper matrices}
\subsubsection{Short description}

The Hyper matrix types and related functions are defined in the header \verb!pnl_matrix.h!.

\begin{verbatim}
typedef struct PnlHMat{
  int ndim; /*!< nb dimensions */ 
  int *dims; /*!< pointer to store the value of the ndim dimensions */ 
  int mn; /*!< product dim_1 *...*dim_ndim */
  double *array; /*!< pointer to store */
} PnlHMat;

typedef struct PnlHMatUint{
  int ndim; /*!< nb dimensions */ 
  int *dims; /*!< pointer to store the value of the ndim dimensions */ 
  int mn; /*!< product dim_1 *...*dim_ndim */
  uint *array; /*!< pointer to store */
} PnlHMatUint;

typedef struct PnlHMatInt{
  int ndim; /*!< nb dimensions */ 
  int *dims; /*!< pointer to store the value of the ndim dimensions */ 
  int mn; /*!< product dim_1 *...*dim_ndim */
  int *array; /*!< pointer to store */
} PnlHMatInt;

typedef struct PnlHMatComplex{
  int ndim; /*!< nb dimensions */ 
  int *dims; /*!< pointer to store the value of the ndim dimensions */ 
  int mn; /*!< product dim_1 *...*dim_ndim */
  dcomplex *array; /*!< pointer to store */
} PnlHMatComplex;
\end{verbatim}
\var{ndim} is the number of dimensions, \var{dim} is an array to store the
size of each dimension and \var{nm} contains the product of the sizes of each
dimension. \var{array} is an array of size \var{mn} containing the data of the
matrix stored linewise.


\subsubsection{Generic Functions}
\paragraph{General functions}
These functions exist for all types of hypermatrices no matter what the basic type
is. The following conventions are used to name functions operating on hypermatrices.
Here is the table of prefixes used for the different basic types.

\begin{center}
  \begin{tabular}[t]{lll}
    type & prefix & BASE\\
    \hline
    double & pnl_hmat & double \\
    \hline
    int & pnl_hmat_int & int \\
    \hline
    uint & pnl_hmat_uint & uint\\
    \hline
    dcomplex & pnl_hmat_complex & dcomplex
  \end{tabular}
\end{center}

In this paragraph we present the functions operating on \refstruct{PnlMat}
which exist for all types. To deduce the prototypes of these functions for
other basic types, one must replace {\tt pnl_mat} and {\tt double} according
the above table.


\subsubsection{Functions}

\paragraph{Constructors and destructors}
\begin{itemize}
\item \describefun{\refstruct{PnlHMat} \ptr }{pnl_hmat_create}{int ndim, const int \ptr dims}
  
\item 
  \describefun{\refstruct{PnlHMat} \ptr }{pnl_hmat_create_from_double}{int ndim, const int \ptr dims, double x}
  
\item 
  \describefun{\refstruct{PnlHMat} \ptr }{pnl_hmat_create_from_ptr}{int ndim, const int \ptr dims, const double \ptr x}
  
\item 
  \describefun{void}{pnl_hmat_free}{\refstruct{PnlHMat} \ptr \ptr H}
  
\item \describefun{\refstruct{PnlHMat} \ptr }{pnl_hmat_copy}{const \refstruct{PnlHMat} \ptr H}
  \sshortdescribe Copies a \refstruct{PnlHMat}.
  
\item \describefun{void}{pnl_hmat_clone}{\refstruct{PnlHMat} \ptr clone, const \refstruct{PnlHMat} \ptr H}
  \sshortdescribe Clones a \refstruct{PnlHMat}.
  
\item \describefun{int}{pnl_hmat_resize}{\refstruct{PnlHMat} \ptr H, int ndim, const int \ptr dims}
  \sshortdescribe Resizes a \refstruct{PnlHMat}.
\end{itemize}  

\paragraph{Accessing elements}

\begin{itemize}
\item   \describefun{void}{pnl_hmat_set}{\refstruct{PnlHMat} \ptr self, int \ptr tab, double x}
  \sshortdescribe Sets the element of index \var{tab} to \var{x}.
  
\item \describefun{double}{pnl_hmat_get}{const \refstruct{PnlHMat} \ptr self, int \ptr tab}
  \sshortdescribe Returns the value of the element of index \var{tab} 
  
\item \describefun{double\ptr }{pnl_hmat_lget}{\refstruct{PnlHMat} \ptr self, int \ptr tab}
  \sshortdescribe Returns the address of self[tab] for use as a lvalue.  
\end{itemize}  

\paragraph{Printing hypermatrices}

\begin{itemize}
\item \describefun{void}{pnl_hmat_print}{const \refstruct{PnlHMat} \ptr H}
  \sshortdescribe Prints an hypermatrix.
\end{itemize}

\paragraph{Accessing elements}

\begin{itemize}
\item \describefun{void}{pnl_hmat_plus_hmat}{\refstruct{PnlHMat} \ptr lhs, const \refstruct{PnlHMat} \ptr rhs}
  \sshortdescribe Computes \var{lhs += rhs}.
  
\item \describefun{void}{pnl_hmat_mult_double}{\refstruct{PnlHMat} \ptr lhs, double x}
  \sshortdescribe Computes \var{lhs *= x} where x is a real number.
\end{itemize}

% \subsection{Morse Matrix}
% \subsubsection{Short Description}

% A system of linear equation is called sparse if only a relatively small number
% of its matrix elements $M_{i, j}$ are nonzero. It is wasteful to use full
% structure to solve the linear system because most of the operations devoted to
% solving the system use elements with values zero. Furthermore, for some 
% high dimensional problems, storing the full matrix with its zero elements is not
% possible because of memory limitations.


% In the following, we propose two structures for Sparse Matrices.  Must of the
% algorithms which use sparse matrices can be divided in two steps.  The first
% step is the construction of the matrix. For this, \refstruct{PnlMorseMat} should
% be used. The second step is the resolution of a sparse linear system. We
% have two ways of doing that. The first one is to use a direct method based on
% matrix-decomposition, like the LU decomposition. The \refstruct{PnlSparseMat} is
% implemented to do that. The second one is to use iterative methods like
% Conjugate Gradient, BICGstab or GMRES. These methods are discussed in the next
% section. If we use iterative methods, we can use \refstruct{PnlMorseMat}. 

% \begin{verbatim}
% typedef struct SpRow{
%   int size;  /*!< size of a row */
%   int Max_size; /*!< max size allocation of a row */
%   int    *Index; /*!< pointer to an int array giving the columns or row i */
%   double *Value; /*!< Pointer on values */
% }SpRow;
% \end{verbatim}
% \var{size} is the number of elements, 
% \var{Max_size} is the size of memory allocation.
% \var{Index}, is the pointer containing the index of row or column, 
% \var{Value}, is the pointer containing the value of row or column.
% So for a \refstruct{SpRow} which contains row $i$ of $M$.
% If $k \leq size $ then
% $$M_{i, Index[k]}=Value[k].$$  

% \begin{verbatim}
% typedef struct PnlMorseMat{
%   int m; /*!< nb rows */ 
%   int n; /*!< nb columns */ 
%   SpRow * array; /*!< pointer in each row or col to store no nul coefficients */
%   int RC; /*!< 0 if we use row-wise storage, 1 if we use column-wise storage */ 
% } PnlMorseMat;
% \end{verbatim}
% \var{m} is the number of rows, \var{n} is the number of columns.
% \var{array} is the pointer containing on SpRow array of size n or m (depend of
% RC).
% \var{RC} is an integer to know if the matrix is stored by row or columns.

% \subsubsection{Functions}
% \paragraph{Constructors and destructors}
% \begin{itemize}
% \item \describefun{\refstruct{PnlMorseMat}\ptr }{pnl_morse_mat_create}{int m, 
%     int n, int Max_size_row, int RC}
%   \sshortdescribe Creates an empty \refstruct{PnlMorseMat} with memory
%   allocated for each component of the array. 
% \item
%   \describefun{\refstruct{PnlMorseMat}\ptr }{pnl_morse_mat_create_fromfull}
%   {\refstruct{PnlMat} \ptr FM, int RC}
%   \sshortdescribe Creates a \refstruct{PnlMorseMat} from  a \refstruct{PnlMat}
%   storing only its nonzero elements.

% \item \describefun{void}{pnl_morse_mat_free}{\refstruct{PnlMorseMat}\ptr \ptr  M}
%   \sshortdescribe Frees a \refstruct{PnlMorseMat}

% \item \describefun{int}{pnl_morse_mat_freeze}{PnlMorseMat\ptr  M}
%   \sshortdescribe Sets Max size equal to size for each SpRow and frees the extra
%   memory.

% \item \describefun{\refstruct{PnlMat} \ptr }{pnl_morse_mat_full}
%   {\refstruct{PnlMorseMat}\ptr  M}
%   \sshortdescribe Creates a full matrix from a morse matrix.
% \end{itemize}


% \paragraph{Accessing elements}
% \begin{itemize}
% \item \describefun{double}{ pnl_morse_mat_get}{PnlMorseMat\ptr  M, int i, int j}
%   \sshortdescribe Return $M_{i, j}$. 
% \item \describefun{int}{ pnl_morse_mat_set}{PnlMorseMat\ptr  M, int i, int
%     j, double Val}
%   \sshortdescribe Do $M_{i, j} = Val$. For example, if $RC=1$ and $(i, j)$ is a valid index, replace
%   $array[i]\rightarrow Value[k]$ with $k$ such that $array[i]\rightarrow Index[k]=j$.
%   If $(i, j)$ is not a valid index, add $j$ to $array[i]\rightarrow Index$ and $Val$ to
%   $array[i] \rightarrow Value$ with memory allocation if needed. 
% \item \describefun{double\ptr }{pnl_morse_mat_lget}{PnlMorseMat\ptr  M, int
%     i, int j}
%   \sshortdescribe Returns the address of $M_{i, j}$. For example, 
%   if $RC=1$ and $(i, j)$ is a valid index, replace return address of
%   $array[i]\rightarrow Value[k]$ with $k$ such that $array[i]\rightarrow
%   Index[k]=j$.  If $(i, j)$ is not a valid index, add $j$ to
%   $array[i]\rightarrow Index$ and add element to $array[i] \rightarrow Value$
%   (with memory allocation if needed), returns address of this element. In
%   practice this function is used to do $M_{i, j} += a$.
% \end{itemize}

% \paragraph{Printing Matrix}
% \begin{itemize}
% \item \describefun{void}{pnl_morse_mat_print}{const \refstruct{PnlMorseMat}\ptr M}
% \end{itemize}

% \paragraph{Standard matrix operations}
% \begin{itemize}
% \item \describefun{void}{pnl_morse_mat_mult_vect_inplace}{\refstruct{PnlVect}
%     \ptr lhs, const \refstruct{PnlMorseMat}\ptr M, const \refstruct{PnlVect}
%     \ptr rhs}
%   \sshortdescribe Compute $ lhs=M \ rhs$.
% \item \describefun{\refstruct{PnlVect}\ptr }{pnl_morse_mat_mult_vect}{const
%     \refstruct{PnlMorseMat}\ptr M, const \refstruct{PnlVect} \ptr vec}
%   \sshortdescribe Compute $ vec=M \ vec$.
% \end{itemize}


% \subsection{Sparse Matrix}

% \refstruct{PnlSparseMat} is the cs structure of the Csparse library written by
% Timothy A.Davis.  For the sake of convenience, we have renamed some functions
% and structures. We have also reduced the number of function parameters for non
% expert users in sparse matrices.  In the following, we only use the LU
% factorisation for sparse systems. If the same operator is used at each time
% step, direct methods relying on factorisations are faster than iterative
% methods. When the PDE coefficients are time dependent, the answer is not so
% clear.

% \subsubsection{Functions}
% \paragraph{Constructors and destructors}
% \begin{itemize}
% \item \describefun{\refstruct{PnlSparseMat}
%     \ptr }{pnl_sparse_mat_create_fromfull}{\refstruct{PnlMat} \ptr M}
%   \sshortdescribe Creates a \refstruct{PnlSparseMat} from  a
%   \refstruct{PnlMat} storing only nonzero elements.
% \item \describefun{\refstruct{PnlSparseMat}
%     \ptr }{pnl_sparse_mat_create_frommorse}{\refstruct{PnlMorseMat}\ptr  M}
%   \sshortdescribe Creates a \refstruct{PnlSparseMat} from  a
%   \refstruct{PnlMorseMat} with $M\rightarrow M->RC =1$.
% \item \describefun{void}{pnl_sparse_mat_free}{\refstruct{PnlSparseMat}
%     \ptr \ptr M}
%   \sshortdescribe Frees a \refstruct{PnlSparseMat}.
% \end{itemize}

% \paragraph{Printing Matrix}
% \begin{itemize}
% \item \describefun{void}{pnl_sparse_mat_print}{\refstruct{PnlSparseMat}
%     \ptr A}
%   \sshortdescribe Prints a \refstruct{PnlSparseMat}.
% \end{itemize}

% \subparagraph{Element wise operations}

% \begin{itemize}
% \item \describefun{void}{pnl_sparse_mat_plus_sparse_mat}{\refstruct{PnlSparseMat} \ptr lhs, 
%     const \refstruct{PnlSparseMat} \ptr rhs} 
%   \sshortdescribe In-place addition  

% \item \describefun{void}{pnl_sparse_mat_minus_sparse_mat}{\refstruct{PnlSparseMat} \ptr lhs, 
%     const \refstruct{PnlSparseMat} \ptr rhs} 
%   \sshortdescribe In-place substraction  

% \item \describefun{void}{pnl_sparse_mat_inv_term}{\refstruct{PnlSparseMat} \ptr lhs}
%   \sshortdescribe In-place term by term inversion  

% \item \describefun{void}{pnl_sparse_mat_div_mat_term}{\refstruct{PnlSparseMat}
%     \ptr lhs, const \refstruct{PnlSparseMat} \ptr rhs} 
%   \sshortdescribe In-place term by term division

% \item \describefun{void}{pnl_sparse_mat_mult_mat_term}{\refstruct{PnlSparseMat}
%     \ptr lhs, const \refstruct{PnlSparseMat} \ptr rhs} 
%   \sshortdescribe In-place term by term multiplication  



% \item \describefun{void}{pnl_sparse_mat_map_inplace}{\refstruct{PnlSparseMat} \ptr M, double(\ptr f)(double)}
%   \sshortdescribe Applies function \var{f} to each entry of \var{M}, which
%   is modified on exit.
% \end{itemize}


% \paragraph{Standard matrix operations}
% \begin{itemize}
% \item \describefun{int}{pnl_sparse_mat_gaxpby}{\refstruct{PnlVect} \ptr lhs, 
%     const \refstruct{PnlSparseMat} \ptr M, const \refstruct{PnlVect}
%     \ptr rhs}
%   \sshortdescribe Computes $lhs=lhs+ M * rhs$.
% \item \describefun{int}{pnl_sparse_mat_mult_vect_inplace}{\refstruct{PnlVect}
%     \ptr lhs, const \refstruct{PnlSparseMat} \ptr M, const
%     \refstruct{PnlVect} \ptr rhs}
%   \sshortdescribe Computes $lhs= M * rhs$.
% \end{itemize}

% \subsubsection{LU structure}

% From the sparse matrix, we extract the LU decomposition stored in \refstruct{PnlSparseFactorization}.
% \paragraph{Constructors and desctructors}
% \begin{itemize}
% \item \describefun{\refstruct{PnlSparseFactorization}
%     \ptr }{pnl_sparse_factorization_lu_create}{const \refstruct{PnlSparseMat} \ptr A, double tol}
%   \sshortdescribe Computes the LU factorisation of \var{A}

% \item \describefun{void}{pnl_sparse_factorization_free}{\refstruct{PnlSparseFactorization} \ptr \ptr  F}
%   \sshortdescribe Frees a \refstruct{PnlSparseFactorization}.
% \end{itemize}

% \paragraph{Solving linear systems}

% \begin{itemize}
% \item \describefun{void}{pnl_sparse_factorization_lu_syslin}{const
%     \refstruct{PnlSparseFactorization} \ptr N, PnlVect \ptr b}
%   \sshortdescribe Solves the linear system \var{Nx = b} and stores the solution \var{x}
%   into \var{b} which means that the r.h.s member of the system is overwritten
%   during the resolution of the system. \var{N} is the decomposition computed by
%   \reffun{pnl_sparse_factorization_lu_create}.
% \end{itemize}

%% solver

\subsection{Solver Functions}
\subsubsection{Short Description}

The structures and functions related to solvers are declared in
\verb!pnl_linalgsolver.h!. 

A Left preconditioner solves the problem :
$$ P M x = P b, $$
and whereas right preconditioner solves
$$ M P y  = b, \quad \quad P y = x.$$

%% With some simplifications, the number of algorithm iterations depends on
%% conditioning. Conditioning is ratio of maximum eigenvalue over minimum
%% eigenvalue of $M$. For GMRES algorithm is depend of conditioning of $M^{T}
%% M$. So if we can find $P_L$ and $P_R$ such that $P_L M P_R$ is closed to
%% identity matrix, then preconditioning problem converge faster than initial
%% problem. We have also to solve $P_R y = x$ so $P_R$ has to be constructed to
%% do that fast.

More information is given in {\em Saad, Yousef (2003). Iterative methods for
  sparse linear systems (2nd ed. ed.). SIAM. ISBN 0898715342. OCLC 51266114}.
The reader will find in this book some discussion about right or/and left
preconditioner and a description of the following algorithms.

These algorithms, we implemented with a left preconditioner. Right preconditioner
can be easily computed changing matrix vector multiplication operator from $M \
x $ to $ M \ P_R \ x$ and solving $P_R y = x$ at the end of algorithm.


\subsubsection{Functions}

Three methods are implemented : Conjugate Gradient, BICGstab and GMRES with
restart. For each of them a structure is created to store temporary vectors
used in the algorithm. In some cases, we have to apply iterative methods more
than once : for example to solve at each time step a discrete form of an
elliptic problem come from parabolic problem. In the cases, do not call the constructor and
destructor at each time, but instead use the initialization and solve procedures.

Formally we have, 
\begin{verbatim}
Create iterative method
For each time step
  Initialisation of iterative method
  Solve linear system link to elliptic problem
end for
free iterative method
\end{verbatim}

In these functions, we don't use any particular matrix structure. We give the
matrix vector multiplication as a parameter of the solver. 

\paragraph{Conjugate Gradient method}

Only available for symmetric and positive matrices.
\begin{itemize}
\item 
  \describefun{\refstruct{PnlCGSolver} \ptr }{pnl_cg_solver_create}{int Size, int max-iter, double tolerance}
  \sshortdescribe Creates a new \refstruct{PnlCGSolver} pointer.  
\item \describefun{void}{pnl_cg_solver_initialisation}{\refstruct{PnlCGSolver} \ptr Solver, const \refstruct{PnlVect} \ptr b}
  \sshortdescribe Initialisation of the solver at the beginning of iterative method.  
\item \describefun{void}{pnl_cg_solver_free}{\refstruct{PnlCGSolver} \ptr \ptr Solver}
  \sshortdescribe Destructor of iterative solver  
\item \describefun{int}{pnl_cg_solver_solve}{void(\ptr matrix vector-product)(const void \ptr , const \refstruct{PnlVect} \ptr , const double, const double, \refstruct{PnlVect} \ptr ), const void \ptr Matrix-Data, void(\ptr matrix vector-product-PC)(const void \ptr , const \refstruct{PnlVect} \ptr , const double, const double, \refstruct{PnlVect} \ptr ), const void \ptr PC-Data, \refstruct{PnlVect} \ptr x, const \refstruct{PnlVect} \ptr b, \refstruct{PnlCGSolver} \ptr Solver}
  \sshortdescribe Solves the linear system matrix vector-product is the matrix vector multiplication function matrix vector-product-PC is the preconditionner function Matrix-Data \& PC-Data is data to compute matrix vector multiplication.  
\end{itemize}
\paragraph{BICG stab}
\begin{itemize}
\item \describefun{\refstruct{PnlBICGSolver} \ptr }{pnl_bicg_solver_create}{int Size, int max-iter, double tolerance}
  \sshortdescribe Creates a new \refstruct{PnlBICGSolver} pointer.  
\item \describefun{void}{pnl_bicg_solver_initialisation}{\refstruct{PnlBICGSolver} \ptr Solver, const \refstruct{PnlVect} \ptr b}
  \sshortdescribe Initialisation of the solver at the beginning of iterative method.  
\item \describefun{void}{pnl_bicg_solver_free}{\refstruct{PnlBICGSolver} \ptr \ptr Solver}
  \sshortdescribe Destructor of iterative solver  
\item \describefun{int}{pnl_bicg_solver_solve}{void(\ptr matrix vector-product)(const void \ptr , const \refstruct{PnlVect} \ptr , const double, const double, \refstruct{PnlVect} \ptr ), const void \ptr Matrix-Data, void(\ptr matrix vector-product-PC)(const void \ptr , const \refstruct{PnlVect} \ptr , const double, const double, \refstruct{PnlVect} \ptr ), const void \ptr PC-Data, \refstruct{PnlVect} \ptr x, const \refstruct{PnlVect} \ptr b, \refstruct{PnlBICGSolver} \ptr Solver}
  \sshortdescribe Solves the linear system matrix vector-product is the matrix vector multiplication function matrix vector-product-PC is the preconditioner function Matrix-Data \& PC-Data is data to compute matrix vector multiplication.  
\end{itemize}

\paragraph{GMRES with restart} See {\em Saad, Yousef (2003)} for discussion
about the restart parameter. For GMRES we need to store at the p-th iteration
$p$ vectors of the same size of the right and side. It could be very expensive
in term of memory allocation. So GMRES with restart algorithm stop if
$p=restart$ and restarts the algorithm with the previously computed solution
as initial guess.

Note that if restart equals $m$, we have a classical GMRES algorithm.

\begin{itemize}
\item \describefun{\refstruct{PnlGMRESSolver} \ptr }{pnl_gmres_solver_create}{int Size, int max-iter, int restart, double tolerance}
  \sshortdescribe Creates a new \refstruct{PnlGMRESSolver} pointer.  
\item \describefun{void}{pnl_gmres_solver_initialisation}{\refstruct{PnlGMRESSolver} \ptr Solver, const \refstruct{PnlVect} \ptr b}
  \sshortdescribe Initialisation of the solver at the beginning of iterative method.  
\item \describefun{void}{pnl_gmres_solver_free}{\refstruct{PnlGMRESSolver} \ptr \ptr Solver}
  \sshortdescribe Destructor of iterative solver  
\item \describefun{int}{pnl_gmres_solver_solve}{void(\ptr matrix vector-product)(const void \ptr , const \refstruct{PnlVect} \ptr , const double, const double, \refstruct{PnlVect} \ptr ), const void \ptr Matrix-Data, void(\ptr matrix vector-product-PC)(const void \ptr , const \refstruct{PnlVect} \ptr , const double, const double, \refstruct{PnlVect} \ptr ), const void \ptr PC-Data, \refstruct{PnlVect} \ptr x, const \refstruct{PnlVect} \ptr b, \refstruct{PnlGMRESSolver} \ptr Solver}
  \sshortdescribe Solves the linear system matrix vector-product is the matrix vector multiplication function matrix vector-product-PC is the preconditionner function Matrix-Data \& PC-Data is data to compute matrix vector multiplication.  
\end{itemize}


In the next paragraph, we write all the solvers for \refstruct{PnlMat}. This will be done as
follow: construct an application matrix vector.
\begin{verbatim}
static void pnl_mat_mult_vect_applied(const void *mat, const PnlVect *vec, 
                                      const double a , const double b, 
                                      PnlVect *lhs)
{pnl_mat_lAxpby(a, (PnlMat*)mat, vec, b, lhs);}
\end{verbatim}
and give this as the parameter of the iterative method
\begin{verbatim}
int pnl_mat_cg_solver_solve(const PnlMat * Matrix, const PnlMat * PC, 
                            PnlVect * x, const PnlVect *b, PnlCGSolver * Solver)
{ return pnl_cg_solver_solve(pnl_mat_mult_vect_applied, 
                             Matrix, pnl_mat_mult_vect_applied, 
                             PC, x, b, Solver);}
\end{verbatim}

In practice, we cannot define all iterative methods for all structures.
With this implementation, the user can easily :
\begin{itemize}
\item implement right precondioner, 
\item implement method with sparse matrix and diagonal preconditioner, or
  special combination of this form $\dots$
\end{itemize}


\paragraph{Iterative algorithms for \refstruct{PnlMat}}


\begin{itemize}
\item \describefun{int}{pnl_mat_cg_solver_solve}{const \refstruct{PnlMat} \ptr M, const \refstruct{PnlMat} \ptr PC, \refstruct{PnlVect} \ptr x, const \refstruct{PnlVect} \ptr b, \refstruct{PnlCGSolver} \ptr Solver}
  \sshortdescribe Solves the linear system \var{M x = b} with preconditionner PC.  
\item \describefun{int}{pnl_mat_bicg_solver_solve}{const \refstruct{PnlMat} \ptr M, const \refstruct{PnlMat} \ptr PC, \refstruct{PnlVect} \ptr x, const \refstruct{PnlVect} \ptr b, \refstruct{PnlBICGSolver} \ptr Solver}
  \sshortdescribe Solves the linear system \var{M x = b} with preconditionner PC.  
\item \describefun{int}{pnl_mat_gmres_solver_solve}{const \refstruct{
      PnlMat} \ptr M, const \refstruct{PnlMat} \ptr PC, 
    \refstruct{PnlVect} \ptr x, \refstruct{PnlVect} \ptr b, \refstruct{PnlGMRESSolver} \ptr Solver}
  \sshortdescribe Solve the linear system \var{M x = b} with preconditionner PC.
\end{itemize}


\section{Probabilistic methods}
\subsection{Cumulative distribution Functions}
\subsubsection{Short Description}
\subsubsection*{Functions}



For various distribution functions, we provide functions named
\var{pnl_cdf_xxx} where \var{xxx} is the abbreviation of the distribution
name. All these functions are based on the same prototype
\begin{equation*}
  p = 1-q; \quad p = \int^x density(u) du 
\end{equation*}

\begin{itemize}
\item \var{which} If \var{which=1}, it computes \var{p} and \var{q}. If
  \var{which=2}, it computes \var{x}. For higher values of \var{which} it
  computes one the parameters characterizing the distribution using all the
  others, \var{p, q, x}.
\item \var{p} the probability $\int^x density(u) du $
\item \var{q} $= 1 - p$
\item \var{x} the upper bound of the integral
\item \var{status} an integer which indicates on exit the success of the
  computation. (0) if calculation completed correctly. (-I) if the input
  parameter number I was out of range. (1) if the answer appears to be lower
  than the lowest search bound.  (2) if the answer appears to be higher than
  the greatest search bound.  (3) if $p + q \ne 1$.
\item \var{bound} is undefined if STATUS is 0.  Bound exceeded by parameter
  number I if STATUS is negative. Lower search bound if STATUS is 1.  Upper
  search bound if STATUS is 2.
\end{itemize}

\begin{itemize}
\item \describefun{void}{pnl_cdf_bet}{int \ptr which, double \ptr p, double
    \ptr q, double \ptr x, double \ptr y, double \ptr a, double \ptr b, 
    int \ptr status, double \ptr bound}
  \sshortdescribe Cumulative Distribution Function BETA distribution.  

\item \describefun{void}{pnl_cdf_bin}{int \ptr which, double \ptr p, double
    \ptr q, double \ptr x, double \ptr xn, double \ptr pr, double
    \ptr ompr, int \ptr status, double \ptr bound}
  \sshortdescribe Cumulative Distribution Function BINa distribution.

\item \describefun{void}{pnl_cdf_chi}{int \ptr which, double \ptr p, double
    \ptr q, double \ptr x, double \ptr df, int \ptr status, double
    \ptr bound}
  \sshortdescribe Cumulative Distribution Function CHI-Square distribution.  

\item \describefun{void}{pnl_cdf_chn}{int \ptr which, double \ptr p, double
    \ptr q, double \ptr x, double \ptr df, double \ptr pnonc, int
    \ptr status, double \ptr bound}
  \sshortdescribe Cumulative Distribution Function Non-central Chi-Square distribution.  

\item \describefun{void}{pnl_cdf_f}{int \ptr which, double \ptr p, double
    \ptr q, double \ptr x, double \ptr dfn, double \ptr dfd, int
    \ptr status, double \ptr bound}
  \sshortdescribe Cumulative Distribution Function F distribution.  

\item \describefun{void}{pnl_cdf_fnc}{int \ptr which, double \ptr p, double
    \ptr q, double \ptr x, double \ptr dfn, double \ptr dfd, double
    \ptr pnonc, int \ptr status, double \ptr bound}
  \sshortdescribe Cumulative Distribution Function Non-central F distribution.  

\item \describefun{void}{pnl_cdf_gam}{int \ptr which, double \ptr p, double
    \ptr q, double \ptr x, double \ptr shape, double \ptr scale, int
    \ptr status, double \ptr bound}
  \sshortdescribe Cumulative Distribution Function GAMma distribution.  

\item \describefun{void}{pnl_cdf_nbn}{int \ptr which, double \ptr p, double
    \ptr q, double \ptr x, double \ptr xn, double \ptr pr, double
    \ptr ompr, int \ptr status, double \ptr bound}
  \sshortdescribe Cumulative Distribution Function Negative BiNomial distribution.  

\item \describefun{void}{pnl_cdf_nor}{int \ptr which, double \ptr p, double
    \ptr q, double \ptr x, double \ptr mean, double \ptr sd, int
    \ptr status, double \ptr bound}
  \sshortdescribe Cumulative Distribution Function NORmal distribution.  

\item \describefun{void}{pnl_cdf_poi}{int \ptr which, double \ptr p, double
    \ptr q, double \ptr x, double \ptr xlam, int \ptr status, double
    \ptr bound}
  \sshortdescribe Cumulative Distribution Function POIsson distribution.  

\item \describefun{void}{pnl_cdf_t}{int \ptr which, double \ptr p, double
    \ptr q, double \ptr x, double \ptr df, int \ptr status, double
    \ptr bound}
  \sshortdescribe Cumulative Distribution Function T distribution.  

\item \describefun{double}{pnl_cdfchi2n}{double x, double df, double ncparam}
  \sshortdescribe Computes the cumulative density function at \var{x} of the
  non central $\chi^2$ distribution with \var{df} degrees of freedom and non
  centrality parameter \var{ncparam}.

\item \describefun{void}{pnl_cdfbchi2n}{double x, double df, double ncparam, double
    beta, double \ptr P}
  \sshortdescribe Stores in \var{P} the cumulative density function at \var{x}
  of the random variable \var{beta \ptr X} where \var{X} is non central $\chi^2$
  random variable with \var{df} degrees of freedom and non
  centrality parameter \var{ncparam}.


\item \describefun{double}{pnl_normal_density}{double x}
  \sshortdescribe Normal density function.

\item \describefun{double}{cdf_nor}{double x}
  \sshortdescribe Cumulative normal distribution function.

\item \describefun{double}{pnl_cdf2nor}{double a, double b, double r}
  \sshortdescribe Cumulative bivariate normal distribution function, returns
  $\frac{1} {2\pi \sqrt{1-r^2}} \int_{- \infty}^a\int_{- \infty}^b e^{-
    \frac{x^2 - 2 r xy+y^2} {2(1-r^2)} } dxdy.$
  
\item \describefun{double}{pnl_inv_cdfnor}{double x}
  \sshortdescribe Inverse of the cumulative normal distribution function.
\end{itemize}

\subsection{Random Number Generators}
\subsubsection{Short Description}

At the moment, there are $14$ random number generators implemented. Each of them is
identified by a macro, actually an integer. One must {\bf NOT} refer to a generator
using directly the value of the macro \var{PNL_RNG_XXX} because there is no warranty
that the order used to store the generators will remain the same in future releases.
Instead, one should call generators directly using their macro names.

\begin{table}[h!]
  \begin{tabular}{l|l|l}
    Random generator & index & Type \\
    \hline
    KNUTH & PNL_RNG_KNUTH & pseudo\\
    MRGK3 & PNL_RNG_MRGK3 & pseudo\\
    MRGK5 & PNL_RNG_MRGK5 & pseudo\\
    SHUFL & PNL_RNG_SHUFL & pseudo\\
    L'ECUYER & PNL_RNG_L_ECUYER & pseudo\\
    TAUSWORTHE & PNL_RNG_TAUSWORTHE & pseudo\\
    MERSENNE & PNL_RNG_MERSENNE & pseudo\\
    MERSENNE (Random Seed) & PNL_RNG_MERSENNE_RANDOM_SEED & pseudo\\
    SQRT & PNL_RNG_SQRT & quasi\\
    HALTON & PNL_RNG_HALTON & quasi\\
    FAURE & PNL_RNG_FAURE & quasi\\
    SOBOL & PNL_RNG_SOBOL & quasi\\
    SOBOL2 & PNL_RNG_SOBOL2 & quasi\\
    NIEDERREITER & PNL_RNG_NIEDERREITER & quasi
  \end{tabular}
  \caption{Indices of the random generators}
  \label{rng_indices}
\end{table}

The initial seed of all the generators is fixed except for the generator
PNL_RNG_MERSENNE_RANDOM_SEED, which is a Mersenne Twister generator with a
random initial seed, read from {\tt /dev/urandom/} on Unix platforms.

\subsubsection{Functions}
\label{std-rng}

Before starting to use random number generators, you must initialize them by
calling 
\begin{itemize}
\item \describefun{int}{pnl_rand_init}{int type_generator, int
    simulation_dim, long samples}
  \sshortdescribe It resets the sample counter to $0$ and checks that the
  generator described by \var{type_generator} can actually generate
  \var{samples} in dimension \var{simulation_dim}.
\end{itemize}

\begin{itemize}
  
\item \describefun{int}{pnl_rand_or_quasi}{int generator}
  \sshortdescribe Returns the type the generator of index \var{type_generator}, 
  \var{MC} or \var{QMC}
  
\item \describefun{const char \ptr }{pnl_rand_name}{int type_generator}
  \sshortdescribe Returns the name of the generator of index \var{type_generator}
\end{itemize}

Once a generator is chosen, there are several functions available in the
library to draw samples according to a given law. 

The following functions return one sample from a specified law.
\begin{itemize}
\item \describefun{int}{pnl_rand_bernoulli}{double p, int type_generator}
  \sshortdescribe Generates a sample from the Bernouilli law on $\{0, 1\}$ with
  parameter \var{p}.
  
\item \describefun{long}{pnl_rand_poisson}{double lambda, int type_generator}
  \sshortdescribe Generates a sample from the Poisson law with
  parameter \var{lambda}.

\item \describefun{double}{pnl_rand_exp}{double lambda, int type_generator}
  \sshortdescribe Generates a sample from the Exponential law with
  parameter \var{lambda}.

\item \describefun{double}{pnl_rand_uni} {int type_generator}
  \sshortdescribe Generates a sample from the Uniform law on $[0, 1]$.

\item \describefun{double}{pnl_rand_uni_ab} {double a, double b, int
    type_generator}
  \sshortdescribe Generates a sample from the Uniform law on $[a, b]$.

\item \describefun{double}{pnl_rand_normal} {int type_generator}
  \sshortdescribe Generates a sample from the standard normal distribution.

\item \describefun{long}{pnl_rand_poisson1}{double lambda, double t, int
    type_generator}
  \sshortdescribe Generates a sample from a Poisson process with intensity
  \var{lambda} at time \var{t}.

\item \describefun{double}{pnl_rand_gamma} {double a, double b, int type_generator}
  \sshortdescribe Generates a sample from the $\Gamma(a, b)$ distribution.  
  
\item \describefun{double}{pnl_rand_chi2} {double n, int type_generator}
  \sshortdescribe Generates a sample from the centered $\chi^2(n)$ distribution.
\end{itemize}

The following functions take an already existing \refstruct{PnlVect}\ptr\  as
its first argument and fill each entry of the vector with a sample from the
specified law. All the entries are independent. The difference between
$n-$samples from a distribution in dimension $1$, and one sample from the same
distribution in dimension $n$ only matters when using a {\bf Quasi} random
number generator.
\begin{itemize}
\item \describefun{void}{pnl_vect_rand_uni}{\refstruct{PnlVect} \ptr G, int
    samples, double a, double b, int type_generator}
  \sshortdescribe \var{G} is a vector of independent and identically distributed
  samples from the uniform distribution on $[a, b]$.

\item \describefun{void}{pnl_vect_rand_normal}{\refstruct{PnlVect} \ptr G, 
    int samples, int generator}
  \sshortdescribe \var{G} is a vector of independent and identically distributed
  samples from the standard normal distribution.

\item \describefun{void}{pnl_vect_rand_uni_d}{\refstruct{PnlVect} \ptr G, int
    d, double a, double b, int type_generator}
  \sshortdescribe \var{G} is a sample from the uniform distribution on $[a, 
  b]^{\text{d}}$.

\item \describefun{void}{pnl_vect_rand_normal_d}{\refstruct{PnlVect} \ptr G, 
    int d, int generator}
  \sshortdescribe \var{G} is a sample from the \var{d}-dimensional
  standard normal distribution.

\end{itemize}

The following functions take an already existing \refstruct{PnlMat}\ptr\  as
first argument and fill each entry of the vector with a sample from the
specified law. All the entries are in-dependant. On return, the matrix \var{M}
is of size \verb!samples x dimension!. The rows of \var{M} are independently
and identically distributed. Each row is a sample from the given law in
dimension \var{dimension}.
\begin{itemize}
\item \describefun{void}{pnl_mat_rand_uni}{\refstruct{PnlMat} \ptr M, int
    samples, int d, const PnlVect \ptr a, const PnlVect \ptr b, int
    type_generator}
  \sshortdescribe \var{M} contains \var{samples} samples from the uniform
  distribution on $\prod_{i=1}^d [a_i, b_i]$.

\item \describefun{void}{pnl_mat_rand_uni2}{\refstruct{PnlMat} \ptr M, int
    samples, int d, double a, double b, int type_generator}
  \sshortdescribe \var{M} contains \var{samples} samples from the uniform
  distribution on $[a, b]^{\text{d}}$.
  
\item \describefun{void}{pnl_mat_rand_normal}{\refstruct{PnlMat} \ptr M, int
    samples, int d, int type_generator}
  \sshortdescribe \var{M} contains \var{samples} samples from the
  \var{d}-dimensional standard normal distribution.
\end{itemize}

Because of the use of {\bf Quasi} random number generators, you may need draw
a set of samples at once because they represent one sample from a distribution
in high dimension. The following function enables to draw one sample from  the
\var{dimension}-dimensional standard normal distribution and store it so
that you can access the elements individually afterwards.
\begin{itemize}
\item \describefun{double}{pnl_rand_gauss}{int d, int
    create_or_retrieve, int index, int type_generator}
  \sshortdescribe The second argument can be either \var{CREATE} (to actually
  draw the sample) or \var{RETRIEVE} (to retrieve that element of index
  \var{index}). With \var{CREATE}, it draws \var{d} random normal variables
  and stores them for future usage. They can be withdrawn using \var{RETRIEVE}
  with the index of the number to be retrieved.
\end{itemize}

\subsubsection{The rng interface}

It is possible to create random number generators each with its own state
variable so that they can evolve independently in a shared memory
environment. These generators are suitable for use with OpenMP. For the
moment, it is only possible to create Mersenne Twister g�n�rators this way.

\begin{verbatim}
typedef struct _PnlRng PnlRng;
struct _PnlRng
{
  PnlObject object;
  int type; /*!< generator type */
  void (*Compute)(PnlRng *g, double *sample); /*!< the function to compute the
                                                next number in the sequence */
  int rand_or_quasi; /*!< can be MC or QMC */
  int dimension; /*!< dimension of the space in which we draw the samples */
  int counter; /*!< counter = number of samples already drawn */
  int has_gauss; /*!< Is a gaussian deviate available? */
  double gauss; /*!< If has_gauss==1, gauss a gaussian sample */
  int size_state; /*!< size in bytes of the state variable */
  void *state; /*!< state of the random generator */
};
\end{verbatim}


\begin{itemize}
\item \describefun{\refstruct{PnlRng}\ptr }{pnl_rng_new}{}
  \sshortdescribe Creates an empty \refstruct{PnlRng}\ptr.
\item \describefun{void}{pnl_rng_free}{\refstruct{PnlRng} \ptr \ptr }
  \sshortdescribe Frees a \refstruct{PnlRng}.
\item \describefun{\refstruct{PnlRng}\ptr }{pnl_rng_create}{int type}
  \sshortdescribe Creates a \refstruct{PnlRng} corresponding to \var{type}
  which must be PNL_RNG_MERSENNE or PNL_RNG_DCMT.
  Once a generator has been created, you {\bf must} call
  \reffun{pnl_rng_sseed} before using it.
\item \describefun{void}{pnl_rng_init}{\refstruct{PnlRng} \ptr rng, int type}
  \sshortdescribe Initializes an empty \refstruct{PnlRng} as returned by
  \reffun{pnl_rng_new} using \var{type} which must be PNL_RNG_MERSENNE or
  PNL_RNG_DCMT. Calling \reffun{pnl_rng_create} is equivalent to calling first
  \reffun{pnl_rng_new} and then \reffun{pnl_rng_init}.
\item \describefun{void}{pnl_rng_sseed}{\refstruct{PnlRng} \ptr rng, unsigned
    long int s}
  \sshortdescribe Sets the seed of the genrator \var{rng} using \var{s}.
\item \describefun{\refstruct{PnlRng} \ptr
    \ptr}{pnl_rng_dcmt_create_array}{int n, ulong seed, int \ptr count} 
  \sshortdescribe Creates an array of \var{n} independent DCMT. \var{seed} is
  the seed used to initialize the Mersenne Twister generator internally used to
  find new DCMT. On exit, \var{count} contains the number of generators actually
  created. Same function as \reffun{pnl_dcmt_create_array} instead that it
  directly returns an array of \refstruct{PnlRng}.
\end{itemize}


The following functions return one sample from a specified law.
\begin{itemize}
\item \describefun{int}{pnl_rng_bernoulli}{double p, \refstruct{PnlRng} \ptr rng}
  \sshortdescribe Generates a sample from the Bernouilli law on $\{0, 1\}$ with
  parameter \var{p}.
  
\item \describefun{long}{pnl_rng_poisson}{double lambda, \refstruct{PnlRng} \ptr rng}
  \sshortdescribe Generates a sample from the Poisson law with
  parameter \var{lambda}.

\item \describefun{double}{pnl_rng_exp}{double lambda, \refstruct{PnlRng} \ptr rng}
  \sshortdescribe Generates a sample from the Exponential law with
  parameter \var{lambda}.

\item \describefun{double}{pnl_rng_uni} {\refstruct{PnlRng} \ptr rng}
  \sshortdescribe Generates a sample from the Uniform law on $[0, 1]$.

\item \describefun{double}{pnl_rng_uni_ab} {double a, double b, int
    type_generator}
  \sshortdescribe Generates a sample from the Uniform law on $[a, b]$.

\item \describefun{double}{pnl_rng_normal} {\refstruct{PnlRng} \ptr rng}
  \sshortdescribe Generates a sample from the standard normal distribution.

\item \describefun{long}{pnl_rng_poisson1}{double lambda, double t, int
    type_generator}
  \sshortdescribe Generates a sample from a Poisson process with intensity
  \var{lambda} at time \var{t}.

\item \describefun{double}{pnl_rng_gamma} {double a, double b, \refstruct{PnlRng} \ptr rng}
  \sshortdescribe Generates a sample from the $\Gamma(a, b)$ distribution.  
  
\item \describefun{double}{pnl_rng_chi2} {double n, \refstruct{PnlRng} \ptr rng}
  \sshortdescribe Generates a sample from the centered $\chi^2(n)$ distribution.
\end{itemize}

The following functions take an already existing \refstruct{PnlVect}\ptr  as
its first argument and fill each entry of the vector with a sample from the
specified law. All the entries are independent. The difference between
$n-$samples from a distribution in dimension $1$, and one sample from the same
distribution in dimension $n$ only matters when using a {\bf Quasi} random
number generator.
\begin{itemize}
\item \describefun{void}{pnl_vect_rng_uni}{\refstruct{PnlVect} \ptr G, int
    samples, double a, double b, \refstruct{PnlRng} \ptr rng}
  \sshortdescribe \var{G} is a vector of independent and identically distributed
  samples from the uniform distribution on $[a, b]$.

\item \describefun{void}{pnl_vect_rng_normal}{\refstruct{PnlVect} \ptr G, 
    int samples, \refstruct{PnlRng} \ptr rng}
  \sshortdescribe \var{G} is a vector of independent and identically distributed
  samples from the standard normal distribution.

\item \describefun{void}{pnl_vect_rng_uni_d}{\refstruct{PnlVect} \ptr G, int
    d, double a, double b, \refstruct{PnlRng} \ptr rng}
  \sshortdescribe \var{G} is a sample from the uniform distribution on $[a, 
  b]^{\text{d}}$.

\item \describefun{void}{pnl_vect_rng_normal_d}{\refstruct{PnlVect} \ptr G, 
    int d, \refstruct{PnlRng} \ptr rng}
  \sshortdescribe \var{G} is a sample from the \var{d}-dimensional
  standard normal distribution.

\end{itemize}

The following functions take an already existing \refstruct{PnlMat}\ptr  as
first argument and fill each entry of the vector with a sample from the
specified law. All the entries are in-dependant. On return, the matrix \var{M}
is of size \verb!samples x dimension!. The rows of \var{M} are independently
and identically distributed. Each row is a sample from the given law in
dimension \var{dimension}.
\begin{itemize}
\item \describefun{void}{pnl_mat_rng_uni}{\refstruct{PnlMat} \ptr M, int
    samples, int d, const PnlVect \ptr a, const PnlVect \ptr b, \refstruct{PnlRng} \ptr rng}
  \sshortdescribe \var{M} contains \var{samples} samples from the uniform
  distribution on $\prod_{i=1}^d [a_i, b_i]$.

\item \describefun{void}{pnl_mat_rng_uni2}{\refstruct{PnlMat} \ptr M, int
    samples, int d, double a, double b, \refstruct{PnlRng} \ptr rng}
  \sshortdescribe \var{M} contains \var{samples} samples from the uniform
  distribution on $[a, b]^{\text{d}}$.
  
\item \describefun{void}{pnl_mat_rng_normal}{\refstruct{PnlMat} \ptr M, int
    samples, int d, \refstruct{PnlRng} \ptr rng}
  \sshortdescribe \var{M} contains \var{samples} samples from the
  \var{d}-dimensional standard normal distribution.
\end{itemize}



\paragraph{Mersenne Twister}

It is possible to create Mersenne Twister random number generators each with
its state variable so that they can evolve independently in a shared memory
environment. These generators are suitable for use with OpenMP
\begin{verbatim}
typedef struct 
{
  unsigned long mt[624];
  int mti;
} mt_state;
typedef unsigned long ulong;
\end{verbatim}

\begin{itemize}
\item \describefun{void}{pnl_mt_sseed}{mt_state \ptr state, unsigned long int
    s}
  \sshortdescribe Sets the initial value of variable \var{state} using \var{s}
\item \describefun{ulong}{pnl_mt_genrand}{mt_state \ptr state}
  \sshortdescribe Returns the following number in the sequence as an unsigned
  long variable. A mask is applied so that only the lowest 32-bits are used.
\item \describefun{double}{pnl_mt_genrand_double}{mt_state \ptr state}
  \sshortdescribe Returns the following number in the sequence as a double.
\end{itemize}


\paragraph{Dynamically created Mersenne Twister}

\begin{verbatim}
typedef struct
{
  ulong aaa;
  int mm,nn,rr,ww;
  ulong wmask,umask,lmask;
  int shift0, shift1, shiftB, shiftC;
  ulong maskB, maskC;
  int i;
  ulong state[17];
} dcmt_state;
\end{verbatim}

Some functions to use ``Dynamically Created Mersenne Twister'' random number
generators (DCMT).
\begin{itemize}
\item \describefun{dcmt_state\ptr}{pnl_dcmt_get_parameter}{ulong seed}
  \sshortdescribe Creates a DCMT. \var{seed} is the seed used to initialize
  the Mersenne Twister generator internally used to find new DCMT.
\item \describefun{dcmt_state \ptr \ptr}{pnl_dcmt_create_array}{int n, ulong seed, int \ptr count}
  \sshortdescribe Creates an array of \var{n} independent DCMT. \var{seed} is
  the seed used to initialize the Mersenne Twister generator internally used to
  find new DCMT. On exit, \var{count} contains the number of generators actually
  created.
\item \describefun{void}{pnl_dcmt_sseed}{dcmt_state \ptr mts, ulong s}
  \sshortdescribe Generates a uniformly distributed unsigned integer
\item \describefun{double}{pnl_dcmt_genrand_double}{dcmt_state \ptr mts}
  \sshortdescribe Generates a uniformly distributed random variable on \var{[0,1]}.
\item \describefun{void}{pnl_dcmt_free}{dcmt_state \ptr \ptr mts}
  \sshortdescribe Frees a dcmt.
\item \describefun{void}{pnl_dcmt_free_array}{dcmt_state \ptr \ptr mts, int count}
  \sshortdescribe Frees an array of dcmt as returned by \reffun{pnl_dcmt_create_array}
\end{itemize}

\clearpage


\section{Deterministic methods}

%% --------------------------------------------------------------------- %%
%% Roots
\subsection{Root finding}
\subsubsection{Short Description}

To provide a uniformed framework to root finding functions, we use two
structures to store functions either returning a single value or computing the
function and its derivative. The pointer \var{params} is used to store the
extra parameters. To evaluate such functions, one can use the two macros
\var{PNL_EVAL_FUNC} and \var{PNL_EVAL_FDF_FUNC}.
\begin{verbatim}
typedef struct {
  double (*function) (double x, void *params);
  void *params;
} PnlFunc ;

#define PNL_EVAL_FUNC(F, x) (*((F)->function))(x, (F)->params)

typedef struct {
  void (*function) (double x, double *f, double *df, void *params);
  void *params;
} PnlFuncDFunc ;

#define PNL_EVAL_FDF_FUNC(F, x, f, df) (*((F)->function))(x, f, df, (F)->params)
\end{verbatim}

\subsubsection{Functions}

\begin{itemize}

\item \describefun{double}{pnl_root_brent}{\refstruct{PnlFunc}$\ast$ F, double
    x1, double  x2, double $\ast$tol}
  \sshortdescribe Finds the root of \var{F} between \var{x1} and \var{x2} with
  an accuracy of order \var{tol}. On exit \var{tol} is an upper bound of the
  error.
  
\item \describefun{int}{pnl_find_root}{\refstruct{PnlFuncDFunc}$\ast$ Func, 
    double x_min, double x_max, double tol, int N_Max, double$\ast$ res}
  \sshortdescribe Finds the root of \var{F} between \var{x1} and \var{x2} with
  an accuracy of order \var{tol} and a maximum of \var{N_max} iterations. On
  exit, the root is stored in \var{res}. Note that the function \var{F} must
  also compute the first derivative of the function.
  

\item \describefun{int}{pnl_root_newton}{\refstruct{PnlFuncDFunc} $\ast$Func, 
    double x0, double epsrel, double epsabs, int N_max, double $\ast$res}
  \sshortdescribe Finds the root of \var{F} starting from \var{x0} with an
  accuracy given both by \var{epsrel} and \var{epsabs} and a maximum number of
  iterations \var{N_max}. On exit, the root is stored in \var{res}.Note that
  the function \var{F} must also compute the first derivative of the function.
  


\item \describefun{int}{pnl_root_bisection}{\refstruct{PnlFunc } $\ast$Func, 
    double xmin, double xmax, double epsrel, double espabs, int N_max, double
    $\ast$res}
  \sshortdescribe Finds the root of \var{F} between \var{x1} and \var{x2} with
  an accuracy given both by \var{epsrel} and \var{epsabs} and a maximum number
  of iterations \var{N_max}. On exit, the root is stored in \var{res}
\end{itemize}


%% --------------------------------------------------------------------- %%
%% Roots
\subsection{Polynomial bases and regression}
\subsubsection{Short Description}

\begin{verbatim}
typedef double(*PnlBasis)(double *x, int index) ;

typedef struct {
  char * label;
  int    space_dim;
  int    max_dim;
  PnlBasis Compute;
} reg_basis;

\end{verbatim}

\begin{table}[h!]
  \begin{describeconst}
    \constentry{CANONICAL}{for the Canonical basis}
    \constentry{HERMITIAN}{for the Hermitian basis}
    \constentry{TCHEBYCHEV}{for the Tchebychev basis}
  \end{describeconst}
  \caption{indices of bases}
  \label{basis_index}
\end{table}

In this section, we provide functions to solve regression problems on
polynomial functions. Let $(x_i, i=1 \dots n)$ be $n$ points in $\R^d$ and a
function $g$ defined by the data $(y_i = g(x_i), i=1 \dots n)$. Assume you
want to approximate the function $g$ by its decomposition on a family of $N$
polynomial functions $(f_j, j=1\dots N)$. Then, we want to compute the vector
$\alpha^\star \in \R^N$ which solves
\begin{equation*} \alpha^\star = \arg\min_\alpha \sum_{i=1}^{n}
  \left(\sum_{j=0}^N \alpha_j f_j(x_i) - y_i\right)^2
\end{equation*}

\subsubsection{Functions}

\begin{itemize}
\item \describefun{PnlBasis}{pnl_init_basis}{int index, int N, int d}
  \sshortdescribe Creates a \refstruct{PnlBasis} for the polynomial family
  defined by \var{index} (see Table~\ref{basis_index}) with at most \var{N}
  elements. \var{d} is the dimension of the space on which the functions are
  defined.

\item \describefun{int}{pnl_fit_least_squares}{\refstruct{PnlVect}$\ast$ coef, 
    \refstruct{PnlMat}$\ast$ x, \refstruct{PnlVect}$\ast$ y, 
    \refstruct{PnlBasis}$\ast$ f, int N}
  \sshortdescribe Computes the coefficients \var{coef} defined by
  \begin{equation*}
    \var{coef} = \arg\min_\alpha \sum_{i=1}^n
    \left( \sum_{j=0}^{\var{N}} \alpha_j  f_j(x_i) - y_i\right)^2
  \end{equation*}
  where \var{N} is the number of functions to regress upon and $n$ is the
  number of points at which we know the value of the original function. $f_j$
  is the $j-th$ basis function. Each row of the matrix \var{x} defines the
  coordinates of one point $x_i$. The function to be approximated is defined
  by the data \var{y} which is the vector of the values taken by the function
  at the points \var{x}.
  
\item \describefun{double}{pnl_basis_eval}{\refstruct{PnlVect}$\ast$ coef, double
    $\ast$x, \refstruct{PnlBasis}$\ast$ f}
  \sshortdescribe Computes the linear combination of \var{f_i(x)} defined by
  \var{coef}. Given the coefficients computed by the function
  \reffun{pnl_fit_least_squares}, this function returns $\sum_{j=0}^n
  \var{coef}_j  f_j(\var{x})$ where \var{x} is a C array.
\end{itemize}

\subsection{Numerical integration}
\subsubsection{Short Description}

Numerical integration methods are designed to numerically evaluate the
integral over an interval (resp. a square) of real valued functions defined on
$\R$ (resp. $\R^2$).

\begin{verbatim}
typedef struct {
  double (*function) (double x, void *params);
  void *params;
} pnl_function ;

typedef struct {
  double (*function) (double x, double y, void *params);
  void *params;
} pnl_function_2D ;
\end{verbatim}

We provide the following two macros to evaluate a \refstruct{pnl_function} at
a given point
\begin{verbatim}
#define PNL_EVAL_FUNC(F, x) (*((F)->function))(x, (F)->params)
#define PNL_EVAL_FUNC2D(F, x, y) (*((F)->function))(x, y, (F)->params)
\end{verbatim}



\subsubsection{Functions}

\begin{itemize}
\item \describefun{int}{ pnl_integration_GK}{const \refstruct{pnl_function} $\ast$F, 
    double x0, double x1, double epsabs, double epsrel, double $\ast$result, 
    double $\ast$abserr,  int $\ast$neval}
  \sshortdescribe Evaluates $\int_{x_0}^{x_1} F$ with an absolute error less than
  \var{espabs} and a relative error less than
  \var{esprel}. The value of the integral is stored in \var{result}, while the
  variables \var{abserr} and \var{neval} respectively contain the absolute
  error and the number of iterations.

\item \describefun{int}{ pnl_integration_GK2D}{const \refstruct{pnl_function_2D} $\ast$F, 
    double x0, double x1, double y0, double y1, double epsabs, double epsrel, 
    double $\ast$result, double $\ast$abserr, int $\ast$neval}
  \sshortdescribe Evaluates $\int_{[x_0, x_1] \times [y_0, y_1]} F$ with an
  absolute error less than \var{espabs} and a relative error less than
  \var{esprel}. The value of the integral is stored in \var{result}, while the
  variables \var{abserr} and \var{neval} respectively contain the absolute
  error and the number of iterations.

\item \describefun{double}{pnl_integration}{const \refstruct{pnl_function} $\ast$F, 
    double x0, double x1, int n, char $\ast$meth}
  \sshortdescribe Evaluates $\int_{x_0}^{x_1} F$ using \var{n} discretisation
  steps. The method used to discretise the integral is defined by \var{meth}
  which can be \var{"rect"} (rectangle rule), \var{"trap"} (trapezoidal rule),
  \var{"simpson"} (Simpson's rule).

\item \describefun{double}{pnl_integration_2D}{const \refstruct{pnl_function_2D} $\ast$F,
    double x0, double x1, double y0, double y1, int nx, int ny, char $\ast$meth}
  \sshortdescribe Evaluates $\int_{[x_0, x_1] \times [y_0, y_1]} F$ using
  \var{nx} (resp. \var{ny}) discretisation steps for \var{[x0, x1]}
  (resp. \var{[y0, y1]}). The method used to discretise the integral is
  defined by \var{meth} which can be \var{"rect"} (rectangle rule),
  \var{"trap"} (trapezoidal rule),   \var{"simpson"} (Simpson's rule).
\end{itemize}


%% FFT function
\subsection{Fast Fourier Transform}
\subsubsection{Short Description}

In the case of Real Fourier transform, the Fourier coefficients satisfy the
following relation
\begin{equation}
  \label{eq:fft-sym}
  z_k = \overline{z_{N-k}}, 
\end{equation}
where $N$ is the number of discretisation points.

A few remarks on the FFT of real functions and its inverse transformation :
\begin{itemize}
\item We only need half of the coefficients.
\item When a value is known to be real the imaginary part is not stored.
So the imaginary part of the zero-frequency component is never stored. It is
known to be zero.
\item For a sequence of even length the imaginary part of the frequency
  $n/2$ is not stored either, since the symmetry (\ref{eq:fft-sym}) implies
  that this is purely real too.
\end{itemize}


\paragraph{FFTPack storage}
\label{sec:fftpack-storage}

The functions use the fftpack storage convention for half-complex sequences.
In this convention, the half-complex transform of a real sequence is stored
with frequencies in increasing order, starting from zero, with the real and
imaginary parts of each frequency in neighboring locations.

The storage scheme is best shown by some examples. The table below shows the
output for an odd-length sequence, $n=5$.  The two columns give the
correspondence between the $5$ values in the half-complex sequence (stored in
a PnlVect $V$) and the values (PnlVectComplex $C$) that would be returned if
the same real input sequence were passed to pnl_dft_complex as a complex
sequence (with imaginary parts set to 0), 
\begin{equation}
  \begin{array}{l}
         C(0) =  V(0) + \imath 0, \\ 
         C(1) =  V(1) + \imath V(2), \\
         C(2) =  V(3) + \imath V(4), \\
         C(3) = V(3) - \imath V(4)=  \overline{C(2)} , \\
         C(4) = V(1) + \imath V(2)=  \overline{C(1)} 
  \end{array}   
\end{equation}

The elements of index greater than $N/2$ of the complex array, as $C(3)$
$C(4)$, are filled in using the symmetry condition.

The next table shows the output for an even-length sequence, $n=6$.
In the even case there are two values which are purely real, 
\begin{equation}
  \begin{array}{l}
         C(0) =  V(0) + \imath 0, \\ 
         C(1) =  V(1) + \imath V(2), \\
         C(2) =  V(3) + \imath V(4), \\
         C(3) = V(5) - \imath 0    =  \overline{C(0)} , \\
         C(4) = V(3) - \imath V(4) =  \overline{C(2)} , \\
         C(5) = V(1) + \imath V(2) =  \overline{C(1)} 
  \end{array}   
 \end{equation}


\subsubsection{Functions}


The following functions comes from a C version of the Fortran FFTPack library
available on \url{http://www.netlib.org/fftpack}.
\begin{itemize}
\item \describefun{int}{pnl_fft_inplace}{\refstruct{PnlVectComplex} $\ast$data}
  \sshortdescribe Computes the FFT of \var{data} in place. The original content
  of \var{data} is lost.

\item \describefun{int}{pnl_ifft_inplace}{\refstruct{PnlVectComplex} $\ast$data}
  \sshortdescribe Computes the inverse FFT of \var{data} in place. The
  original content of \var{data} is lost.

\item \describefun{int}{pnl_fft}{const \refstruct{PnlVectComplex} $\ast$in, 
    \refstruct{PnlVectComplex} $\ast$out}
  \sshortdescribe Computes the FFT of \var{in} and stores it into \var{out}.

\item \describefun{int}{pnl_ifft}{const \refstruct{PnlVectComplex} $\ast$in, 
    \refstruct{PnlVectComplex} $\ast$out}
  \sshortdescribe Computes the inverse FFT of \var{in} and stores it into \var{out}.

\item \describefun{int}{pnl_fft2}{double $\ast$re, double $\ast$im, int n}
  \sshortdescribe Computes the FFT of the vector of length \var{n} whose real
  (resp. imaginary) parts are given by the arrays \var{re}
  (resp. \var{im}). The real and imaginary parts of the FFT are respectively
  stored in \var{re} and \var{im} on output.

\item \describefun{int}{pnl_ifft2}{double $\ast$re, double $\ast$im, int n}
  \sshortdescribe Computes the inverse FFT of the vector of length \var{n}
  whose real (resp. imaginary) parts are given by the arrays \var{re}
  (resp. \var{im}). The real and imaginary parts of the inverse FFT are
  respectively stored in \var{re} and \var{im} on output.

\item \describefun{int}{pnl_real_fft}{const \refstruct{PnlVect} $\ast$in, 
    \refstruct{PnlVectComplex} $\ast$out}
  \sshortdescribe Computes the FFT of the real valued sequence \var{in} and
  stores it into \var{out}.

\item \describefun{int}{pnl_real_ifft}{const \refstruct{PnlVect} $\ast$in, 
    \refstruct{PnlVectComplex} $\ast$out}
  \sshortdescribe Computes the inverse FFT of \var{in} and stores it into \var{out}.

\item \describefun{int}{pnl_real_fft_inplace}{double $\ast$data, int n}
  \sshortdescribe Computes the FFT of the real valued vector \var{data} of
  length \var{n}. The result is stored in \var{data} using the FFTPack storage
  described above, see~\ref{sec:fftpack-storage}.

\item \describefun{int}{pnl_real_ifft_inplace}{double $\ast$data, int n}
  \sshortdescribe Computes the inverse FFT of the vector \var{data} of length
  \var{n}. \var{data} is supposed to be the FFT coefficients a real valued
  sequence stored using the FFTPack storage. On output, \var{data} contains
  the inverse FFT.

\item \describefun{int}{pnl_real_fft2}{double $\ast$re, double $\ast$im, int n}
  \sshortdescribe Computes the FFT of the real vector \var{re} of length \var{n}.
  \var{im} is only used on output to store the imaginary part the FFT. The
  real part is stored into \var{re}
  
\item \describefun{int}{pnl_real_ifft2}{double $\ast$re, double $\ast$im, int n}
  \sshortdescribe Computes the inverse FFT of the vector \var{re + i * im} of
  length \var{n}, which is supposed to be the FFT of a real valued
  sequence. On exit, \var{im} is unused. 
\end{itemize}

%% The following functions are deprecated and will be removed in future
%% releases. Use the above functions instead.
%% \begin{itemize}
%% \item \describefun{void}{pnl_dft_complex}{const \refstruct{ PnlVectComplex}
%%     $\ast$data, \refstruct{ PnlVectComplex} $\ast$result}
%%   \sshortdescribe classical FFT with complexity  $ O(N^2)$  

%% \item \describefun{void}{pnl_dft_complex_backward}{const \refstruct{
%%       PnlVectComplex} $\ast$data, \refstruct{ PnlVectComplex} $\ast$result}
%%   \sshortdescribe classical inverse FFT without $ L_1$ normalisation 
%%   and with complexity $ O(N^2)$  

%% \item \describefun{void}{pnl_dft_complex_inverse}{const \refstruct{
%%       PnlVectComplex} $\ast$data, \refstruct{ PnlVectComplex} $\ast$result}
%%   \sshortdescribe classical inverse FFT on complex in $ O(N^2)$  

%% \item \describefun{void}{pnl_dft_complex_transform}{const
%%     \refstruct{PnlVectComplex} $\ast$data, \refstruct{ PnlVectComplex} $\ast$result, const int sign}
%%   \sshortdescribe classical FFT

%% \item \describefun{void}{real_fft}{\refstruct{ PnlVect} $\ast$a, int fft-size, 
%%     \refstruct{ boolean} inversefft}
%%   \sshortdescribe in-place FFT for real functions. Extracted from
%%   Numerical Recipes Paragraph 12.3 and adapted.
%% \end{itemize}

%% Laplace transform
\subsection{Inverse Laplace transform}
\subsubsection{Short Description}

For a real valued function $f$ such that $t \longmapsto f(t) \expp{- \sigma_c
  t}$ is integrable over $\R^+$, we can define its Laplace transform
\begin{equation*}
  \hat{f}(\lambda) = \int_0^\infty f(t) \expp{- \lambda t} dt \qquad
  \mbox{for $\lambda \in \C$ with $\real{\lambda} \ge \sigma_c$}.
\end{equation*}

\subsubsection{Functions}
\begin{itemize}
\item \describefun{double}{pnl_ilap_euler}{\refstruct{PnlCmplxFunc}
    $\ast$fhat, double t, int N, int M}
  \sshortdescribe Computes $f(\var{t})$ where $f$ is given by its Laplace
  transform \var{fhat} by numerically inverting the Laplace transform using
  Euler's summation. The values \var{N = M = 15} usually give a very good
  accuracy. For more details on the accuracy of the method, see \cite{aw, ll}. 

\item \describefun{double}{pnl_ilap_cdf_euler}{\refstruct{PnlCmplxFunc}
    $\ast$fhat, double t, int N, int M}
  \sshortdescribe Computes the cumulative distribution function $F(\var{t})$
  where $F(x) = \int_0^x f(t) dt$ and $f$ is a density function with values on
  the positive real linegiven by its Laplace transform \var{fhat}. The
  computation is carried out by numerical inversion of the Laplace transform
  using Euler's summation. The values \var{N = M = 15} usually give a very
  good accuracy. The parameter \var{h} is the discretisation step, the
  algorithm is very sensitive to the choice of \var{h}.

\item \describefun{double}{pnl_ilap_fft}{\refstruct{PnlVect} $\ast$res,
    \refstruct{PnlCmplxFunc} $\ast$fhat, double T, double eps}
  \sshortdescribe Computes $f(t)$ for $t \in [h, \var{T}]$ on a regular grid
  and stores the values in \var{res}, where $h = T / {\mathrm size}(res)$. The
  function $f$ is defined by its Laplace transform \var{fhat}, which is
  numerically inverted using a Fast Fourier Transform algorithm. The size of
  \var{res} is related to the choice of the relative precision \var{eps}
  required on the value of $f(t)$ for all $t \le T$.

\item \describefun{double}{pnl_ilap_gs}{\refstruct{PnlFunc} $\ast$fhat, double
    t, int n}
  \sshortdescribe Computes $f(\var{t})$ where $f$ is given by its Laplace
  transform \var{fhat} by numerically inverting the Laplace transform using a
  weighted combination of different Gaver Stehfest's algorithms. Note that
  this function does not need the comple valued Laplace transform but only the
  real valued one. \var{n} is the number of terms used in the weighted combination.

\item \describefun{double}{pnl_ilap_gs_basic}{\refstruct{PnlFunc}
    $\ast$fhat, double t, int n}
  \sshortdescribe Computes $f(\var{t})$ where $f$ is given by its Laplace
  transform \var{fhat} by numerically inverting the Laplace transform using
  Gaver Stehfest's method, see~\cite{aw}. Note that this function does not
  need the comple valued Laplace transform but only the real valued
  one. \var{n} is the number of iterations of the algorithm.
  {\bf Note : }~This function is provided only for test purposes, even though
  the function \reffun{pnl_ilap_gs} gives far more accurate results.
\end{itemize}

%% pde tools
\subsection{PDE tools}
\subsubsection{Short Description}
\subsubsection{Functions}
\begin{itemize}
\item 
\describefun{\refstruct{PnlPDEBoundary}$\ast$}{pnl_pde_boundary_create}{double X0, double X1}
  \sshortdescribe creates a \refstruct{PnlPDEBoundary}  
\item \describefun{double}{pnl_pde_boundary_real_variable}{const \refstruct{ PnlPDEBoundary} BP, double X}
\item 
\describefun{double}{pnl_pde_boundary_unit_interval}{const \refstruct{ PnlPDEBoundary} BP, double X}
\item 
\describefun{\refstruct{PnlPDEDimBoundary} $\ast$}{pnl_pde_dim_boundary_create_from_int}{int dim}
  \sshortdescribe creates a \refstruct{PnlPDEBoundary} with Left down corner is $ (0, \dots, 0)$ and right up corner is $ (1, \dots, 1)$  
\item \describefun{\refstruct{ PnlPDEDimBoundary} $\ast$}{pnl_pde_dim_boundary_create}{const \refstruct{ PnlVect} $\ast$X0, const \refstruct{ PnlVect} $\ast$X1}
  \sshortdescribe 
\item \describefun{void}{pnl_pde_dim_boundary_free}{\refstruct{ PnlPDEDimBoundary} $\ast$$\ast$\refstruct{ v}}
  \sshortdescribe frees a \refstruct{PnlPDEDimBoundary}  
\item \describefun{double}{pnl_pde_dim_boundary_eval_from_unit}{double($\ast$f)(const \refstruct{ PnlVect} $\ast$), const \refstruct{ PnlPDEDimBoundary} $\ast$BP, const \refstruct{ PnlVect} $\ast$X}
\item 
\describefun{void}{pnl_pde_dim_boundary_from_unit_to_real_variable}{const \refstruct{ PnlPDEDimBoundary} $\ast$BP, \refstruct{ PnlVect} $\ast$X}
\item 
\describefun{double}{pnl_pde_dim_boundary_get_step}{const \refstruct{ PnlPDEDimBoundary} $\ast$BP, int i}
\item 
\describefun{double}{standard_time_repartition}{int i, int N-T}
\item \describefun{\refstruct{ PnlPDETimeGrid} $\ast$}{pnl_pde_time_grid}{const
  double T, const int $N-T$, double($\ast$ repartition)(int i, int NN)}
  \sshortdescribe creates a \refstruct{PnlPDETimeGrid}
\item \describefun{\refstruct{ PnlPDETimeGrid} $\ast$}{pnl_pde_time_homogen_grid}{const double T, const int N-T}
  \sshortdescribe creates a \refstruct{PnlPDETimeGrid}
\item \describefun{void}{pnl_pde_time_grid_free}{\refstruct{ PnlPDETimeGrid} $\ast$$\ast$TG}
  \sshortdescribe frees a \refstruct{PnlPDETimeGrid}
\item \describefun{void}{pnl_pde_time_start}{\refstruct{ PnlPDETimeGrid} $\ast$TG}
  \sshortdescribe initialise \refstruct{PnlPDETimeGrid}
\item \describefun{int}{pnl_pde_time_grid_increase}{\refstruct{ PnlPDETimeGrid} $\ast$TG}
  \sshortdescribe go to the next time step  
\item \describefun{double}{pnl_pde_time_grid_step}{const \refstruct{ PnlPDETimeGrid} $\ast$TG}
  \sshortdescribe GET function on current step.  
\item \describefun{double}{pnl_pde_time_grid_time}{const \refstruct{ PnlPDETimeGrid} $\ast$TG}
  \sshortdescribe GET function on current time.
\end{itemize}


%%% Local Variables: 
%%% mode: latex
%%% TeX-master: "pnl-premia-manual"
%%% End: 


\printindex

\end{document}
