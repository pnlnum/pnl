\section{Root finding}
\subsection{Overview}
\label{sec:PnlFunc}

To provide a uniformed framework to root finding functions, we use several
structures for storing different kind of functions. The pointer
\var{params} is used to store the extra parameters. These new types come
with dedicated macros starting in \verb!PNL_EVAL!  to evaluate the function
and their Jacobian.
\describestruct{PnlFunc}
\begin{verbatim}
/*
 * f: R --> R
 * The function  pointer returns f(x)
 *
typedef struct {
  double (*F) (double x, void *params);
  void *params;
} PnlFunc ;
#define PNL_EVAL_FUNC(Fstruct, x) (*((Fstruct)->F))(x, (Fstruct)->params)
\end{verbatim}

\describestruct{PnlFunc2D}
\begin{verbatim}
/*
 * f: R^2 --> R
 * The function pointer returns f(x)
 *
typedef struct {
  double (*F) (double x, double y, void *params);
  void *params;
} PnlFunc2D ;
#define PNL_EVAL_FUNC2D(Fstruct, x, y) (*((Fstruct)->F))(x, y, (Fstruct)->params)
\end{verbatim}

\describestruct{PnlFuncDFunc}
\begin{verbatim}
/*
 * f: R --> R
 * The function pointer computes f(x) and Df(x) and stores them in fx
 * and dfx respectively
 *
typedef struct {
  void (*F) (double x, double *fx, double *dfx, void *params);
  void *params;
} PnlFuncDFunc ;
#define PNL_EVAL_FUNC_DF(Fstruct, x, fx, dfx) (*((Fstruct)->F))(x, fx, dfx, (Fstruct)->params)
\end{verbatim}

\describestruct{PnlRnFuncR}
\begin{verbatim}
/*
 * f: R^n --> R
 * The function pointer returns f(x)
 *
typedef struct {
  double (*F) (const PnlVect *x, void *params);
  void *params;
} PnlRnFuncR ;
#define PNL_EVAL_RNFUNCR(Fstruct, x) (*((Fstruct)->F))(x, (Fstruct)->params)
\end{verbatim}

\describestruct{PnlRnFuncRm}
\describestruct{PnlRnFuncRn}
\begin{verbatim}
/*
 * f: R^n --> R^m
 * The function pointer computes the vector f(x) and stores it in
 * fx (vector of size m)
 *
typedef struct {
  void (*F) (const PnlVect *x, PnlVect *fx, void *params);
  void *params;
} PnlRnFuncRm ;
#define PNL_EVAL_RNFUNCRM(Fstruct, x, fx) (*((Fstruct)->F))(x, fx, (Fstruct)->params)

/*
 * Synonymous of PnlRnFuncRm for f:R^n --> R^n
 *
typedef PnlRnFuncRm PnlRnFuncRn;
#define PNL_EVAL_RNFUNCRN  PNL_EVAL_RNFUNCRM
\end{verbatim}

\describestruct{PnlRnFuncRmDFunc}
\describestruct{PnlRnFuncRnDFunc}
\begin{verbatim}
/*
 * f: R^n --> R^m
 * The function pointer computes the vector f(x) and stores it in fx
 * (vector of size m)
 * The Dfunction pointer computes the matrix Df(x) and stores it in dfx
 * (matrix of size m x n)
 *
typedef struct {
  void (*F) (const PnlVect *x, PnlVect *fx, void *params);
  void (*DF) (const PnlVect *x, PnlMat *dfx, void *params);
  void (*FDF) (const PnlVect *x, PnlVect *fx, PnlMat *dfx, void *params);
  void *params;
} PnlRnFuncRmDFunc ;
#define PNL_EVAL_RNFUNCRM_DF(Fstruct, x, dfx) \
    (*((Fstruct)->Dfunction))(x, dfx, (Fstruct)->params)
#define PNL_EVAL_RNFUNCRM_FDF(Fstruct, x, fx, dfx) \
    (*((Fstruct)->F))(x, fx, dfx, (Fstruct)->params)

/*
 * Synonymous of PnlRnFuncRmDFunc for f:R^n --> R^m
 *
typedef PnlRnFuncRmDFunc PnlRnFuncRnDFunc;
#define PNL_EVAL_RNFUNCRN_DF PNL_EVAL_RNFUNCRM_DF
#define PNL_EVAL_RNFUNCRN_FDF PNL_EVAL_RNFUNCRM_FDF
\end{verbatim}

\subsection{Functions}

To use the following functions, you should include \verb!pnl/pnl_root.h!.

For finding the zero of a real valued function we provide the following
functions.
\begin{itemize}
\item \describefun{double}{pnl_root_brent}{\refstruct{PnlFunc} \ptr F, double
    x1, double  x2, double \ptr tol}
  \sshortdescribe Find the root of \var{F} between \var{x1} and \var{x2} with
  an accuracy of order \var{tol}. On exit \var{tol} is an upper bound of the
  error.

\item \describefun{int}{pnl_root_newton_bisection}{\refstruct{PnlFuncDFunc} \ptr  Func,
    double x_min, double x_max, double tol, int N_Max, double \ptr res}
  \sshortdescribe Find the root of \var{F} between \var{x1} and \var{x2} with
  an accuracy of order \var{tol} and a maximum of \var{N_max} iterations. On
  exit, the root is stored in \var{res}. Note that the function \var{Func} must
  also compute the first derivative of the function. This function relies on
  combining Newton's approach with a bisection technique.


\item \describefun{int}{pnl_root_newton}{\refstruct{PnlFuncDFunc} \ptr Func,
    double x0, double epsrel, double epsabs, int N_max, double \ptr res}
  \sshortdescribe Find the root of \var{F} starting from \var{x0} with an
  accuracy given both by \var{epsrel} and \var{epsabs} and a maximum number of
  iterations \var{N_max}. On exit, the root is stored in \var{res}.Note that
  the function \var{F} must also compute the first derivative of the function.

\item \describefun{int}{pnl_root_bisection}{\refstruct{PnlFunc} \ptr Func,
    double xmin, double xmax, double epsrel, double espabs, int N_max, double
    \ptr res}
  \sshortdescribe Find the root of \var{F} between \var{x1} and \var{x2} with
  an accuracy given both by \var{epsrel} and \var{epsabs} and a maximum number
  of iterations \var{N_max}. On exit, the root is stored in \var{res}
\end{itemize}

Searching for the zero of a multivariate and vector valued function is a
complicated problem and we rely on minpack for doing this. Here, we provide
two wrappers for calling minpack routines.
\begin{itemize}
\item \describefun
  {int}{pnl_root_fsolve}{\refstruct{PnlRnFuncRnDFunc} \ptr f,
    \refstruct{PnlVect} \ptr x, \refstruct{PnlVect} \ptr fx, double xtol,
    int maxfev, int \ptr nfev, \refstruct{PnlVect} \ptr scale, int
    error_msg}
  \sshortdescribe Compute the root of a function $f:\R^n \longmapsto
  \R^n$. Note that the number of components of \var{f} must be equal to the
  number of variates of \var{f}. This function returns \var{OK} or
  \var{FAIL} if something went wrong.
  \parameters
  \begin{itemize}
  \item \var{f} is a pointer to a \refstruct{PnlRnFuncRnDFunc} used to
    store the function whose root is to be found. \var{f} can also
    store the Jacobian of the function, if not it is computed using
    finite differences (see the file \url{examples/minpack_test.c} for
    a usage example). \var{f->FDF} can be NULL because it is
    not used in this function.
  \item  \var{x} contains on input the starting point of the search and
    an approximation of the root of \var{f} on output,
  \item \var{xtol} is the precision required on \var{x}, if set to 0 a
    default value is used.
  \item \var{maxfev} is the maximum number of evaluations of the function
    \var{f} before the algorithm returns, if set to 0, a coherent
    number is determined internally.
  \item \var{nfev} contains on output the number of evaluations of
    \var{f} during the algorithm,
  \item \var{scale} is a vector used to rescale \var{x} in a way that
    each coordinate of the solution is approximately of order 1 after
    rescaling. If on input \var{scale=NULL}, a scaling vector is
    computed internally by the algorithm.
  \item \var{error_msg} is a boolean
    (\var{TRUE} or \var{FALSE}) to specify if an error message should be
    printed when the algorithm stops before having converged.
  \item On output, \var{fx} contains \var{f(x)}.
  \end{itemize}

\item \describefun {int}{pnl_root_fsolve_lsq}{\refstruct{PnlRnFuncRmDFunc}
    \ptr f, \refstruct{PnlVect} \ptr x, int m, \refstruct{PnlVect} \ptr fx,
    double xtol, double ftol, double gtol, int maxfev, int \ptr nfev,
    \refstruct{PnlVect} \ptr scale, int error_msg}
  \sshortdescribe Compute the root of $x \in \R^n \longmapsto
  \sum_{i=1}^m f_i(x)^2$, note that there is no reason why \var{m} should
  be equal to \var{n}.
  \parameters
  \begin{itemize}
  \item \var{f} is a pointer to a \refstruct{PnlRnFuncRmDFunc} used to
    store the function whose root is to be found. \var{f} can also
    store the Jacobian of the function, if not it is computed using
    finite differences (see the file \url{examples/minpack_test.c} for
    a usage example). \var{f->FDF} can be NULL because it is
    not used in this function.
  \item  \var{x} contains on input the starting
    point of the search and an approximation of the root of \var{f} on
    output,
  \item \var{m} is the number of components of \var{f},
  \item \var{xtol} is the precision required on \var{x}, if set to 0 a
    default value is used.
  \item \var{ftol} is the precision required on \var{f}, if set to 0 a
    default value is used.
  \item \var{gtol} is the precision required on the Jacobian of
    \var{f}, if set to 0 a default value is used.
  \item \var{maxfev} is the maximum number of evaluations of the function
    \var{f} before the algorithm returns, if set to 0, a coherent
    number is determined internally.
  \item \var{nfev} contains on output the number of evaluations of
    \var{f} during the algorithm,
  \item \var{scale} is a vector used to rescale \var{x} in a way that
    each coordinate of the solution is approximately of order 1 after
    rescaling.  If on input \var{scale=NULL}, a scaling vector is
    computed internally by the algorithm.
  \item \var{error_msg} is a boolean (\var{TRUE} or \var{FALSE}) to
    specify if an error message should be printed when the algorithm
    stops before having converged.
  \item On output, \var{fx} contains \var{f(x)}.
  \end{itemize}
\end{itemize}

% vim:spelllang=en:spell:

