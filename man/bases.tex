\section{Function bases and regression}
\subsection{Overview}

To use these functionalities, you should include \verb!pnl/pnl_basis.h!.

\describestruct{PnlBasis}
\begin{lstlisting}
struct _PnlBasis
{
  /**
    * Must be the first element in order for the object mechanism to work
    * properly. This allows any PnlBasis pointer to be cast to a PnlObject
    */
  PnlObject     object;
  /** The basis type */
  int           id;
  /** The string to label the basis */
  const char   *label;
  /** The number of variates */
  int           nb_variates;
  /** The total number of elements in the basis */
  int           nb_func;
  /** The sparse Tensor matrix */
  PnlSpMatInt  *SpT;
  /** The number of functions in the tensor @p SpT */
  int           len_T;
  /** Compute the i-th element of the one dimensional basis.  As a convention, (*f)(x, 0) MUST be equal to 1 */
  double      (*f)(double x, int i, int dim, void *params);
  /** Compute the first derivative of i-th element of the one dimensional basis */
  double      (*Df)(double x, int i, int dim, void *params);
  /** Compute the second derivative of the i-th element of the one dimensional basis */
  double      (*D2f)(double x, int i, int dim, void *params);
  /** PNL_TRUE if the basis is reduced */
  int           isreduced;
  /** The center of the domain */
  double       *center;
  /** The inverse of the scaling factor to map the domain to [-1, 1]^nb_variates */
  double       *scale;
  /** An array of additional functions */
  PnlRnFuncR   *func_list;
  /** The number of functions in @p func_list */
  int           len_func_list;
  /** Extra parameters to pass to basis functions */
  void         *f_params;
  /** Size of params in bytes to be passed to malloc */
  size_t        f_params_size;
  /** Non linear mapping of the data */
  double       (*map)(double x, int dim, void *params);
  /** First derivative of the non linear mapping  */
  double       (*Dmap)(double x, int dim, void *params);
  /** Second derivate of the linear mapping */
  double       (*D2map)(double x, int dim, void *params);
  /** Extra parameters for map, Dmap and D2map */
  void          *map_params;
  /** Size of @p map_params in bytes to be passed to malloc */
  size_t         map_params_size;
};
\end{lstlisting}

A \refstruct{PnlBasis} is a family of multivariate functions with real values. Two different kinds of functions can be stored in these families: tensor functions --- originally, this was the only possibility --- and standard multivariate function typed as \refstruct{PnlRnFuncR}.

\subsubsection{Tensor functions}

Tensor functions are built as a tensor product of one dimensional elements. Hence, we only need a tensor matrix \var{T} to describe a multi-dimensional basis in terms of the one dimensional one. These tensors functions can be easily evaluated and differentiated twice, see \reffun{pnl_basis_eval},
\reffun{pnl_basis_eval_vect}, \reffun{pnl_basis_eval_D}, \reffun{pnl_basis_eval_D_vect},
\reffun{pnl_basis_eval_D2}, \reffun{pnl_basis_eval_D2_vect},
\reffun{pnl_basis_eval_derivs}, \reffun{pnl_basis_eval_derivs_vect}.

Three bases are already registered as listed in Table~\ref{tab:basis_index}. A new tensor basis can be registered using the function
  \reffun{pnl_basis_type_register}.

\begin{table}[h!]
  \begin{describeconst}
    \constentry{PNL_BASIS_CANONICAL}{for the Canonical polynomials}
    \constentry{PNL_BASIS_HERMITE}{for the Hermite polynomials}
    \constentry{PNL_BASIS_TCHEBYCHEV}{for the Tchebychev polynomials}
  \end{describeconst}
  \caption{Names of the bases.}
  \label{tab:basis_index}
\end{table}

The Hermite polynomials are defined by
\begin{equation*}
  H_n(x) = (-1)^n \expp{\frac{x^2}{2}} \frac{d^n}{dx^n} \expp{-\frac{x^2}{2}}.
\end{equation*}
If $G$ is a real valued standard normal random variable, ${\mathbb E}[H_n(G) H_m(G)] = n!  \ind{n = m}$.


The two tensors \var{T} and \var{SpT} do actually store the same information --- \var{T(i,j)} is the \emph{degree} w.r.t the \var{j}-th variable in the \var{i}-th function. Originally, we were only using the dense representation \var{T}, which is far more convenient to use when building the basis but it slows down the evaluation of the basis by a great deal. To overcome this lack of efficiency, a sparse storage was added. Such a basis can be created using one of the following functions.

\begin{itemize}
  \item \describefun{\PnlBasis *}{pnl_basis_create}{int index, int
      nb_func, int nb_variates}
    \sshortdescribe Create a \PnlBasis for the family
    defined by \var{index} (see Table~\ref{tab:basis_index} and
    \reffun{pnl_basis_type_register}) with \var{nb_variates}
    variates. The basis will contain \var{nb_func}.

  \item \describefun{\PnlBasis *}{pnl_basis_create_from_degree}{int
      index, int degree, int nb_variates}
    \sshortdescribe Create a \PnlBasis for the family
    defined by \var{index} (see Table~\ref{tab:basis_index} and \reffun{pnl_basis_type_register}) with total degree less
    or equal than \var{degree} and \var{nb_variates} variates. The total degree is the sum of the partial degrees.
  \item \describefun{\PnlBasis *}{pnl_basis_create_from_prod_degree}{int index, int degree, int nb_variates}
    \sshortdescribe Create a \PnlBasis for the family
    defined by \var{index} (see Table~\ref{tab:basis_index} and \reffun{pnl_basis_type_register}) with total degree less
    or equal than \var{degree} and \var{nb_variates} variates. The total degree is
    the product of \var{MAX(1, d_i)} where the \var{d_i} are the partial degrees.

  \item  \describefun{\PnlBasis\ptr}{pnl_basis_create_from_hyperbolic_degree}
    {int index, double degree, double q, int n}
    \sshortdescribe Create a sparse basis of polynomial with \var{n}
    variates. We give the example of the Canonical basis. A canonical polynomial
    with \var{n} variates writes $X_1^{\alpha_1} X_2^{\alpha_2} \dots
    X_n^{\alpha_n}$. To be a member of the basis, it must satisfy $\left(\sum_{i=1}^n {\alpha_i}^q \right)^{1/q} \leq degree$. This kind of basis based on an hyperbolic set of indices gives priority to polynomials associated to low
    order interaction.

  \item \describefun{\PnlBasis *}{pnl_basis_create_from_tensor}{int
      index, PnlMatInt \ptr T}
    \sshortdescribe Create a \PnlBasis for the polynomial family
    defined by \var{index} (see Table~\ref{tab:basis_index}) using the basis
    described by the tensor matrix \var{T}. The number of lines of \var{T} is
    the number of functions of the basis whereas the numbers of columns of
    \var{T} is the number of variates of the functions.
    \\
    Here is an example of a tensor matrix. Assume you are working with three
    variate functions, the basis \verb!{ 1, x, y, z, x^2, xy, yz, z^3}! is
    decomposed in the one dimensional canonical basis using the following tensor
    matrix
    \[ \left(
      \begin{array}{ccc}
        0 & 0 & 0 \\
        1 & 0 & 0 \\
        0 & 1 & 0 \\
        0 & 0 & 1 \\
        2 & 0 & 0 \\
        1 & 1 & 0 \\
        0 & 1 & 1\\
        0 & 0 & 3
      \end{array}
    \right) \]

  \item \describefun{\PnlBasis *}{pnl_basis_create_from_sparse_tensor}{int
      index, PnlSpMatInt \ptr SpT}
    \sshortdescribe Create a \PnlBasis for the polynomial family
    defined by \var{index} (see Table~\ref{tab:basis_index}) using the basis
    described by the sparse tensor matrix \var{SpT}. The number of lines of \var{SpT} is the number of functions of the basis whereas the numbers of columns of \var{SpT} is the number of variates of the functions.

  \item \describefun{void }{pnl_basis_set_from_tensor}{\PnlBasis \ptr
      b, int index, const \PnlMatInt \ptr T}
    \sshortdescribe Set an alredy existing basis \var{b} to a polynomial family
    defined by \var{index} (see Table~\ref{tab:basis_index}) using the basis
    described by the tensor matrix \var{T}. The number of lines of \var{T} is
    the number of functions of the basis whereas the numbers of columns of
    \var{T} is the number of variates of the functions. \\
    Same function as \reffun{pnl_basis_create_from_tensor} except that it
    operates on an already existing basis.

\end{itemize}


\subsubsection{Local basis functions}

A local basis is a family of indicator functions of a Cartesian partition of ${[-1,1]}^d$. Let $(n_i)_{i \in \{1, \dots, d\}}$ be the number of interval along each direction. An element of the partition write \[A_k = \prod_{i=1}^d \left[-1 + \frac{k_i}{n_i}, -1 + \frac{k_i + 1}{n_i}\right]\] where $k$ is the multi-index defined by $k_i \in \{0, \dots, n_i - 1\}$ for all $i\in \{1,\dots, d\}$. These functions are orthogonal for the standard $L^2$ scalar product; this property is used by \reffun{pnl_basis_fit_ls}. Note that they are not differentiable.

We provide the helper function \reffun{pnl_basis_local_get_index} to compute the linear representation of the multi-index $k$.

\begin{itemize}
  \item \describefun{\PnlBasis\ptr}{pnl_basis_local_create}{int *n_intervals, int space_dim}
  \sshortdescribe Create a local basis with \var{n_intervals[i - 1]} intervals along dimension $i$ for all $i \in \{1, \dots, \var{space\_dim}\}$
  \item \describefun{\PnlBasis\ptr}{pnl_basis_local_create_regular}{int n_intervals, int space_dim}
  \sshortdescribe Equivalent to calling \reffun{pnl_basis_local_create} with an array of size \var{space_dim} filled with \var{n_intervals}.
  \item \describefun{int}{pnl_basis_local_get_index}{const \PnlBasis *basis, const double *x}
  \sshortdescribe Return the linear index of the cell containing \var{x}. It is an integer between $0$ and $(\prod_{i=1}^d n_i) - 1$.
\end{itemize}

If the domain you want to consider is not ${[-1,1]}^d$, use the functions \reffun{pnl_basis_set_domain}, \reffun{pnl_basis_set_reduced} and \reffun{pnl_basis_set_map} to map your product space into ${[-1,1]}^d$.

\subsubsection{Standard multivariate functions}

These functions are supposed to be \refstruct{PnlRnFuncR}.

To make this toolbox more complete, it is now possible to add some extra functions, which are not tensor product. They are stored using an independent mechanism in \var{func_list}.  These additional functions are only taken into account by the methods \reffun{pnl_basis_i}, \reffun{pnl_basis_i_vect}, \reffun{pnl_basis_eval} and \reffun{pnl_basis_eval_vect}. Note in particular that it is not possible to differentiate these functions and that they are not sensitive to the \var{isreduced} attribute. To add an extra function to an existing \refstruct{PnlBasis}, call the function
\begin{itemize}
  \item \describefun{void}{pnl_basis_add_function}{\PnlBasis \ptr b, \PnlRnFuncR \ptr f}
  \sshortdescribe Add the function \var{f} to the already existing basis \var{b}.
\end{itemize}

\subsection{Functions}

\begin{itemize}

\item \describefun{\PnlBasis *}{pnl_basis_new}{}
  \sshortdescribe Create an empty \PnlBasis.

\item \describefun{void}{pnl_basis_clone}{\PnlBasis \ptr dest, const \PnlBasis \ptr src}
  \sshortdescribe Clone \var{src} into \var{dest}. The basis \var{dest} must
  already exist before calling this function. On exit, \var{dest} and \var{src}
  are identical and independent.
\item \describefun{\PnlBasis\ptr }{pnl_basis_copy}{const \PnlBasis \ptr B}
  \sshortdescribe Create a copy of \var{B}.

\item  \describefun{void}{pnl_basis_free}{\PnlBasis \ptr\ptr basis}
  \sshortdescribe Free a \PnlBasis created by
  \reffun{pnl_basis_create}. Beware that \var{basis} is the address of a
  \PnlBasis\ptr.

\item \describefun{void}{pnl_basis_del_elt}{\PnlBasis \ptr B, const \PnlVectInt \ptr d}
  \sshortdescribe Remove the function defined by the tensor product \var{d} from
  an existing basis \var{B}.

\item \describefun{void}{pnl_basis_del_elt_i}{\PnlBasis \ptr B, int i}
  \sshortdescribe Remove the \var{i-th} element of basis \var{B}.

\item \describefun{void}{pnl_basis_add_elt}{\PnlBasis \ptr B, const \PnlVectInt \ptr d}
  \sshortdescribe Add the function defined by the tensor \var{d} to the Basis \var{B}.

\item \describefun{int}{pnl_basis_type_register}{const char *name, double (*f)(double, int, int, void), double (*Df)(double, int, int, void), double (*D2f)(double, int, int, void), int is_orthogonal}
\sshortdescribe Register a new basis type and return the index to be passed to
\reffun{pnl_basis_create}. The variable \var{name} is a unique
string identifier of the family. The variables \var{f}, \var{Df}, \var{D2f} are
the one dimensional basis functions, its first and second order derivatives.
Each of these functions must return a \var{double} and take two arguments : the
first one is the point at which evaluating the basis functions, the second one
is the index of function. The var{is_orthogonal} if the elements of the basis are orthogonal for the $L^2$ scalar product. Here is a toy example to show how the canonical basis
is registered (this family is actually already available with the id
PNL_BASIS_CANONICAL, so the following example may look a little fake)
\begin{lstlisting}
  double f(double x, int n, int dim, void *params) { return pnl_pow_i(x, n); }
  double Df(double x, int n, int dim, void *params) { return n * pnl_pow_i(x, n-1); }
  double f(double x, int n, int dim, void *params) { return n * (n-1) * pnl_pow_i(x, n-2); }

  int id = pnl_basis_register ("Canonic", f, Df, D2f, PNL_FALSE);
  /*
   * B is the Canonical basis of polynomials with degree less or equal than 2 in dimension 5.
   */
  PnlBasis *B = pnl_basis_create_from_degree (id, 2, 5);
  \end{lstlisting}

\item \describefun{void}{pnl_basis_print}{const \PnlBasis \ptr B}
  \sshortdescribe Print the characteristics of a basis.

\end{itemize}


Functional regression based on a least square approach often leads to ill conditioned linear systems. One way of improving the stability of the system is to use centered and renormalized polynomials so that the original domain of interest $\cD$ (a subset of $\R^d$) is mapped to $[-1,1]^d$. If the domain $\cD$ is rectangular and writes $[a, b]$ where $a,b \in \R^d$, the \emph{reduction} mapping is done by 
\begin{equation}
  \label{basis_reduced}
  x \in \cD \longmapsto \left(\frac{x_i - (b_i+a_i)/2}{(b_i - a_i)/2}
  \right)_{i=1,\cdots,d}
\end{equation}
\begin{itemize}
\item \describefun{void}{pnl_basis_set_domain}{\PnlBasis \ptr B, 
  const \PnlVect \ptr a, const \PnlVect \ptr b}
  \sshortdescribe This function declares \var{B} as a centered and normalized basis
  as defined by Equation~\ref{basis_reduced}. Calling this function is equivalent to
  calling \reffun{pnl_basis_set_reduced} with \var{center=(b+a)/2} and
  \var{scale=(b-a)/2}.
\item \describefun{void}{pnl_basis_set_reduced}{\PnlBasis \ptr B,
  const \PnlVect \ptr center, const \PnlVect \ptr scale}
  \sshortdescribe This function declares \var{B} as a centered and normalized basis using the mapping
  \begin{equation*}
    x \in \cD \longmapsto \left(\frac{x_i - \var{center}_i }{\var{scale}_i}
    \right)_{i=1,\dots,d}
  \end{equation*}
\item \describefun{void}{pnl_basis_reset_reduced}{\PnlBasis \ptr B}
  \sshortdescribe Reset the reduction settings.
\end{itemize}
Note that this renormalization does not apply to the extra functions by \reffun{pnl_basis_add_function} but only to the functions defined by the tensor \var{T}.

It is also possible to apply a non linear map $\varphi_i:\R \to \R$ to the $i-$th coordinate of the input variable before the reduction operation if any. Then, the input variables are transformed according to
\begin{equation}
  \label{basis_non_linear_reduced}
  x \in \cD \longmapsto \left(\frac{\varphi_i(x_i) - (b_i+a_i)/2}{(b_i - a_i)/2}
  \right)_{i=1,\dots,d}
\end{equation}


\begin{itemize}
\item \describefun{void}{pnl_basis_set_map}{\PnlBasis \ptr B, double (\ptr map)(double x, int i, void\ptr params), double (\ptr Dmap)(double x, int i, void\ptr params), double (\ptr D2map)(double x, int i, void\ptr params), void \ptr params, size_t size_params}
\sshortdescribe Define the non linear applied to the input variables before the reduction operation. The parameters \var{Dmap} and \var{D2map} can be \var{NULL}. The three functions \var{map}, \var{Dmap} and \var{D2map} take as second argument the index of the coordinate, while their third argument is used to passe extra parameters defined by \var{params}.
\end{itemize}

\begin{itemize}
\item \describefun{int}{pnl_basis_fit_ls}{\PnlBasis \ptr P, \PnlVect \ptr  coef, \PnlMat \ptr  x, \PnlVect \ptr  y}
  \sshortdescribe Compute the coefficients \var{coef} defined by
  \begin{equation*}
    \var{coef} = \arg\min_\alpha \sum_{i=1}^n
    \left( \sum_{j=0}^{\var{N}} \alpha_j  P_j(x_i) - y_i\right)^2
  \end{equation*}
  where \var{N} is the number of functions to regress upon and $n$ is the number
  of points at which the values of the original function are known. $P_j$ is the
  $j-th$ basis function. Each row of the matrix \var{x} defines the coordinates
  of one point $x_i$. The function to be approximated is defined by the vector
  \var{y} of the values of the function at the points \var{x}.

\item \describefun{double}{pnl_basis_ik_vect}{const \PnlBasis \ptr b, const \PnlVect \ptr x, int i, int k}
  \sshortdescribe An element of a basis writes $\prod_{l=0}^{\var{nb_variates}}
  \phi_l(x_l)$ where the $\phi$'s are one dimensional polynomials. This
  functions computes the therm $\phi_k$ of the \var{i-th} basis function at the
  point \var{x}.
\item \describefun{double}{pnl_basis_i_vect}{const \PnlBasis \ptr b, const \PnlVect \ptr x, int i}
  \sshortdescribe If \var{b} is composed of $f_0, \dots, f_{\var{nb\_func}-1}$,
  then this function returns $f_i(x)$. 

\item \describefun{double}{pnl_basis_i_D_vect}{const \PnlBasis \ptr b, const \PnlVect \ptr x, int i, int j}
  \sshortdescribe If \var{b} is composed of $f_0, \dots, f_{\var{nb\_func}-1}$,
  then this function returns $\partial_{x_{\var{j}}} f_i(x)$.

  
\item \describefun{double}{pnl_basis_i_D2_vect}{const \PnlBasis \ptr b, const \PnlVect \ptr x, int i, int j1, int j2}
  \sshortdescribe If \var{b} is composed of $f_0, \dots, f_{\var{nb\_func}-1}$,
  then this function returns $\partial^2_{x_{\var{j1}}, x_{\var{j2}}}
  f_i(x)$.


\item \describefun{void}{pnl_basis_eval_derivs_vect}{const \PnlBasis \ptr b, const \PnlVect \ptr coef, const \PnlVect \ptr x, double \ptr fx, \PnlVect \ptr Dfx, \PnlMat \ptr D2fx}
  \sshortdescribe Compute the function, the gradient and the Hessian matrix
  of $\sum_{k=0}^n \var{coef}_k  P_k(\cdot)$ at the point \var{x}.
  On output, \var{fx} contains the value of the function, \var{Dfx} its
  gradient and \var{D2fx} its Hessian matrix. This function is optimized and
  performs much better than calling \reffun{pnl_basis_eval},
  \reffun{pnl_basis_eval_D} and \reffun{pnl_basis_eval_D2} sequentially.

\item \describefun{double}{pnl_basis_eval_vect}{const \PnlBasis \ptr basis, const \PnlVect \ptr coef, const \PnlVect \ptr x}
  \sshortdescribe Compute the linear combination of \var{P_k(x)} defined by
  \var{coef}. Given the coefficients computed by the function
  \reffun{pnl_basis_fit_ls}, this function returns $\sum_{k=0}^n
  \var{coef}_k  P_k(\var{x})$.

\item \describefun{double}{pnl_basis_eval_D_vect}{const \PnlBasis \ptr basis, const \PnlVect \ptr coef, const \PnlVect \ptr x, int i}
  \sshortdescribe Compute the derivative with respect to \var{x_i} of the
  linear combination of \var{P_k(x)} defined by \var{coef}. Given the
  coefficients computed by the function \reffun{pnl_basis_fit_ls}, this function
  returns $\partial_{x_i} \sum_{k=0}^n \var{coef}_k  P_k(\var{x})$ The index
  \var{i} may vary between \var{0} and \var{P->nb_variates - 1}.


\item \describefun{double}{pnl_basis_eval_D2_vect}{const \PnlBasis \ptr basis, const \PnlVect \ptr coef, const \PnlVect \ptr x, int i, int j}
  \sshortdescribe Compute the derivative with respect to \var{x_i} of the
  linear combination of \var{P_k(x)} defined by \var{coef}. Given the
  coefficients computed by the function \reffun{pnl_basis_fit_ls}, this function
  returns $\partial_{x_i} \partial_{x_j} \sum_{k=0}^n \var{coef}_k
  P_k(\var{x})$.  The indices \var{i} and \var{j} may vary between \var{0} and
  \var{P->nb_variates - 1}.

\end{itemize}
The following functions are provided for compatibility purposes but are marked as
deprecated. Use the functions with the \verb!_vect! extension.
\begin{itemize}
\item \describefun{double}{pnl_basis_ik}{const \PnlBasis \ptr b, const double \ptr x, int i, int k}
  \sshortdescribe Same as function \reffun{pnl_basis_ik_vect} but takes a
  C array as the point of evaluation.
\item  \describefun{double}{pnl_basis_i}{\PnlBasis \ptr b, double \ptr x, int i}
  \sshortdescribe Same as function \reffun{pnl_basis_i_vect} but takes a
  C array as the point of evaluation.
\item \describefun{double}{pnl_basis_i_D}{ const \PnlBasis \ptr b, const double \ptr x, int i, int j }
  \sshortdescribe Same as function \reffun{pnl_basis_i_D_vect} but takes a
  C array as the point of evaluation.
\item \describefun{double}{pnl_basis_i_D2}{const \PnlBasis \ptr b, const double \ptr x, int i, int j1, int j2}
  \sshortdescribe Same as function \reffun{pnl_basis_i_D2_vect} but takes a
  C array as the point of evaluation.
\item \describefun{double}{pnl_basis_eval}{\PnlBasis \ptr P, \PnlVect\ptr  coef, double \ptr x}
  \sshortdescribe Same as function \reffun{pnl_basis_eval_vect} but takes a
  C array as the point of evaluation.
\item \describefun{double}{pnl_basis_eval_D}{\PnlBasis \ptr P, \PnlVect \ptr  coef, double \ptr x, int i}
  \sshortdescribe Same as function \reffun{pnl_basis_eval_D_vect} but takes a
  C array as the point of evaluation.
\item \describefun{double}{pnl_basis_eval_D2}{\PnlBasis \ptr  P, \PnlVect \ptr  coef, double \ptr x,  int i, int j}
  \sshortdescribe Same as function \reffun{pnl_basis_eval_D2_vect} but takes a 
  C array as the point of evaluation.
\item \describefun{void}{pnl_basis_eval_derivs}{\PnlBasis \ptr P, \PnlVect\ptr coef, double \ptr x, double \ptr fx, \PnlVect \ptr Dfx, \PnlMat \ptr D2fx}
  \sshortdescribe Same as function \reffun{pnl_basis_eval_derivs_vect} but takes a
  C array as the point of evaluation.
\end{itemize}


% vim:spelllang=en:spell:

