\section{Fast Fourier Transform}
\subsection{Overview}

The coefficients of the Fourier transform of a rela function satisfy the
following relation
\begin{equation}
  \label{eq:fft-sym}
  z_k = \overline{z_{N-k}},
\end{equation}
where $N$ is the number of discretization points.

A few remarks on the FFT of real functions and its inverse transform:
\begin{itemize}
\item We only need half of the coefficients.
\item When a value is known to be real, its imaginary part is not stored.
  So the imaginary part of the zero-frequency component is never stored as it is
  known to be zero.
\item For a sequence of even length the imaginary part of the frequency
  $n/2$ is not stored either, since the symmetry (\ref{eq:fft-sym}) implies
  that this is purely real too.
\end{itemize}


\paragraph{FFTPack storage}
\label{sec:fftpack-storage}

The functions use the fftpack storage convention for half-complex sequences.
In this convention, the half-complex transform of a real sequence is stored
with frequencies in increasing order, starting from zero, with the real and
imaginary parts of each frequency in neighboring locations.

The storage scheme is best shown by some examples. The table below shows the
output for an odd-length sequence, $n=5$.  The two columns give the
correspondence between the $5$ values in the half-complex sequence (stored in
a PnlVect $V$) and the values (PnlVectComplex $C$) that would be returned if
the same real input sequence were passed to pnl_dft_complex as a complex
sequence (with imaginary parts set to 0),
\begin{equation}
  \begin{array}{l}
    C(0) =  V(0) + \imath 0, \\
    C(1) =  V(1) + \imath V(2), \\
    C(2) =  V(3) + \imath V(4), \\
    C(3) = V(3) - \imath V(4)=  \overline{C(2)} , \\
    C(4) = V(1) + \imath V(2)=  \overline{C(1)}
  \end{array}
\end{equation}

The elements of index greater than $N/2$ of the complex array, as $C(3)$
$C(4)$, are filled in using the symmetry condition.

The next table shows the output for an even-length sequence, $n=6$.
In the even case there are two values which are purely real,
\begin{equation}
  \begin{array}{l}
    C(0) =  V(0) + \imath 0, \\
    C(1) =  V(1) + \imath V(2), \\
    C(2) =  V(3) + \imath V(4), \\
    C(3) = V(5) - \imath 0    =  \overline{C(0)} , \\
    C(4) = V(3) - \imath V(4) =  \overline{C(2)} , \\
    C(5) = V(1) + \imath V(2) =  \overline{C(1)}
  \end{array}
\end{equation}


\subsection{Functions}

To use the following functions, you should include \verb!pnl/pnl_fft.h!.

The following functions comes from a C version of the Fortran FFTPack library
available on \url{http://www.netlib.org/fftpack}.
\begin{itemize}
\item \describefun{int}{pnl_fft_inplace}{\refstruct{PnlVectComplex} \ptr data}
  \sshortdescribe Computes the FFT of \var{data} in place. The original content
  of \var{data} is lost.

\item \describefun{int}{pnl_ifft_inplace}{\refstruct{PnlVectComplex} \ptr data}
  \sshortdescribe Computes the inverse FFT of \var{data} in place. The
  original content of \var{data} is lost.

\item \describefun{int}{pnl_fft}{const \refstruct{PnlVectComplex} \ptr in,
    \refstruct{PnlVectComplex} \ptr out}
  \sshortdescribe Computes the FFT of \var{in} and stores it into \var{out}.

\item \describefun{int}{pnl_ifft}{const \refstruct{PnlVectComplex} \ptr in,
    \refstruct{PnlVectComplex} \ptr out}
  \sshortdescribe Computes the inverse FFT of \var{in} and stores it into \var{out}.

\item \describefun{int}{pnl_fft2}{double \ptr re, double \ptr im, int n}
  \sshortdescribe Computes the FFT of the vector of length \var{n} whose real
  (resp. imaginary) parts are given by the arrays \var{re}
  (resp. \var{im}). The real and imaginary parts of the FFT are respectively
  stored in \var{re} and \var{im} on output.

\item \describefun{int}{pnl_ifft2}{double \ptr re, double \ptr im, int n}
  \sshortdescribe Computes the inverse FFT of the vector of length \var{n}
  whose real (resp. imaginary) parts are given by the arrays \var{re}
  (resp. \var{im}). The real and imaginary parts of the inverse FFT are
  respectively stored in \var{re} and \var{im} on output.

\item \describefun{int}{pnl_real_fft}{const \refstruct{PnlVect} \ptr in,
    \refstruct{PnlVectComplex} \ptr out}
  \sshortdescribe Computes the FFT of the real valued sequence \var{in} and
  stores it into \var{out}.

\item \describefun{int}{pnl_real_ifft}{const \refstruct{PnlVect} \ptr in,
    \refstruct{PnlVectComplex} \ptr out}
  \sshortdescribe Computes the inverse FFT of \var{in} and stores it into \var{out}.

\item \describefun{int}{pnl_real_fft_inplace}{double \ptr data, int n}
  \sshortdescribe Computes the FFT of the real valued vector \var{data} of
  length \var{n}. The result is stored in \var{data} using the FFTPack storage
  described above, see~\ref{sec:fftpack-storage}.

\item \describefun{int}{pnl_real_ifft_inplace}{double \ptr data, int n}
  \sshortdescribe Computes the inverse FFT of the vector \var{data} of length
  \var{n}. \var{data} is supposed to be the FFT coefficients a real valued
  sequence stored using the FFTPack storage. On output, \var{data} contains
  the inverse FFT.

\item \describefun{int}{pnl_real_fft2}{double \ptr re, double \ptr im, int n}
  \sshortdescribe Computes the FFT of the real vector \var{re} of length \var{n}.
  \var{im} is only used on output to store the imaginary part the FFT. The
  real part is stored into \var{re}

\item \describefun{int}{pnl_real_ifft2}{double \ptr re, double \ptr im, int n}
  \sshortdescribe Computes the inverse FFT of the vector \var{re + i * im} of
  length \var{n}, which is supposed to be the FFT of a real valued
  sequence. On exit, \var{im} is unused.
\end{itemize}

% vim:spelllang=en:spell:

